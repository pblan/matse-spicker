\section{Punktoperatoren}

\begin{defi}{Punktoperator}
    Als \emph{Punktoperatoren} bezeichnet man eine umfangreiche Klasse von Bildverarbeitungsoperationen in der digitalen Bildverarbeitung, die sich -- gegenüber lokalen und globalen Operatoren -- dadurch auszeichnen, dass bei allen Verfahren dieser Klasse ein neuer Farb- oder Grauwert eines Pixels allein in Abhängigkeit von seinem eigenen bisherigen Farb- oder Grauwert und seiner eigenen bisherigen Position im Bild berechnet wird, ohne sich dabei um seine Nachbarschaft und/oder den Kontext des Pixels zu kümmern.

    Ein Punktoperator $T$ ordnet einem Eingabebild $f$ durch Transformation der Grauwerte der einzelnen Pixel ein Ergebnisbild $f^*$ zu.
    Der Grauwert $f(x,y)$ eines Pixels $(x,y)$ wird dabei nur in Abhängigkeit vom Grauwert selbst und eventuell von der Position des Pixels im Bild modifiziert:
    \[
        f^*(x, y) = T_{xy}(f(x, y))
    \]

    Ist die Transformation von der Position des Pixels im Bild abhängig, so heißt sie \emph{inhomogen}.
    Die Indizes $x$ und $y$ von $T$ sollen diese Abhängigkeit verdeutlichen.

    In der Mehrheit der Fälle kommen jedoch \emph{homogene} Transformationen zum Einsatz, bei denen diese Abhängigkeit nicht gegeben ist.
    Die Indizes werden dann überflüssig:
    \[
        f^*(x, y) = T(f(x, y))
    \]
\end{defi}

\begin{defi}[Homogener Punktoperator]{Helligkeit}
    Um die \emph{Helligkeit} zu verändern, wird zu jedem Pixel ein konstanter Wert $c$ addiert:
    \[
        T_{\text{Brightness}}(f) = f + c
    \]

    TODO: Beispiel

    Eine Änderung der Helligkeit führt zu einer \emph{Histogrammverschiebung}.

    TODO: Beispiel
\end{defi}

\begin{defi}[Homogener Punktoperator]{Kontrast}
    Um den \emph{Kontrast} zu verändern, wird jeder Pixel mit einem konstanten Wert $c$ multipliziert:
    \[
        T_{\text{Contrast}}(f) = f \cdot c
    \]

    TODO: Beispiel

    Eine Änderung des Kontrasts führt zu einer \emph{Histogrammspreizung}.

    TODO: Beispiel
\end{defi}

\begin{defi}[Homogener Punktoperator!Kontrast]{Automatischer Kontrast}
    Die \emph{automatische Kontrastanpassung} soll Pixelwerte so verändern, dass der verfügbare Wertebereich ausgenutzt wird.

    Unter den Annahmen, dass
    \begin{itemize}
        \item der Wertebereich durch $[v_{\min}; v_{\max}]$ beschränkt ist, und
        \item $v_{\low}$ und $v_{\high}$ der kleinste bzw. größte vorhandene Pixelwert sind,
    \end{itemize}
    gilt:
    \[
        T_{\text{Auto-Contrast}}(f) = v_{\min} + (f - v_{\low}) \cdot \frac{v_{\max} - v_{\min}}{v_{\high} - v_{\low}}
    \]

    Diese Variante des automatischen Kontrasts ist stark beeinflussbar durch wenige Pixel, die nicht repräsentativ sind (\enquote{Ausreißer}).

    TODO: Beispiel

    TODO: Beispiel
\end{defi}

\begin{defi}[Homogener Punktoperator!Kontrast]{Verbesserter automatischer Kontrast}
    Die \emph{verbesserte automatische Kontrastanpassung} bestimmt einen Prozentsatz $s_{\low}$ bzw. $s_{\high}$, der Pixel am oberen und unteren Ende in die Sättigung treiben soll.

    Analog zu diesen Perzentilen müssen zwei Schwellwerte $v'_{\low}$ und $v'_{\high}$ gefunden werden:
    \[
        v'_{\low} = \min \{ i \mid H(i) \geq \width \cdot \height \cdot s_{\low} \}
    \]
    \[
        v'_{\high} = \min \{ i \mid H(i) \geq \width \cdot \height \cdot (1 - s_{\high}) \}
    \]
    Werte außerhalb von $v'_{\low}$ und $v'_{\high}$ werden dann auf $v_{\min}$ bzw. $v_{\high}$ abgebildet, Werte dazwischen linear auf das Intervall $[v_{\min}; v_{\max}]$:
    \[
        T_{\text{Improved-Auto-Contrast}}(f) =
        \begin{cases}
            v_{\min}                                                                            & f \leq v'_{\low}           \\
            v_{\min} + (f - v'_{\low}) \cdot \frac{v_{\max} - v_{\min}}{v'_{\high} - v'_{\low}} & v'_{\low} < f < v'_{\high} \\
            v_{\max}                                                                            & f \geq v'_{\high}
        \end{cases}
    \]

    TODO: Beispiel

    TODO: Beispiel
\end{defi}

\begin{defi}[Homogener Punktoperator]{Invertierung}
    Bei der \emph{Invertierung} werden alle Pixel negiert und zum maximalen Wert $c_{\max}$ des Wertebereichs addiert:
    \[
        T_{\text{Invert}}(f) = -f + c_{\max} = c_{\max} - f
    \]

    TODO: Beispiel

    TODO: Beispiel
\end{defi}

\begin{defi}[Homogener Punktoperator]{Thresholding}
    \emph{Schwellwertverfahren} (\emph{Thresholding}) teilen die Pixelwerte abhängig von einem Schwellwert $c$ in zwei Klassen ein
    \[
        T_{\text{Threshold}}(f) =
        \begin{cases}
            v_a & f > c    \\
            v_b & f \leq c
        \end{cases}
    \]

    TODO: Beispiel

    TODO: Beispiel
\end{defi}

\begin{defi}[Homogener Punktoperator]{Transferfunktion}
    TODO

    TODO: Beispiel

    TODO: Beispiel
\end{defi}

\begin{defi}[Homogener Punktoperator]{Windowing}
    TODO

    TODO: Beispiel

    TODO: Beispiel
\end{defi}

\begin{defi}[Homogener Punktoperator]{Histogrammausgleich}
    Das Ziel des \emph{Histogrammausgleichs} ist, eine Punktoperation zu finden und anzuwenden, so dass die Histogramme der veränderten Bilder eine Gleichverteilung approximieren.

    Nur approximieren daher, dass globale Punktoperatoren Histogrammeinträge nur verschieben und mit anderen Einträgen verschmelzen, aber niemals trennen können.
    Dadurch können z. B. individuelle Peaks im Histogramm nicht beseitigt werden.

    Ein Histogrammausgleich kann die Farbzusammensetzung durcheinander bringen.
    Bilder wirken dadurch unnatürlich.

    Ein Histogrammausgleich eignet sich am besten für Grauwertbilder, oder in einem anderen Farbraum (dort nur auf dem Luminanzkanal).

    TODO: Beispiel

    TODO: Beispiel
\end{defi}

\begin{example}[Homogener Punktoperator!Histogrammausgleich]{Gleichverteilung}
    Bei einem \emph{gleichverteilten} Histogramm ist das kumulative Histogramm ungefähr linear.

    Das Problem des Histogrammausgleichs lässt sich also so umformulieren, dass stattdessen ein Punktoperator gefunden werden soll, der das kumulative Histogramm ungefähr linear werden lässt.

    Die Einträge können im kumulativen Histogramm wie folgt nach rechts oder links verschoben werden:
    \[
        T_{\text{Equalize}}(f) = \left\lfloor H(f) \cdot \frac{K - 1}{\width \cdot \height} \right\rfloor
    \]

    TODO: Beispiel

    TODO: Beispiel
\end{example}

\begin{defi}[Inhomogener Punktoperator]{Gammakorrektur}
    TODO

    TODO: Beispiel

    TODO: Beispiel
\end{defi}