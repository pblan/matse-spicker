\section{Einleitung}

\begin{defi}{Digitale Bildverarbeitung}
    \emph{Digitale Bildverarbeitung} ist das Entwerfen, Entwickeln und/oder Erweitern von Software oder Algorithmen zur Verarbeitung von Bildern mit einem Digitalrechner.

    Dagegen steht die Bildbearbeitung, die das (interaktive) Bearbeiten von Bildern mit Hilfe von Software wie Adobe Photoshop oder GIMP bezeichnet.
\end{defi}

\begin{bonus}[Digitale Bildverarbeitung]{Verwandte Gebiete}
    Verwandte Gebiete zur digitalen Bildverarbeitung sind z. B.:
    \begin{itemize}
        \item \emph{Computergrafik}: Erzeugung von Bildern aus geometrischen Modellen
        \item \emph{Signalverarbeitung}: Verarbeitung von Signalen (z.B. Audio, Video, Sprache, etc.)
        \item \emph{Augmented Reality}: Fusionierung realer und virtueller Bilder
        \item \emph{Virtual Realiity}: Erzeugung von virtuellen Welten
        \item \emph{Algorithmische Geometrie}: Algorithmen und Datenstrukturen zur Verarbeitung geometrischer Objekte
    \end{itemize}

    Generell lassen sich die Gebiete unter dem Oberbegriff \emph{Visual Computing} zusammenfassen.
\end{bonus}

\begin{bonus}[Digitale Bildverarbeitung]{Anwendungsgebiete}
    Digitale Bildverarbeitung kommt z. B. in folgenden Bereichen zum Einsatz:
    \begin{itemize}
        \item \emph{Aufbereitung von Bildmaterial}: z. B. Rauschunterdrückung, Kontrastanpassung, Farbkorrektur, etc.
        \item \emph{Fusionierung von Animationen mit realem Filmmaterial}: z. B. in der Filmindustrie
        \item \emph{Medizinische Bildverarbeitung}: z. B. zur Erkennung von Tumoren, Bestrahlungsplanung, etc.
        \item \emph{Astronomie}: z. B. zur Erkennung von Objekten im Weltall
        \item \emph{Automobilindustrie}: z. B. zur Erkennung von Verkehrszeichen, Fahrspuren, etc.
    \end{itemize}
\end{bonus}

