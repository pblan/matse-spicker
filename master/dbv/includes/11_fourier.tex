\section{Fourier}

\begin{defi}{Diskrete Fourier-Transformation}
    % TODO: https://de.wikipedia.org/wiki/Diskrete_Fourier-Transformation#Diskrete_Fourier-Transformation_(DFT)
    Die \emph{Diskrete Fourier-Transformation} (\emph{DFT}) ist eine Transformation aus dem Bereich der Fourier-Analysis.
    Sie bildet ein zeitdiskretes endliches Signal, das periodisch fortgesetzt wird, auf ein diskretes, periodisches Frequenzspektrum ab, das auch als Bildbereich bezeichnet wird.

    Die DFT besitzt in der digitalen Signalverarbeitung zur Signalanalyse große Bedeutung.
    Hier werden optimierte Varianten in Form der schnellen Fourier-Transformation (engl. \emph{Fast Fourier Transform}, \emph{FFT}) und ihrer Inversen angewandt.

    Die DFT wird in der Signalverarbeitung für viele Aufgaben verwendet, so z. B.
    \begin{itemize}
        \item zur Bestimmung der in einem abgetasteten Signal hauptsächlich vorkommenden Frequenzen,
        \item zur Bestimmung der Amplituden und der zugehörigen Phasenlage zu diesen Frequenzen,
        \item zur Implementierung digitaler Filter mit großen Filterlängen.
    \end{itemize}
\end{defi}

\begin{defi}[Diskrete Fourier-Transformation]{Mathematische Definition}
    % TODO: https://de.wikipedia.org/wiki/Diskrete_Fourier-Transformation#Diskrete_Fourier-Transformation_(DFT)
    Die \emph{Diskrete Fourier-Transformation} verarbeitet eine Folge von Zahlen $a = (a_0, \ldots, a_{N-1})$ die zum Beispiel als zeitdiskrete Messwerte entstanden sind. Dabei wird angenommen, dass diese Messwerte einer Periode eines periodischen Signals entsprechen.

    Das Ergebnis der Transformation ist eine Zerlegung der Folge in harmonische (sinusförmige) Anteile, sowie einen \emph{Gleichanteil} $\hat{a}_0$, der dem Mittelwert der Eingangsfolge entspricht.
    Das Ergebnis nennt man \emph{diskrete Fourier-Transformierte} $\hat{a} = ( \hat{a}_0, \ldots, \hat{a}^{N-1} ) \in \mathbb{C}^N$ von $a$.
    Die Koeffizienten von $\hat{a}$ sind die Amplituden der Zerlegungsanteile.
    Man nennt $\hat{a}_k$ auch \emph{Fourierkoeffizienten} oder \emph{Fourierkomponenten}.

    Üblicherweise wird bei der Bestimmung der Frequenzanteile bzw. Phasenlage die kompakte mathematische Schreibweise der Polarform verwendet (Eulersche Formel):
    \[
        e^{i\phi} = \cos(\phi) + i \sin(\phi)
    \]

    Die Fourierkoeffizienten $\hat{a}_{k}$ werden damit aus der Eingangsfolge berechnet durch:
    \[
        \hat{a}_{k} = \sum _{j=0}^{N-1} e^{-2\pi i \cdot \frac {jk}{N}} \cdot a_{j}
    \]
\end{defi}
