\section{Fourier}

\begin{defi}{Diskrete Fourier-Transformation}
    % TODO: https://de.wikipedia.org/wiki/Diskrete_Fourier-Transformation#Diskrete_Fourier-Transformation_(DFT)
    Die \emph{Diskrete Fourier-Transformation} (\emph{DFT}) ist eine Transformation aus dem Bereich der Fourier-Analysis.
    Sie bildet ein zeitdiskretes endliches Signal, das periodisch fortgesetzt wird, auf ein diskretes, periodisches Frequenzspektrum ab, das auch als Bildbereich bezeichnet wird.

    Die DFT besitzt in der digitalen Signalverarbeitung zur Signalanalyse große Bedeutung.
    Hier werden optimierte Varianten in Form der schnellen Fourier-Transformation (engl. \emph{Fast Fourier Transform}, \emph{FFT}) und ihrer Inversen angewandt.

    Die DFT wird in der Signalverarbeitung für viele Aufgaben verwendet, so z. B.
    \begin{itemize}
        \item zur Bestimmung der in einem abgetasteten Signal hauptsächlich vorkommenden Frequenzen,
        \item zur Bestimmung der Amplituden und der zugehörigen Phasenlage zu diesen Frequenzen,
        \item zur Implementierung digitaler Filter mit großen Filterlängen.
    \end{itemize}
\end{defi}

\begin{defi}[Diskrete Fourier-Transformation]{Mathematische Definition}
    % TODO: https://de.wikipedia.org/wiki/Diskrete_Fourier-Transformation#Diskrete_Fourier-Transformation_(DFT)
    Die \emph{Diskrete Fourier-Transformation} verarbeitet eine Folge von Zahlen $a = (a_0, \ldots, a_{N-1})$ die zum Beispiel als zeitdiskrete Messwerte entstanden sind. Dabei wird angenommen, dass diese Messwerte einer Periode eines periodischen Signals entsprechen.

    Das Ergebnis der Transformation ist eine Zerlegung der Folge in harmonische (sinusförmige) Anteile, sowie einen \emph{Gleichanteil} $\hat{a}_0$, der dem Mittelwert der Eingangsfolge entspricht.
    Das Ergebnis nennt man \emph{diskrete Fourier-Transformierte} $\hat{a} = ( \hat{a}_0, \ldots, \hat{a}_{N-1} ) \in \mathbb{C}^N$ von $a$.
    Die Koeffizienten von $\hat{a}$ sind die Amplituden der Zerlegungsanteile.
    Man nennt $\hat{a}_k$ auch \emph{Fourierkoeffizienten} oder \emph{Fourierkomponenten}.

    Üblicherweise wird bei der Bestimmung der Frequenzanteile bzw. Phasenlage die kompakte mathematische Schreibweise der Polarform verwendet (Eulersche Formel):
    \[
        e^{i\phi} = \cos(\phi) + i \sin(\phi)
    \]

    Die Fourierkoeffizienten $\hat{a}_{k}$ werden damit aus der Eingangsfolge berechnet durch:
    \[
        \hat{a}_{k} = \sum _{j=0}^{N-1} e^{-2\pi i \cdot \frac {jk}{N}} \cdot a_{j}
    \]

    Die Gleichung kann auch als Matrix-Vektor-Produkt geschrieben werden:
    \[
        \hat{a} = W \cdot a \quad W_{k, j} = e^{-2\pi i \cdot \frac {jk}{N}}
    \]
    Die symmetrische Transformationsmatrix $W$ mit Dimension $N \times N$ wird Fourier-Matrix genannt.
\end{defi}

\begin{bonus}{Nyquist-Frequenz}
    % TODO: https://de.wikipedia.org/wiki/Nyquist-Frequenz 
    Die \emph{Nyquist-Frequenz} ist ein Begriff aus der Signaltheorie.

    Sie ist definiert als die halbe Abtastfrequenz eines zeitdiskreten Systems:
    \[
        f_{\text{nyquist}} = \frac {1}{2} \cdot f_{\text{abtast}} > f_{\text{signal}}
    \]

    Falls dieses Kriterium nicht eingehalten wird, entstehen nichtlineare Verzerrungen, die auch als Alias-Effekt bezeichnet werden. Diese lassen sich nicht wieder herausfiltern.
    Die untere Grenze für eine Alias-freie Abtastung wird auch als Nyquist-Rate bezeichnet.
\end{bonus}

\begin{defi}{Inverse Diskrete Fourier-Transformation}
    % TODO: https://de.wikipedia.org/wiki/Diskrete_Fourier-Transformation#Diskrete_Fourier-Transformation_(DFT)
    Mit der \emph{inversen DFT}, kurz \emph{iDFT} kann aus den Frequenzanteilen das Signal im Zeitbereich rekonstruiert werden.

    Durch Kopplung von DFT und iDFT kann ein Signal im Frequenzbereich manipuliert werden, wie es beim Equalizer angewandt wird.

    Die Diskrete Fourier-Transformation ist von der verwandten Fouriertransformation für zeitdiskrete Signale zu unterscheiden, die aus zeitdiskreten Signalen ein kontinuierliches Frequenzspektrum bildet.
\end{defi}

\begin{defi}[Inverse Diskrete Fourier-Transformation]{Mathematische Definition}
    % TODO: https://de.wikipedia.org/wiki/Diskrete_Fourier-Transformation#Diskrete_Fourier-Transformation_(DFT)
    Die Summe der sinusförmigen Zerlegungsanteile ergibt wiederum die ursprüngliche Eingangsfolge $a$.

    Dafür wird das Transformationsergebnis $\hat{a}$ als Koeffizienten eines Polynoms verwendet, wobei $\hat{a}_k$ die Amplituden von den zugehörigen harmonischen Schwingungen $z_k$ darstellen:
    \[
        A(z_{k}) = \frac{1}{N} \left({\hat{a}}_{0}z_{k}^{0}+{\hat {a}}_{1}z_{k}^{1}+\ldots +{\hat {a}}_{N-1}z_{k}^{N-1}\right)
    \]

    Die Argumente $z_0, z_1, \ldots, z_{N-1}$ sind $N$ gleich verteilte Punkte auf dem Einheitskreis der komplexen Zahlenebene, d. h. die $N$-ten Einheitswurzeln
    \[
        z_{k} = e^{{\frac{2\pi i }{N}}k} = \cos \left({\frac{2\pi}{N}}k\right)+i \sin \left({\frac{2\pi}{N}}k\right)
    \]

    Die Koeffizienten der ursprünglichen Folge $a$ lassen sich mit der iDFT aus den Fourierkoeffizienten $\hat{a}_{j}$ bestimmen:
    \[
        a_k = \frac{1}{N} \sum _{j=0}^{N-1} e^{2\pi i \cdot \frac {jk}{N}} \cdot \hat{a}_{j}
    \]

    In der Schreibweise als Matrix-Vektor-Produkt:
    \[
        a = W^{-1} \cdot  \hat{a} \quad W^{-1}_{k, j} = \frac{1}{N} e^{2\pi i \cdot \frac {jk}{N}}
    \]
    wobei hier $\hat {a}$ mit der inversen Matrix von $W$ multipliziert wird.
\end{defi}

\begin{defi}{Mehrdimensionale Diskrete Fourier-Transformation}
    % TODO: https://de.wikipedia.org/wiki/Diskrete_Fourier-Transformation
    Die Diskrete Fourier-Transformation kann leicht auf \emph{mehrdimensionale Signale} erweitert werden.
    Sie wird dann je einmal auf alle Koordinatenrichtungen angewendet.

    Im wichtigen Spezialfall von zwei Dimensionen (Bildverarbeitung) gilt etwa:
    \[
        \hat{a}_{k,l} = \sum_{m=0}^{M-1} \sum_{n=0}^{N-1} a_{m,n} \cdot e^{-2\pi i \cdot \frac {mk}{M}} e^{-2\pi i \cdot \frac {nl}{N}}, \quad \text{für} \; k = 0, \ldots, M-1 \; \text{und} \;  l = 0, \ldots, N-1
    \]
    Die Rücktransformation lautet entsprechend:
    \[
        \hat{a}_{k,l} = \frac{1}{MN} \sum_{k=0}^{M-1} \sum_{l=0}^{N-1} \hat{a}_{k,l} \cdot e^{2\pi i \cdot \frac {mk}{M}} e^{2\pi i \cdot \frac {nl}{N}}, \quad \text{für} \; m = 0, \ldots, M-1 \; \text{und} \;  n = 0, \ldots, N-1
    \]
\end{defi}

\begin{bonus}{Fast Fourier-Transformation}
    % TODO: https://de.wikipedia.org/wiki/Schnelle_Fourier-Transformation
    Die \emph{schnelle Fourier-Transformation} (englisch \emph{Fast Fourier Transform}, daher meist \emph{FFT} abgekürzt) ist ein Algorithmus zur effizienten Berechnung der diskreten Fourier-Transformation (DFT).
    Mit ihr kann ein zeitdiskretes Signal in seine Frequenzanteile zerlegt und dadurch analysiert werden.

    Analog gibt es für die diskrete inverse Fourier-Transformation die inverse schnelle Fourier-Transformation (IFFT).

    Es kommen bei der IFFT die gleichen Algorithmen, aber mit konjugierten Koeffizienten zur Anwendung.
\end{bonus}

TODO: Beispiele

