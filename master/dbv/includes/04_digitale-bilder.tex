\section{Digitale Bilder}

\begin{defi}{Bild}
    Ein \emph{Bild} ist eine Abbildung $f$ von räumlichen Koordinaten $\mathcal{P}$ auf einen Wertebereich $\mathcal{C}$
    \[
        f: \mathcal{P} \to \mathcal{C}
    \]

    Ein $n$-dimensionales \emph{digitales Bild} mit $m$ Komponenten hat endliche diskrete Bereiche
    \[
        \mathcal{P} \subseteq \mathbb{N}^n, \quad \mathcal{C} \subseteq \mathbb{W}^m, \quad \mathbb{W} \subseteq \{ x \mid 0 \leq x \leq 2^b - 1 \}
    \]

    \begin{tabularx}{\linewidth}{lXl}
        \toprule
        $(n, m, b)$  & Wertebereich                                                                                                                               & Beispiel                 \\
        \midrule
        $(2, 1, 1)$  & $\mathcal{P} \subseteq \mathbb{N}^2, \quad \mathcal{C} \subseteq \mathbb{W}, \quad \mathbb{W} \subseteq \{ x \mid 0 \leq x \leq 1 \}$      & Binärbild                \\
        $(2, 1, 8)$  & $\mathcal{P} \subseteq \mathbb{N}^2, \quad \mathcal{C} \subseteq \mathbb{W}, \quad \mathbb{W} \subseteq \{ x \mid 0 \leq x \leq 255 \}$    & Graustufenbild           \\
        $(3, 1, 16)$ & $\mathcal{P} \subseteq \mathbb{N}^3, \quad \mathcal{C} \subseteq \mathbb{W}, \quad \mathbb{W} \subseteq \{ x \mid 0 \leq x \leq  65535 \}$ & Typisch für CT           \\
        $(2, 3, 8)$  & $\mathcal{P} \subseteq \mathbb{N}^2, \quad \mathcal{C} \subseteq \mathbb{W}^3, \quad \mathbb{W} \subseteq \{ x \mid 0 \leq x \leq 255 \}$  & True Color Fotos         \\
        $(2, 4, 8)$  & $\mathcal{P} \subseteq \mathbb{N}^2, \quad \mathcal{C} \subseteq \mathbb{W}^4, \quad \mathbb{W} \subseteq \{ x \mid 0 \leq x \leq 255 \}$  & Zusätzlicher Alpha-Kanal \\
        \bottomrule
    \end{tabularx}
\end{defi}

\begin{defi}{Tagged Image File Format (TIFF)}
    TODO
\end{defi}

\begin{defi}{Graphics Interchange Format (GIF)}
    TODO
\end{defi}

\begin{defi}{Windows Bitmap (BMP)}
    TODO
\end{defi}

\begin{defi}{Joint Photographic Expert Group (JPEG)}
    TODO
\end{defi}

\begin{defi}{Portable Network Graphics (PNG)}
    TODO
\end{defi}

\begin{defi}{Pixel}
    Die einzelnen Elemente eines 2D-Bildes heißen \emph{Pixel}.
\end{defi}

\begin{defi}{Voxel}
    Die einzelnen Elemente eines 3D-Bildes heißen \emph{Voxel}.
\end{defi}

\begin{defi}{Bildgröße}
    Die \emph{Bildgröße} wird in Pixel (2D) bzw. Voxel (3D) angegeben.
\end{defi}

\begin{defi}{Bildauflösung}
    Die \emph{Bildauflösung} im physikalischen Sinne wird als \emph{Punktdichte} angegeben.

    Die Punktdichte bestimmt die Ausgabegröße in Abhängigkeit der Bildgröße:
    \[
        \si{\ppi} = \frac{\text{Input Size} \ [\si{\pixel}]}{\text{Output Size} \ [\si{\cm}]} \cdot \SI{2,54}{\cm}
    \]
\end{defi}

\begin{defi}[Bild]{Nachbarschaft}
    Oftmals werden in der Bildverarbeitung nur bestimmte Nachbarn eines Pixels betrachtet.

    Die \emph{4-Nachbarschaft} $N_4(p)$ eines Pixels $p$ sind die vertikalen und horizontalen Nachbarn.

    Die \emph{diagonalen Nachbarn} eines Pixels $p$ heißen $N_D(p)$.

    Die \emph{8-Nachbarschaft} $N_8(p)$ eines Pixels $p$ besteht aus $N_4(p)$ und $N_D(p)$.

    TODO: Grafik
\end{defi}


\begin{bonus}[Bild]{Rechenoperationen}
    Digitale Bilder der Höhe $m$ und der Breite $n$ können als Matrizen oder Arrays $M$ interpretiert werden:
    \[
        M_{m.n} =
        \begin{pmatrix}
            m_{0, 0}   & m_{0, 1}   & \cdots & m_{0, n-1}   \\
            m_{1, 0}   & m_{1, 1}   & \cdots & m_{1, n-1}   \\
            \vdots     & \vdots     & \ddots & \vdots       \\
            m_{m-1, 0} & m_{m-1, 1} & \cdots & m_{m-1, n-1}
        \end{pmatrix}
    \]

    Entsprechend können auch elementweise arithmetische Operationen (z. B. Addition, Subtraktion, Multiplikation, Division) durchgeführt werden.

    Konkret beachten muss man, dass \emph{Datentypen} eingehalten werden, durch z. B.
    \begin{itemize}
        \item Konvertierungen in einen passenden Datentypen,
        \item Clipping oder
        \item andere Rechenvorschriften.
    \end{itemize}
\end{bonus}
