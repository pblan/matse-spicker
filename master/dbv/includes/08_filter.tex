\section{Filter}

\begin{defi}{Filter}
    % TODO: https://de.wikipedia.org/wiki/Bildverarbeitung#Operationen_der_Bildverarbeitung 
    \emph{Nachbarschaftsoperationen} (\emph{Filter}) verwenden sowohl einen Punkt als auch eine bestimmte Menge seiner Nachbarn als Eingabe, errechnen aus ihnen ein Ergebnis und schreiben dieses an die Koordinate des Referenzpunktes in das Zielbild.

    Eine sehr verbreitete Art von Nachbarschaftsoperationen sind die Faltungsfilter (\emph{Convolutional Filter}).
    Hierbei werden die Helligkeit- oder Farbwerte gemäß einem Filterkern miteinander verrechnet um das Ergebnis zu bilden.

    Bei Filtern wird ein Pixel für die Berechnung von mehr als einem neuen Pixel benötigt -- im Kontrast zu Punktfiltern.
    Daher benötigt es einer Kopie.

    Filterkerne haben üblicherweise eine ungerade Anzahl an $(2k + 1)$ Spalten und $(2l + 1)$ Reihen ($k, l \in \mathbb{N} \setminus \{0\}$).
    Für ein Bild der Größe $m \times n$ kann ein Filter für die Bildkoordinaten $(x, y)$ berechnet werden, wenn
    \begin{itemize}
        \item $k \leq x \leq m - k - 1$ und
        \item $l \leq y \leq n - l - 1$
    \end{itemize}

    TODO: Linear vs. nicht-linear
\end{defi}

\begin{defi}[Filter]{Glättung}
    Bei einem \emph{Glättungsfilter} werden lokale Intensitäten \enquote{geglättet}.

    TODO
\end{defi}

\begin{defi}[Filter]{Ableitung}
    Bei \emph{Differenzfiltern} (\emph{Ableitung}) sind einige Koeffizienten negativ.

    Das Ergebnis kann interpretiert werden als Summe von allen Pixeln mit korrespondierenden positiven Koeffizienten minus Summe von allen Pixeln mir korrespondierenden negativen Koeffizienten -- analog zur Berechnung von Ableitungen.

    Da Bilder keine kontinuierlichen Funktionen sind, werden Ableitung mit Finiten Differenzen berechnet.

    TODO
\end{defi}

\begin{example}[Filter!Ableitung]{Laplace-Operator}
    Der einfachste isotropische 2. Ableitungsoperator ist der \emph{Laplace-Operator}:
    \[
        \nabla^2 f = \frac{\partial^2 f}{\partial x^2} + \frac{\partial^2 f}{\partial y^2}
    \]

    Nach Diskretisierung mit finiten Differenzen erhält man:
    \begin{alignat*}{2}
        \frac{\partial^2 f}{\partial x^2} & = f(x + 1, y) + f(x - 1, y) - 2 f(x, y)                           \\
        \frac{\partial^2 f}{\partial y^2} & = f(x, y + 1) + f(x, y - 1) - 2 f(x, y)                           \\
        \nabla^2 f                        & = f(x + 1, y) + f(x - 1, y) f(x, y + 1) + f(x, y - 1) - 4 f(x, y)
    \end{alignat*}

    TODO
\end{example}

\begin{defi}[Filter]{Glättung}
    Bei einem \emph{Glättungsfilter} werden lokale Intensitäten \enquote{geglättet}.

    TODO
\end{defi}

\begin{defi}[Filter]{Unscharfmaskierung}
    % TODO: https://de.wikipedia.org/wiki/Unscharfmaskierung 
    \emph{Unscharfmaskierung} (\enquote{Selektive Schärfe}) bezeichnet eine Filtermethode, bei der der Schärfeeindruck eines Fotos mit Hilfe einer unscharfen Kopie dieses Fotos erhöht wird.

    TODO
\end{defi}

\begin{defi}{Separierbarkeit}
    \emph{Separierbarkeit} bezeichnet die Möglichkeit einen 2D Filterkern in zwei 1D Filterkerne zu trennen.

    Ein 2D Filterkern ist separierbar, wenn er als dyadisches Produkt ausgedrückt werden kann:

    TODO
\end{defi}

\begin{defi}{Randbehandlung}
    Ohne spezielle \emph{Randbehandlung} würde jeder Filter die Größe des Bildes reduzieren.

    Es existieren verschiedene Methoden:
    \begin{itemize}
        \item Pixel am Rand, die nicht verarbeitet wurden, auf einen konstanten Wert setzen (z. B. schwarz).
              \begin{itemize}
                  \item einfache Methode
                  \item Bildgröße wird reduziert
              \end{itemize}
        \item Pixel am Rand, die nicht verarbeitet wurden, auf den Originalwert des Bildes setzen.
              \begin{itemize}
                  \item einfache Methode
                  \item sichtbarer Unterschied
              \end{itemize}
        \item Bild um zusätzliche Pixel erweitern, dann normal filtern und anschließend beschneiden.
              \subitem Erweitern mit:
              \begin{itemize}
                  \item konstantem Wert
                        \begin{itemize}
                            \item sichtbare Artefakte
                        \end{itemize}
                  \item kopierten Randpixeln
                        \begin{itemize}
                            \item geringe Artefakte
                            \item oft genutzt
                        \end{itemize}
                  \item gespiegeltem Bild
                        \begin{itemize}
                            \item nur bei großen Kernen Vorteile
                        \end{itemize}
                  \item periodischer Fortsetzung des Bildes
                        \begin{itemize}
                            \item sichtbare Artefakte
                            \item theoretische Vorteile bzgl. Spektralanalyse
                        \end{itemize}
              \end{itemize}
    \end{itemize}

    TODO
\end{defi}

\begin{defi}[Nicht-linearer Filter]{Minimum- und Maximum-Filter}
    \emph{Minimum-} und \emph{Maximum-Filter} ersetzen den aktuellen Pixel mit dem Minimum bzw. Maximum in der Filterregion.

    TODO: Salt and Pepper Noise
\end{defi}

\begin{defi}[Nicht-linearer Filter]{Median-Filter}
    Ein \emph{Median-Filter} ersetzt den aktuellen Pixel mit dem Median der Filterregion.

    Median-Filter eignen sich gut, um Salt- and Pepper-Rauschen zu entfernen, sind aber weniger gut bei Gaußschem Rauschen.
    Außerdem glätten sie das Bild im Vergleich zu linearen Filtern viel weniger und erhalten Kanten.

    TODO: Salt and Pepper Noise
\end{defi}