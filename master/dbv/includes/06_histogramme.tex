\section{Histogramme}

\begin{defi}{Histogramm}
    % TODO: https://de.wikipedia.org/wiki/Histogramm#Histogramm_in_der_Bildverarbeitung 
    In der digitalen Bildverarbeitung versteht man unter einem \emph{Histogramm} die statistische Häufigkeit der Grauwerte bzw. der Farbwerte in einem Bild.

    Das Histogramm eines Bildes erlaubt eine Aussage über die vorkommenden Grau- bzw. Farbwerte und über Kontrastumfang und Helligkeit des Bildes.

    In einem farbigen Bild kann entweder ein Histogramm über alle möglichen Farben oder Histogramme über die einzelnen Farbkanäle erstellt werden.
    Letzteres ist meist sinnvoller, da die meisten Verfahren auf Grauwertbildern basieren und so die sofortige Weiterverarbeitung möglich ist.

    Die Anzahl der Farbkanäle in einem Bild ist abhängig vom Modus, das heißt pro Farbauszug gibt es einen Kanal.
    Daher haben CMYK-Bilder vier Farbkanäle, RGB-Farbbilder nur drei.
\end{defi}

\begin{defi}{Grauwert-Histogramm}
    Ein \emph{Grauwert-Histogramm} ist eine visuelle Darstellung der Häufigkeitsverteilung der Intensitätswerte.


\end{defi}

\begin{defi}[Histogramm!Interpretation]{Belichtung}

\end{defi}

\begin{defi}[Histogramm!Interpretation]{Kontrast}

\end{defi}

\begin{defi}[Histogramm!Interpretation]{Kontrastumfang}

\end{defi}

\begin{defi}[Histogramm]{Mittelwert}

\end{defi}

\begin{example}[Histogramm]{Mittelwert}

\end{example}

\begin{defi}[Histogramm]{Standardabweichung}

\end{defi}

\begin{example}[Histogramm]{Standardabweichung}

\end{example}

\begin{defi}{Farb-Histogramm}

\end{defi}

\begin{defi}{Kumulatives Histogramm}

\end{defi}

