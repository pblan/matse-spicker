\section{Morphologische Filter}

\begin{defi}{Morphologische Bildverarbeitung}
    % TODO: https://de.wikipedia.org/wiki/Mathematische_Morphologie 
    Die \emph{morphologische Bildverarbeitung} ist ein Teilgebiet der computergestützten Bildverarbeitung und kann als Technik zur Analyse von Strukturen in Bildern verstanden werden.

    Morphologie ist die Lehre der Gestalt oder der Form.
    Diese nichtlineare Bildverarbeitungsmethode dient dazu, die Struktur von Bildern zu analysieren und zu beeinflussen.
    Sie ist ein Konzept, das auf der Mengenlehre, der Topologie und der Verbandstheorie basiert.

    Es sind sowohl Binär- als auch Grauwertbilder zulässig, da auch Binärbilder bereits die Form und Gestalt eines Objektes wiedergeben können.

    Ein Ziel der morphologischen Bildverarbeitung kann einerseits ein neues Bild sein, das Relevantes hervorhebt.
    Ein weiteres Ziel kann eine Liste sein, die mit aus dem Bild bestimmten Messgrößen gefüllt wird.
\end{defi}

\begin{defi}{Morphologischer Filter}
    \emph{Morphologische Filter} greifen in die Struktur und Form von Bildern ein.

    Im Kontext der morphologischen Filter werden Binärbilder als Menge von Koordinaten (\emph{Punktmengen}) interpretiert:
    \[
        Q_I = \{ p \mid I(p) = 1 \}
    \]

    Das \emph{strukturierende Element} ist eine Menge, die nur die Werte $0$ und $1$ enthält:
    \[
        S(x, y) \in \{ 0; 1 \}
    \]

\end{defi}

\begin{defi}[Morphologischer Filter]{Dilatation}
    % TODO: https://de.wikipedia.org/wiki/Dilatation_(Bildverarbeitung) 
    \emph{Dilatation} (von lat.: dilatare = ausdehnen, erweitern) ist eine Basisoperation der morphologischen Bildverarbeitung.
    Die entgegengesetzte Operation ist die Erosion.

    In ihrer einfachsten Variante ersetzt sie jeden Bildpunkt durch das hellste Pixel innerhalb einer gewissen Umgebung, was dazu führt, dass helle Bereiche des Bilds vergrößert werden und dunkle verkleinert.
    In der digitalen Bildverarbeitung wird die Dilatation im Allgemeinen mittels eines strukturierenden Elements angewandt:
    \[
        I \oplus S = \{ (p + q) \mid \forall p \in I, \forall q \in S \}
    \]
    Interpretation: Wenn man den Ursprung von $S$ auf einer Position $p$ in $I$ platziert, dann werden überall dort (neue) Positionen erzeugt, wo ein Punkt in $S$, aber nicht in $I$, enthalten ist.

    Die Dilatation:
    \begin{itemize}
        \item ist kommutativ: $I \oplus S = S \oplus I$
        \item ist assoziativ: $(I_1 \oplus I_2) \oplus I_3 = I_1 \oplus (I_2 \oplus I_3)$
        \item hat ein neutrales Element: $I \oplus \delta = \delta \oplus I = I$
    \end{itemize}

    TODO: Grafik
\end{defi}

\begin{defi}[Morphologischer Filter]{Erosion}
    % TODO: https://de.wikipedia.org/wiki/Erosion_(Bildverarbeitung)
    \emph{Erosion} (von lat.: erodere = abnagen) ist eine Basisoperation der morphologischen Bildverarbeitung.
    Die entgegengesetzte Operation ist die Dilatation.

    \[
        I \ominus S = \{ p \in \mathbb{Z}^2 \mid (p + q) \in I, \forall q \in S \}
    \]
    Interpretation: Eine Position $p$ ist im Ergebnis enthalten, wenn das strukturierende Element $S$ vollständig in $I$ enthalten ist, wenn man den Ursprung von $S$ an der Position $p$ platziert.

    Die Erosion ist nicht kommutativ.

    TODO: Grafik
\end{defi}

\begin{bonus}{Eigenschaften von Dilatation und Erosion}
    Die Dilatation des Vordergrunds ist identisch mit der Erosion des Hintergrunds und umgekehrt:
    \[
        I \ominus S = \overline{(\overline{I} \oplus S^*)}
    \]
    \[
        I \oplus S = \overline{(\overline{I} \ominus S^*)}
    \]
\end{bonus}

\begin{defi}[Morphologischer Filter]{Opening}
    % TODO: https://de.wikipedia.org/wiki/Opening_(Bildverarbeitung)
    \emph{Opening} ist eine Basisoperation der morphologischen Bildverarbeitung.

    Das Öffnen dient u. a. der Unterdrückung lokaler Störungen durch helle Bildpunkte oder dem Ausfiltern kleiner Strukturen.

    Opening ist eine Erosion gefolgt von einer Dilatation mit demselben strukturierenden Element:
    \[
        I \circ S = (I \ominus S) \oplus S
    \]

    TODO: Grafik
\end{defi}

\begin{defi}[Morphologischer Filter]{Closing}
    % TODO: https://de.wikipedia.org/wiki/Closing_(Bildverarbeitung)
    \emph{Closing} ist eine Basisoperation der morphologischen Bildverarbeitung.

    Anwendung findet der Operator beim Filtern von Bildern;
    durch das Schließen lassen sich lokal begrenzte dunkle Störungen in einem Bild unterdrücken oder kleine dunkle Strukturen gezielt herausfiltern.

    Closing ist eine Dilatation gefolgt von einer Erosion mit demselben strukturierenden Element:
    \[
        I \bullet S = (I \oplus S) \ominus S
    \]

    TODO: Grafik
\end{defi}

\begin{bonus}{Eigenschaften von Opening und Closing}
    Opening und Closing sind idempotent.

    Das Opening des Vordergrunds ist identisch mit dem Closing des Hintergrunds und umgekehrt:
    \[
        I \circ S = \overline{(\overline{I} \bullet S)}
    \]
    \[
        I \oplus S = \overline{(\overline{I} \circ S)}
    \]
\end{bonus}

\begin{defi}{Grauwert-Morphologie}
    \emph{Grauwert-Morphologie} ist eine Generalisierung der binären Morphologie.

    Morphologische Filter für Grauwertbilder funktionieren auch mit binären Bildern.

    Bei Farbbildern werden die Filter auf jedem Farbkanal separat angewendet.

    Das \emph{strukturierende Element} ist eine reellwertige Funktion:
    \[
        S(x, y) \in \mathbb{R}
    \]

    $0$-Elemente tragen zum Ergebnis bei.
    Es muss daher explizit zwischen $0$ und \enquote{leer} unterschieden werden.
\end{defi}

\begin{defi}[Morphologischer Filter]{Grauwert-Dilatation}
    % TODO: https://de.wikipedia.org/wiki/Dilatation_(Bildverarbeitung) 
    Das Ergebnis einer \emph{Grauwert-Dilatation} ist definiert als das Maximum der Werte des strukturierenden Elements addiert zu den Werten des aktuellen Bildausschnitts:
    \[
        I \oplus S(x, y) = \max_{(i, j) \in S} \{ I(x + i, y + k) + S(i, j) \}
    \]

    TODO: Grafik
\end{defi}

\begin{defi}[Morphologischer Filter]{Grauwert-Erosion}
    % TODO: https://de.wikipedia.org/wiki/Erosion_(Bildverarbeitung)
    Das Ergebnis einer \emph{Grauwert-Erosion} ist definiert als das Minimum der Werte des strukturierenden Elements subtrahiert den Werten des aktuellen Bildausschnitts:
    \[
        I \ominus S(x, y) = \min_{(i, j) \in S} \{ I(x + i, y + k) - S(i, j) \}
    \]

    TODO: Grafik
\end{defi}

\begin{defi}[Morphologischer Filter]{Top-Hat}
    Das Ziel einer \emph{Top-Hat-Transformation} ist das Erkennen oder Entfernen von Strukturen einer bestimmten Größe.

    Eine \emph{White Top-Hat-Transformation} gibt ein Bild zurück mit den Elementen, die:
    \begin{itemize}
        \item \enquote{kleiner} als das strukturierende Element sind und
        \item heller als die Umgebung sind.
    \end{itemize}

    Die White Top-Hat-Transformation ist definiert als:
    \[
        I - (I \circ S)
    \]

    Eine \emph{Black Top-Hat-Transformation} gibt ein Bild zurück mit den Elementen, die:
    \begin{itemize}
        \item \enquote{kleiner} als das strukturierende Element sind und
        \item dunkler als die Umgebung sind.
    \end{itemize}

    Die Black Top-Hat-Transformation ist definiert als:
    \[
        (I \bullet S) - I
    \]

    TODO: Grafik
\end{defi}