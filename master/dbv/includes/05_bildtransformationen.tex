\section{Bildtransformationen}

\begin{defi}{Affine Transformation}
    % TODO: https://de.wikipedia.org/wiki/Affine_Abbildung
    In der Geometrie und in der Linearen Algebra ist eine \emph{affine Abbildung} (auch \emph{affine Transformation} genannt, insbesondere bei einer bijektiven affinen Abbildung) eine Abbildung zwischen zwei affinen Räumen, bei der \emph{Kollinearität}, \emph{Parallelität} und \emph{Teilverhältnisse} bewahrt bleiben oder gegenstandslos werden.

    Präziser formuliert:
    \begin{enumerate}
        \item Die Bilder von Punkten, die auf einer Geraden liegen (d. h. kollinear sind), liegen wieder auf einer Geraden (\emph{Invarianz der Kollinearität}). Dabei können auch alle -- aber dann alle und nicht nur einige -- Punkte einer Geraden auf einen Punkt abgebildet werden.
        \item Die Bilder zweier paralleler Geraden sind parallel, wenn keine der beiden Geraden auf einen Punkt abgebildet wird.
        \item Drei verschiedene Punkte, die auf einer Geraden liegen (kollineare Punkte), werden so abgebildet, dass das Teilverhältnis ihrer Bildpunkte mit dem der Urbildpunkte übereinstimmt -- es sei denn, alle drei werden auf denselben Bildpunkt abgebildet.
    \end{enumerate}

    Eine bijektive affine Abbildung eines affinen Raumes auf sich selbst wird \emph{Affinität} genannt.
\end{defi}

\begin{bonus}[Affine Transformation]{Taxonomie}

\end{bonus}

\begin{defi}{Homogene Koordinaten}
    Zweidimensionale lineare Transformationen lassen sich einheitlich durch eine $3 \times 3$-Matrix darstellen.

    Die Transformation eines Punktes $(x, y)$ zu einem neuen Punkt $(x', y')$ durch eine beliebige Folge von Translationen, Rotationen und Skalierungen wird beschrieben durch:
    \[
        \begin{pmatrix}
            a & b & c \\
            d & e & f \\
            0 & 0 & 1
        \end{pmatrix}
        \begin{pmatrix}
            x \\ y \\ 1
        \end{pmatrix}
        =
        \begin{pmatrix}
            x' \\ y' \\ 1
        \end{pmatrix}
    \]

    Dabei beschreibt die Teilmatrix aus den Komponenten $a$, $b$, $d$ und $e$ Rotation und Skalierung und die Teilmatrix bestehend aus $c$ und $f$ eine Translation.

    Zwei oder mehrere affine Transformationen, die hintereinander ausgeführt werden sollen, lassen sich Verketten (\emph{Konkatenation}) und durch die Anwendung einer Transformationsmatrix ausführen.

    Homogene Koordinaten sind invariant gegenüber Skalierung.
\end{defi}

\begin{defi}[Affine Transformation]{Translation}
    % TODO: https://www.gm.th-koeln.de/~konen/WPF-BV/BV08.pdf 
    Die Position $p'$ eines Pixels $p$ nach einer Verschiebung (\emph{Translation}) um $t_x$ in positive $x$-Richtung und $t_y$ in positive $y$-Richtung wird wie folgt berechnet:
    \[
        p' = p + t =
        \begin{pmatrix}
            x + t_x \\
            y + t_y
        \end{pmatrix}
    \]

    Die Transformationsgleichung für die Translation sieht wie folgt aus:
    \[
        \begin{pmatrix}
            1 & 0 & t_x \\
            0 & 1 & t_y \\
            0 & 0 & 1
        \end{pmatrix}
        \begin{pmatrix}
            x \\ y \\ 1
        \end{pmatrix}
        =
        \begin{pmatrix}
            x' \\ y' \\ 1
        \end{pmatrix}
    \]

    Die Translation ist eine affine Abbildung.

    TODO: Grafik
\end{defi}

\begin{defi}[Affine Transformation]{Rotation}
    % TODO: https://www.gm.th-koeln.de/~konen/WPF-BV/BV08.pdf 
    Die Position $p'$ eines Pixels $p$ nach einer Drehung (\emph{Rotation}) um den Ursprung $(0, 0)$ mit dem Winkel $\alpha$ wird wie folgt berechnet:
    \[
        p' =
        \begin{pmatrix}
            \cos \alpha & -\sin \alpha \\
            \sin \alpha & \cos \alpha
        \end{pmatrix}
        \begin{pmatrix}
            x \\
            y
        \end{pmatrix}
    \]

    Die Transformationsgleichung für die Rotation sieht wie folgt aus:
    \[
        \begin{pmatrix}
            \cos \alpha  & \sin \alpha & 0 \\
            -\sin \alpha & \cos \alpha & 0 \\
            0            & 0           & 1
        \end{pmatrix}
        \begin{pmatrix}
            x \\ y \\ 1
        \end{pmatrix}
        =
        \begin{pmatrix}
            x' \\ y' \\ 1
        \end{pmatrix}
    \]

    Die Rotation ist eine affine Abbildung.

    TODO: Grafik
\end{defi}

\begin{defi}[Affine Transformation]{Skalierung}
    % TODO: https://www.gm.th-koeln.de/~konen/WPF-BV/BV08.pdf 
    Die Position $p'$ eines Pixels $p$ nach einer \emph{Skalierung} relativ zum Ursprung $(0, 0)$, mit dem horizontalen Skalierungsfaktor $s_x$ und dem vertikalen Skalierungsfaktor $s_y$ wird wie folgt berechnet:
    \[
        p' = p \circ s =
        \begin{pmatrix}
            x \cdot s_x \\
            y \cdot s_y
        \end{pmatrix}
    \]

    Die Transformationsgleichung für die Skalierung sieht wie folgt aus:
    \[
        \begin{pmatrix}
            s_x & 0   & 0 \\
            0   & s_y & 0 \\
            0   & 0   & 1
        \end{pmatrix}
        \begin{pmatrix}
            x \\ y \\ 1
        \end{pmatrix}
        =
        \begin{pmatrix}
            x' \\ y' \\ 1
        \end{pmatrix}
    \]

    Die Skalierung ist eine affine Abbildung.

    TODO: Grafik
\end{defi}

\begin{defi}[Affine Transformation]{Scherung}
    % TODO: https://www.gm.th-koeln.de/~konen/WPF-BV/BV08.pdf 
    Die Position $p'$ eines Pixels $p$ nach einer \emph{Scherung} (Transvektion) TODO
    \[
        p' =
        \begin{pmatrix}
            1   & c_x \\
            c_y & 1
        \end{pmatrix}
        \begin{pmatrix}
            x \\
            y
        \end{pmatrix}
    \]

    Die Transformationsgleichung für die Scherung sieht wie folgt aus:
    \[
        \begin{pmatrix}
            1   & c_x & 0 \\
            c_y & 1   & 0 \\
            0   & 0   & 1
        \end{pmatrix}
        \begin{pmatrix}
            x \\ y \\ 1
        \end{pmatrix}
        =
        \begin{pmatrix}
            x' \\ y' \\ 1
        \end{pmatrix}
    \]

    Die Scherung ist eine affine Abbildung.

    TODO: Grafik
\end{defi}

\begin{defi}[TODO]{Forward Mapping}
    TODO

    TODO: Grafik
\end{defi}

\begin{defi}[TODO]{Backward Mapping}
    TODO

    TODO: Grafik
\end{defi}

\begin{defi}[TODO]{Interpolation}
    TODO

    TODO: Grafik
\end{defi}

\begin{defi}{Projektive Transformation}
    \emph{Projektive Abbildungen} sind Abbildungen, welche Geraden in Geraden überführen.

    \begin{itemize}
        \item Lösung als Lineares Gleichungssystem
              \begin{itemize}
                  \item 4 Punktepaare
                  \item Ein Element (unten rechts) wird auf einen festen Wert gesetzt (meist auf 1)
                  \item Lösung des Gleichungssystems nutzt Pseudoinverse
              \end{itemize}
        \item Lösung mit Singulärwertzerlegung (SVD)
              \begin{itemize}
                  \item Optimierungsproblem mit N Punktepaare
                  \item Algebraische Distanz wird minimiert
                  \item SVD wird zur Lösung verwendet
              \end{itemize}
        \item Nicht-lineare Lösung
              \begin{itemize}
                  \item Optimierungsproblem mit N Punktepaaren
                  \item Euklidische Distanzen zw. Punktepaaren werden minimiert
                  \item Lösung wird iterativ gesucht
              \end{itemize}
    \end{itemize}

    TODO: Grafik, Formulierung
\end{defi}