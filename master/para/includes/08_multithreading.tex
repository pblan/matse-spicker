\section{Multithreading}

\begin{defi}{Thread}
    % TODO: https://de.wikipedia.org/wiki/Thread_(Informatik) (Quelle)
    Ein \emph{Thread} bezeichnet einen Ausführungsstrang oder eine Ausführungsreihenfolge in der Abarbeitung eines Programms.

    Ein Thread ist Teil eines Prozesses.
    Threads innerhalb des gleichen Prozesses können verschiedenen Prozessoren zugeordnet sein.

    Jeder Thread besitzt einen eigenen sogenannten \emph{Threadkontext}:
    \begin{itemize}
        \item unabhängiger Registersatz inkl. Befehlszeiger (Instruction Pointer),
        \item einen eigenen Stack, jedoch meist im gemeinsamen Prozess-Adressraum.
    \end{itemize}

    Andere Betriebsmittel werden von allen Threads gemeinsam verwendet.
    Durch die gemeinsame Nutzung von Betriebsmitteln kann es auch zu Konflikten kommen.
    Diese müssen durch den Einsatz von Synchronisationsmechanismen aufgelöst werden.
\end{defi}

\begin{defi}{Multithreading}
    % TODO: https://de.wikipedia.org/wiki/Multithreading (Quelle)
    \emph{Multithreading} bezeichnet das gleichzeitige (oder quasi-gleichzeitige) Abarbeiten mehrerer Threads (Ausführungsstränge) innerhalb eines einzelnen Prozesses oder eines Tasks (ein Anwendungsprogramm).

    Im Gegensatz zum Multitasking, bei dem mehrere unabhängige Programme voneinander abgeschottet quasi-gleichzeitig ausgeführt werden, sind die Threads eines Anwendungsprogramms nicht voneinander abgeschottet und können somit durch sogenannte Race Conditions Fehler verursachen, die durch Synchronisation vermieden werden müssen.

    Ein Vorteil von Multithreading ist, dass die gemeinsame Nutzung des Adressraums zu einem schnelleren \emph{Context Switch} (Umschalten des Prozessors von einem Thread auf den anderen) führt.

    Allerdings erfordert Multithreading eine Synchronisation für den Zugriff auf globale bzw. geteilte Daten.
\end{defi}

\begin{bonus}{Simultaneous Multithreading}
    % TODO: https://de.wikipedia.org/wiki/Simultaneous_Multithreading (Quelle)
    Der Begriff \emph{Simultaneous Multithreading} (\emph{SMT}) bezeichnet die Fähigkeit eines Mikroprozessors, mittels getrennter Pipelines und/oder zusätzlicher Registersätze mehrere Threads gleichzeitig auszuführen.
    Hiermit stellt SMT eine Form des hardwareseitigen Multithreadings dar.

    Die derzeit wohl bekannteste Form des SMT ist Intels Hyper-Threading-Technik (HTT) für Pentium 4, Xeon, Atom und Core i und neuer, aber auch Prozessoren anderer Hersteller verfügen über SMT, z. B. Cell, Power ab POWER5 und POWER6 von IBM und die Prozessorserien von AMD ab der Zen-Architektur, Ryzen und EPYC.
\end{bonus}