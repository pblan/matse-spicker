\section{Multithreading}

\begin{defi}{Thread}
    Ein Thread ist \ldots
    \begin{itemize}[\ldots]
        \item ein Stück Code zusammen mit einem Zustand,
              gespeichert in:
              \begin{itemize}
                  \item Register File
                  \item Stack
                  \item Program pointer
              \end{itemize}
        \item eine \enquote{leichtgewichtige} Einheit für den Ablauf von Befehlen (light-weight)
        \item in vielen Betriebssystemen die kleinste Einheit,
              die verwaltet (schedule) werden kann.
        \item die Basis für ein Programmiermodell bei parallelen Rechnern (SMP-Rechner: OpenMP).
    \end{itemize}
\end{defi}

\begin{defi}{Multithreading}
    \emph{Multithreading} beschreibt viele \enquote{gleichzeitig} verwaltbare und ablaufbare Threads bei \ldots
    \begin{itemize}[\ldots]
        \item gleichzeitig bearbeiteten Programmen (multi-programming)
        \item einem parallelen Programm
        \item verschiedenen, in der Regel unabhängigen, Teilen eines sequentiellen Programms,
              z. B. verschiedene Iterationen einer Schleife
    \end{itemize}
\end{defi}

\begin{defi}{Task vs Thread}
    \emph{Process = Task} $\neq$ \emph{Thread}
    \begin{itemize}
        \item Ein Task enthält mindestens einen Thread
        \item Multiple Threads können \enquote{gleichzeitig} in einem Prozessor bearbeitet werden
        \item Tasks besitzen unabhängige Adressräume (address spaces),
              wohingegen Threads einen Adressraum eines Tasks gemeinsam nutzen
              \begin{itemize}
                  \item \underline{Vorteil:} Die gemeinsame Nutzung des Adressraums führt zu einem schnelleren \enquote{context switch}
                        (Umschalten des Prozessors von einem Thread zum anderen)
                  \item \underline{Nachteil:} Multithreading erfordert Synchronisation für den Zugriff auf globale Daten
              \end{itemize}
    \end{itemize}
\end{defi}

\begin{example}{Threads Anwendung und Einsatz}
    Anwendungen für Threads:
    \begin{itemize}
        \item Web-Server: ein Thread für eine Anfrage
        \item Router: ein Thread für ein Paket
        \item Vordergrund / Hintergrund: ein Thread für ein GUI,
              ein anderer zur Berechnung der Aufgabe
    \end{itemize}
    Library:
    \begin{itemize}
        \item Pthreads (nach einem POSIX-Standard)
        \item Routinen für
              \begin{itemize}
                  \item Thread-Erzeugung, Beendigung
                  \item Wechselseitiger Ausschluss und Semaphoren zur Synchronisation
              \end{itemize}
    \end{itemize}
\end{example}

\begin{defi}[Threads]{Simultaneous Multithreading (SMT)}
    Bei Einprozessor-Systemen: \\
    \begin{tabularx}{\textwidth}{|lX|}
        \toprule
        Fakt:    & Superskalare und VLIW Prozessoren holen (fetch) und führen (execute) mehrere Befehle pro Clock Cycle aus \\
        \midrule
        Problem: & Ein einzelner Thread besitzt nicht genug \enquote{instruction-level parallelism}                         \\
        \midrule
        Lösung:  & Ausführen der Befehle von vielen parallelen Threads in einem Clock Cycle                                 \\
        \midrule
        Vorteil: & Die Leistung (performance) der einzelnen Threads wird nicht von anderen Threads beeinträchtigt           \\
        \bottomrule
    \end{tabularx}
    \\
    $\to$ bei Intel wird SMT \enquote{Hyper-threading} genannt.
\end{defi}

\section{Aufgaben}

\begin{aufgabe}{Multithreading}
    Was ist ein Thread?
    \tcblower
    Ein Thread \ldots
    \begin{itemize}[$\ldots$]
        \item ist ein Stück Code zusammen mit einem Zustand, gespeichert in:
              \begin{itemize}
                  \item Register File
                  \item Stack
                  \item Program pointer
              \end{itemize}
        \item ist eine \enquote{leichtgewichtige} Einheit für den Ablauf von Befehlen (light-weight)
        \item ist in vielen Betriebssystemen die kleinste Einheit, die verwaltet (schedule) werden kann
        \item bildet die Basis für ein Programmiermodell bei parallelen Rechnern (SMP-Rechner: OpenMP)
    \end{itemize}
\end{aufgabe}

\begin{aufgabe}{Multithreading}
    Wodurch unterscheidet sich ein Task (Prozess) von einem Thread?
    \tcblower
    \begin{itemize}
        \item Ein Prozess enthält mindestens einen Thread
        \item Multiple Threads können \enquote{gleichzeitig} in einem Prozessor bearbeitet werden
        \item Prozesse besitzen unabhängige Adressräume (address spaces), wohingegen Threads einen Adressraum eines Prozesses gemeinsam nutzen
    \end{itemize}
\end{aufgabe}