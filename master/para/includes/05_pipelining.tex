
\section{Pipelining}

\begin{defi}{Pipelining}
    Durch Aufteilen einer Instruktion in Teile, 
    die in separaten Funktionseinheiten bearbeitet werden können, 
    ist eine überlappte (pseudo-parallele) Verarbeitung der Instruktionen möglich.
\end{defi}

\begin{defi}[Pipelining]{Funktionseinheiten}
    Eine 5-stufige Pipeline (z.B. MIPS-Prozessor) wird aus folgenden Funktionseinheiten gebildet:
    \begin{center}
        \begin{tabular}{ll}
            IF:  & Instruction Fetch  \\
            ID:  & Instruction Decode \\
            EX:  & Execution          \\
            MEM: & Memory Access      \\
            WB:  & Write Back         \\
        \end{tabular}
    \end{center}
\end{defi}

\begin{defi}[Pipelining]{Latenz}
    Wie groß ist die Laufzeit für eine einzelne Instruktion?
    \[T_\text{instr} = T_\text{IF} + T_\text{ID} + T_\text{EX} + T_\text{MEM} + T_{WB}\]
\end{defi}

\begin{defi}[Pipelining]{Durchsatz}
    Wie oft pro Zeiteinheit wird eine Instruktion fertig ?
    \[X = \frac{1}{\max(T_\text{IF}, T_\text{ID}, T_\text{EX}, T_\text{MEM}, T_{WB})}\]
\end{defi}

\begin{defi}{Superskalarität}
    
\end{defi}

\begin{defi}{Instruction Level Parallelism (ILP)}\label{defi:ilp}
    \emph{Instruction Level Parallelism (ILP)} oder Parallelität auf Anweisungsebene ist die parallele oder gleichzeitige Ausführung einer Folge von Anweisungen in einem Computerprogramm. 
    Genauer gesagt bezieht sich ILP auf die durchschnittliche Anzahl der Anweisungen, die pro Schritt dieser parallelen Ausführung ausgeführt werden.
\end{defi}

\begin{defi}{Hazards}
    \begin{itemize}
        \item \underline{Structural hazards:}
              Die HW kann nicht eine bestimmte Aufeinanderfolge von Instruktionen ausführen.
        \item \underline{Data hazards:}
              Instruktionen hängen vom Ergebnis vorheriger Schritte ab, 
              die noch nicht die Pipeline verlassen haben.
        \item \underline{Control hazards:}
              Zwischen dem \enquote{fetch} von Instruktionen und dem Ändern des Kontrollflusses (branch) entstehen Verzögerungen.
    \end{itemize}
\end{defi}

\begin{defi}{Schleifenparallelisierung}
    Ziel ist es, 
    den Schleifenkörper ohne stottern durch die vorhandenen Funktionsinheiten zu schicken.
    Dazu sind Abhängigkeiten zwischen den Anweisungen zu finden und, 
    wenn möglich, zu beseitigen.
\end{defi}

\begin{defi}{Abhängigkeitsgraph}
    
\end{defi}

\begin{defi}{Predication}
    Mit der Predication-Methode können bestimmte Control Hazards (branches) aufgelöst werden, 
    z. B. 'If-then-else'-Konstrukte
\end{defi}

\begin{defi}{Data Speculation}
    Durch Umstellen der Befehlsreihenfolge (z. B. Laden von Daten) kann Parallelität erhöht werden.
\end{defi}

\begin{defi}{Control Speculation}
    Durch Umstellen der Befehlsreihenfolge (z. B. Ausführung von Operationen) kann Parallelität erhöht werden.
\end{defi}

\begin{defi}{Attached-Processor-Systeme}
    Eine zusätzliche Einheit, 
    die in einer Multiprozessor-Umgebung an die primäre CPU angeschlossen ist. 
    Sie arbeitet als Erweiterung der primären CPU und 
    nutzt die Systemsoftware und Peripheriegeräte gemeinsam.
\end{defi}