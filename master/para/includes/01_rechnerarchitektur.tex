\section{Rechnerarchitekturen}

\subsection{Einführung}

\begin{defi}{Struktur}
    Unter \emph{Struktur} versteht man die Art der Verknüpfung der verschiedenen Hardwarekomponenten eines Rechner miteinander.

    Sie ist in der Regel \emph{statisch}.
\end{defi}

\begin{defi}{Organisation}
    \emph{Organisation} steht für die zeitabhängigen Wechselwirkungen zwischen Komponenten und die Steuerung dieser Komponenten.

    Dise Wechselwirkungen können \emph{dynamisch} sein.
\end{defi}

\begin{defi}{Implementierung}
    \emph{Implementierung} bezeichnet die Ausgestaltung einzelner Bausteine.

    Sie gibt die \emph{Größe} eines Systems an.
\end{defi}

\begin{defi}{Leistung}
    \emph{Leistung} beschreibt das nach außen hin sichtbare Systemverhalten.

    Sie gibt die \emph{Geschwindigkeit} eines Systems an.
\end{defi}

\subsection{Von-Neumann-Rechner}

\begin{defi}{Von-Neumann-Rechner}
    Der \emph{Von-Neumann-Rechner} besteht aus folgenden Werken:
    \begin{itemize}
        \item \emph{Eingabe- bzw. Ausgabewerk:}
              \begin{itemize}
                  \item Schnittstelle zur Außenwelt
              \end{itemize}
        \item \emph{Leitwerk:}
              \begin{itemize}
                  \item interpretiert Befehle
                  \item steuert Abläufe
              \end{itemize}
        \item \emph{Haupt- bzw. Arbeitsspeicher:}
              \begin{itemize}
                  \item Speicher für Daten und Befehle
              \end{itemize}
        \item \emph{Rechenwerk:}
              \begin{itemize}
                  \item führt arithmetische und logische Operationen aus
              \end{itemize}
    \end{itemize}

    Die Struktur des Rechners ist unabhängig von der Aufgabe, die er lösen soll.
    Hardware und Software sind voneinander \emph{getrennt}.
\end{defi}

\begin{defi}[Von-Neumann-Rechner]{Struktur}

\end{defi}

\begin{defi}[Von-Neumann-Rechner]{Organisation}

\end{defi}

\begin{defi}[Von-Neumann-Rechner]{Arbeitsweise}

\end{defi}

\begin{defi}{Flaschenhals}

\end{defi}

\subsection{Rechenleistung}

\begin{defi}{Clock-Rate}

\end{defi}

\begin{defi}{FLOPS}

\end{defi}

\begin{bonus}{Top500}

\end{bonus}

\begin{bonus}{Transistoren und Clock-Rate}

\end{bonus}

\begin{bonus}{Leistungslücke Prozessor-Memory}

\end{bonus}

\begin{defi}{Rekurrenzgleichung}

\end{defi}

\begin{example}{Rekurrenzgleichung}

\end{example}

\begin{defi}{Master-Theorem}

\end{defi}

\begin{example}{Master-Theorem}

\end{example}

