\section{Rechnerarchitekturen}\label{sec:rechnerarchitekturen}

\subsection{Einführung}\label{subsec:einfuehrung}

\begin{defi}{Struktur}
    Unter \emph{Struktur} versteht man die Art der Verknüpfung der verschiedenen Hardwarekomponenten eines Rechner miteinander.
    
    Sie ist in der Regel \emph{statisch}.
\end{defi}

\begin{defi}{Organisation}
    \emph{Organisation} steht für die zeitabhängigen Wechselwirkungen zwischen Komponenten und die Steuerung dieser Komponenten.
    
    Diese Wechselwirkungen können \emph{dynamisch} sein.
\end{defi}

\begin{defi}{Implementierung}
    \emph{Implementierung} bezeichnet die Ausgestaltung einzelner Bausteine.
    
    Sie gibt die \emph{Größe} eines Systems an.
\end{defi}

\begin{defi}{Leistung}
    \emph{Leistung} beschreibt das nach außen hin sichtbare Systemverhalten.
    
    Sie gibt die \emph{Geschwindigkeit} eines Systems an.
\end{defi}

\subsection{Von-Neumann-Rechner}\label{subsec:von-neumann-rechner}

\begin{defi}{Von-Neumann-Rechner}
    Der \emph{Von-Neumann-Rechner} besteht aus folgenden Werken:
    \begin{itemize}
        \item \emph{Eingabe- bzw. Ausgabewerk:}
              \begin{itemize}
                  \item Schnittstelle zur Außenwelt
              \end{itemize}
        \item \emph{Leitwerk:}
              \begin{itemize}
                  \item interpretiert Befehle
                  \item steuert Abläufe
              \end{itemize}
        \item \emph{Haupt- bzw. Arbeitsspeicher:}
              \begin{itemize}
                  \item Speicher für Daten und Befehle
              \end{itemize}
        \item \emph{Rechenwerk:}
              \begin{itemize}
                  \item führt arithmetische und logische Operationen aus
              \end{itemize}
    \end{itemize}
    
    Die Struktur des Rechners ist unabhängig von der Aufgabe, die er lösen soll.
    Hardware und Software sind voneinander \emph{getrennt}.
\end{defi}

\begin{defi}[Von-Neumann-Rechner]{Struktur}
    
\end{defi}

\begin{defi}[Von-Neumann-Rechner]{Organisation}
    
\end{defi}

\begin{defi}[Von-Neumann-Rechner]{Arbeitsweise}
    \begin{itemize}
        \item Das Programm besteht aus einer \emph{Folge von Befehlen} (Instruktion),
              die nacheinander (sequentiell) ausgeführt werden.
        \item Von der Folge kann durch bedingte und unbedingte Sprungbefehle abgewichen werden, die die Programmfortsetzung an einer anderen Zelle bewirken.
              (Die bedingten Sprünge sind von gespeicherten Werten abhängig)
        \item Die Maschine benutzt Binärcodes, Zahlen werden dual dargestellt.
              Befehle und andere Daten müssen geeignet kodiert werden.
              Bitfolgen im Speicher sind nicht selbstidentifizierend.
    \end{itemize}
\end{defi}

\begin{defi}{Flaschenhals}
    \begin{itemize}
        \item Prozessor/Speicherschnittstelle ist kritisch für die Leistung des Gesamtsystems
        \item Daten werden vom Prozessor schneller verarbeitet als sie aus dem Speicher gelesen oder in den Speicher geschrieben werden können
        \item Durch Fortschritte in der Halbleitertechnik wurde die Diskrepanz zwischen Prozessorgeschwindigkeit und Speicherzugriffszeiten immer größer $\to$ \textbf{Flaschenhals (bottleneck)}
    \end{itemize}
\end{defi}

\subsection{Rechenleistung}\label{subsec:rechenleistung}

\begin{defi}{Leistung}
    Physikalisch: $P = \frac{W}{t}$, mit Leistung $P$, Arbeit $W$ und Zeit $t$
\end{defi}

\begin{defi}{Arbeit}
    Die Bearbeitung \ldots
    \begin{itemize}
        \item \ldots einer Instruktion (rechnerabhängig)
        \item \ldots einer Gleitkomma-Operation
        \item \ldots eines standardisierten Programms (Benchmark)
    \end{itemize}
\end{defi}

\begin{defi}[Leistungsmaß]{Clock-Rate}
    Die Taktfrequenz (engl.\ Clock-Rate) misst die Anzahl der Takte oder Zyklen, die von der CPU pro Sekunde durchgeführt werden in GHz (Gigahertz).
\end{defi}

\begin{defi}[Leistungsmaß]{FLOPS}
    Floating Point Operations Per Second
\end{defi}

\begin{bonus}{Top500}
    \begin{itemize}
        \item Erstellung der Liste der 500 schnellsten Rechner der Welt (2x pro Jahr)
        \item Leistungskriterium: Benchmark-Programme aus dem Gebiet der linearen Algebra (Linpack)
    \end{itemize}
    \underline{Aufbau:}\\
    \begin{tabularx}{\textwidth}{l|X}
        Manufacturer      & Manufacturer or vendor                     \\
        Computer Type     & Indicated by manufacturer or vendor        \\
        Installation Site & Customer                                   \\
        Location          & Location and country                       \\
        Year              & Year of installation / last major update   \\
        Customer Segment  & Academic, Research, Industry, Vendor Class \\
        \# Processors     & Number of processors                       \\
        R max             & Maximal LINPACK performance achieved       \\
        R peak            & Theoretical peak performance               \\
        N max             & Problem size for achieving R max           \\
        N 1/2             & Problem size for achieving half of R max   \\
        N world           & Position within the TOP500 ranking         \\
    \end{tabularx}
\end{bonus}

\begin{bonus}{Transistoren und Clock-Rate}
    Die Weiterentwicklung bei der Chipherstellung (Lithographie) führt zu feineren Strukturen auf einem Chip.
    
    Was passiert, wenn Leiterbahnen und Schaltelemente um einen Faktor $x$ schrumpfen ?
    \begin{itemize}
        \item Clock-Rate $f$ wächst um Faktor $x$ (Stromverbrauch, Abwärme $T \sim f^2$)
        \item Die Anzahl der Transistoren pro Fläche wächst mit $x^2$
        \item Die Rechenleistung des Chips wächst mit $x^4$ (aber der Zuwachs um $x^3$ beruht auf Architektur)
    \end{itemize}
\end{bonus}

\begin{defi}{Moore's Law}
    \label{defi:moores_law}
    \emph{Gordon Moore} (Mitbegründer von Intel) sagte 1965 voraus:
    \fbox{Die Anzahl der Transistoren auf einem Halbleiterchip verdoppelt sich ungefähr alle 18 Monate.}
    \[N = a\cdot 10^{\frac{1}{5}\cdot t}\]
\end{defi}

\begin{bonus}{Leistungslücke Prozessor-Memory}
    Aufgrund von~\nameref{defi:moores_law} wächst der \enquote{Performance Gap} (die Lesitungslücke) um ca.\ 50\% pro Jahr,
    da bisher die CPU-Performance um ca.\ 60\% pro Jahr und die DRAM-Performance nur um ca.\ 7\% pro Jahr gewachsen ist.
\end{bonus}
