\section{Mehrprozessorsysteme}

\begin{defi}{Flynn'sche Klassifikation}
    Von Flynn wurde 1972 eine sehr grobe aber heute noch häufig genutzte Unterscheidung von Parallelrechnern eingeführt.
    Sie orientiert sich an der Anzahl der gleichzeitig vorhandenen Instruktions- und Datenströme.
    
    Parallelrechner:
    \begin{itemize}
        \item[SISD] Single Instruction Single Data (Von Neumann-Rechner $\to$ kein paralleles System)
        \item[SIMD] Single Instruction Multiple Data
        \item[MISD] Multiple Instruction Single Data $\to$ \textbf{irrelevant!}
        \item[MIMD] Multiple Instruction Multiple Data
    \end{itemize}
\end{defi}

\begin{bonus}[Flynn'sche Klassifikation]{Erweiterungen}
    \begin{itemize}
        \item[SIMD] Single Instruction Multiple Data
        \begin{itemize}
            \item Vektor-Prozessoren (auch Mischformen mit MIMD vorhanden!)
            \item Array-Prozessoren
        \end{itemize}
        \item[MIMD] Multiple Instruction Multiple Data
        \begin{itemize}
            \item Speichergekoppelte Multiprozessoren
            \begin{itemize}
                \item Unified Memory Architecture (UMA)
                \item Non-Uniform Memory Access (NUMA)
            \end{itemize}
            \item Nachrichtengekoppelte Multiprozessoren
            \begin{itemize}
                \item Massively Parallel Processing (MPP)
                \item Cluster Of Workstations (COW)
            \end{itemize}
        \end{itemize}
    \end{itemize}
\end{bonus}

\subsection{Speichergekoppelte Systeme}

\begin{defi}{Speichergekoppelte Systeme}
    Bei speichergekoppelten Multiprozessoren arbeiten alle Prozessoren in einem einheitlichen Adressraum.
    
    Je nach physiklischer Speicherorganisation unterscheidet man:
    \begin{itemize}
        \item Symmetrische Multiprozessoren (SMP, symmetric multiprocessor), 
        bei denen gleichartige Prozessoren über ein Verbindungsnetzwerk mit einem gemeinsamen Speicher verbunden sind
        \item Distributed-Shared-Memory Systeme, 
        bei denen zwar ein einheitlicher Adressraum existiert, 
        aber die Speicher physikalisch auf einzelnen Verarbeitungsknoten verteilt sind
    \end{itemize}

    \emph{Bemerkungen:}
    \begin{itemize}
        \item Speichergekoppelte Multiprozessoren gelten als einfacher programmierbar gegenüber nachrichtengekoppelten Multiprozessoren
        \item Nutzbare Parallelität reicht von der Programmebene bis zur Blockebene
        \item Autoparallelisierende Compiler nutzen insbesondere die Parallelisierung der Schleifenebene (einzelne Iterationen nebenläufig abarbeitbar)
    \end{itemize}
\end{defi}

\begin{defi}{Uniform Memory Access}
    Kapitel 5 Seite 8
\end{defi}

\begin{defi}{Non-Uniform Memory Access}
    Kapitel 5 Seite 9
\end{defi}

\begin{defi}{Nachrichtengekoppelte Systeme}
    Kapitel 5 Seite 10
\end{defi}

\subsection{Verbindungsnetzwerke}

\begin{defi}{Verbindungsnetzwerk}
     Ein Verbindungsnetzwerk ermöglicht den Datenaustausch (und die Verteilung des Programms) zwischen den Prozessoren.

     Um einen hohen Datentransfer zu erhalten, 
     wird eine große Anzahl von Drähten benötigt!

     Ein VN gleicht einer Straße:
     \begin{itemize}
        \item Link = Straße
        \item Switch = Kreuzung 
        \item Distance/Hops = Anzahl der zurückgelegten Straßenblöcke
        \item Routing algorithm = Reiseplan
     \end{itemize}

\end{defi}

\begin{defi}[Verbindungsnetzwerk]{Latenz}
    Zeit für den Transfer zwischen den Knoten
\end{defi}

\begin{defi}[Verbindungsnetzwerk]{Bandbreite}
    \[\frac{\text{transferierte Daten}}{\text{Zeit}}\]
    Bandbeite eines Link: $\text{bw} = w \cdot \frac{1}{t}\frac{\text{bit}}{\text{sec}}$
    mit $w$: Anzahl der Drähte
\end{defi}

\begin{defi}[Verbindungsnetzwerk]{Durchmesser}
    maximale Distanz zwischen zwei beliebigen Prozessoren
\end{defi}

\begin{defi}[Verbindungsnetzwerk]{Topologie}
    Wie sind die Nachbarknoten angeordnet und erreichbar?
\end{defi}

\subsubsection{Statische Verbindungsnetzwerke}

\begin{defi}{Lineare Anordnung}
    
\end{defi}

\begin{defi}{Ring}
    
\end{defi}

\begin{defi}{Torus}
    
\end{defi}

\begin{defi}{Gitter}
    
\end{defi}

\begin{defi}{Hypercube}
    
\end{defi}

\begin{defi}{Baum}
    
\end{defi}

\subsubsection{Dynamische Verbindungsnetzwerke}

\begin{defi}{Bus}
    
\end{defi}

\begin{defi}{Crossbar}
    
\end{defi}

\begin{defi}{Zeilenbasierte Systeme}
    
\end{defi}

\begin{defi}{Einstufige Netzwerke}
    
\end{defi}

\begin{defi}{Mehrstufige Netzwerke}
    
\end{defi}

\begin{defi}{Omega-Netzwerk}
    
\end{defi}

\begin{defi}{Benes-Netzwerk}
    
\end{defi}

\subsubsection{Cluster-Interconnects}

\begin{defi}{Cluster-Interconnect}
    
\end{defi}

\begin{defi}{Infiniband}
    
\end{defi}

\begin{defi}{Gigabit Ethernet}
    
\end{defi}

\begin{defi}[Verbindungsnetzwerk]{Klassifikation}
    
\end{defi}

