\section{Quantencomputer}

\begin{defi}[Quantencomputer]{Qubit}
    \begin{itemize}
        \item Digitaler Computer: Bit, Zustände: 0 oder 1
        \item Quantencomputer: Quantum Bit = Qubit,
              $\text{Zustände: } \alpha|0\rangle + \beta|1\rangle,$
              wobei $|0\rangle$ und $|1\rangle$ Quantenzustände und
              $\alpha, \beta$ Komplexe Zahlen mit $|\alpha|^2 + |\beta|^2 = 1$ sind.
    \end{itemize}
\end{defi}

\begin{defi}[Quantencomputer]{Operationen}

\end{defi}

\begin{defi}[Quantencomputer]{Nutzung}

\end{defi}

\begin{defi}[Quantencomputer]{Typische Programme}
    \begin{itemize}
        \item Shor: Faktorisierung in Primzahlen
              \begin{itemize}
                  \item \underline{Wichtig für:} Verschlüsselung / Kryptographie und andere sicherheitsrelevante Bereiche
                  \item \underline{Problem:} Für $n=p\cdot q$ finde Primfaktor $p$
                        (digitaler Computer: praktisch unmöglich ab 1024-bit)
                  \item \underline{Speedup durch Quantencomputer:} exponentiell
                  \item \underline{Warum:} Quantenparallelismus $|00\ldots00\rangle+$ $|00\ldots01\rangle+$ $|00\ldots10\rangle+$ $\ldots+$ $|11\ldots11\rangle$
              \end{itemize}
        \item Grover: Suche in Datenbanken
              \begin{itemize}
                  \item Problem: In $m$ Elementen $x_1, \ldots, x_m$ finde $x$ (digitaler Computer: $O(m)$)
                  \item Speedup durch Quantencomputer: quadratisch $O(\sqrt{m})$
              \end{itemize}
        \item Prinzipiell alle digitalen Algorithmen, aber nicht unbedingt schneller! $\to$ Beispiel: 2-bit Addition
    \end{itemize}
\end{defi}

\begin{defi}[Quantencomputer]{D-WAVE QUANTENANNEALER}
    Löst spezielles Optimierungsproblem \enquote{QUBO} schnell und effizient:
    \begin{align*}
        \min\limits_{q_i = 0,1}\left(\sum\limits_i a_i q_i + \sum\limits_{i<j} b_{ij} q_i q_j\right)
    \end{align*}
\end{defi}

\begin{bonus}[Quantencomputer]{Zusammenfassung}
    \begin{itemize}
        \item Quantencomputer = Idee aus Quantentheorie
        \item Kleinste Recheneinheit: 1 \nameref{defi:Qubit}
        \item Messung von \nameref{defi:Qubit} liefert Bit 0 oder 1
        \item Program = Schaltung aus Quantengattern
        \item Besondere Form: D-Wave Quantenannealer
        \item Werden digitale Computer durch Quantencomputer ersetzt? Nein!
        \item Nutzermodell: Programme schicken spezielle Teilprobleme an Quantencomputer
        \item Relevant für z. B. Faktorisierung, Suche in Datenbanken, Optimierung, \ldots
    \end{itemize}
\end{bonus}

\begin{jsc}{JUNIQ - Jülicher Nutzer-Infrastruktur für Quantencomputing}
    Das Ziel ist die Bereitstellung von \ldots
    \begin{itemize}[\ldots]
        \item Quantencomputer-Simulatoren (JUQCS, Atos QLM-30) über Cloud-Zugriff
        \item Zugängen zu verfügbaren Systemen für Anwender aus Wissenschaft und Industrie
              \begin{itemize}
                  \item Für diese sind Systeme mit verschiedenen technologischen Reifegraden (QTRL) verfügbar
                  \item Benutzer können erstmals das bevorzugte System auf einer QTRL-Rampe wählen:
                        \begin{itemize}
                            \item Zugriff auf aufkommende experimentelle Systeme
                            \item Zugriff auf Multi-Qubit-Systeme für Quantencomputing ohne Fehlerkorrektur (IBM, Google, Rigetti)
                            \item Hosting und Betrieb eines Systems für Quantenannealing (D-Wave Systems)
                        \end{itemize}
              \end{itemize}
        \item hochkarätiger Anwenderunterstützung in allen Aspekten der HPC- und Quantencomputer-Nutzung
              \begin{itemize}
                  \item Einrichtung eines unterstützenden Simulationslabors \enquote{Quantencomputing}
                  \item Forschungsteams für Quantenalgorithmen, Protokolle, Workflows und Software-Werkzeuge
              \end{itemize}
    \end{itemize}
\end{jsc}

\subsection{Aufgaben}

\begin{aufgabe}[Quantencomputer]{Qubits}
    Kreuzen Sie die richtigen Aussagen an.\\
    Qubits \ldots
    \begin{itemize}[label={\Square}]
        \item können drei Zustände annehmen.
        \item werden durch eine Messung in einen ihrer möglichen Zustände gezwungen.
        \item sind physhikalisch schwer zu erzeugen und ggf. instabil.
    \end{itemize}
    \tcblower
    Qubits \ldots
    \begin{itemize}[label={\Square}]
        \item können drei Zustände annehmen.
        \item [\XBox] werden durch eine Messung in einen ihrer möglichen Zustände gezwungen.
        \item [\XBox] sind physhikalisch schwer zu erzeugen und ggf. instabil.
    \end{itemize}
\end{aufgabe}

\begin{aufgabe}{Quantencomputer}
    Kreuzen Sie die richtigen Aussagen an.\\
    Quantencomputer \ldots
    \begin{itemize}[label={\Square}]
        \item können bestimmte Computeralgorithmen um ein Vielfaches schneller ausführen als herkömmliche Rechner.
        \item können herkömmliche Rechner komplett ersetzen.
        \item stellen eine Bedrohung für derzeitige Verschlüsselungsalgorithmen dar.
    \end{itemize}
    \tcblower
    Quantencomputer \ldots
    \begin{itemize}[label={\Square}]
        \item [\XBox] können bestimmte Computeralgorithmen um ein Vielfaches schneller ausführen als herkömmliche Rechner.
        \item können herkömmliche Rechner komplett ersetzen.
        \item [\XBox] stellen eine Bedrohung für derzeitige Verschlüsselungsalgorithmen dar.
    \end{itemize}
\end{aufgabe}