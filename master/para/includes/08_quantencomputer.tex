\section{Quantencomputer}

\begin{defi}[Quantencomputer]{Qubit}
    \begin{itemize}
        \item Digitaler Computer: Bit, Zustände: 0 oder 1
        \item Quantencomputer: Quantum Bit = Qubit,
              $$\text{Zustände: } \alpha|0\rangle + \beta|1\rangle,$$
              wobei $|0\rangle$ und $|1\rangle$ Quantenzustände und
              $\alpha, \beta$ Komplexe Zahlen mit $|\alpha|^2 + |\beta|^2 = 1$ sind.
    \end{itemize}
\end{defi}

\begin{defi}[Quantencomputer]{Operationen}

\end{defi}

\begin{defi}[Quantencomputer]{Nutzung}

\end{defi}

\begin{defi}[Quantencomputer]{Typische Programme}
    \begin{itemize}
        \item Shor: Faktorisierung in Primzahlen
              \begin{itemize}
                  \item \underline{Wichtig für:} Verschlüsselung / Kryptographie und andere sicherheitsrelevante Bereiche
                  \item \underline{Problem:} Für $n=p\cdot q$ finde Primfaktor $p$
                        (digitaler Computer: praktisch unmöglich ab 1024-bit)
                  \item \underline{Speedup durch Quantencomputer:} exponentiell
                  \item \underline{Warum:} Quantenparallelismus $|00\ldots00\rangle+$ $|00\ldots01\rangle+$ $|00\ldots10\rangle+$ $\ldots+$ $|11\ldots11\rangle$
              \end{itemize}
        \item Grover: Suche in Datenbanken
              \begin{itemize}
                  \item Problem: In $m$ Elementen $x_1, \ldots, x_m$ finde $x$ (digitaler Computer: $O(m)$)
                  \item Speedup durch Quantencomputer: quadratisch $O(\sqrt{m})$
              \end{itemize}
        \item Prinzipiell alle digitalen Algorithmen, aber nicht unbedingt schneller! $\to$ Beispiel: 2-bit Addition
    \end{itemize}
\end{defi}

\begin{defi}[Quantencomputer]{D-WAVE QUANTENANNEALER}

\end{defi}

\begin{bonus}[Quantencomputer]{Zusammenfassung}

\end{bonus}