\section{Parallelverarbeitung}\label{sec:parallelverarbeitung}

\begin{defi}{Parallelverarbeitung}
    Durch Aufteilung einer Aufgabe in Teilaufgaben,
    die gleichzeitig bearbeitet werden können,
    soll eine Leistungssteigerung erzielt werden.\\
    Man unterscheidet:
    \begin{itemize}
        \item[Sequentielle Verarbeitung:] nacheinander ausführen der Teilaufgaben
        \item[Parallele Verarbeitung:] es gibt Zeitabschnitte in der Verarbeitung, in der Teilaufgaben gleichzeitig bearbeitet werden
    \end{itemize}
    \underline{Voraussetzungen:}
    \begin{enumerate}
        \item Aufteilungsmöglichkeit existiert und wird erkannt
        \item mehrere Bearbeiter stehen zur Verfügung
        \item gleichzeitig $\to$ möglichst unabhängig
    \end{enumerate}
\end{defi}

\begin{defi}[Parallelverarbeitung]{Domain Decomposition}
    Statische oder dynamische Aufteilung eines Datengebiets in Bereiche gleicher Prozessorarbeit.
    Alle Gebiete werden mit demselben Programmteil parallel verarbeitet.
    \begin{itemize}[$\to$]
        \item Kommunikation erfordert \enquote{enggekoppelte} Prozessoren
        \item geeignet für homogene Plattformen
        \item am häufigsten verwendet in HPC
    \end{itemize}
\end{defi}

\begin{defi}[Parallelverarbeitung]{Functional Decomposition}
    Unterschiedliche Programmteile, Unterprogramme,
    Module werden auf Prozessoren verteilt und parallel bearbeitet.
    \begin{itemize}[$\to$]
        \item verschiedene Rechnerarchitekturen für unterschiedliche Programmteile einsetzbar
        \item aufwändiger zu Implementieren
    \end{itemize}
\end{defi}

\begin{bonus}{Modular Supercomputing Architecture}

\end{bonus}

\begin{defi}{Speedup}
    \ldots ist ein Maß für die Bewertung der Verarbeitung eines einzelnen Auftrages (Passagiers).
\end{defi}

\begin{defi}{Durchsatz}
    \ldots ist ein Maß für die Bewertung der Verarbeitung einer gesamten Arbeitslast
    (mit welchem Flugzeug kann eine Fluggesellschaft mehr Personen befördern ?).
\end{defi}

\begin{defi}{Amdahl's Law}
    \begin{align*}
        \text{Speedup }S(p)=
        \frac{\text{Laufzeit (sequentiell)}}{\text{Laufzeit (parallel)}} =
        \frac{T_s}{T_S \cdot f + (1-f) \cdot \frac{T_s}{p}}
    \end{align*}
    mit:
    \begin{itemize}
        \item $p:$ Anzahl der parallelen Bearbeitungselemente (Prozessoren)
        \item $T_s:$ Laufzeit für die sequentielle Bearbeitung der gesamten Aufgabe
        \item $f:$ Anteil der Laufzeit, die für Teile verbraucht wird, die nicht parallel ablaufen können
    \end{itemize}
    für:
    \begin{itemize}
        \item $p\to\infty: S(\infty) = \frac{1}{f} \text{, oder} T(\infty) = f \cdot T_s$
        \item $f\to 0: S(p) = p \text{, oder} T(p) = \frac{T_s}{p}$
    \end{itemize}
\end{defi}

\begin{defi}{Effizienz}
    Effizienz definiert das Verhältnis von Speedup und eingesetzter Prozessoranzahl.
    Sie drückt aus, welcher Anteil der Prozessorleistung nutzbar ist.
    \begin{align*}
        E(p) = \frac{S(p)}{p} =
        \frac{T(\text{sequentiell})}{p\cdot T(\text{parallel})},
        0<E(p)\leq 1
    \end{align*}
\end{defi}

\begin{example}[Speedup]{Skalarprodukt}
    Paralleler Algorithmus:
    \begin{enumerate}
        \item Aufteilen in $p$ Teilvektoren der Länge $\frac{n}{p}$
        \item jeder Prozessor berechnet das Skalarprodukt des Teilvektors $T_* (p) = \left(2 \cdot \frac{n}{p} - 1\right) \cdot t_\text{flop}$
        \item die Ergebnisse werden zu Prozessor 1 übertragen $T_t (p) = (p - 1) \cdot (t_\text{startup} + t_\text{word})$
        \item und Addition der Teilergebnisse (reduzieren) $T_+ (p) = (p - 1) \cdot t_\text{flop}$
    \end{enumerate}
    für $p=2^d$:
    \begin{align*}
        S(p) = \frac{T(\text{seq})}{T(p)} = \ldots =
        \frac{p}{\frac{p(\frac{2n}{p} + d-1)}{2n-1} + \frac{p\cdot d\cdot (t_\text{startup} + t_\text{word})}{(2n - 1)t_\text{flop}}}
    \end{align*}
    bei $n\gg p$:
    \begin{align*}
        \approx\frac{p}{
            \underbrace{1 + p\cdot \log_2 p}_{\text{Algorithmus}} \cdot
            \underbrace{\frac{1}{2n}}_{\text{Problemgröße}} \cdot
            \underbrace{\frac{t_\text{startup}}{t_\text{flop}}}_{\text{Hardware}}}
    \end{align*}
\end{example}

\begin{example}[Effizienz]{Skalarprodukt}
    \begin{align*}
        E(p)\approx\frac{1}{1 + p\cdot\log_2 p\cdot \frac{1}{2n} \cdot \frac{t_\text{startup}}{t_\text{flop}}}
    \end{align*}
    Die Effizienz \emph{steigt} mit wachsender Problemgröße $n$ (bei $p$ fest) und
    \emph{sinkt} bei größerer Prozessoranzahl $p$ (mit $n$ fest)
\end{example}

\begin{defi}{Skalierbarkeit}
    Eine Rechnerarchitektur bzw.\ ein Programm ist skalierbar,
    wenn die Effizienz der Programmbearbeitung bei wachsender Prozessorzahl gleich bleibt.
\end{defi}

\begin{defi}{Pipelining}
    Henry Ford hat 1913 mit Produktionsbeginn der \enquote{Tin Lizzy} als erster das Fließband in die industrielle Fabrikation eingeführt.
    Prinzip: Ein komplexer Vorgang wird in eine Sequenz von einfacheren Teilaufgaben zerlegt. \\
    Vorteile:
    \begin{itemize}
        \item Höherer Durchsatz ($\frac{\text{Lizzy}}{h}$)
        \item Vereinfachung (und damit auch eine schnellere Bearbeitung) der Teilaufgaben (CISC $\to$ RISC)
    \end{itemize}
\end{defi}