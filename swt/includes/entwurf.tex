\section{Entwurf}

\begin{bonus}{Bestandteile einer Softwarearchitektur}
    \begin{itemize}
        \item Anwendungsspezifische Funktionen (Applikationslogik)
        \item Details der Benutzerschnittstelle (GUI-Bibliotheken)
        \item Ablaufsteuerung (Transaktionen, Workflowmanagement)
        \item Datenhaltung (in Dateien, Datenbanken)
        \item Infrastrukturdienste für
              \begin{itemize}
                  \item Objektverwaltung (Freispeichersammlung, verteilte Objekte)
                  \item Prozesskommunikation
              \end{itemize}
        \item Sicherheitsfunktionen (Verschlüsselung, Passwortschutz)
        \item Zuverlässigkeitsfunktionen (Fehlererkennung und -behebung)
        \item Systemadministration (Statistiken, Installation, Sicherung)
        \item Weitere Basisbibliotheken (arithmetische Funktionen)
        \item \ldots
    \end{itemize}
\end{bonus}

\begin{bonus}{Funktion der Architektur im Software-Lebenszyklus}
    \begin{tabularx}{\textwidth}{|>{\bfseries}l|X|}
        \hline
        Bauplan             & Spezifikation des zu implementierenden Systems sowohl auf grober als auch auf feiner Ebene   \\
        \hline
        Projektplanung      & Definition von Arbeitspaketen zum Implementieren und Testen                                  \\
                            & Definition von Meilensteinen                                                                 \\
                            & Aufteilung des Projektteams nach Architekturkomponenten                                      \\
                            & Fortschrittskontrolle                                                                        \\
        \hline
        Testen              & Festlegung von Teststrategien                                                                \\
                            & Ableitung von Testfällen                                                                     \\
        \hline
        Nachvollziehbarkeit & Management von Beziehungen zu Anwendungsfällen bzw. Funktionen der Anforderungsspezifikation \\
        \hline
        Wartung             & Verstehen des Systems                                                                        \\
                            & Analyse der Auswirkung von Änderungen                                                        \\
                            & Planung von Änderungen                                                                       \\
        \hline
    \end{tabularx}
\end{bonus}