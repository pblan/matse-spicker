\section{Anforderungsanalyse}

\begin{defi}{Anforderungsanalyse}
    Die \emph{Anforderungsanalyse} (engl. \enquote{Requirements Engineering}) ist ein kooperativer, iterativer, inkrementeller Prozess, dessen Ziel es ist, zu gewährleisten, dass:
    \begin{enumerate}
        \item alle relevanten \emph{Anforderungen bekannt} und in dem erforderlichen Detaillierungsgrad \emph{verstanden} sind,
        \item die involvierten Stakeholder eine \emph{ausreichende Übereinstimmung} über die bekannten Anforderungen erzielen,
        \item alle Anforderungen konform zu den Dokumentationsvorschriften \emph{dokumentiert} sind.
    \end{enumerate}

    Es umfasst zudem die Querschnittsaktivitäten Management und Validierung.
\end{defi}

\begin{defi}{Stakeholder}
    Ein \emph{Stakeholder} ist eine Person, die ein potenzielles Interesse an dem zukünftigen System hat und somit Anforderungen an das System stellt.

    Eine Person kann dabei die Interessen von mehreren Personen vertreten, d.h. mehrere Rollen einnehmen.
\end{defi}

\begin{defi}{Anforderung}
    \begin{enumerate}
        \item Eine Eigenschaft, die ein System oder eine Person benötigt, um ein Problem zu lösen.
        \item Eine Eigenschaft, die ein System aufweisen muss, um einen Vertrag zu erfüllen oder einem anderen formell auferlegten Dokument zu genügen.
        \item Eine dokumentierte Repräsentation einer Bedingung wie in 1. oder 2. definiert.
    \end{enumerate}
\end{defi}

\begin{defi}{Anforderungsspezifikation}
    Eine \emph{Anforderungsspezifikation} ist ein Dokument, das spezifizierte Anforderungen enthält, d.h. Anforderungen, die definierten Spezifikationskriterien genügen.

    \begin{itemize}
        \item Bei einer Auftragsentwicklung dient eine Anforderungsspezifikation als \emph{Kontrakt} zwischen KundIn und Auftragnehmenden.
        \item Ferner dient eine Anforderungsspezifikation als \emph{Vorgabe} für die Entwicklung des Systems durch den Auftragnehmenden.
    \end{itemize}
\end{defi}

\begin{defi}{Systemgrenze}
    Die \emph{Systemgrenze} separiert das geplante System von seiner Umgebung.

    Sie grenzt das System von den Teilen der Umgebung ab, die durch den Entwicklungsprozess nicht verändert werden können.
\end{defi}

\begin{defi}{Systemkontext}
    Der \emph{Systemkontext} ist der Teil der Umgebung des Systems, der für die Definition und das Verständnis der Anforderungen an das System relevant ist.
\end{defi}

\subsection{Erhebung}

\begin{defi}{Funktionale und Nicht-Funktionale Anforderung}
    Funktionale Anforderungen sind z.B.:
    \begin{itemize}
        \item \emph{Was} soll das System leisten?
        \item \emph{Welche Dienste} soll das System bereitstellen?
        \item \emph{Welche Eingaben} werden verarbeitet? Wie sehen die \emph{Ausgaben} aus?
        \item \emph{Wie verhält sich das System} in Situationen X, Y?
    \end{itemize}

    Nicht-Funktionale Anforderungen sind z.B.:
    \begin{itemize}
        \item \emph{Wie} soll das System arbeiten?
        \item Welche Anforderungen bzgl. der \emph{Benutzerschnittstelle} bestehen?
        \item Wie \emph{performant} oder \emph{verfügbar} muss das System sein?
        \item \emph{Welchen Qualitätsanforderungen} soll das System genügen?
    \end{itemize}
\end{defi}

\begin{defi}{Grundlegende Techniken zur Anforderungserhebung}
    \begin{tabularx}{\textwidth}{|>{\bfseries}l|X|}
        \hline
        Interview                   & \emph{Befragung} von Stakeholdern in Einzel- oder Gruppeninterviews in explorativer oder standardisierter Form                                                                               \\
        \hline
        Workshop                    & \emph{Erarbeitung} von Anforderungen an das System durch eine \emph{Gruppe} von Stakeholdern , z.B. mit Brainstorming, Diskussionen, Kleingruppenarbeit oder interaktiver Szenariodefinition \\
        \hline
        Beobachtung                 & \emph{Beobachtung von Stakeholdern} bei der Durchführung von Arbeitsprozessen in direkter oder ethnografischer (teilnehmender) Form                                                          \\
        \hline
        Schriftliche Befragung      & Befragung von Stakeholdern mit Hilfe von Fragebögen, die offene oder geschlossene Fragen enthalten können                                                                                    \\
        \hline
        Perspektivenbasiertes Lesen & Gezieltes, ggf. selektives Lesen von Dokumenten aus einer bestimmten Perspektive (z.B. Nutzungs-, Gegenstands-, IT-System- oder Entwicklungsperspektive)                                     \\
        \hline
    \end{tabularx}

    Besonders relevant sind hier \emph{Interview}, \emph{Workshop} und \emph{Beobachtung}.
\end{defi}

\begin{defi}{Hilfstechniken zur Anforderungserhebung}
    \begin{tabularx}{\textwidth}{|>{\bfseries}l|X|}
        \hline
        Brainstorming & Generierung einer großen Zahl von Ideen in einer Gruppe, Visualisierung z.B. mit Whiteboards, Flipcharts oder Pinnwänden                                 \\
        \hline
        Prototypen    & Entwicklung einer initialen Version des Systems, um die Funktionalität und Benutzerschnittstelle zu demonstrieren                                        \\
        \hline
        Kartenabfrage & Synchrones Abfragen des Wissens aller Teilnehmer einer Gruppe, die jeweils Karten ausfüllen, die anschließend auf Pinnwänden thematisch gruppiert werden \\
        \hline
        Mind Maps     & Strukturierung von Informationen in einem baumartigen Diagramm                                                                                           \\
        \hline
        Checklisten   & Vorbereitete Listen von Fragen oder Aussagen, die zur Gewinnung von Anforderungen verwendet werden                                                       \\
        \hline
    \end{tabularx}

    Besonders relevant ist hier der \emph{Prototyp}.
\end{defi}

\subsection{Dokumentation}

\begin{defi}{Szenario}
    Ein \emph{Szenario} ist ein konkretes Beispiel für die Erfüllung bzw. Nichterfüllung mehrerer Ziele.

    Ein Szenario enthält typischerweise eine Folge von Interaktionsschritten und setzt diese in Bezug zum Systemkontext.

    Ein Szenario kann z.B. (strukturiert) textuell, tabellarisch oder graphisch mit UML Use-Case-Diagrammen dargestellt werden.
\end{defi}

\begin{defi}{Datenlexikon}
    Die Motivation eines \emph{Datenlexikons} liegt darin, die fachliche Struktur von Daten festzuhalten.

    Es dient dazu, den Aufbau der Daten in textueller Notation darzustellen.
    Dabei beschreibt die Grammatik den Aufbau:

    \begin{center}
        \begin{tabular}{r|l}
            \bfseries Symbol            & \bfseries Bedeutung                                                       \\
            \hline
            \texttt{name = beschr}      & Definition (besteht aus)                                                  \\
            \texttt{teil1 + teil 2}     & Sequenz (und)                                                             \\
            \texttt{[optA | optB]}      & Alternative (entweder oder)                                               \\
            \texttt{min \{beschr\} max} & Iteration (\texttt{min}-\texttt{max} mal, \texttt{max}-Default: $\infty$) \\
            \texttt{(beschr)}           & Option (muss nicht, entspricht \texttt{0 \{...\} 1})                      \\
            \texttt{name}               & Benennung eines atomaten Datenobjektes
        \end{tabular}
    \end{center}
\end{defi}

\begin{example}{Datenlexikon}
    Wir beziehen uns hier auf das Lagerungssystem aus Pokémon\footnote{Für mehr Informationen, siehe:  \href{https://www.pokewiki.de/Pokémon-Lagerungssystem}{https://www.pokewiki.de/Pokémon-Lagerungssystem}}.
    \begin{itemize}
        \item \texttt{Box = 0 \{Pokémon\} 30}
        \item \texttt{Pokémon = 1 \{Typ\} 2 + (Item) + 1 \{Level\} 100}
    \end{itemize}
\end{example}

\begin{bonus}{Pseudocode}
    Mit \emph{Pseudocode} kann man Minispezifikationen (Operationsweise atomarer Prozesse) beschreiben.

    Dabei ist man nicht an eine exakte Syntax - also offensichtlich auch nicht an eine bestimmte Programmiersprache - gebunden.
    Insbesondere kann es Teile geben, die in natürlicher Sprache geschrieben sind.

    Üblicherweise werden Konstrukte wie Fallunterscheidungen und Schleifen verwendet.

    Das richtige Niveau erfordert dabei viel Übung:
    \begin{itemize}
        \item Überspezifikation (wenig Pseudo, viel Code): einschränkend, wenig lesbar
        \item Unterspezifikation (viel Pseudo, wenig Code): zu vage, versteckt Probleme
    \end{itemize}
\end{bonus}

\begin{bonus}{Lastenheft}
    In dem \emph{Lastenheft} legt der Auftraggebende fest, was dieser sich von dem Projekt erwartet.
    Es wird also alles niedergeschrieben und es werden die  gesamten Anforderungen an das Projekt definiert.
    Demnach lässt sich das Lastenheft auch als Kundenspezifikation oder Anforderungskatalog bezeichnen.

    Durch das Lastenheft ist der Auftraggebende intensiv mit der Aufgabe konfrontiert.
    Es müssen sich also intensive Gedanken darüber gemacht werden, was eigentlich vom Auftrag erwartet wird.
    Das macht es dem Auftragnehmenden leichter, auf das Ziel hinzuarbeiten, welches sich gewünscht wird.
\end{bonus}

\begin{bonus}{Pflichtenheft}
    Erst wenn das Lastenheft erstellt wurde, kann der Projektdurchführende das \emph{Pflichtenheft} erstellen.
    Dieses bildet sozusagen die Antwort auf das Lastenheft.

    In dem Pflichtenheft stellt der Dienstleistende in ganz konkreter Form dar, wie er das Projekt für den Auftraggebenden umsetzen kann und würde.
\end{bonus}