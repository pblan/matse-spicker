\documentclass[german]{spicker}

\usepackage{amsmath}

\title{Analysis 1}
\author{Patrick Gustav Blaneck, Felix Racz}

\begin{document}
\maketitle
\tableofcontents
\newpage

\section{Grundlagen}
\subsection{Funktionen}
\subsubsection{Eigenschaften von Funktionen}
\begin{thirdboxl}
    \begin{defi}{Injektivität}
        $f(x) = f(x')\implies x = x'$
    \end{defi}
\end{thirdboxl}%
\begin{thirdboxm}
    \begin{defi}{Surjektivität}
        $\forall y, \exists x: x = f(y)$
    \end{defi}
\end{thirdboxm}%
\begin{thirdboxr}
    \begin{defi}{Bijektivität}
        $\forall y, \exists! x: x = f(y)$
    \end{defi}
\end{thirdboxr}%

\begin{algo}{Beweisen der Injektivität}
    \begin{enumerate}
        \item Behauptung: $f(x) = f(x')$
        \item Umformen auf eine Aussage der Form $x = x'$
    \end{enumerate}
\end{algo}

\begin{algo}{Beweisen der Surjektivität}
    \begin{enumerate}
        \item Aufstellen der Umkehrfunktion
        \item Zeigen, dass diese Umkehrfunktion auf dem gesamten Definitionsbereich definiert ist
    \end{enumerate}
\end{algo}

\begin{algo}{Beweisen der Bijektivität}
    \begin{enumerate}
        \item Injektivität beweisen
        \item Surjektivität beweisen
    \end{enumerate}
\end{algo}

\begin{bonus}{Tipps und Tricks}
    \begin{enumerate}
        \item Gilt eine Eigenschaft nicht, ist ein Gegenbeispiel oft einfach gefunden.
        \item Gilt eine Eigenschaft nicht, ist die Abbildung auch nicht bijektiv.
    \end{enumerate}
\end{bonus}

\subsection{Polynome}

\subsection{Gebrochen rationale Funktionen}

\subsection{Gleichungen und Ungleichungen}

\subsection{Komplexe Zahlen}
\subsubsection{Rechenregeln für komplexe Zahlen in Polarkoordinaten}

\subsubsection{Eigenschaften von $e^{i\pi}$}

\subsubsection{Radizieren von komplexen Zahlen}

\subsubsection{Faktorisierung von Polynomen mit komplexen Koeffizienten}

\section{Folgen und Reihen}

\section{Konvergenz von Folgen, Reihen und Funktionen}

\section{Differentialrechnung}

\section{Integration}

\end{document}