\section{Internet und IP-Adressen}

\begin{defi}{Internet}
    Das Internet, ist ein weltweiter Verbund von Rechnernetzwerken, den autonomen Systemen.
    Router werden als koppelnde Elemente zwischen den Teilnetzen genutzt.
\end{defi}

\begin{defi}{Netzwerk (ISO/OSI)}
    \includegraphics[width=\textwidth]{includes/figures/defi_iso_osi_network.pdf}
\end{defi}

\begin{defi}{TCP/IP}
    \emph{TCP/IP} ist eine protokollunabhängige Ausmodellierung des konzeptionellen \emph{ISO/OSI} Modells.

    \begin{center}
        \includegraphics[width=0.75\textwidth]{includes/figures/defi_tcp_ip.pdf}
    \end{center}
\end{defi}

\begin{bonus}{Aufbau des Internets}
    Die Basis des Internets bilden \emph{Tier-1} Internet Service Provider (ISPs).
    Diese treten gleichberechtigt untereinander auf und sind durch vertraglich regulierte Verbindungen angebunden.

    Als Verbindung zwischen den ISPs dienen \emph{Network Access Points} (NAPs) oder \emph{Internet Exchange Points} (IXPs)

    \begin{center}
        \includegraphics[width=0.75\textwidth]{includes/figures/bonus_aufbau_internet_1.pdf}
    \end{center}
\end{bonus}

\begin{bonus}{Aufbau des Internets}
    \emph{Tier 2} ISPs sind national bzw. regional und sind immer an einen oder mehrere \emph{Tier 1} ISPs angeschlossen.

    Sie treten dabei gegenüber \emph{Tier 1} ISPs als KundIn auf.

    \begin{center}
        \includegraphics[width=0.75\textwidth]{includes/figures/bonus_aufbau_internet_2.pdf}
    \end{center}
\end{bonus}

\begin{bonus}{Aufbau des Internets}
    \emph{Tier 3} ISPs binden den Kunden beispielsweise über DSL direkt an das gesamte Netzwerk an.

    Sie treten dabei gegenüber \emph{Tier 2} ISPs selber als KundIn auf.

    \begin{center}
        \includegraphics[width=0.75\textwidth]{includes/figures/bonus_aufbau_internet_3.pdf}
    \end{center}
\end{bonus}

\begin{bonus}{Aufbau des Internets}
    \begin{center}
        \includegraphics[width=0.75\textwidth]{includes/figures/bonus_aufbau_internet_4.pdf}
    \end{center}
\end{bonus}

\begin{bonus}{Nachrichtenzustellung}
    \begin{wrapfigure}{r}{0.25\textwidth}
        \begin{center}
            \includegraphics[width=0.2\textwidth]{includes/figures/bonus_iso_osi_vermittlung.pdf}
        \end{center}
    \end{wrapfigure}
    %
    Die Herausforderung in diesem großen Netz aus Netzen ist es Datenpakete von einem beliebigen Endgerät zu einem bestimmten Ziel zu versenden.

    Damit dies gelingt, wird in der Vermittlungsschicht einer der möglichen Wege zum Zielgerät gewählt.
    Die Wegwahl findet durch \emph{Routing Protokolle} statt.

    Die Entscheidung erfolgt durch \emph{Routing Tabellen}.
    In diesen lassen sich - je nach Protokoll - definierte Wege finden, unter denen man das autonome System des Zielgeräts findet bzw. welcher Weg befolgt werden soll, wenn das Zielgerät komplett unbekannt ist.

    In der Vermittlungsschicht wird größtenteils das IP-Protokoll genutzt, um die Pakete zu adressieren und Regeln zur Paketbehandlung aufzustellen.
\end{bonus}

\begin{defi}{Kapselung von Protokoll-Headern}
    Zur erfolgreichen Nachrichtenzustellung erweitern die verschiedenen Protokollinstanzen ein von einer Anwendung produziertes Datenpaket um verschiedene fest definierte Header.

    \begin{center}
        \includegraphics[width=0.75\textwidth]{includes/figures/defi_header_kapselung.pdf}
    \end{center}

    \begin{itemize}
        \item TCP bzw. UDP-Header dienen zur eindeutigen Prozessadressieren durch Ports.
        \item TCP sichert darüber hinaus die Datenübertragung
        \item IP-Header identifizieren ein eindeutiges Quell und Zielsystem
    \end{itemize}

    Demnach entsteht folgendes versandfertiges Datagramm:

    \begin{center}
        \includegraphics[width=0.75\textwidth]{includes/figures/defi_datagramm.pdf}
    \end{center}
\end{defi}

\begin{defi}{Das Internet-Protokoll (IP)}
    \emph{IP} bietet eine Ende-zu-Ende Kommunikation zwischen Endgeräten im Internet.
    Derzeit wird flächendeckend \emph{IPv4} bzw. \emph{IPv6} eingesetzt.

    IP ist paketvermittelnd, leitet also in sich geschlossene, unabhängige Dateneinheiten bzw. Datagramme ungesichert zwischen zwei Endpunkten weiter.
    Jedes Paket wird dabei in dem jeweiligen Router zwischengespeichert, überprüft und dementsprechend aussortiert oder weitergeleitet.
    Da die Pakete parallel von den Routern abgearbeitet werden, kann es passieren, dass:
    \begin{itemize}
        \item Datagramme verloren gehen
        \item Datagramme sich gegenseitig überholen
        \item Datagramme mehrfach ankommen
    \end{itemize}

    \emph{Sitzungen} werden durch einen exklusiven Kommunikationskanal über TCP simuliert.
\end{defi}

\begin{defi}{Wegwahl im Internet-Protokoll}
    Die IP-Implementierung eines Rechners oder Routers entscheidet eigenständig, wohin dieser ein Datagramm übertragen soll.

    Diese Entscheidung wird auf Basis von Routing bzw. Forwarding-Tabellen getroffen\footnote{Wie Routingtabellen aufgebaut werden, wird zu einem späteren Zeitpunkt erwähnt.}.
    Dazu werden die Informationen, normalerweise ausschließlich die Zieladresse des Pakets, aus dem IP-Header in der jeweiligen Tabelle abgefragt.

    Router sind dabei meist an mehreren Netzen angeschlossen.
    Sie müssen zusätzlich noch überprüfen, ob sich das Zielgerät in einem anderen Netz befindet als das Quellgerät bzw. wie dieses Netz erreicht werden kann.
\end{defi}

\begin{example}{Wegwahl im Internet Protokoll}
    In folgendem Beispiel möchte \emph{Rechner A} an Paket an \emph{Rechner F} senden.

    \begin{center}
        \includegraphics[width=0.75\textwidth]{includes/figures/example_ip_routing.pdf }
    \end{center}

    \begin{minipage}{0.25\textwidth}
        \begin{center}
            \emph{Rechner A}

            \begin{tabular}{|c|l|}
                \hline
                \textbf{Ziel} & \textbf{Next Hop} \\\hline
                > * <         & R1                \\\hline
            \end{tabular}
        \end{center}
    \end{minipage}
    \begin{minipage}{0.25\textwidth}
        \begin{center}
            \emph{R1}

            \begin{tabular}{|c|l|}
                \hline
                \textbf{Ziel} & \textbf{Next Hop} \\\hline
                A             & A                 \\
                B             & B                 \\
                C             & R4                \\
                D             & R2                \\
                E             & R2                \\
                > F <         & R2                \\\hline
            \end{tabular}
        \end{center}
    \end{minipage}
    \begin{minipage}{0.25\textwidth}
        \begin{center}
            \emph{R2}

            \begin{tabular}{|c|l|}
                \hline
                \textbf{Ziel} & \textbf{Next Hop} \\\hline
                A             & R1                \\
                B             & R1                \\
                C             & R4                \\
                D             & R3                \\
                E             & R3                \\
                > F <         & R5                \\\hline
            \end{tabular}
        \end{center}
    \end{minipage}
    \begin{minipage}{0.25\textwidth}
        \begin{center}
            \emph{R5}

            \begin{tabular}{|c|l|}
                \hline
                \textbf{Ziel} & \textbf{Next Hop} \\\hline
                A             & R2                \\
                B             & R2                \\
                C             & R4                \\
                D             & R3                \\
                E             & R3                \\
                > F <         & F                 \\\hline
            \end{tabular}
        \end{center}
    \end{minipage}
\end{example}

\begin{defi}{IPv4-Adresse}
    IP-Adressen sind 32 bit bzw. 4 Byte lang.

    Sie sind unterteilt in einen Bereich um das Netz zu adressieren und einen Bereich um diesen Rechner in dem angegebenen Netz zu identifizieren.
    In Kombination ist die IP-Adresse (theoretisch) Weltweit eindeutig.

    Zur einfachen und übersichtlichen Darstellung werden IP-Adressen in vier Blöcke je ein Byte aufgeteilt, von denen jeder eine dezimale Zahl darstellt\footnote{Führende Nullen können bei der Visualisierung weggelassen werden.}:

    z.B. \texttt{192.168.0.1}
\end{defi}

\begin{bonus}{Probleme IPv4}
    Da nur ca. 4.3 Mrd. ($2^{32}$) Adressen zur Verfügung stehen, kann nicht jedes Gerät eine eigene weltweit eindeutige IP-Adresse erhalten.

    Um dieses Problem zu lösen, bzw. die resultierenden Auswirkungen aufzuschieben, wurden Adressklassen definiert.

    \begin{enumerate}[label=Class \Alph*:, leftmargin=*]
        \item Für Netze mit bis zu 16 Mio Knoten (Knoten-ID: $0_{10}\text{-}127_{10}$)
    \end{enumerate}

    \begin{center}
        \includegraphics[width=0.75\textwidth]{includes/figures/bonus_class_a.pdf}
    \end{center}

    \begin{enumerate}[label=Class \Alph*:, leftmargin=*, start=2]
        \item Für Netze mit bis zu 65.536 Knoten (Knoten-ID: $128_{10}\text{-}191_{10}$)
    \end{enumerate}

    \begin{center}
        \includegraphics[width=0.75\textwidth]{includes/figures/bonus_class_b.pdf}
    \end{center}

    \begin{enumerate}[label=Class \Alph*:, leftmargin=*, start=3]
        \item Für Netze mit bis zu 256 Knoten (Knoten-ID: $192_{10}\text{-}223_{10}$)
    \end{enumerate}

    \begin{center}
        \includegraphics[width=0.75\textwidth]{includes/figures/bonus_class_c.pdf}
    \end{center}

    \begin{enumerate}[label=Class \Alph*:, leftmargin=*, start=4]
        \item Für Gruppenkommunikation (Knoten-ID: $224_{10}\text{-}239_{10}$)
    \end{enumerate}

    \begin{center}
        \includegraphics[width=0.75\textwidth]{includes/figures/bonus_class_d.pdf}
    \end{center}

    \begin{enumerate}[label=Class \Alph*:, leftmargin=*, start=5]
        \item Reserviert für zukünftige Anwendungen (Knoten-ID: $240_{10}\text{-}255_{10}$)
    \end{enumerate}

    \begin{center}
        \includegraphics[width=0.75\textwidth]{includes/figures/bonus_class_e.pdf}
    \end{center}

    In allen Netzen ist die Koten-ID 0..0 ($0_{10}$) für das Netz selber reserviert.

    Des weiteren wird für Broadcast Nachrichten an alle teilnehmenden Geräte im Netz die Knoten-ID 1..1 ($\text{A: } 127_{10} \text{ bzw. B: } 191_{10} \text{ bzw. C: } 223_{10} \text{ bzw.  D:} 239_{10} \text{ bzw. E: } 255_{10}$) genutzt
\end{bonus}

\begin{defi}{IP-Header}
    \begin{center}
        \includegraphics[width=0.75\textwidth]{includes/figures/defi_ip_header_kapselung.pdf}

        \includegraphics[width=0.75\textwidth]{includes/figures/defi_ip_header.pdf}
    \end{center}
\end{defi}