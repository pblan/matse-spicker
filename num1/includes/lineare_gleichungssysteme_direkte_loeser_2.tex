\section{Lineare Gleichungssysteme - Direkte Löser II}

\subsection{Fehlerbetrachtung für lineare Gleichungssysteme}

\begin{defi}{Konditionszahl einer Matrix}
    Sei $\| \cdot \|$ eine Norm auf $\R^n$, $A$ regulär und $\| \Delta A \| < \nicefrac{1}{\| A^{-1} \|}$, wobei die durch $\| \cdot \|$ induzierte Matrixnorm benutzt wird.
    
    Dann ist $\tilde{A}$ regulär und es gilt
    \[
        \frac{\| \Delta x \|}{\| x \|} \leq \frac{\varkappa(A)}{1 - \varkappa(A) \frac{\| \Delta A \|}{\| A \|}} \cdot \left( \frac{\| \Delta A \|}{\| A \|} + \frac{\| \Delta b \|}{\| b \|} \right)
    \]
    
    mit $\varkappa(A) = \| A \| \| A^{-1} \|$.
    
    $\varkappa(A)$ heißt \emph{Konditionszahl der Matrix} $A$ und hat folgende Eigenschaften:
    \begin{itemize}
        \item $\varkappa$ hängt von der verwendeten Norm ab
        \item ist $\alpha \| B \|' \leq \| B \| \leq \beta \| B \|'$ so gilt $\alpha^2 \varkappa'(B) \leq \varkappa(B) \leq \beta^2 \varkappa' (B)$, d.h. die Äquivalenz der Normen überträgt sich
        \item $\varkappa(A) \geq 1$, denn $1 = \| I \| = \| AA^{-1} \| \leq \|A^{-1}A \| = \varkappa(A)$
        \item $\varkappa(A) = \varkappa(A^{-1})$
        \item $\varkappa(AB) \leq \varkappa(A)\varkappa(B)$
    \end{itemize}
    
    $\| \Delta A \|$ ist üblicherweise sehr viel kleiner als $\| A \|$, d. h. der Zähler der Gleichung oben dominiert, so dass der Eingangsfehler im wesentlichen mit $\varkappa(A)$ verstärkt wird.
    Daher sollte $\varkappa(A)$ so klein wie möglich sein.
\end{defi}

\begin{defi}{Vorkonditioniertes System}
    $\varkappa(A) = \| A \| \| A^{-1} \|$ ist wegen $\| A^{-1} \|$ in der Regel nicht berechenbar.
    
    Generell sollte man vor der Berechnung versuchen, die Kondition des Problems zu verbessern.
    
    Betrachte $Ax = b$ und $U$, $V$ regulär:
    \[
        Ax = b \quad \iff \quad \underbrace{UAV}_{\tilde{A}} \underbrace{V^{-1}x}_{\tilde{x}} = \underbrace{Ub}_{\tilde{b}}
    \]
    
    $\tilde{A} \tilde{x} = \tilde{b}$ heißt \emph{vorkonditioniertes System}.
\end{defi}

\begin{example}{Vorkonditioniertes System}
    Ein optimales, aber nicht praktikables, vorkonditioniertes System für $Ax = b$ ist z. B.:
    \[
        U = A^{-1}, V = I \quad \implies \quad \tilde{A} = I \quad \implies \quad \varkappa(\tilde{A}) = 1
    \]
    \[
        Ax = b \quad \iff \quad \underbrace{UAV}_{\tilde{A}} \underbrace{V^{-1}x}_{\tilde{x}} = \underbrace{Ub}_{\tilde{b}} \quad \iff \quad I x = A^{-1} b
    \]
\end{example}

\begin{bonus}{Äquilibrieren}
    In der Praxis werden für $U$ bzw. $V$ Näherungen für $A^{-1}$ gesucht, die billig zu berechnen sind.
    
    Ein einfacher Ansatz ist Skalieren von Zeilen und Spalten.
    Dafür können
    \begin{itemize}
        \item Zeilen skaliert werden:
              \[
                  U =
                  \begin{pmatrix}
                      u_1    & 0      & \cdots & 0      \\
                      0      & \ddots & \ddots & \vdots \\
                      \vdots & \ddots & \ddots & 0      \\
                      0      & \cdots & 0      & u_n
                  \end{pmatrix},
                  \quad V = I
              \]
        \item Spalten skaliert werden:
              \[
                  U = I,
                  \quad V =
                  \begin{pmatrix}
                      v_1    & 0      & \cdots & 0      \\
                      0      & \ddots & \ddots & \vdots \\
                      \vdots & \ddots & \ddots & 0      \\
                      0      & \cdots & 0      & v_n
                  \end{pmatrix}
              \]
        \item Zeilen und Spalten skaliert werden:
              \[
                  U =
                  \begin{pmatrix}
                      u_1    & 0      & \cdots & 0      \\
                      0      & \ddots & \ddots & \vdots \\
                      \vdots & \ddots & \ddots & 0      \\
                      0      & \cdots & 0      & u_n
                  \end{pmatrix},
                  \quad V =
                  \begin{pmatrix}
                      v_1    & 0      & \cdots & 0      \\
                      0      & \ddots & \ddots & \vdots \\
                      \vdots & \ddots & \ddots & 0      \\
                      0      & \cdots & 0      & v_n
                  \end{pmatrix}
              \]
    \end{itemize}
    
    Eine vernünftige Skalierung sollte erfüllen:
    \begin{itemize}
        \item alle Gleichungen sollten Koeffizienten gleicher Größenordnung haben $\implies$ Zeilen gleicher Norm
        \item alle Unbekannten sollten mit gleichem Gewicht auftauchen $\implies$ Spalten gleicher Norm
    \end{itemize}
    
    Spalten bzw. Zeilen auf gleiche Norm bringen nennt man \emph{Äquilibrieren}.
    
    Spalten und Zeilen gleichzeitig zu äquilibrieren ist sehr aufwendig, deswegen werden in der Regel entweder Zeilen oder Spalten äquilibriert.
\end{bonus}

\begin{example}{Äquilibrieren}
    TODO
\end{example}

\begin{defi}{Verfahren von Prager-Oettli}
    Wir betrachten dabei das folgende Szenario:
    \begin{itemize}
        \item $A$ und $b$ sind nur bis auf Störungen $B$ bzw. $c$ bekannt
        \item $Ax = b$ ist also mit \enquote{gewisser Unschärfe} gegeben
        \item für vorgelegte Näherungslösung $\tilde{x}$ soll \emph{Plausabilitätsprüfung} durchgeführt werden
    \end{itemize}
    
    Für $x \in \R^n$, $A \in \R^{n \times n}$ seien
    \[
        |x| =
        \begin{pmatrix}
            |x_1| \\ \vdots \\ |x_n|
        \end{pmatrix}, \quad
        |A| =
        \begin{pmatrix}
            |a_{11}| & \cdots & |a_{1n}| \\
            \vdots   & \ddots & \vdots   \\
            |a_{n1}| & \cdots & |a_{nn}|
        \end{pmatrix}
    \]
    
    Ist $\det(A) \neq 0$ und $\tilde{x}$ eine Näherungslösung von $Ax = b$ mit \emph{Residuum}
    \[
        \tilde{r} = b - A\tilde{x}
    \]
    sowie
    \[
        B \in \R^{n \times n}, c \in \R^n, \quad \forall i, j: b_{ij} \geq 0, c_i \geq 0
    \]
    mit
    \[
        |\tilde{r}| \leq B |\tilde{x}| + c
    \]
    
    Dann gibt es $|\Delta A| \leq B$ und $|\Delta b| \leq c$ so, dass $\tilde{x}$ exakte Lösung von
    \[
        (A + \Delta A)\tilde{x} = b + \Delta b
    \]
    ist.\footnote{Die Ungleichheitszeichen sind jeweils komponentenweise anzuwenden.}
    
    \begin{itemize}
        \item kennt man $B$, $c$ so ist $\tilde{x}$ exakte Lösung eines gestörten Ausgangsproblems
        \item über $B$, $c$ weiß man auch, wie weit das gestörte Problem höchstens vom Ausgangsproblem entfernt ist
        \item sind $B$, $c$ komponentenweise klein, ist $\tilde{x}$ eine gute Näherung von $x$
    \end{itemize}
\end{defi}

\begin{example}{Verfahren von Prager-Oettli}
    TODO
\end{example}

\subsection{LU-Zerlegung mit Pivot-Strategie}

\begin{defi}{LU-Zerlegung mit Pivot-Strategie}
    Sei $A \in \R^{n \times n}$ regulär.
    Es lässt sich zeigen, dass eine LU-Zerlegung die geringste Kondition (und damit die geringste Fehleranfälligkeit) hat, wenn die \emph{betragsgrößten Elemente} der Spalte nach einem Schritt der Zerlegung auf die Diagonale sind.
    
    Das heißt nach jedem Zeilentausch gilt:
    \[
        | \tilde{a}_{kk}^{(k-1)} | \geq | \tilde{a}_{ik}^{(k-1)} | \quad \forall i > k
    \]
    
    Dieses Verfahren nennt sich \emph{Spalten-Pivot-Suche}.
\end{defi}

\begin{example}{LU-Zerlegung mit Pivot-Strategie}
    TODO
\end{example}

\subsection{Orthogonale Transformationen}

\begin{bonus}{Orthonormalität}
    $Q \in \R^{n \times n}$ heißt \emph{orthonormal}, falls die Spaltenvektoren $q_i$ paarweise senkrecht stehen und Länge $1$ haben, d. h.
    \[
        \langle q_i, q_j \rangle = \delta =
        \begin{cases}
            1 & \text{für} \ i = j, \\
            0 & \text{sonst}
        \end{cases}
    \]
    
    Sind $Q, \tilde{Q} \in \R^{n \times n}$ orthonormal, dann gilt:
    \begin{enumerate}
        \item $QQ^T = Q^TQ = I \quad \iff \quad Q^{-1} = Q^T$
        \item $\varkappa_2(Q) = 1$
        \item $Q\tilde{Q}$ ist orthonormal
    \end{enumerate}
\end{bonus}

\subsubsection{Verfahren von Givens}

\begin{defi}{Spaltentransformation mithilfe von Drehungen und Spiegelungen}
    Es gilt:
    \begin{itemize}
        \item Spaltentransformation auf Vielfaches des Einheitsvektors ist Elementaroperation bei der Matrixzerlegung
        \item für orthonormales $Q$ ist $\varkappa_2(Q) = 1$ (minimal) und damit $\varkappa_2(U) \leq \varkappa_2(A)$
        \item Drehungen und Spiegelungen sind orthonormal
        \item im $\R^2$ können Spalten mit Drehungen und Spiegelungen eliminiert werden
    \end{itemize}
\end{defi}

\begin{bonus}{Verfahren von Givens (Idee)}
    \setcounter{MaxMatrixCols}{20}
    
    TODO Grafik von Folie 202/203
    
    Sei ein Vektor $u$ gegeben mit
    \[
        u =
        \begin{pmatrix}
            u_1 & \cdots & u_{i-1} & u_{i} & u_{i-1} & \cdots & u_{n}
        \end{pmatrix}^T
        \in \R^n
    \]
    
    Die Eliminierung des Eintrags $u_{j}$ mithilfe des Eintrags $u_{i}$ erfolgt dann wie folgt:\footnote{später ist wird ein Eintrag $u_{ij}$ mithilfe des zugehörigen Diagonalelements $u_{ii}$ eliminiert.}
    \begin{itemize}
        \item sei $w = \sqrt{u_i^2 + u_j^2}$ mit $j > i$
              \begin{itemize}
                  \item $w = 0$: setze $Q_{ij} = I$
                  \item $w \neq 0$: setze
                        \[
                            Q_{ij} =
                            \begin{pmatrix}
                                1 &        &   &    &   &        &   &   &   &        &   \\
                                  & \ddots &   &    &   &        &   &   &   &        &   \\
                                  &        & 1 &    &   &        &   &   &   &        &   \\
                                  &        &   & c  &   &        &   & s &   &        &   \\
                                  &        &   &    & 1 &        &   &   &   &        &   \\
                                  &        &   &    &   & \ddots &   &   &   &        &   \\
                                  &        &   &    &   &        & 1 &   &   &        &   \\
                                  &        &   & -s &   &        &   & c &   &        &   \\
                                  &        &   &    &   &        &   &   & 1 &        &   \\
                                  &        &   &    &   &        &   &   &   & \ddots &   \\
                                  &        &   &    &   &        &   &   &   &        & 1
                            \end{pmatrix},
                            \quad
                            c = \pm \frac{u_i}{w},
                            \quad
                            s = \pm \frac{u_j}{w}
                        \]
              \end{itemize}
        \item dann ist $Q_{ij}$ orthonormal und es werden nur $u_i$ und $u_j$ verändert:
              \[
                  Q_{ij}u =
                  \begin{pmatrix}
                      u_1 & \cdots & u_{i-1} & \pm w & u_{i+1} & \cdots & u_{j-1} & 0 & u_{j-1} & \cdots & u_{n}
                  \end{pmatrix}^T
              \]
    \end{itemize}
\end{bonus}

\begin{defi}{Verfahren von Givens}
    In der Praxis wird das \emph{Givens-Verfahren} wie folgt durchgeführt:
    \begin{itemize}
        \item Analog zur LU-Zerlegung:\footnote{Handhabung von $Q$ ist komplizierter, da $Q$ keine untere Dreiecksmatrix.}
              \[
                  Ax = b, \quad A = QR \quad \iff \quad Qy = b, \quad Rx = y
              \]
        \item Elimination des Koeffizienten an Position $i$, $j$:
              \begin{itemize}
                  \item alle vorhergehenden Koeffizienten sind schon eliminiert ($A \to \tilde{A}$)
                  \item benutze $Q_{ij}$ mit geeignetem $c$ bzw. $s$:
                        \[
                            c = \pm \frac{u_i}{w} = \pm \frac{u_i}{\sqrt{u_i^2 + u_j^2}}, \quad s = \pm \frac{u_j}{w} = \pm \frac{u_j}{\sqrt{u_i^2 + u_j^2}}
                        \]
                  \item $Q_{ij} \tilde{A}$ verändert nur die $i$-te und die $j$-te Zeile von $\tilde{A}$:\footnote{Bei der LU-Zerlegung wurde nur eine Zeile verändert.}
                        \[
                            \tilde{a}_{jk} \to c \cdot \tilde{a}_{jk} + s \cdot \tilde{a}_{ik}, \quad \tilde{a}_{ik} \to - s \cdot \tilde{a}_{jk} + c \cdot \tilde{a}_{ik}
                        \]
                  \item $Q_{ij}$ eliminiert $\tilde{a}_{ij}$, d.h. $\tilde{a}_{ij} \to 0$
                  \item $Q_{ij}$ ist durch reelle Zahl (Drehwinkel) $\rho_{ij}$ definiert
                  \item wähle Vorzeichen so, dass $c \geq 0$, d.h. $\rho_{ij} \in \left[ -\nicefrac{\pi}{2}, \nicefrac{\pi}{2} \right]$
                  \item speichere $\rho_{ij} = s$ an der Stelle von $\tilde{a}_{ij}$
                  \item wir erhalten untere Dreiecksmatrix mit Drehwinkeln $\rho_{ij}$ aus denen Matrizen $Q_{ij}$ rekonstruiert werden können
              \end{itemize}
        \item Lösen von $Qy = b$:
              \begin{itemize}
                  \item Es gilt
                        \[
                            y = Q^Tb =  Q_{n-1} \cdot \ldots \cdot Q_1 b = Q_{nn-1} \cdot \ldots \cdot Q_{21} b
                        \]
                  \item betrachte $\rho_{ij}$ in Eliminationsreihenfolge und rekonstruiere $Q_{ij}$ über
                        \[
                            s = \rho_{ij}, \quad c = \sqrt{1 - s^2}
                        \]
                  \item führe Matrix-Vektor-Produkte aus\footnote{Dabei verändert $Q_{ij} v$, $v \in \R^n$ nur $v_i$ und $v_j$.}
              \end{itemize}
        \item Auflösen von $Rx = y$ erfolgt durch Rückwärtseinsetzen
    \end{itemize}
    
    Der Aufwand des Verfahrens von Givens ist etwa viermal höher als der einer LU-Zerlegung.
\end{defi}

\begin{example}{Verfahren von Givens}
    TODO
\end{example}

\begin{bonus}{Verfahren von Givens (stabilisierte Variante)}
    Für $|s| \approx 1$ liefert $c = \sqrt{1 - s^2}$ sehr schlechte Werte für $c$ durch Auslöschung. 
    Deswegen benutzt man folgende stabilisierte Variante, um die Informationen zu den Drehmatrizen zu speichern: 
    \begin{itemize}
        \item es sei $\alpha$ der Drehwinkel, $c = \cos(\alpha)$ und $s = \sin(\alpha)$
        \item wähle freies Vorzeichen wieder so, dass $c \geq 0$, d. h. $\alpha \in \left[ -\nicefrac{\pi}{2}, \nicefrac{\pi}{2} \right]$
        \item Fallunterscheidung:
              \begin{enumerate}
                  \item $\alpha \in \left[ -\nicefrac{\pi}{4}, \nicefrac{\pi}{4} \right]$:
                        \begin{itemize}
                            \item hier gilt:
                                  \[
                                      |s| \leq \frac{\sqrt{2}}{2} \leq c
                                  \]
                            \item Auslöschungen wegen $|s| \leq \nicefrac{\sqrt{2}}{2} < 0.8$ bei $c = \sqrt{1 - s^2}$ unkritisch
                            \item wir speichern:
                                  \[
                                      \rho = s    
                                  \]
                            \item $c$ und $s$ berechnen sich durch
                                  \[
                                      s = \rho, \quad c = \sqrt{1 - s^2}
                                  \]
                        \end{itemize}
                  \item $\alpha \in \left( -\nicefrac{\pi}{2}, -\nicefrac{\pi}{4} \right) \cup \left( \nicefrac{\pi}{4}, \nicefrac{\pi}{2} \right)$:
                        \begin{itemize}
                            \item hier gilt:
                                  \[
                                      |s| > \frac{\sqrt{2}}{2} > c > 0
                                  \]
                            \item Auslöschungen wegen $0 < c < \nicefrac{\sqrt{2}}{2} < 0.8$ bei $|s| = \sqrt{1 - c^2}$ unkritisch
                            \item damit günstiger $c$ statt $s$ abzuspeichern
                            \item $c$ legt nur den Betrag von $s$ fest, das Vorzeichen muss separat behandelt werden
                            \item um zusätzlich diesen Fall vom vorherigen zu unterscheiden benutzen wir:\footnote{Der Fall ist unterscheidbar, da $|p| = \nicefrac{1}{c} > \nicefrac{2}{\sqrt{2}} = \sqrt{2} > 1$, sonst ist $\rho \leq 1$.}
                                  \[
                                      \rho = \frac{\sign(s)}{c}
                                  \]
                            \item $c$ und $s$ berechnen sich durch
                                  \[
                                      c = \frac{1}{\rho}, \quad s = \sign(\rho) \cdot \sqrt{1 - c^2}
                                  \]
                        \end{itemize}
                  \item $\alpha \in \{ -\nicefrac{\pi}{2}, \nicefrac{\pi}{2} \}$:
                        \begin{itemize}
                            \item hier gilt:
                                  \[
                                      s = \pm 1, \quad c = 0    
                                  \]
                            \item wir speichern:
                                  \[
                                      \rho = s    
                                  \]
                            \item $c$ und $s$ berechnen sich durch
                                  \[
                                      s = \rho, \quad c = \sqrt{1 - s^2} = 0  
                                  \]
                        \end{itemize}
              \end{enumerate}
    \end{itemize}
\end{bonus}

\subsubsection{Verfahren von Householder}

\begin{bonus}{Verfahren von Householder (Idee)}
    Sei eine Matrix $A \in \R^{n \times n}$ gegeben. 
    Wir versuchen die erste Spalte $a = \begin{pmatrix} a_{11} & a_{12} & \cdots & a_{n1} \end{pmatrix}^T$ von $A$ auf einen Einheitsvektor $\pm \|a\|_2 e_1$ zu spiegeln.\footnote{Zunächst wird nur $+\|a\| e_1$ betrachtet. Negatives Vorzeichen funktioniert analog.}
    
    $a$ legt die Spiegelebene $E_a$ fest, da $E_a$ senkrecht auf $v$ steht mit
    \[
        v = a - \|a\|_2 e_1
    \]
    Um ein beliebiges $x$ an $E_a$ zu spiegeln, wird:
    \begin{itemize}
        \item der Anteil von $x$ senkrecht $E_a$ bestimmt:
              \[
                  p = w \langle w, x \rangle, \quad w = \frac{v}{\|v\|_2}
              \]
        \item $p$ zweimal von $x$ subtrahiert:
              \[ 
                  x' = x - 2 p = x - 2 w \langle w, x \rangle = x - 2 ww^T x = \underbrace{(I - 2 ww^T)}_{Q_+} x = Q_+ x
              \]
    \end{itemize}
    
    Insgesamt ist eine Spiegelung an der Spiegelebene $E_a$ also eine lineare Abbildung mit Matrix:
    \[ 
        Q_+ = I - 2 ww^T, \quad w = \frac{v}{\|v\|_2}, \quad v = a - \|a\|_2 e_1
    \]
    
    bzw. für die Spiegelung auf $-\|a\|_2 e_1$:
    \[
        Q_- = I - 2 ww^T, \quad w = \frac{v}{\|v\|_2}, \quad v = a + \|a\|_2 e_1  
    \]
    
    Betrachten wir $Q_1 = Q_+$ bzw. $Q_1 = Q_-$ zu $a = a_1$. 
    Es gilt: 
    \begin{itemize}
        \item $Q_1$ mit $\|w\|_2 = 1$ definiert eine Spiegelung an der Hyperebene $\langle w, x \rangle = 0$
        \item $Q_1 = Q_1^T = Q_1^{-1}$ $\implies$ $Q_1$ orthonormal $\implies$ $\varkappa_2(Q_1) = 1$
    \end{itemize}
    
    Multiplizieren beider Seiten von $Ax = b$ mit $Q_1$ ergibt:
    \[
        Q_1 A = Q_1 b, \quad Q_1 A = 
        \begin{pmatrix}
            *      & *      & \cdots & *      \\
            0      & *      & \cdots & *      \\
            \vdots & \vdots &        & \vdots \\
            0      & *      & \cdots & *
        \end{pmatrix}
    \]
    
    Wiederholung auf die jeweilige $(n-1) \times (n-1)$-Untermatrix ergibt schließlich
    \[ 
        Q_{n-1} \cdot \ldots \cdot Q_1 Ax = Q_{n-1} \cdot \ldots \cdot Q_1 b \quad \text{mit} \quad Q_{n-1} \cdot \ldots \cdot Q_1 A = R
    \]
    
    Damit ist auch 
    \[
        A = QR, \quad Q = (Q_{n-1} \cdot \ldots \cdot Q_1)^{-1} = Q_1 \cdot \ldots \cdot Q_{n-1}
    \]
\end{bonus}

\begin{defi}{Verfahren von Householder}
    In der Praxis wird das \emph{Householder-Verfahren} wie folgt durchgeführt: 
    \begin{itemize}
        \item Analog zum LU und Givens-Verfahren:\footnote{$Q = (Q_{n-1} \cdot \ldots \cdot Q_1)^{-1} = Q_1 \cdot \ldots \cdot Q_{n-1}$}
              \[
                  Ax = b, \quad A = QR \quad \iff \quad Qy = b, \quad Rx = y \quad \iff \quad Rx = Q^T b
              \]
        \item Betrachte $A = \begin{pmatrix} a_1 & \ldots & a_n \end{pmatrix} \in \R^{n \times n}$.
        \item Berechnen und Speichern der $Q_i$ am Beispiel von $Q_1$:
              \begin{itemize}
                  \item berechne $\|a_1\|_2$ und speichere Ergebnis zwischen
                  \item bestimme
                        \[
                            v = a_1 \mp \|a_1\|_2 e_1 = 
                            \begin{pmatrix}
                                a_{11} \mp \|a_1\|_2 \\ 
                                a_{21}               \\ 
                                \vdots               \\
                                a_{n1}
                            \end{pmatrix}
                        \]
                  \item $v$ unterscheidet sich nur in der erste Komponente von der ersten Spalte der Ausgangsmatrix $\implies$ speichere $v$ über erste Spalte
                  \item wähle freies Vorzeichen, so dass Auslöschungen vermieden werden:
                        \[
                            a_{11} < 0 \quad \implies \quad - \quad \implies \quad Q_1a_1 = \|a_1\|_2 e_1 
                        \]
                        \[
                            a_{11} \geq 0 \quad \implies \quad + \quad \implies \quad Q_1a_1 = - \|a_1\|_2 e_1
                        \]
                  \item für $Q_1$ brauchen wir auch $\|v\|_2$:
                        \begin{alignat*}{1}
                            \|v\|_2^2 & = ( \mp | a_{11} \mp \|a_1\|_2 )^2 + a_{21}^2 + \cdots + a_{n1}^2                                         \\ 
                                      & = \|a_1\|_2^2 + 2 |a_{11}| \|a_1\|^2 + \underbrace{a_{11}^2 + a_{21}^2 + \cdots + a_{n1}^2}_{\|a_1\|_2^2} \\ 
                                      & = 2 (\|a_1\|_2^2 + |a_{11}| \|a_1\|_2)
                        \end{alignat*}
              \end{itemize}
        \item die Produkte $Q_1z$ werden wie folgt effizient berechnet:
              \begin{itemize}
                  \item betrachte
                        \[
                            Q_1z = \left( I - \frac{2}{\|v\|_2^2} vv^T \right) z = z - \frac{2}{\|v\|_2^2} v \underbrace{v^Tz}_{\langle v, z \rangle} = z - \frac{2 \langle v, z \rangle}{\|v\|_2^2} v
                        \]
                  \item berechne zuerst $\langle v, z \rangle$ und dann
                        \[ 
                            \alpha = \frac{2 \langle v, z \rangle}{\|v\|_2^2}
                        \]
                  \item schließlich ist
                        \[ 
                            Q_1 z = z - \alpha v
                        \]
              \end{itemize}
        \item Speicherproblem:
              \begin{itemize}
                  \item $v$ belegt ganze Spalte ab Diagonalelement abwärts
                  \item auf Diagonale sollte aber $\pm \|a_1\|_2$ stehen
                  \item speichere Diagonale in seperatem Vektor ab
              \end{itemize}
    \end{itemize}
\end{defi}

\begin{example}{Householder-Verfahren}
    TODO
\end{example}