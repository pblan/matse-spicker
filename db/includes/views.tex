\section{Views}

\begin{sql}{VIEW}
    Eine \texttt{VIEW} erzeugt aus einem \texttt{SELECT} eine Tabelle.

    Dementsprechend kann man wie folgt eine temporäre Tabelle aller Feuer-Pokémon erstellen\footnote{Das Schlüsselwort \texttt{OR REPLACE} überschreibt eine eventuell existierende Tabelle}, um bei den folgenden Abfragen Laufzeit zu sparen.\footnote{Wenn der innere \texttt{SELECT} z.B. mit einem oder mehreren \texttt{JOIN}-Befehlen kombiniert wird oder die Bedingung in der Selektion eine gewisse Komplexität überschreitet und die selbe Abfrage mehrfach benötigt wird}

    \begin{lstlisting}[language=mysql]
        -- Erstelle View feuerpokemon
        CREATE OR REPLACE VIEW feuerpokemon
        AS (
            SELECT *
            FROM pokemon
            WHERE
                PrimaerTyp = 'Feuer'
                OR SekundaerTyp = 'Feuer'
        );
        -- Gebe Entitaeten der View feuerpokemon aus
        SELECT * FROM feuerpokemon;
    \end{lstlisting}

    \begin{tabular}{I|ITIIITT}
        \rowcolor{gray!35}
        & \multicolumn{1}{T}{ID} & \multicolumn{1}{T}{Name} & \multicolumn{1}{T}{Groesse} & \multicolumn{1}{T}{Gewicht} & \multicolumn{1}{T}{Generation} & \multicolumn{1}{T}{PrimaerTyp} & \multicolumn{1}{T}{SekundaerTyp} \\\hline
        1 & 4 & Glumanda & 0.6 & 8.5 & 1 & Feuer & NULL \\
        2 & 5 & Glutexo & 1.1 & 19 & 1 & Feuer & NULL \\
        3 & 6 & Glurak & 1.7 & 90.5 & 1 & Feuer & Flug \\
        \multicolumn{1}{c|}{\dots} & \multicolumn{7}{c}{\dots} \\
        71 & 851 & Infernopod & 3 & 120 & 8 & Feuer & Käfer \\
    \end{tabular}

    Um die View zu entfernen, nutzt man \texttt{DROP}:

    \begin{lstlisting}[language=mysql]
        DROP VIEW feuerpokemon;
    \end{lstlisting}
\end{sql}