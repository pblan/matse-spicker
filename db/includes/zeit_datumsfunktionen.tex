\section{Zeit- und Datumsfunktionen}

\begin{sql}{Zeit- und Datumsfunktionen}
    MySQL nutzt \texttt{session} und \texttt{global} Konfigurationen.

    Die \texttt{session} Konfiguration wird beim Abschalten der Datenbank auf \texttt{global} zurückgesetzt.

    \begin{lstlisting}[language=mysql]
        SELECT @@session.time_zone, @@global.time_zone;;
    \end{lstlisting}

    \setcounter{rownum}{0}
    \begin{tabular}{I|TT}
        \rowcolor{gray!35}
        & \multicolumn{1}{T}{@@session.time\_zone} & \multicolumn{1}{T}{@@global.time\_zone} \\\hline
        1 & SYSTEM & SYSTEM \\
    \end{tabular}

    Die Zeitzone kann mit folgendem Befehl angepasst werden:

    \begin{lstlisting}[language=mysql]
        SET SESSION TIME_ZONE = <Zeitzone>
    \end{lstlisting}

    Dabei kann als Zeitzone folgendes eingestellt werden:

    \begin{itemize}
        \item Eine exakte Zeitzone: \lstinline[language=mysql]{'Europe/Berlin'}
        \item Ein Offset: \lstinline[language=mysql]{'+02:00'}
    \end{itemize}

    Einige Datumstypen sind:

    \begin{itemize}
        \item \texttt{date}
        \item \texttt{time}
        \item \texttt{datetime}
    \end{itemize}

    Ein Cast in verschiedene Typen ist wie folgt möglich:

    \begin{lstlisting}[language=mysql]
        SELECT
            NOW() AS 'Session',
            cast(NOW() AS DATE) AS 'Date',
            cast(NOW() AS TIME) AS 'Time',
            cast(NOW() AS DATETIME) AS 'Datetime';
    \end{lstlisting}

    \setcounter{rownum}{0}
    \begin{tabular}{I|TTTT}
        \rowcolor{gray!35}
        & \multicolumn{1}{T}{Session} & \multicolumn{1}{T}{Date} & \multicolumn{1}{T}{Time} & \multicolumn{1}{T}{Datetime} \\\hline
        1 & 1996-02-27 10:00:00 & 1996-02-27 & 10:00:00 & 1996-02-27 10:00:00 \\
    \end{tabular}

    Zusätzlich ist es ebenfalls möglich von Strings zu einem Datum zu casten:

    \begin{lstlisting}[language=mysql]
        SELECT
            STR_TO_DATE('27.02.1996', GET_FORMAT(DATE, 'EUR')) AS 'STR_TO_DATE',
            DATE_FORMAT(NOW(), GET_FORMAT(DATE, 'EUR')) AS 'DATE',
            -- %W = Wochentag, %D Tag im Monat, %M = Monat, %Y = Jahr
            DATE_FORMAT(NOW(), '%W, %D %M %Y') AS 'CUSTOMIZED DATE';
    \end{lstlisting}

    \setcounter{rownum}{0}
    \begin{tabular}{I|TTT}
        \rowcolor{gray!35}
        & \multicolumn{1}{T}{STR\_TO\_DATE} & \multicolumn{1}{T}{DATE} & \multicolumn{1}{T}{CUSTOMIZED\_DATE} \\\hline
        1 & 1996-02-27 & 27.02.1996 & Tuesday, 27th February 1996 \\
    \end{tabular}

    Ein Datum kann über \lstinline[language=mysql]{DATE_ADD()}, oder per \texttt{+} bzw. \texttt{-}, modifiziert werden:

    \begin{lstlisting}[language=mysql]
        SELECT
            NOW() !+! INTERVAL 1 YEAR,
            NOW() + INTERVAL 1 HOUR,
            !DATE_ADD!(NOW(), INTERVAL 1 YEAR);
    \end{lstlisting}

    Alternativ kann ebenfalls eine Zeitspanne ermittelt werden:

    \begin{lstlisting}[language=mysql]
        SELECT
            DATEDIFF(DATE_ADD(NOW(), INTERVAL 1 YEAR), NOW()),
            TIMEDIFF(DATE_ADD(NOW(), INTERVAL 1 HOUR), NOW());
    \end{lstlisting}
\end{sql}