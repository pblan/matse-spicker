\section{Iterative Löser für lineare Gleichungssysteme}

\subsection{Grundlagen}

\begin{defi}{Iteratives Verfahren}
    Unter einem \emph{iterativem Verfahren} versteht man ein Verfahren zur schrittweisen Annäherung an die Lösung einer Gleichung unter Anwendung eines sich wiederholenden Rechengangs.

    Ein allgemeines iteratives Verfahren funktioniert wie folgt:
    \begin{itemize}
        \item starte mit Anfangsnäherung $x_0$ von $x$
        \item erzeuge Folge von Näherungen
              \[
                  x_0 \to x_1 \to \ldots \to x_k \to \ldots
              \]
              so lange bis $x_k$ nahe genug an $x$ liegt
    \end{itemize}

    Erzeugung von $x_k$:
    \begin{itemize}
        \item benutzt man zur Berechnung von $x_{k+1}$ nur den direkten Vorgänger $x_k$, so heißt das Verfahren \emph{einstufig}, sonst \emph{mehrstufig}
        \item wir betrachten nur einstufige Verfahren, d. h.
              \[
                  x_{k+1} = \Phi_k(x_k)
              \]
              mit \emph{Verfahrensvorschrift} $\Phi_k: \R^n \to \R^n$
        \item ist $\Phi_k = \Phi$, d. h. wird in jedem Schritt die selbe Vorschrift benutzt, so heißt das Verfahren \emph{stationär}, sonst \emph{instationär}
    \end{itemize}
\end{defi}

\begin{defi}{Fehler und Residuum}
    Ist $x_{k+1} = \Phi_k(x_k)$ ein iteratives Verfahren zur Lösung von $Ax = b$, dann sind der \emph{Fehler} $e_k$ und das \emph{Residuum} $r_k$ im $k$-ten Schritt gegeben durch:\footnote{$Ae_k = A(x - x_k) = Ax - Ax_k = b - Ax_k = r_k$}
    \[
        e_k = x - x_k = A^{-1} r_k
    \]
    \[
        r_k = b - A x_k = A e_k
    \]

    Von einem guten Verfahren $\Phi_k$ erwartet man, dass $e_k$ für große $k$ klein wird und zwar unabhängig vom Startwert $x_0$.

    $\Phi_k$ heißt \emph{konvergent}, falls $\lim_{k \to \infty} e_k = 0$ für alle $x_0 \in \R^n$.
\end{defi}

\subsection{Lineare stationäre Verfahren}

\begin{bonus}{Residueniteration mit Vorkonditionierer (Idee)}
    Für ein $\Phi$ gilt:
    \[
        e_k = x - x_k = A^{-1} r_k
    \]
    d. h. aus $x_k$ und $r_k$ kann $x$ durch
    \[
        x = x_k + A^{-1} r_k
    \]
    bestimmt werden.

    Aber:
    \begin{itemize}
        \item $A^{-1}$ ist unbekannt
        \item $A^{-1}$ zu Berechnen ist genauso aufwendig wie das Lösen des Ausgangsproblems $Ax = b$
    \end{itemize}

    Die Idee ist nun $A^{-1}$ durch ein $M^{-1}$ mit $M \approx A$ zu ersetzen, welches aber \enquote{einfach} berechenbar ist.
\end{bonus}

\begin{defi}{Residueniteration mit Vorkonditionierer}
    Die \emph{Residueniteration} mit \emph{Vorkonditionierer} $M$ ist gegeben durch
    \[
        x_{k+1} = x_k + M^{-1} r_k
    \]

    $M$ wird \emph{Vorkonditionierer} genannt, da
    \[
        x_{k+1} = x_k + M^{-1} r_k = x_k + M^{-1} A e_k = x_k + M^{-1} A (x - x_k)
    \]
    Mit $M^{-1} A \approx I$ folgt
    \[
        x_{k+1} \approx x_k + k - x_k = x
    \]

    In der Praxis wird das Verfahren wie folgt durchgeführt:
    \begin{itemize}
        \item Berechne pro Schritt:\footnote{Der Aufwand besteht im Matrix-Vektor=Produkt $Ax_k$ und dem Invertieren von $M$, wobei $M$ aber einfach invertierbar sein soll.}
              \begin{itemize}
                  \item $r_k = b - Ax_k$
                  \item löse $M p_k = r_k$, d.h. $p_k = M^{-1} r_k$
                  \item $x_{k+1} = x_k + p_k$
              \end{itemize}
    \end{itemize}
\end{defi}

\begin{defi}{Stationäres lineares Iterationsverfahren (Einstufig)}
    Ein \emph{einstufiges, stationäres lineares Iterationsverfahren} ist gegeben durch
    \[
        x_{k+1} = \Phi(x_k) = B x_k + d
    \]

    wobei $B$ die \emph{Iterationsmatrix} und $d$ der \emph{Iterationsvektor} ist.
\end{defi}

\begin{bonus}{Residueniteration mit Vorkonditionierer als einstufiges, stationäres lineares Iterationsverfahren}
    Bringen wir die Formel für die Residueniteration mit Vorkonditionierer $M$ in die Standardform, so erhalten wir:
    \begin{alignat*}{2}
        x_{k+1} & = x_k + M^{-1} r_k                                                \\
                & = x_k + M^{-1} A e_k                                              \\
                & = x_k + M^{-1} A (x - x_k)                                        \\
                & = x_k + M^{-1} (b - A x_k)                                        \\
                & = x_k + M^{-1} b - M^{-1} A x_k                                   \\
                & = x_k - M^{-1} A x_k + M^{-1} b                                   \\
                & = \underbrace{(I - M^{-1} A)}_{B} x_k + \underbrace{M^{-1} b}_{d} \\
                & = B x_k + d                                                       \\
                & = \Phi(x_k)
    \end{alignat*}

    Damit ist die Residueniteration mit Vorkonditionierer $M$ ein Spezialfall eines einstufigen, stationären linearen Iterationsverfahrens.
\end{bonus}

\begin{defi}{Fixpunkteigenschaft}
    Alle Residueniterationen $\Phi(x)$ mit Vorkonditionierer $M$ besitzen die \emph{Fixpunkteigenschaft}:
    \[
        x = \Phi(x) \quad \iff \quad Ax = b
    \]
\end{defi}

\begin{defi}{Konvergenz}
    % Für die Fehlerfortpflanzung gilt:
    % \begin{alignat*}{1}
    %     e_{k+1} & = x - x_{k+1}             \\
    %             & = \Phi(x) - x_{k+1}       \\
    %             & = Bx + d - (Bx_{k+1} + d) \\
    %             & = B(x - x_{k+1})          \\
    %             & = Be_k
    % \end{alignat*}

    % und damit:
    % \[
    %     e_k = B^k e_0, \quad e_0 = x - x_0
    % \]
    Für \emph{Konvergenz} muss gelten:\footnote{Wir erkennen, dass das $d$ in $\Phi(x)$ die Konvergenz nicht beeinflusst. Es stellt nur die Fixpunkteigenschaft sicher.}
    \[
        e_k \xrightarrow{k \to \infty} 0, \quad \forall x_0 \in \mathbb{R}^n \quad \iff \quad B^k e_0 \xrightarrow{k \to \infty} 0, \quad \forall e_0 \in \mathbb{R}^n
    \]

    Sei $\| \cdot \|$ eine belieblige Vektornorm auf $\R^n$ und $\| \cdot \|_M$ eine beliebige Matrixnorm auf $\R^{n \times n}$.
    Für
    \[
        x_{k+1} = \Phi(x_k) = Bx_k + d
    \]

    gilt dann\footnote{Die Konvergenz des Verfahrens ist also unabhängig von der benutzten Matrix- und Vektornorm.}
    \[
        \lim_{k \to \infty} \|e_k\| = 0 \quad \forall e_0 \in \mathbb{R}^n \quad \iff \quad \lim_{k \to \infty} \|B^k\|_M = 0
    \]

    Für den Spektralradius $\rho(A) = \max \{ | \lambda | \}$ (betragsgrößter Eigenwert) gilt:
    \begin{itemize}
        \item $\rho(A) \leq \|A\|$ für alle induzierten Matrixnormen
        \item $\forall \epsilon > 0$ existiert eine induzierte Matrixnorm mit $\rho(A) \leq \|A\| \leq \rho(A) + \epsilon$
        \item Für jede Matrixnorm gilt
              \[
                  \rho(A) = \lim_{k \to \infty} \| A^k \|^{\frac{1}{k}}
              \]
    \end{itemize}

    Sei $A$ regulär, $x_{k+1} = \Phi(x_k) = Bx_k + d$.
    $\Phi$ \emph{konvergiert} genau dann, wenn $\rho(B) < 1$.

    $\Phi$ konvergiert monoton, wenn $\|B\| < 1$.

    Je nach Matrixnorm kann $\| B \| \geq 1$ gelten, obwohl $\rho(B) < 1$.
    Das Verfahren konvergiert hier auch, aber nicht notwendig monoton.
\end{defi}

\begin{example}{Konvergenz}
    TODO
\end{example}

\begin{defi}{Jacobi-Verfahren (Gesamtschritt-Verfahren)}
    Das \emph{Jacobi-Verfahren} ist ein lineares, stationäres Iterationsverfahren.
    Die Matrix $A$ des linearen Gleichungssystems $Ax = b$ wird hierzu in einer Diagonalmatrix $D$, eine strikte untere Dreiecksmatrix $E$ und eine strikte obere Dreiecksmatrix $F$ zerlegt, so dass gilt:
    \[
        A = D - E - F
    \]

    Ausführlich heißt das:
    {
    \footnotesize
    \[
        \underbrace{
            \begin{pmatrix}
                a_{11} & a_{12} & \cdots & a_{1n} \\
                a_{21} & a_{22} & \cdots & a_{2n} \\
                \vdots & \vdots & \ddots & \vdots \\
                a_{n1} & a_{n2} & \cdots & a_{nn}
            \end{pmatrix}
        }_A
        =
        \underbrace{
            \begin{pmatrix}
                a_{11} & 0      & \cdots & 0      \\
                0      & a_{22} & \cdots & 0      \\
                \vdots & \vdots & \ddots & \vdots \\
                0      & 0      & \cdots & a_{nn}
            \end{pmatrix}
        }_D
        -
        \underbrace{
            \begin{pmatrix}
                0        & \cdots & \cdots     & 0      \\
                - a_{21} & \ddots &            & \vdots \\
                \vdots   & \ddots & \ddots     & \vdots \\
                - a_{n1} & \cdots & - a_{nn-1} & 0
            \end{pmatrix}
        }_E
        -
        \underbrace{
            \begin{pmatrix}
                0      & - a_{12} & \cdots & - a_{1n}   \\
                \vdots & \ddots   & \ddots & \vdots     \\
                \vdots &          & \ddots & - a_{n-1n} \\
                0      & \cdots   & \cdots & 0
            \end{pmatrix}
        }_F
    \]
    }

    Die komponentenweise Iterationsvorschrift lässt sich dann folgendermaßen für den kompletten Vektor darstellen (wir benutzen $M = D$):
    \begin{alignat*}{1}
        x^{(k+1)} & = x^{(k)} + M^{-1} r^{(k)}                                                   \\
                  & = x^{(k)} + D^{-1} \left( b - (D - E - F) x^{(k)} \right)                    \\
                  & = x^{(k)} + D^{-1} \left( b + \left( E + F \right) x^{(k)} \right) - x^{(k)} \\
                  & = D^{-1} \left( b + (E + F) x^{(k)} \right)
    \end{alignat*}

    Wegen
    \[
        D^{-1} =
        \begin{pmatrix}
            \frac{1}{a_{11}} & 0                & \cdots & 0                \\
            0                & \frac{1}{a_{22}} & \cdots & 0                \\
            \vdots           & \vdots           & \ddots & \vdots           \\
            0                & 0                & \cdots & \frac{1}{a_{nn}}
        \end{pmatrix}
    \]
    erhalten wir komponentenweise
    \[
        x_i^{(k+1)} = \frac{1}{a_{ii}} \left( b_i - \sum_{j \neq i} a_{ij} x_j^{(k)} \right), \quad i = 1, \ldots, n
    \]

    Wir erhalten die Iterationsmatrix bzw. -vektor
    \[
        B_{J} = I - M^{-1} A = I - D^{-1} (D - E - F) = D^{-1} (E + F), \quad d_{J} = M^{-1} b = D^{-1} b
    \]
\end{defi}

\begin{example}{Jacobi-Verfahren (Gesamtschritt-Verfahren)}
    Gegeben sei eine Matrix $A$, ein Vektor $b$ und ein Startpunkt $x^{(0)}$:
    \[
        A = \begin{pmatrix}
            2  & 3 \\
            -1 & 2
        \end{pmatrix}
        , \quad
        b = \begin{pmatrix}
            1 \\
            3
        \end{pmatrix}
        , \quad
        x^{(0)} = \begin{pmatrix}
            0 \\
            0
        \end{pmatrix}
    \]

    Führen Sie das Jacobi-Verfahren für drei Iterationen aus.

    \exampleseparator

    Es gilt die komponentenweise Iterationsvorschrift:
    \[
        x_i^{(k+1)} = \frac{1}{a_{ii}} \left( b_i - \sum_{j \neq i} a_{ij} x_j^{(k)} \right), \quad i = 1, \ldots, n
    \]

    Für $i = 1$:
    \[
        x_1^{(k+1)} = \frac{1}{a_{11}} \left( b_1 - \sum_{j \neq 1} a_{1j} x_j^{(k)} \right) = \frac{1}{a_{11}} \left( b_1 - a_{12} x_2^{(k)} \right) = \frac{1}{2} \left( 1 - 3 x_2^{(k)} \right) = \frac{1}{2} - \frac{3}{2}x_2^{(k)}
    \]

    Für $i = 2$:
    \[
        x_2^{(k+1)} = \frac{1}{a_{22}} \left( b_2 - \sum_{j \neq 2} a_{2j} x_j^{(k)} \right) = \frac{1}{a_{22}} \left( b_2 - a_{21} x_1^{(k)} \right) = \frac{1}{2} \left( 3 - (-1) x_1^{(k)} \right) = \frac{3}{2} + \frac{1}{2}x_1^{(k)}
    \]

    Die Iterationen sind dann wie folgt:
    \[
        x^{(0)} = \begin{pmatrix}
            0 \\
            0
        \end{pmatrix}
        \to
        x^{(1)} = \begin{pmatrix}
            \frac{1}{2} - \frac{3}{2}x_2^{(0)} \\
            \frac{3}{2} + \frac{1}{2}x_1^{(0)}
        \end{pmatrix}
        = \begin{pmatrix}
            \frac{1}{2} - \frac{3}{2} \cdot 0 \\
            \frac{3}{2} + \frac{1}{2} \cdot 0
        \end{pmatrix}
        = \begin{pmatrix}
            \nicefrac{1}{2} \\
            \nicefrac{3}{2}
        \end{pmatrix}
    \]
    \[
        x^{(1)} = \begin{pmatrix}
            \nicefrac{1}{2} \\
            \nicefrac{3}{2}
        \end{pmatrix}
        \to
        x^{(2)} = \begin{pmatrix}
            \frac{1}{2} - \frac{3}{2}x_2^{(1)} \\
            \frac{3}{2} + \frac{1}{2}x_1^{(1)}
        \end{pmatrix}
        = \begin{pmatrix}
            \frac{1}{2} - \frac{3}{2} \cdot \frac{3}{2} \\
            \frac{3}{2} + \frac{1}{2} \cdot \frac{1}{2}
        \end{pmatrix}
        = \begin{pmatrix}
            -\nicefrac{7}{2} \\
            \nicefrac{7}{2}
        \end{pmatrix}
    \]
    \[
        x^{(2)} = \begin{pmatrix}
            -\nicefrac{7}{2} \\
            \nicefrac{7}{2}
        \end{pmatrix}
        \to
        x^{(3)} = \begin{pmatrix}
            \frac{1}{2} - \frac{3}{2}x_2^{(2)} \\
            \frac{3}{2} + \frac{1}{2}x_1^{(2)}
        \end{pmatrix}
        = \begin{pmatrix}
            \frac{1}{2} - \frac{3}{2} \cdot \frac{7}{2} \\
            \frac{3}{2} + \frac{1}{2} \cdot (-\frac{7}{2})
        \end{pmatrix}
        = \begin{pmatrix}
            -\nicefrac{17}{8} \\
            \nicefrac{5}{8}
        \end{pmatrix}
    \]
\end{example}

\begin{defi}{Diagonaldominante Matrix}
    Eine Matrix $A \in \R^{n \times n}$ heißt \emph{streng diagonaldominant}, falls die Beträge ihrer Diagonalelemente $a_{ii}$ jeweils größer sind als die Summe der Beträge der restlichen jeweiligen Zeileneinträge $a_{ij}$, d.h.
    \[
        |a_{ii}| > \sum_{j \neq i} |a_{ij}|, \quad \forall i = 1, \ldots, n
    \]

    Bei einigen Verfahren zum Lösen von Gleichungssystemen (z. B. Gauß-Seidel-, Jacobi-Verfahren) bietet die Diagonaldominanz der Systemmatrix, ein hinreichendes Kriterium für die Konvergenz des Verfahrens.

    Eine Matrix $A \in \R^{n \times n}$ heißt \emph{schwach diagonaldominant}, falls die Beträge ihrer Diagonalelemente $a_{ii}$ jeweils größer oder gleich der Summe der Beträge der restlichen jeweiligen Zeileneinträge $a_{ij}$ sind, d.h.
    \[
        |a_{ii}| \geq \sum_{j \neq i} |a_{ij}|, \quad \forall i = 1, \ldots, n
    \]

    Reelle, symmetrische, schwach diagonaldominante Matrizen mit nichtnegativen Diagonaleinträgen sind positiv semidefinit (spd).
\end{defi}

\begin{defi}{Gauß-Seidel-Verfahren (Einzelschritt-Verfahren)}
    Das \emph{Gauß-Seidel-Verfahren} ist ein lineares, stationäres Iterationsverfahren.
    Gegeben ist ein lineares Gleichungssystem in $n$ Variablen mit den $n$ Gleichungen
    \[
        \begin{matrix}
            a_{11} x_1 & + & \ldots & + & a_{1n} x_n & =      & b_1 \\
            a_{21} x_1 & + & \ldots & + & a_{2n} x_n & =      & b_2 \\
                       &   &        &   &            & \vdots &     \\
            a_{n1} x_1 & + & \ldots & + & a_{nn} x_n & =      & b_n
        \end{matrix}
    \]

    Um dieses zu lösen, wird ein Iterationsverfahren durchgeführt.
    Gegeben sei ein Näherungsvektor $x^{(k)}$ an die exakte Lösung.
    Nun wird die $k$-te Gleichung nach der $k$-ten Variablen $x_k$ aufgelöst, wobei die vorher berechneten Werte des aktuellen Iterationsschritts mit verwendet werden, im Gegensatz zum Jacobi-Verfahren, bei dem nur die Werte des letzten Iterationsschrittes verwendet werden.

    Das heißt für den $(k+1)$-ten Iterationsschritt:
    \[
        x_i^{(k+1)} = \frac{1}{a_{ii}} \left( b_i - \sum_{j < i} a_{ij} x_j^{(k+1)} - \sum_{j > i} a_{ij} x_j^{(k)} \right), \quad i = 1, \ldots, n
    \]

    Das Ergebnis der Rechnung ist ein neuer Näherungsvektor $x^{(k+1)}$ für den gesuchten Lösungsvektor $x$.
    Wiederholt man diesen Vorgang, gewinnt man eine Folge von Werten, die sich dem Lösungsvektor im Falle der Konvergenz immer mehr annähern:
    \[
        x^{(0)}, x^{(1)}, x^{(2)}, \ldots \to x
    \]

    Wir erhalten die Iterationsmatrix bzw. -vektor (wir verwenden $M = D - E$)
    \[
        B_{GS} = I - M^{-1} A = I - (D - E)^{-1} (D - E - F) = (D - E)^{-1} F, \quad d_{GS} = M^{-1} b = (D - E)^{-1} b
    \]

    Ist A spd, so konvergiert das Gauß-Seidel-Verfahren.\footnote{Die Konvergenz ist monoton in der Norm $\| \cdot \|_A$.}
\end{defi}

\begin{example}{Gauß-Seidel-Verfahren (Einzelschritt-Verfahren)}
    Gegeben sei eine Matrix $A$, ein Vektor $b$ und ein Startpunkt $x^{(0)}$:
    \[
        A = \begin{pmatrix}
            2  & 3 \\
            -1 & 2
        \end{pmatrix}
        , \quad
        b = \begin{pmatrix}
            1 \\
            3
        \end{pmatrix}
        , \quad
        x^{(0)} = \begin{pmatrix}
            0 \\
            0
        \end{pmatrix}
    \]

    Führen Sie das Gauß-Seidel-Verfahren für drei Iterationen aus.

    \exampleseparator

    Es gilt die komponentenweise Iterationsvorschrift:
    \[
        x_i^{(k+1)} = \frac{1}{a_{ii}} \left( b_i - \sum_{j < i} a_{ij} x_j^{(k+1)} - \sum_{j > i} a_{ij} x_j^{(k)} \right), \quad i = 1, \ldots, n
    \]

    Für $i = 1$:
    \[
        x_1^{(k+1)} = \frac{1}{a_{11}} \left( b_1 - \underbrace{\sum_{j < 1} a_{1j} x_j^{(k+1)}}_{= 0} - \sum_{j > 1} a_{1j} x_j^{(k)} \right) = \frac{1}{2} \left( 1 - 3 x_2^{(k)} \right) = \frac{1}{2} - \frac{3}{2} x_2^{(k)}
    \]

    Für $i = 2$:
    \[
        x_2^{(k+1)} = \frac{1}{a_{22}} \left( b_2 - \sum_{j < 2} a_{2j} x_j^{(k+1)} - \underbrace{\sum_{j > 2} a_{2j} x_j^{(k)}}_{= 0} \right) = \frac{1}{2} \left( 3 - (-1) x_1^{(k+1)} \right) = \frac{3}{2} + \frac{1}{2} x_1^{(k+1)}
    \]

    Die Iterationen sind dann wie folgt:
    \[
        x^{(0)} = \begin{pmatrix}
            0 \\
            0
        \end{pmatrix}
        \to
        x^{(1)} = \begin{pmatrix}
            \frac{1}{2} - \frac{3}{2} x_2^{(0)} \\
            \frac{3}{2} + \frac{1}{2} x_1^{(1)}
        \end{pmatrix}
        = \begin{pmatrix}
            \frac{1}{2} - \frac{3}{2} \cdot 0 \\
            \frac{3}{2} + \frac{1}{2} x_1^{(1)}
        \end{pmatrix}
        = \begin{pmatrix}
            \nicefrac{1}{2} \\
            \frac{3}{2} + \frac{1}{2} \cdot \frac{1}{2}
        \end{pmatrix}
        = \begin{pmatrix}
            \nicefrac{1}{2} \\
            \nicefrac{7}{4}
        \end{pmatrix}
    \]
    \[
        x^{(1)} = \begin{pmatrix}
            \nicefrac{1}{2} \\
            \nicefrac{7}{4}
        \end{pmatrix}
        \to
        x^{(2)} = \begin{pmatrix}
            \frac{1}{2} - \frac{3}{2}x_2^{(1)} \\
            \frac{3}{2} + \frac{1}{2}x_1^{(2)}
        \end{pmatrix}
        = \begin{pmatrix}
            \frac{1}{2} - \frac{3}{2} \cdot \frac{7}{4} \\
            \frac{3}{2} + \frac{1}{2}x_1^{(2)}
        \end{pmatrix}
        = \begin{pmatrix}
            -\nicefrac{17}{8} \\
            \frac{3}{2} + \frac{1}{2} \cdot (-\frac{17}{8})
        \end{pmatrix}
        = \begin{pmatrix}
            -\nicefrac{17}{8} \\
            \nicefrac{7}{16}
        \end{pmatrix}
    \]
    \[
        x^{(2)} = \begin{pmatrix}
            -\nicefrac{17}{8} \\
            \nicefrac{7}{16}
        \end{pmatrix}
        \to
        x^{(3)} = \begin{pmatrix}
            \frac{1}{2} - \frac{3}{2}x_2^{(2)} \\
            \frac{3}{2} + \frac{1}{2}x_1^{(3)}
        \end{pmatrix}
        = \begin{pmatrix}
            \frac{1}{2} - \frac{3}{2} \cdot \frac{7}{16} \\
            \frac{3}{2} + \frac{1}{2}x_1^{(3)}
        \end{pmatrix}
        = \begin{pmatrix}
            -\nicefrac{5}{32} \\
            \frac{3}{2} + \frac{1}{2} \cdot (-\frac{5}{32})
        \end{pmatrix}
        = \begin{pmatrix}
            -\nicefrac{5}{32} \\
            \nicefrac{91}{64}
        \end{pmatrix}
    \]
\end{example}

\begin{defi}{Satz von Stein und Rosenberg}
    Die Iterationsmatrizen vom Jacobi- und Gauß-Seidel-Verfahren sind gegeben durch
    \[
        B_{J} = D^{-1} (E + F), \quad B_{GS} = (D - E)^{-1} F
    \]

    Sei $A = D - E - F$, $D$ invertierbar, $E + F > 0$ (komponentenweise) und $\rho(B_{J}) < 1$.
    Dann gilt:
    \[
        \rho(B_{GS}) < \rho(B_{J})
    \]

    Für allgemeines $A$ gibt es Gegenbeispiele mit:
    \begin{itemize}
        \item Jacobi konvergent, Gauß-Seidel nicht
        \item Gauß-Seidel konvergent, Jacobi nicht
        \item Jacobi schneller als Gauß-Seidel
        \item Gauß-Seidel schneller als Jacobi
    \end{itemize}
\end{defi}

\begin{defi}{Nachiteration}
    Je besser $M$ die Matrix $A$ approximiert desto schneller konvergiert die Residueniteration.

    Wir betrachten einen direkten Löser, wie z.B. die LU-Zerlegung.
    Wegen Rundungsfehlern wird eine ungenaue Zerlegung $\tilde{L} \tilde{U} \approx A$ bestimmt und $\tilde{x}$ mit $A \tilde{x} \approx b$ berechnet.

    Wir erzeugen eine lineare stationäre Iteration mit $M = \tilde{L} \tilde{U}$, die sogenannte \emph{Nachiteration}, die die Qualität von $\tilde{x}$ verbessert.

    In der Praxis wird das Verfahren wie folgt durchgeführt:
    \begin{enumerate}
        \item Bestimme mit LU-Zerlegung $\tilde{x}$, $\tilde{L}$,  $\tilde{U}$
        \item Setze $x_0 = \tilde{x}$
        \item Wiederhole, bis $\frac{\| p_k \|}{\|x_{k+1} \|} < \epsilon$:
              \begin{enumerate}
                  \item $r_k = b - A \tilde{x}_k$
                  \item $p_k = M^{-1} r_k$, d. h. $\tilde{L} \tilde{U} p_k = r_k$
                  \item $x_{k+1} = x_k + p_k$
              \end{enumerate}
    \end{enumerate}

    Ohne Rundungsfehler wäre $M = LU = A$ und $M^{-1} = A^{-1}$ d.h. nach einem Schritt hätten wir die exakte Lösung.

    Mit Rundungsfehlern können wir das nicht erwarten, aber trotzdem ist in der Regel
    \[
        B = I - M^{-1} A = I - (\tilde{L} \tilde{U})^{-1} A
    \]
    nahe an der Nullmatrix und damit $\rho(B) \ll 1$, d.h. der Iterationsprozess sollte sehr schnell konvergieren.

    Eine schlechte Konvergenz ist ein Indikator für schlechte Kondition des Ausgangsproblems $Ax = b$.

    $r_k$ liegt nahe bei $0$, d.h. Fehler durch Auslöschung spielen eventuell eine große Rolle, deswegen wird $r_k$ oft in höherer Genauigkeit berechnet.

    Ein $r_k$ mit hoher Genauigkeit führt dazu, dass eine Iteration ausreicht.

    Ein $r_k$ mit leicher Genauigkeit wie LU-Zerlegung lohnt sich ebenfalls, da die Stabilitätseigenschaften verbessert werden.
\end{defi}

\begin{example}{Nachiteration}
    Gegeben sei eine Matrix $A$ ein Vektor $b$, eine ungenaue Zerlegung $\tilde{L} \tilde{U} \approx A$ mit:
    \[
        A = \begin{pmatrix}
            2 & 2 & 0 \\
            2 & 4 & 2 \\
            0 & 2 & 4
        \end{pmatrix}
        , \quad
        b = \begin{pmatrix}
            2 \\
            0 \\
            2
        \end{pmatrix}
        , \quad
        \tilde{L} = \begin{pmatrix}
            1 & 0 & 0 \\
            1 & 1 & 0 \\
            1 & 1 & 1
        \end{pmatrix}
        , \quad
        \tilde{U} = \begin{pmatrix}
            2 & 2 & 1 \\
            0 & 2 & 2 \\
            0 & 0 & 2
        \end{pmatrix}
    \]

    Berechnen Sie die Näherungslösung $x_0$ für $A x_0 \approx b$ und führen Sie eine Nachiteration durch.

    \exampleseparator

    \footnotesize

    Es gilt:
    \[
        \tilde{L} \tilde{U} x_0 \approx b \quad \implies \quad \tilde{L} y_0 = b, \quad \tilde{U} x_0 = y_0
    \]
    \[
        \tilde{L} y_0 = b \quad \iff \quad \begin{pmatrix}
            1 & 0 & 0 \\
            1 & 1 & 0 \\
            1 & 1 & 1
        \end{pmatrix}
        \begin{pmatrix}
            y_1 \\
            y_2 \\
            y_3
        \end{pmatrix}
        =
        \begin{pmatrix}
            2 \\
            0 \\
            2
        \end{pmatrix}
        \quad \implies \quad
        y_0 = \begin{pmatrix}
            2  \\
            -2 \\
            2
        \end{pmatrix}
    \]
    \[
        \tilde{U} x_0 = y_0 \quad \iff \quad \begin{pmatrix}
            2 & 2 & 1 \\
            0 & 2 & 2 \\
            0 & 0 & 2
        \end{pmatrix}
        \begin{pmatrix}
            x_1 \\
            x_2 \\
            x_3
        \end{pmatrix}
        =
        \begin{pmatrix}
            2  \\
            -2 \\
            2
        \end{pmatrix}
        \quad \implies \quad
        x_0 = \begin{pmatrix}
            \nicefrac{5}{2} \\
            -2              \\
            1
        \end{pmatrix}
    \]

    Es gilt:
    \begin{itemize}
        \item $r_k = b - A x_k$
        \item $\tilde{L} \tilde{U} p_k = r_k$
        \item $x_{k+1} = x_k + p_k$
    \end{itemize}

    $k = 0$:
    \[
        r_0 = b - A x_0 =
        \begin{pmatrix}
            2 \\
            0 \\
            2
        \end{pmatrix}
        -
        \begin{pmatrix}
            2 & 2 & 0 \\
            2 & 4 & 2 \\
            0 & 2 & 4
        \end{pmatrix}
        \begin{pmatrix}
            \nicefrac{5}{2} \\
            -2              \\
            1
        \end{pmatrix}
        =
        \begin{pmatrix}
            2 \\
            0 \\
            2
        \end{pmatrix}
        -
        \begin{pmatrix}
            1  \\
            -1 \\
            0
        \end{pmatrix}
        =
        \begin{pmatrix}
            1 \\
            1 \\
            2
        \end{pmatrix}
    \]
    \[
        \tilde{L} \tilde{U} p_0 = r_0 \quad \implies \quad \tilde{L} z_0 = r_0, \quad \tilde{U} p_0 = z_0
    \]
    \[
        \tilde{L} z_0 = r_0 \quad \iff \quad \begin{pmatrix}
            1 & 0 & 0 \\
            1 & 1 & 0 \\
            1 & 1 & 1
        \end{pmatrix}
        \begin{pmatrix}
            z_1 \\
            z_2 \\
            z_3
        \end{pmatrix}
        =
        \begin{pmatrix}
            1 \\
            1 \\
            2
        \end{pmatrix}
        \quad \implies \quad
        z_0 = \begin{pmatrix}
            1 \\
            0 \\
            0
        \end{pmatrix}
    \]
    \[
        \tilde{U} p_0 = z_0 \quad \iff \quad \begin{pmatrix}
            2 & 2 & 1 \\
            0 & 2 & 2 \\
            0 & 0 & 2
        \end{pmatrix}
        \begin{pmatrix}
            p_1 \\
            p_2 \\
            p_3
        \end{pmatrix}
        =
        \begin{pmatrix}
            1 \\
            0 \\
            0
        \end{pmatrix}
        \quad \implies \quad
        p_0 = \begin{pmatrix}
            \nicefrac{3}{4}  \\
            -\nicefrac{1}{2} \\
            \nicefrac{1}{2}
        \end{pmatrix}
    \]
    \[
        x_1 = x_0 + p_0 =
        \begin{pmatrix}
            \nicefrac{5}{2} \\
            -2              \\
            1
        \end{pmatrix}
        +
        \begin{pmatrix}
            \nicefrac{3}{4}  \\
            -\nicefrac{1}{2} \\
            \nicefrac{1}{2}
        \end{pmatrix}
        =
        \begin{pmatrix}
            \nicefrac{13}{4} \\
            -\nicefrac{5}{2} \\
            \nicefrac{3}{2}
        \end{pmatrix}
    \]

    Kennen wir die exakte Lösung $x = \begin{pmatrix} 4 & -3 & 2 \end{pmatrix}^T$ für $A x = b$, dann können wir folgende Fehlerbetrachtungen durchführen:
    \[
        e_0 = x - x_0 =
        \begin{pmatrix}
            4  \\
            -3 \\
            2
        \end{pmatrix}
        -
        \begin{pmatrix}
            \nicefrac{5}{2} \\
            -2              \\
            1
        \end{pmatrix}
        =
        \begin{pmatrix}
            \nicefrac{3}{2} \\
            -1              \\
            1
        \end{pmatrix}
        \quad \implies \quad
        \|e_0\|_\infty = \frac{3}{2} = 1.5
    \]
    \[
        e_1 = x - x_1 =
        \begin{pmatrix}
            4  \\
            -3 \\
            2
        \end{pmatrix}
        -
        \begin{pmatrix}
            \nicefrac{13}{4} \\
            -\nicefrac{5}{2} \\
            \nicefrac{3}{2}
        \end{pmatrix}
        =
        \begin{pmatrix}
            \nicefrac{3}{4}  \\
            -\nicefrac{1}{2} \\
            \nicefrac{1}{2}
        \end{pmatrix}
        \quad \implies \quad
        \|e_1\|_\infty = \frac{3}{4} = 0.75
    \]
\end{example}

\subsubsection{Relaxationsverfahren}

\begin{defi}{Relaxiertes Jacobi-Verfahren}
    Nach dem Jacobi-Verfahren gilt:
    \[
        x^{(k+1)} = x^{(k)} + D^{-1} r^{(k)}
    \]

    Fügen wir vor dem Korrekturterm einen Parameter $\omega \in \R$ ein und versuchen darüber die Konvergenz günstig zu beeinflussen so erhalten wir das \emph{relaxierte Jacobi-Verfahren} $J_{\omega}$
    \[
        x^{(k+1)} = x^{(k)} + \omega D^{-1} r^{(k)}
    \]

    Wir erhalten die Iterationsmatrix bzw. -vektor analog zum Jacobi-Verfahren:
    \[
        B_{J_{\omega}} = (1 - \omega) I + \omega B_{J} =  (1 - \omega) I + \omega D^{-1} (E + F), \quad d_{J_{\omega}} = \omega d_{J} = \omega D^{-1} b
    \]

    Komponentenweise gilt dann
    \[
        x_i^{(k+1)} = (1 - \omega) x_i^{(k)} + \frac{\omega}{a_{ii}} \left( b_i - \sum_{j \neq i} a_{ij} x_j^{(k)} \right), \quad i = 1, \ldots, n
    \]

    Die Konvergenz wird durch $\rho(B_{J_{\omega}})$ bestimmt.
    Es gilt:
    \[
        B_{J_{\omega}} = (1 - \omega) I + \omega B_{J} \quad \implies \quad \lambda_i(B_{J_{\omega}}) = 1 - \omega + \omega \lambda_i(B_{J})
    \]

    Ist $\omega \in \left(0, 1\right]$, und konvergiert das Jacobi-Verfahren, dann konvergiert auch das relaxierte Jacobi-Verfahren.

    Wähle $\omega$ so, dass $\rho(B_{J_{\omega}})$ möglichst klein ist, d. h. $\omega_{\text{opt}}$ ist die Lösung von
    \[
        \max_{i = 1, \ldots, n} | 1 - \omega + \omega \lambda_i(B_{J}) | \to \min
    \]

    Hat $B_{J}$ reelle Eigenwerte $-1 < \lambda_1 \leq \ldots \leq \lambda_n < 1$, so ist $\rho(B_{J_{\omega}})$ minimal für
    \footnote{Die Eigenwerte $\lambda_i$ sind in der Regel nicht bekannt, sie werden deswegen entweder abgeschätzt oder $\omega$ wird empirisch bestimmt.}
    \[
        \omega_{\text{opt}} = \frac{2}{2 - \lambda_1 - \lambda_n}
    \]
\end{defi}

\begin{defi}{SOR-Verfahren}
    Das \emph{Successive Over-Relaxation}-Verfahren (Überrelaxationsverfahren) oder \emph{SOR-Verfahren} ist ein Algorithmus der numerischen Mathematik zur näherungsweisen Lösung von linearen Gleichungssystemen.

    Wendet man den Gauß-Seidel-Trick auf das relaxierte Jacobi-Verfahren an, so erhält man das SOR-Verfahren mit
    \[
        x_i^{(k+1)} = (1 - \omega) x_i^{(k)} + \frac{\omega}{a_{ii}} \left( b_i - \sum_{j < i} a_{ij} x_j^{(k+1)} - \sum_{j > i} a_{ij} x_j^{(k)} \right), \quad i = 1, \ldots, n
    \]

    Für das SOR-Verfahren gilt also:
    \[
        M_{SOR} = \frac{D - E}{\omega}
    \]
    und damit
    \[
        B_{SOR} = (I - \omega D^{-1}E)^{-1} ((1 - \omega) I + \omega D^{-1} F), \quad d_{SOR} = \omega M_{SOR}^{-1} b
    \]
\end{defi}

\begin{defi}{Satz von Kahan}
    Es gilt
    \[
        \rho(B_{SOR}) \geq |\omega - 1|
    \]
    d. h. das SOR-Verfahren kann für $\omega \notin (0, 2)$ \emph{nicht konvergent} sein.
\end{defi}

\begin{defi}{Satz von Ostrowski und Reich}
    Ist $A$ spd dann konvergiert das SOR-Verfahren genau dann wenn $\omega \in (0, 2)$.

    Die Konvergenz ist monoton in $\| \cdot \|_A$.
\end{defi}

\subsubsection{Symmetrische Varianten}

\begin{bonus}{Symmetrisches Jacobi-Verfahren}
    Für das Jacobi-Verfahren hatten wir:
    \[
        x_i^{(k+1)} = \frac{1}{a_{ii}} \left( b_i - \sum_{j \neq i} a_{ij} x_j^{(k)} \right), \quad i = 1, \ldots, n
    \]

    Durchlaufen wir den Index $i$ in umgekehrter Reihenfolge, so ändert sich nichts.

    Das Jacobi-Verfahren ist also symmetrisch.
\end{bonus}

\begin{defi}{Symmetrisches Gauß-Seidel-Verfahren}
    Für das Gauß-Seidel-Verfahren hatten wir:
    \[
        x_i^{(k+1)} = \frac{1}{a_{ii}} \left( b_i - \sum_{j < i} a_{ij} x_j^{(k+1)} - \sum_{j > i} a_{ij} x_j^{(k)} \right), \quad i = 1, \ldots, n
    \]

    Durchlaufen wir den Index $i$ von $1$ bis $n$, so benutzen wir für $x_i^{(k+1)}$ alle $x_j^{(k+1)}$ mit $j < i$ und erhalten die Gleichung oben.

    Durchlaufen wir den Index $i$ von $n$ bis $1$, so benutzen wir für $x_i^{(k+1)}$ alle $x_j^{(k+1)}$ mit $j > i$ und erhalten als neues Verfahren das \emph{Rückwärts-Gauß-Seidel-Verfahren}:
    \[
        x_i^{(k+1)} = \frac{1}{a_{ii}} \left( b_i - \sum_{j < i} a_{ij} x_j^{(k)} - \sum_{j > i} a_{ij} x_j^{(k+1)} \right), \quad i = n, \ldots, 1
    \]

    Vorkonditionierer, Iterationsmatrix und Iterationsvektor sind dann gegeben durch:
    \[
        M_{RGS} = D - F, \quad B_{RGS} = (D - F)^{-1} E, \quad d_{RGS} = M_{RGS}^{-1} b
    \]

    Für das \emph{symmetrische Gauß-Seidel-Verfahren (SGS)} führt man zunächst einen Gauß-Seidel Schritt und dann einen Rückwarts-Gauß-Seidel Schritt durch:
    \[
        x^{(k)} \xrightarrow{GS} B_{GS} x^{(k)} + d_{GS} \xrightarrow{RGS} \underbrace{ B_{RGS} ( B_{GS} x^{(k)} + d_{GS} ) + d_{RGS} }_{x^{(k+1)}}
    \]
    \[
        \implies x^{(k+1)} = \underbrace{ B_{RGS} B_{GS} }_{B_{SGS}} x^{(k)} + \underbrace{ B_{RGS} d_{GS} + d_{RGS} }_{d_{SGS}}
    \]

    Damit erhalten wir:
    \[
        B_{SGS} = (D - F)^{-1} E (D-E)^{-1} F, \quad M_{SGS} = (D - E) D^{-1} (D-F), \quad d_{SGS} = M^{-1}_{SGS} b
    \]

    Ist $A$ symmetrisch und $a_{ii} > 0$, $i = 1, \ldots, n$ so ist $M_{SGS}$ spd.
\end{defi}

\subsection{Nichtstationäre Iterationen}

\subsubsection{Steepest-Descent-Verfahren}

\begin{defi}{Steepest-Descent-Verfahren}
    Das \emph{Steepest-Descent-Verfahren} generiert ausgehend von einem Startpunkt $x_0 \in \R^n$ eine Folge von Punkten $x_k \in \R^n$ gemäß der folgenden Iteration:

    Starte mit $x_0$ gegeben, $r_0 = b - Ax_0$ und wiederhole für $M = I$:
    \[
        p_k = M^{-1} r_k = r_k
    \]
    \[
        s_k = A p_k
    \]
    \[
        \alpha_k = \frac{\langle p_k, r_k \rangle}{\langle p_k, s_k \rangle}
    \]
    \[
        x_{k+1} = x_k + \alpha_k p_k
    \]
    \[
        r_{k+1} = r_k - \alpha_k s_k
    \]

    d. h. man geht von $x_k$ in Richtung des steilsten Abstiegs $-\nabla f(x_k)$, bis $\| e_{k+1} \|_A$ minimal wird, also die Folge zu einem stationären Punkt von $f$ konvergiert.\footnote{Wählt man $f$ entsprechend, kann damit auch die Lösung für ein $Ax = b$ gefunden werden. Siehe \emph{Konjugierte Gradienten}.}
\end{defi}

\begin{bonus}{Vorkonditionierte Steepest-Descent-Verfahren}
    Im Steepest-Descent-Verfahren gilt $M = I$.

    Für ein $M \neq I$ spd erhalten wir das \emph{vorkonditionierte Steepest-Descent-Verfahren} mit:
    \[
        x_{k+1} = M x_k + \alpha_k r_k, \quad r_k = - \nabla f(x_k)
    \]
\end{bonus}

\begin{example}{Steepest-Descent-Verfahren}
    Gegeben sei das System $Ax = b$ mit dem Startvektor $x_0$:
    \[
        A =
        \begin{pmatrix}
            2 & 1 \\
            1 & 2
        \end{pmatrix}
        , \quad
        b =
        \begin{pmatrix}
            2 \\ 1
        \end{pmatrix}
        , \quad
        x_0 =
        \begin{pmatrix}
            0 \\ 0
        \end{pmatrix}
    \]

    Wenden Sie das Steepest-Descent-Verfahren auf das System an.

    \exampleseparator

    $k = 0$:
    \[
        p_0 = r_0 = b - Ax_0 =
        \begin{pmatrix}
            2 \\ 1
        \end{pmatrix}
        -
        \begin{pmatrix}
            2 & 1 \\
            1 & 2
        \end{pmatrix}
        \begin{pmatrix}
            0 \\ 0
        \end{pmatrix}
        =
        \begin{pmatrix}
            2 \\ 1
        \end{pmatrix}
    \]
    \[
        s_0 = A p_0 =
        \begin{pmatrix}
            2 & 1 \\
            1 & 2
        \end{pmatrix}
        \begin{pmatrix}
            2 \\ 1
        \end{pmatrix}
        =
        \begin{pmatrix}
            5 \\ 4
        \end{pmatrix}
    \]
    \[
        \alpha_0 = \frac{\langle p_0, r_0 \rangle}{\langle p_0, s_0 \rangle} = \frac{\langle
            \begin{pmatrix}
                2 \\ 1
            \end{pmatrix},
            \begin{pmatrix}
                2 \\ 1
            \end{pmatrix}
            \rangle}{\langle
            \begin{pmatrix}
                2 \\ 1
            \end{pmatrix},
            \begin{pmatrix}
                5 \\ 4
            \end{pmatrix}
            \rangle} = \frac{5}{14}
    \]
    \[
        x_{1} = x_0 + \alpha_0 p_0 =
        \begin{pmatrix}
            0 \\ 0
        \end{pmatrix}
        + \frac{5}{14}
        \begin{pmatrix}
            2 \\ 1
        \end{pmatrix}
        =
        \begin{pmatrix}
            \nicefrac{5}{7} \\ \nicefrac{5}{14}
        \end{pmatrix}
    \]
    \[
        r_{1} = r_0 - \alpha_0 s_0 =
        \begin{pmatrix}
            2 \\ 1
        \end{pmatrix}
        - \frac{5}{14}
        \begin{pmatrix}
            5 \\ 4
        \end{pmatrix}
        = \begin{pmatrix}
            \nicefrac{3}{14} \\ -\nicefrac{3}{7}
        \end{pmatrix}
    \]

    $k = 1$:
    \[
        p_1 = r_1 = b - Ax_1 =
        \begin{pmatrix}
            \nicefrac{3}{14} \\ -\nicefrac{3}{7}
        \end{pmatrix}
        -
        \begin{pmatrix}
            2 & 1 \\
            1 & 2
        \end{pmatrix}
        \begin{pmatrix}
            \nicefrac{5}{7} \\ \nicefrac{5}{14}
        \end{pmatrix}
        =
        \begin{pmatrix}
            \nicefrac{3}{14} \\ -\nicefrac{3}{7}
        \end{pmatrix}
    \]
    \[
        s_1 = A p_1 =
        \begin{pmatrix}
            2 & 1 \\
            1 & 2
        \end{pmatrix}
        \begin{pmatrix}
            \nicefrac{3}{14} \\ -\nicefrac{3}{7}
        \end{pmatrix}
        =
        \begin{pmatrix}
            0 \\ -\nicefrac{9}{14}
        \end{pmatrix}
    \]
    \[
        \alpha_1 = \frac{\langle p_1, r_1 \rangle}{\langle p_1, s_1 \rangle} = \frac{\langle
            \begin{pmatrix}
                \nicefrac{3}{14} \\ -\nicefrac{3}{7}
            \end{pmatrix},
            \begin{pmatrix}
                \nicefrac{3}{14} \\ -\nicefrac{3}{7}
            \end{pmatrix}
            \rangle}{\langle
            \begin{pmatrix}
                \nicefrac{3}{14} \\ -\nicefrac{3}{7}
            \end{pmatrix},
            \begin{pmatrix}
                0 \\ -\nicefrac{9}{14}
            \end{pmatrix}
            \rangle} = \frac{5}{6}
    \]
    \[
        x_{2} = x_1 + \alpha_1 p_1 =
        \begin{pmatrix}
            \nicefrac{5}{7} \\ \nicefrac{5}{14}
        \end{pmatrix}
        +\frac{5}{6}
        \begin{pmatrix}
            \nicefrac{3}{14} \\ -\nicefrac{3}{7}
        \end{pmatrix}
        =
        \begin{pmatrix}
            \nicefrac{25}{28} \\ 0
        \end{pmatrix}
    \]
    \[
        r_{2} = r_1 - \alpha_1 s_1 =
        \begin{pmatrix}
            \nicefrac{3}{14} \\ -\nicefrac{3}{7}
        \end{pmatrix}
        - \frac{5}{6}
        \begin{pmatrix}
            0 \\ -\nicefrac{9}{14}
        \end{pmatrix}
        = \begin{pmatrix}
            \nicefrac{3}{14} \\ \nicefrac{3}{28}
        \end{pmatrix}
    \]

    usw.
\end{example}

\subsubsection{Konjugierte Gradienten}

\begin{defi}{CG-Verfahren}
    Das \emph{CG-Verfahren} (\enquote{conjugate gradients}) bzw. das Verfahren der konjugierten Gradienten ist eine effiziente numerische Methode zur Lösung von großen linearen Gleichungssystemen der Form $A x = b$ mit spd Systemmatrix $A$.

    Die Idee des CG-Verfahrens besteht darin, dass für $A$ spd das Minimieren der quadratischen Form
    \[
        f(x) = \frac{1}{2} \langle Ax, x \rangle - \langle b, x \rangle
    \]
    äquivalent zum Lösen von $A x = b$ ist.

    Das Verfahren funktioniert wie folgt:
    \begin{enumerate}
        \item $x_0$ gegeben, $p_0 = r_0 = b - Ax_0$
        \item Wiederhole für $k \geq 0$:
              \[ \alpha_k = \frac{\langle r_k, r_k \rangle}{\langle p_k, A p_k \rangle} \]
              \[ x_{k+1} = x_k + \alpha_k p_k \]
              \[ r_{k+1} = r_k - \alpha_k A p_k \]
              \[ \beta_k = \frac{\langle r_{k+1}, r_{k+1} \rangle}{\langle r_k, r_k \rangle} \]
              \[ p_{k+1} = r_{k+1} + \beta_k p_k \]
    \end{enumerate}

    Nach höchstens $n$ Schritten ist $e_k = 0$.
\end{defi}

\begin{example}{CG-Verfahren}
    Gegeben sei das System $Ax = b$ mit dem Startvektor $x_0$:
    \[
        A =
        \begin{pmatrix}
            2 & 1 \\
            1 & 2
        \end{pmatrix}
        , \quad
        b =
        \begin{pmatrix}
            2 \\ 1
        \end{pmatrix}
        , \quad
        x_0 =
        \begin{pmatrix}
            0 \\ 0
        \end{pmatrix}
    \]

    Wenden Sie das Konjugierte Gradienten-Verfahren auf das System an.

    \exampleseparator

    $k = 0$:
    \[
        p_0 = r_0 = b - Ax_0 =
        \begin{pmatrix}
            2 \\ 1
        \end{pmatrix}
        -
        \begin{pmatrix}
            2 & 1 \\
            1 & 2
        \end{pmatrix}
        \begin{pmatrix}
            0 \\ 0
        \end{pmatrix}
        =
        \begin{pmatrix}
            2 \\ 1
        \end{pmatrix}
    \]
    \[
        \alpha_0 = \frac{\langle r_0, r_0 \rangle}{\langle p_0, A p_0 \rangle}
        =
        \frac{\langle
            \begin{pmatrix}
                2 \\ 1
            \end{pmatrix},
            \begin{pmatrix}
                2 \\ 1
            \end{pmatrix}
            \rangle}{\langle
            \begin{pmatrix}
                2 \\ 1
            \end{pmatrix},
            \begin{pmatrix}
                2 & 1 \\
                1 & 2
            \end{pmatrix}
            \begin{pmatrix}
                2 \\ 1
            \end{pmatrix}
            \rangle}
        = \frac{5}{14}
    \]
    \[
        x_{1} = x_0 + \alpha_0 p_0 =
        \begin{pmatrix}
            0 \\ 0
        \end{pmatrix}
        + \frac{5}{14}
        \begin{pmatrix}
            2 \\ 1
        \end{pmatrix}
        =
        \begin{pmatrix}
            \nicefrac{5}{7} \\ \nicefrac{5}{14}
        \end{pmatrix}
    \]
    \[
        r_{1} = r_0 - \alpha_0 A p_0 =
        \begin{pmatrix}
            2 \\ 1
        \end{pmatrix}
        - \frac{5}{14}
        \begin{pmatrix}
            2 & 1 \\
            1 & 2
        \end{pmatrix}
        \begin{pmatrix}
            2 \\ 1
        \end{pmatrix}
        =
        \begin{pmatrix}
            \nicefrac{3}{14} \\ -\nicefrac{3}{7}
        \end{pmatrix}
    \]
    \[
        \beta_0 = \frac{\langle r_{1}, r_{1} \rangle}{\langle r_0, r_0 \rangle} =
        \frac{\langle
            \begin{pmatrix}
                \nicefrac{3}{14} \\ -\nicefrac{3}{7}
            \end{pmatrix},
            \begin{pmatrix}
                \nicefrac{3}{14} \\ -\nicefrac{3}{7}
            \end{pmatrix}
            \rangle}{\langle
            \begin{pmatrix}
                2 \\ 1
            \end{pmatrix},
            \begin{pmatrix}
                2 \\ 1
            \end{pmatrix} \rangle}
        =
        \frac{9}{196}
    \]
    \[
        p_{1} = r_{1} + \beta_0 p_0 =
        \begin{pmatrix}
            \nicefrac{3}{14} \\ -\nicefrac{3}{7}
        \end{pmatrix}
        + \frac{9}{196}
        \begin{pmatrix}
            2 \\ 1
        \end{pmatrix}
        =
        \begin{pmatrix}
            \nicefrac{15}{49} \\ -\nicefrac{75}{196}
        \end{pmatrix}
    \]

    usw.
\end{example}

\begin{defi}{Vorkonditioniertes CG-Verfahren}
    Die Rekursionsvorschrift für $M \neq I$ ist etwas aufwendiger und liefert das \emph{vorkonditionierte CG-Verfahren} bzw. PCG-Verfahren (\enquote{preconditioned conjugate gradient}) für $M$, $A$ spd:

    \begin{enumerate}
        \item $x_0$ gegeben, $r_o = b - Ax_0$, $p_0 = q_0 = M^{-1} r_0$
        \item Wiederhole für $k \geq 0$:
              \[ x_{k+1} = x_k + \alpha_k p_k \]
              \[ r_{k+1} = r_k - \alpha_k A p_k, \quad \alpha_k = \frac{\langle q_k, r_k \rangle}{\langle p_k, A p_k \rangle} \]
              \[ p_{k+1} = q_{k+1} + \beta_k p_k, \quad \beta_k = \frac{\langle q_{k+1}, r_{k+1} \rangle}{\langle q_k, r_k \rangle}\]
    \end{enumerate}

    Als Vorkonditionierer $M$ benutzt man unter anderem:
    \begin{itemize}
        \item $M = \diagonal(A) = M_J$
        \item $M$ aus einem symmetrischen, stationären Verfahren (z. B. symmetrisches Gauß-Seidel-Verfahren):
              \[
                  M = M_{SGS} = (D-E)D^{-1} (D-E)^T
              \]
        \item unvollständige Cholesky Zerlegung
    \end{itemize}

    PCG ist (neben Multigrid) \emph{das} Verfahren für große, dünn besetzte spd Matrizen A.
\end{defi}

\subsection{Dünn besetzte Matrizen}

\begin{bonus}{Problem dünn besetzter Matrizen}
    Bei (fast) allen Iterationsverfahren besteht der wesentliche Aufwand in einem Matrix-Vektor-Produkt und dieser ist proportional zu der Anzahl der Matrixelemente.

    Bei vollbesetzten Matrizen in $\R^{n \times n}$ erhalten wir also $\bigo(n^2)$.

    Geht man davon aus, das man höchstens $n$ Iterationen durchführt, so erhält man wieder, wie bei den direkten Lösern, einen Gesamtaufwand von $\bigo(n^3)$.

    In Anwendungen treten aber oft Gleichungssysteme auf, deren Matrix dünn besetzt ist (\enquote{sparse}), d.h. die Anzahl $n_{nz}$ der Elemente ungleich $0$ ist sehr gering (deutlich unter 1\%).

    Berücksichtigt man bei Matrix-Vektor-Produkten die dünne Besetzungsstruktur der Matrix, so reduziert sich der Aufwand auf $\bigo(n_{nz})$.
    In vielen realen Anwendungen ist $n_{nz} \approx n$, so dass der Aufwand dann nur $\bigo(n)$ ist.

    Dünn besetzte Matrizen werden nicht im Standardformat als zweidimensionales Array abgelegt, sondern in einem angepassten Format, das einerseits das einfache Implementieren von Matrix-Vektor-Produkten zulässt und andererseits möglichst wenig Speicherplatz benötigt.
\end{bonus}

\begin{defi}{Compressed Sparse Row}
    Ein gängiges Datenformat für dünn besetzte Matrizen (\enquote{sparse matrices}) ist \emph{Compressed Sparse Row (CSR)}.

    Bei CSR werden nur die Nichtnullelemente der Matrix abgelegt, sowie entsprechende Indexinformationen:
    \begin{itemize}
        \item im Vektor \texttt{val} stehen alle Nichtnullelemente der Matrix in zeilenweiser Anordnung (deshalb \emph{Compressed Sparse Row})
        \item der Vektor \texttt{col\_ind} enthält für jedes Element in \texttt{val} den zugehörigen Spaltenindex $j$
        \item die Einträge in \texttt{row\_ind} geben an, ab dem wievielten Element in \texttt{val} jeweils eine neue Zeile beginnt ($i$-Index)
    \end{itemize}
\end{defi}

\begin{example}{Compressed Sparse Row}
    Betrachten wir die Matrix
    \[
        \begin{pmatrix}
            1 & 0 & 7 & 0 & 8 \\
            6 & 3 & 0 & 0 & 0 \\
            0 & 0 & 0 & 2 & 0 \\
            4 & 5 & 0 & 0 & 0
        \end{pmatrix}
    \]

    In CSR werden dafür folgende Daten gespeichert:

    \begin{center}
        \begin{tabular}{|T|TTTTTTTT|}
            \hline
            val      & 1 & 7 & 8 & 6 & 3 & 2 & 4 & 5 \\
            \hline
            col\_ind & 1 & 3 & 5 & 1 & 2 & 4 & 1 & 2 \\
            \hline
            row\_ind & 1 & 4 & 6 & 7 & 9 &   &   &   \\
            \hline
        \end{tabular}
    \end{center}
\end{example}