\section{Einleitung}

\begin{defi}{Numerik}
    Lloyd Trefethen 1992:
    \begin{displayquote}
        Numerical analysis is the study of algorithms for the problems of continuous mathematics.
    \end{displayquote}

    \enquote{continuous mathematics} heißt, dass reelle oder komplexe Zahlen beteiligt sind, in Abgrenzung zur diskreten Mathematik.
    Dies umfasst die Analysis und Teile der linearen Algebra.

    Die \emph{Numerik} beschäftigt sich mit der Entwicklung von (effizienten) Berechnungsmethoden, da viele mathematische Objekte zwar wohldefiniert sind, aber nicht endlich berechenbar.
\end{defi}

\begin{example}{Auswertung der Exponentialfunktion}
    Zu berechnen sei $e^{\frac{1}{2}}$, wie Exponentialfunktion bekanntermaßen wie folgt definiert ist:
    \[
        e^x = \sum_{k=0}^\infty \frac{x^k}{k!}
    \]

    Die unendliche Summe ist nicht direkt berechenbar, daher wird ein Ersatzproblem gelöst:
    \[
        e^x \approx \sum_{k=0}^n \frac{x^k}{k!}
    \]
    Dabei ist $n$ ein neuer freier Parameter.
    Es gilt:
    \begin{itemize}
        \item kleines $n$ $\implies$ eventuell zu viel vernachlässigt, Wert ungenau approximiert.
        \item großes $n$ $\implies$ Berechnung aufwändig, Gewinn an Genauigkeit potentiell gering.
    \end{itemize}

    Ein weiteres Problem kann durch eine nicht-exakte Arithmetik aufkommen.
    Seien die beiden Summationsreihenfolgen $s_1(x)$ und $s_2(x)$ gegeben:
    \[
        s_1(x) = 1 + x + \frac{x^2}{2} + \ldots + \frac{x^n}{n!}
    \]
    \[
        s_2(x) = \frac{x^n}{n!} + \ldots + \frac{x^2}{2} + x + 1
    \]

    Bei einer Floating-Point-Rechnung mit 3 Stellen Genauigkeit gilt:
    \begin{center}
        \begin{tabular}{|C||C|C|C|C|C|}
            \hline
            n        & 1   & 2    & 3    & 4    & 5    \\
            \hline
            \hline
            s_1(0.5) & 1.5 & 1.62 & 1.64 & 1.64 & 1.64 \\
            \hline
            s_2(0.5) & 1.5 & 1.62 & 1.65 & 1.65 & 1.65 \\
            \hline
        \end{tabular}
    \end{center}

    Die Summationsreihenfolge beeinflusst also das Ergebnis.

    Ein Abbruch bei $n=4$ ist für $x = \nicefrac{1}{2}$ angemessen.
\end{example}


\begin{example}{Rechteckregel und Trapezregel}
    Zu approximieren sei die Funktion
    \[
        f(x) = \int_{0}^{1} \frac{1}{1+x^2} \diff x
    \]

    \emph{Rechteckregel} und \emph{Trapezregel} approximieren diese Funktion wie folgt:

    % Rechteckregel
    \begin{tikzpicture}
        \begin{axis}[
                axis lines = left,
                %xlabel = $x$,
                %ylabel = $f(x)$,
                xmin = 0,
                xmax = 1.05,
                ymin = 0,
                ymax = 1.05,
            ]

            \addplot[name path=f,domain = 0:1.05,samples = 100] { 1/(1+x^2) };

            \addplot[dashed] coordinates{(1,0) (1,1)};

            \path[name path=axis] (axis cs:0,0) -- (axis cs:1,0);
            \path[name path=approx] (axis cs:0,1) -- (axis cs:1,1);

            \addplot[fill=red,fill opacity=0.1] fill between[of=approx and axis,soft clip={domain=0:1}];
        \end{axis}
    \end{tikzpicture}
    % Trapezregel
    \begin{tikzpicture}
        \begin{axis}[
                axis lines = left,
                %xlabel = $x$,
                %ylabel = $f(x)$,
                xmin = 0,
                xmax = 1.05,
                ymin = 0,
                ymax = 1.05,
            ]

            \addplot[name path=f,domain = 0:1.05,samples = 100] { 1/(1+x^2) };

            \addplot[dashed] coordinates{(1,0) (1,1)};

            \path[name path=axis] (axis cs:0,0) -- (axis cs:1,0);
            \path[name path=approx] (axis cs:0,1) -- (axis cs:1,0.5);

            \addplot[fill=blue,fill opacity=0.1] fill between[of=approx and axis,soft clip={domain=0:1}];
        \end{axis}
    \end{tikzpicture}

    TO DO
\end{example}

\begin{defi}{Numerische Simulation}
    \emph{Numerische Simulation} beschreibt die Verhaltensvorhersage eines physikalischen Systems durch mathematische Modelle und numerische Verfahren.

    Numerische Simulationen sind z.B. anwendbar in physikalischen Systemen (beliebige Bauteile) und daher sehr wichtig für die Produktentwicklung in der Industrie.

    Das Vorgehen ist wie folgt:
    \begin{itemize}
        \item virtueller Prototyp: vollständige Analyse durch numerische Simulation
        \item davon ausgehend: endgültige Auslegung der Konstruktion
        \item am Ende: Validierung durch einen realen Prototypen
    \end{itemize}

    In der Regel spart numerische Simulation Geld und Zeit.
\end{defi}