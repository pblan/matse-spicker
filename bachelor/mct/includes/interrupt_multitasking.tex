\section{Interrupts und Multitasking}

\begin{defi}{Interrupt}
    Ein \emph{Interrupt} ist die automatische Unterbrechung bzw. Verzweigung des Programmflusses auf Grund eines Hardwareereignisses (\emph{Trigger}), das asynchron zur Programmausführung auftritt.

    Ein Interrupt besitzt eine Priorität, die den Vorrang mehrerer Interrupts regelt, sowie eine Nummer, die die Trigger-Quelle identifiziert.

    Mögliche Trigger sind:
    \begin{itemize}
        \item \emph{Externe Trigger} (Hardware):
              \begin{itemize}
                  \item Beim PC: Ein- bzw. Ausgabe-Gerät (Tastatur, Maus)
                  \item Mikrocontroller: Pegeländerung z. B. an einem GPIO-Pin bzw. an einem analogen Eingang
              \end{itemize}
        \item \emph{Interne Trigger} (Hardware bzw. Software):
              \begin{itemize}
                  \item periodischer Timer-Interrupt
                  \item Speicherverletzung oder Division durch 0
                  \item softwareerzeugter Interrupt
              \end{itemize}
    \end{itemize}

    Ein Interrupt ist eine Alternative zum \enquote{Polling} (regelmäßige Abfrage der CPU auf neue Daten)
    Er ist meist immer damit verbunden, dass entweder der Prozessor neue Daten liefern soll, oder Daten zur Abholung bereit stehen.

    Der Interrupt passiert i. d. R. asynchron zur Programmausführung.
    Die \emph{Interrupt Service Routine (ISR)} muss den aktuellen Zustand der CPU \enquote{retten} und am Ende wieder herstellen.

    Die Latenz ist definiert als $\Delta t$ zwischen Interrupt und Start der ISR.

    Ein Interrupt, der aktiviert wurde, aber noch nicht verarbeitet wurde, nennt man einen \emph{pending Interrupt}.
    Ausnahme sind hier \emph{Non Maskable Interrupts (NMIs)}, die i. d. R. zum Signalisieren von kritischen Zustandsänderungen verwendet.

    Häufig wird das Mapping von Interrupt zu ISR über eine Tabelle in \texttt{ROM} oder \texttt{RAM}.
    Die CPU nutzt dann die Interrupt-Nummer als Index in dieser Tabelle, und springt zur entsprechenden ISR.
\end{defi}

\begin{defi}{Multithreading}
    Interrupts sind die Grundlage für \emph{Multithreading}.

    Wenn ein Interrupt regelmäßig von einem Timer erzeugt wird, dann muss der Prozessor in der entsprechenden ISR den aktuellen Zustand der CPU \enquote{retten} und könnte danach einen anderen Programmkontext wiederherstellen, und bis zum nächsten Interrupt das entsprechende Programmsegment ausführen.
    Die IST wird dann zum sog. \emph{Scheduler}.
\end{defi}

\begin{defi}{Booten}
    Beim \emph{Booten} wird der \texttt{Reset\_Handler} ausgeführt.
    Danach folgt ein Sprung zur \texttt{main()}-Methode.
\end{defi}

\begin{defi}{PRIMASK}
    Um bestimmte Interrupts auszuschalten wird das Prozessor-Register \texttt{PRIMASK} auf 1 gesetzt.
\end{defi}

\begin{defi}{Cortex M4-Interrupt}
    Der Cortex M4-Prozessor besitzt darüber hinaus einen \emph{Nested Vectored Interrupt Controller (NCIV)}.

    Er besitzt u. a. folgende Funktionalitäten:
    \begin{itemize}
        \item Programmbasierte Prioritäten (0-255) für alle Interrupts
        \item Gruppierung von Interrupts in Priorisierungs- und Untergruppen
    \end{itemize}

    Sobald ein Interrupt ausgeführt wird, passieren folgende Schritte:
    \begin{enumerate}
        \item Die Register werden auf dem Stack abgelegt.
        \item Register \texttt{LR} wird auf \texttt{0xFFFF FFF9} gesetzt.
        \item \texttt{IPSR} wird auf die entsprechende Interrupt-Nummer gesetzt.
        \item \texttt{PC} wird auf den entsprechenden Interrupt-Vektor gesetzt.
        \item Die ISR wird ausgeführt.
        \item Danach werden die nötigen Operationen in der ISR ausgeführt.
        \item Am Ende wird zurück in die ursprüngliche Methode gesprungen und die Register werden geladen.
        \item Es wird an der unterbrochenen Stelle weiter ausgeführt.
    \end{enumerate}

    Er erfolgt dabei folgenden Regeln:
    \begin{itemize}
        \item Der Code sollte möglichst schnell ablaufen, damit möglichst bald ein weiterer Interrupt bearbeitet werden kann.
        \item Kommunikation zwischen Hauptprogramm und ISR über globale Variablen
    \end{itemize}
\end{defi}

\begin{defi}{Multitasking}
    \emph{Multitasking} bedeutet die Fähigkeit einen (Betriebs-)Systems, mehrere Aufgaben (\emph{Tasks}) quasi nebenläufig auszuführen.

    Ein zentraler Bestandteil ist der Scheduler, der die Steuerung der Ausführung übernimmt.

    \begin{itemize}
        \item \emph{Kooperatives Multitasking}: Die einzelnen Tasks werden von Scheduler aufgerufen, unterscheiden selber, wann sie die Ausführung beenden und zum Scheduler zurückkehren.
        \item \emph{Preemptives Multitasking}: Der Scheduler unterbricht den Task nach einer gewissen Zeit und speichert den Zustand dieses Tasks.
              Danach übergibt er die Kontrolle an einen anderen Task, ggf. nach Analyse einer sinnvollen Reihenfolge.
              Im einfachsten Fall wird die \emph{Round Robin}\footnote{Tasks werden im Kreis abgearbeitet.}-Strategie angewendet.

              Hier benötigt jeder Task einen eigenen Stack.
              Die \texttt{SP}-Werte der einzelnen Tasks werden in sog. \emph{Task Control Block (TCB)} (eine Art verkettete Liste) gespeichert.
    \end{itemize}
\end{defi}