\section{Serielle Schnittstelle}

\begin{defi}{Schnittstelle}
    Schnittstellen werden genutzt, um zwischen einem Mikrocontroller und Sensoren oder einem Host System zu kommunizieren.

    Eigenschaften:
    \begin{itemize}
        \item \emph{Physikalische Schnittstelle}:
              \begin{itemize}
                  \item Spannungspegel
                  \item Anzahl der Kanäle
              \end{itemize}
        \item \emph{Arbeitsweise}:
              \begin{itemize}
                  \item \emph{Synchron} (mit Taktsignal)
                  \item \emph{Asynchron} (ohne Taktsignal)
              \end{itemize}
        \item \emph{Kommunikationsrichtung}:
              \begin{itemize}
                  \item \emph{simplex}: Ausschließlich eine Richtung
                  \item \emph{halb-duplex}: Abwechselnd in eine Richtung
                  \item \emph{voll-duplex}: zeitgleich in beide Richtungen
              \end{itemize}
        \item \emph{Flussteuerung}
        \item \emph{Topologie}:
              \begin{itemize}
                  \item Punkt-zu-Punkt
                  \item Bus
              \end{itemize}
    \end{itemize}
\end{defi}

\begin{defi}{Parallele Schnittstelle}
    \emph{Parallele Schnittstellen} übertragen Bits \emph{parallel} über mehrere, parallele Leitungen.
\end{defi}

\begin{defi}{Serielle Schnittstelle}
    \emph{Serielle Schnittstellen} übertragen Bits nacheinander auf einer Leitung.

    In der Praxis zeigt sich, dass serielle Schnittstellen schneller sind, da sie keine Probleme wie Laufzeit der Signale, \enquote{übersprechen} oder Leitungskapazitäten haben.
\end{defi}

\begin{bonus}{Baudrate vs. Bitrate}
    Wenn die Zeitdauer (Schrittdauer) eines Symbols bzw. Codeelements $T$ ist, ist die Schrittgeschwindigkeit bzw. \emph{Baudrate}:
    \[
        v_s = \frac{1}{T} \text{ [Baud]}
    \]

    Die dazugehörige Übertragungsgeschwindigkeit bzw. \emph{Bitrate} ist dann:\footnote{$n$ = Anzahl diskreter Zustände des Codeelements}
    \[
        v_u = v_s \log_2 n
    \]

    Bei binären Codeelementen stimmen somit Bitrate und Baudrate überein, falls nur Codeelementen für Daten übermittelt werden.
\end{bonus}

\subsection{UART und SPI}

\begin{defi}{UART}
    \emph{Universal Asynchronous Receiver/Transmitter (UART)} ist eine asynchrone Punkt-zu-Punkt Verbindung, die grundsätzlich voll-duplex Betrieb erlaubt.

    Im einfachsten Fall existieren nur 2 Leitungen, die über Kreuz zwischen zwei Geräten angeschlossen werden:
    \begin{itemize}
        \item \texttt{TX} (Senden)
        \item \texttt{RX} (Empfangen)
    \end{itemize}

    Zu Beginn der Kommunikation muss man sich auf folgende Einstellungen einigen:
    \begin{itemize}
        \item 7 oder 8 Datenbits
        \item Paritätsbit 1 wenn Anzahl 1en gerade oder ungerade.
        \item 1 oder 2 Stopbits
    \end{itemize}

    Mit UART lassen sich z. B. 150m bei 9600 Baud überbrücken.

    Die UART Schnittstelle verwendet manchmal spezielle Leitungen zur Datenflusssteuerung zwischen dem \emph{Data Communication Equipment (DCE)} und dem \emph{Data Terminal Equipment (DTE)}:
    \begin{itemize}
        \item \emph{Ready To Send (RTS)}
        \item \emph{Clear To Send (CTS)}
        \item \emph{Data Terminal Ready (DTR)}
        \item \emph{Data Set Read (DSR)}
    \end{itemize}

    In der Praxis werden diese Leitungen jedoch nicht verwendet, sondern die Flussteuerung z. B. per Software realisiert.
    Daher sieht man auf dem Launchpad auch nur \texttt{RX} und \texttt{TX}
\end{defi}

\begin{defi}{eUSCI}
    \emph{enhanced Universal Serial Communication Interface (eUSCI)} Einheiten können im \emph{UART} Modus betrieben werden.

    \texttt{eUSCI\_A0} ist normalerweise über zwei Jumper an das Backchannel-UART angeschlossen.
\end{defi}

\begin{bonus}{Empfangen einer Nachricht}
    Zum Empfangen benötigen wir verschiedene Register.

    Zu Beginn wird die Nachricht in das \emph{Receive Shift Register} aufgenommen.

    Dieses leitet die empfangenen Daten in den \emph{Receive Buffer} weiter.
    Der Buffer nimmt nur Daten an, wenn er leer ist.

    Dabei können verschiedene Fehler auftreten:
    \begin{itemize}
        \item \emph{Break} \texttt{(BRK)} wird gesetzt, wenn das Empfangssignal länger als die Dauer eines Datenrahmens auf logisch 0 liegt.
        \item \emph{Overrun Error} \texttt{(OE)} wird gesetzt, wenn \texttt{UCSxRXBUF} bereits gefüllt ist, und ein komplettes neues Datenwort Shift-Register eingegangen ist.
    \end{itemize}
\end{bonus}

\begin{defi}{SPI}
    Ein \emph{Serial Peripheral Interface (SPI)} ist ein synchrones voll-duplex Bussystem mit einer 1:N Topologie.

    Es gibt u. a. folgende variable Größen:
    \begin{itemize}
        \item Wortlänge
        \item LSB bzw MSB first
        \item Taktflanke
        \item Frequenz
    \end{itemize}

    Dazu werden vier Signale genutzt:
    \begin{itemize}
        \item \emph{Serial CLocK} \texttt{(SCLK)}, gesetzt von Main
        \item \emph{Main Out, Sub In} \texttt{(MOSI)}
        \item \emph{Main In, Sub Out} \texttt{(MISO)}
        \item \emph{Sub Select \texttt{(/SS)}}
    \end{itemize}

    Jedes Subsystem hat ein eigenes \emph{Sub Select}-Signal.
    Alle anderen Leitungen werden gemeinsam genutzt.

    \emph{/SS} ist \texttt{LOW}-Aktiv, d. h. eine logische 0 selektiert den Sub.
\end{defi}

\subsection{I2C}

\begin{defi}{I2C}
    \emph{Inter-Integrated Curcuit (I2C)} ist ein populäres Bussystem zur Anbindung von integrierten Schaltungen.

    Es ist eine synchrone, halb-duplex Schnittstelle mit einer 1:N Topologie.

    Es werden lediglich zwei Leitungen genutzt:
    \begin{itemize}
        \item \emph{Serial CLock (SCL)}
        \item \emph{Serial DAta (SDA)}
    \end{itemize}

    \includegraphics[width=\textwidth]{includes/figures/defi_i2c.pdf}
\end{defi}

\begin{bonus}{Besonderheiten I2C}
    \begin{itemize}
        \item Main- bzw Sub-Architektur wie bei SPI
        \item Main initiiert die Kommunikation
        \item Nur eine Datenleitung, trotzdem Datenaustausch in beide Richtungen möglich
        \item Auch die Taktleitung kann von beiden Kommunikationspartnern genutzt werden.
    \end{itemize}
\end{bonus}

\begin{defi}{Open-Drain}
    Wenn ein Kommunikationspartner eine Leitung auf \texttt{HIGH} und ein anderer auf \texttt{LOW} setzt, kann es zu einem Kurzschluss kommen.

    Um dem entgegenzuwirken wurden \emph{Open-Drai- Ausgangsstufen} implementiert:

    \begin{center}
        \includegraphics[width=0.5\textwidth]{includes/figures/defi_open_drain.pdf}
    \end{center}

    Der Ruhezustand ist \texttt{HIGH}.
    Wenn ein Gerät die Leitung selber nicht auf \texttt{LOW} zieht, kann es den Wert einer Leitung auslesen.
\end{defi}

\begin{bonus}{Taktraten}
    Die Pullup-Widerstände müssen richtig dimensioniert werden, damit die Pegelwechsel auf den Leitungen schnell genug passieren.

    I2C arbeitet durch die Open-Drain Ausgänge langsamer als SPI.

    Üblich sind Taktraten von 100kHz bus 1MHz.

    Im Gegensatz zu SPI werden bei I2C immer 8 Bit und ein \texttt{ACK}-bit versendet, die entweder als Geräteadresse oder Daten interpretiert werden.
\end{bonus}

\begin{defi}{Datenaustausch}
    Jede Datenübertragung wird von einer sogenannten \texttt{START}- und \texttt{STOP}-Bedingung eingerahmt.

    Bei \texttt{START} wird zuerst \texttt{SDA}, dann \texttt{SCL} auf \texttt{LOW} gesetzt.
    Bei \texttt{STOP} wird zuerst \texttt{SCL}, dann \texttt{SDA} auf \texttt{HIGH} gesetzt.

    Nach der \texttt{START}-Bedingung erfolgt die Übertragung von 8 bit, wobei \texttt{SDA} nur seinen Pegel ändern darf, wenn \texttt{SCL} \texttt{LOW} ist.

    Der Wert von \texttt{SDA} wird übernommen, wenn \texttt{SCL} \texttt{LOW} ist.

    Beim \texttt{ACK} lässt der Main die \texttt{SDA}-Leitung lost, d. h. sie wäre normalerweise auf \texttt{HIGH}.
    Das angesprochene Sub-System muss die Leitung auf \texttt{LOW} setzten, damit das Main-System weiß, dass das Paket empfangenen wurde.

    Der Empfänger kann nach dem \texttt{ACK} die Leitung auf \texttt{LOW} lassen, damit der Sender langsamer sendet.
\end{defi}

\begin{defi}{Adressierung}
    Jedes I2C-Gerät wird über eine 7 bit Adresse adressiert.
    Ein weiteres bit ($\nicefrac{R}{\bar{W}}$) gibt an, ober das Main-System die Daten schreiben oder lesen möchte.

    Beim \emph{lesenden} Zugriff wird zunächst auch die Adresse des I2C-Geräts und die Adresse des Registers vom Main-System gesendet.
    Danach wird eine weitere \texttt{START}-Bedingung gesendet und nach dem Schreiben der Register-Adresse übernimmt das Sub-System die \texttt{SDA}-Leitung und versendet die Daten.

    Das Main-System signalisiert mit einem \texttt{NACK}, dass es keine weiteren Daten empfangen möchte.
\end{defi}

\begin{bonus}{Empfangen einer Nachricht}
    Zum Empfangen benötigen wir verschiedene Register.

    Zu Beginn wird die Nachricht in das \emph{Receive Shift Register} aufgenommen.

    Dieses leitet die empfangenen Daten in den \emph{Receive Buffer} weiter.
    Der Buffer nimmt nur Daten an, wenn er leer ist.
\end{bonus}

\begin{bonus}{Senden einer Nachricht}
    Zum Senden einer Nachricht schreiben wir die Daten in  den \emph{Transmit Buffer}.

    Zu Beginn wird die Nachricht in des \emph{Transmit Shift Register} aufgenommen.
    Die Sub-Adresse befindet sich in einem eigenen Register.

    Solange Daten im Transmit Buffer liegen, wird das Transmit Shift Register gefüllt.
\end{bonus}