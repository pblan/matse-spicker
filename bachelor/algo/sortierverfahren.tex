\section{Sortierverfahren}

\begin{defi}{Klassifikationskriterien für Sortieralgorithmen}
    \begin{itemize}
        \item Effizienz
              \begin{itemize}
                  \item Schlechter als $\bigo(n^2)$: nicht ganz ernst gemeinte Verfahren
                  \item $\bigo(n^2)$: Elementare Sortierverfahren
                        \begin{itemize}
                            \item Bubble-Sort
                            \item Insertion-Sort
                            \item Selection-Sort
                        \end{itemize}
                  \item Zwischen $\bigo(n^2)$ und $\bigo(n \log n)$
                        \begin{itemize}
                            \item Shell-Sort
                        \end{itemize}
                  \item $\bigo(n \log n)$: Höhere Sortierverfahren
                        \begin{itemize}
                            \item Heap-Sort
                            \item Merge-Sort
                            \item Quick-Sort
                        \end{itemize}
                  \item Besser als $\bigo(n \log n)$: Spezialisierte Sortierverfahren
                        \begin{itemize}
                            \item Radix-Sort
                        \end{itemize}
              \end{itemize}
        \item Speicherverbrauch
        \item Intern vs. Extern
        \item Stabil vs. Instabil
        \item allgemein vs. spezialisiert
    \end{itemize}
\end{defi}

\begin{bonus}{Beste Sortierverfahren}
    \begin{itemize}
        \item normalerweise Quick-Sort
        \item Merge-Sort, falls:
              \begin{itemize}
                  \item die Datenmenge zu groß für den Hauptspeicher ist.
                  \item die Daten als verkettete Liste vorliegen.
                  \item ein stabiles Verfahren nötig ist.
              \end{itemize}
        \item Insertion-Sort, falls:
              \begin{itemize}
                  \item wenige Elemente u sortieren sind.
                  \item die Daten schon vorsortiert sind.
              \end{itemize}
        \item Radix-Sort, falls:
              \begin{itemize}
                  \item sich ein hoher Programmieraufwand für ein sehr schnelles Verfahren lohnt.
              \end{itemize}
    \end{itemize}
\end{bonus}

\subsection{Elementare Sortierverfahren}

\begin{algo}{Simple-Sort}
    Das Suchen und Vertauschen wir bei \emph{Simple-Sort} wie folgt realisiert:
    \begin{enumerate}
        \item Gehe vom Element \texttt{i} aus nach rechts
        \item Jedes Mal, wenn ein kleineres Element als das auf Position \texttt{i} auftaucht, vertausche es mit dem Element \texttt{i}
        \item Wiederholen bis das Array sortiert ist
    \end{enumerate}

    Simple-Sort ergibt einen besonders einfachen Code, ist aber langsamer als z.B. Selection-Sort.
    Dann zu nutzen, wenn:
    \begin{itemize}
        \item Sie keine Zeit oder Lust zum Nachdenken haben.
        \item die Felder so klein sind, dass der Algorithmus nicht effektiv sein muss.
        \item niemand sonst Ihren Code zu sehen bekommt.
    \end{itemize}
\end{algo}

\begin{code}{Simple-Sort}
    \lstinputlisting{simplesort.java}
\end{code}

\begin{algo}{Selection-Sort}
    \emph{Selection-Sort} funktioniert ähnlich wie Simple-Sort:
    \begin{enumerate}
        \item Suchen des kleinsten Elements im unsortierten Bereich
        \item Vertauschen mit Position \texttt{i}
        \item Wiederholen bis das Array sortiert ist
    \end{enumerate}

    Selection-Sort ist eines der wichtigeren elementaren Verfahren, ist aber meistens etwas langsamer als Insertion-Sort.

    Selection-Sort ist instabil und bietet keine Vorteile, wenn das Feld schon vorsortiert ist.
\end{algo}

\begin{code}{Selection-Sort}
    \lstinputlisting{selectionsort.java}
\end{code}

\begin{example}{Selection-Sort}
    \centering

    \begin{tikzpicture}[
        start chain,
        node distance = 0em,
        ArrayBlock/.style={draw, minimum width=2em, minimum height=2em, outer sep=0pt, on chain},
        sorted/.style={fill=blue!25},
        i/.style={fill=teal!25, label=above:\texttt{i}},
        min/.style={fill=red!25, label=above:\texttt{min}},
        i min/.style={fill=teal!25, label=above:\texttt{i,min}},
        ]
        { start chain = going right

        \node [ArrayBlock, i] (0_3) {$3$};
        \node [ArrayBlock] (0_8a) {$8_a$};
        \node [ArrayBlock] (0_8b) {$8_b$};
        \node [ArrayBlock] (0_6) {$6$};
        \node [ArrayBlock] (0_4) {$4$};
        \node [ArrayBlock, min] (0_2) {$2$};

        { [continue chain = going right]
        \chainin (0_3);

        \node [ArrayBlock, xshift=-2em, yshift=-4em, sorted] (1_2) {$2$};
        \node [ArrayBlock, i] (1_8a) {$8_a$};
        \node [ArrayBlock] (1_8b) {$8_b$};
        \node [ArrayBlock] (1_6) {$6$};
        \node [ArrayBlock] (1_4) {$4$};
        \node [ArrayBlock, min] (1_3) {$3$};
        }

        { [continue chain = going right]
        \chainin (0_3);

        \node [ArrayBlock, xshift=-2em, yshift=-8em, sorted] (2_2) {$2$};
        \node [ArrayBlock, sorted] (2_3) {$3$};
        \node [ArrayBlock, i] (2_8b) {$8_b$};
        \node [ArrayBlock] (2_6) {$6$};
        \node [ArrayBlock, min] (2_4) {$4$};
        \node [ArrayBlock] (2_8a) {$8_a$};
        }

        { [continue chain = going right]
        \chainin (0_3);

        \node [ArrayBlock, xshift=-2em, yshift=-12em, sorted] (3_2) {$2$};
        \node [ArrayBlock, sorted] (3_3) {$3$};
        \node [ArrayBlock, sorted] (3_4) {$4$};
        \node [ArrayBlock, i min] (3_6) {$6$};
        \node [ArrayBlock] (3_8b) {$8_b$};
        \node [ArrayBlock] (3_8a) {$8_a$};
        }

        { [continue chain = going right]
        \chainin (0_3);

        \node [ArrayBlock, xshift=-2em, yshift=-16em, sorted] (4_2) {$2$};
        \node [ArrayBlock, sorted] (4_3) {$3$};
        \node [ArrayBlock, sorted] (4_4) {$4$};
        \node [ArrayBlock, sorted] (4_6) {$6$};
        \node [ArrayBlock, i min] (4_8b) {$8_b$};
        \node [ArrayBlock] (4_8a) {$8_a$};
        }

        { [continue chain = going right]
        \chainin (0_3);

        \node [ArrayBlock, xshift=-2em, yshift=-20em, sorted] (5_2) {$2$};
        \node [ArrayBlock, sorted] (5_3) {$3$};
        \node [ArrayBlock, sorted] (5_4) {$4$};
        \node [ArrayBlock, sorted] (5_6) {$6$};
        \node [ArrayBlock, sorted] (5_8b) {$8_b$};
        \node [ArrayBlock, i min] (5_8a) {$8_a$};
        }

        { [continue chain = going right]
        \chainin (0_3);

        \node [ArrayBlock, xshift=-2em, yshift=-24em, sorted] (6_2) {$2$};
        \node [ArrayBlock, sorted] (6_3) {$3$};
        \node [ArrayBlock, sorted] (6_4) {$4$};
        \node [ArrayBlock, sorted] (6_6) {$6$};
        \node [ArrayBlock, sorted] (6_8b) {$8_b$};
        \node [ArrayBlock, sorted] (6_8a) {$8_a$};
        }
        }

        \draw[<->] (0_3) -- ++(0,-1.5em) -| (0_2);
        \draw[<->] (1_3) -- ++(0,-1.5em) -| (1_8a);
        \draw[<->] (2_4) -- ++(0,-1.5em) -| (2_8b);
        \draw[<->] (3_6) -- ++(0,-1.5em) -| (3_6);
        \draw[<->] (4_8b) -- ++(0,-1.5em) -| (4_8b);
        \draw[<->] (5_8a) -- ++(0,-1.5em) -| (5_8a);
    \end{tikzpicture}
\end{example}

\begin{algo}{Insertion-Sort}
    Die Grundidee ist:
    \begin{enumerate}
        \item Starte mit einem Wert im Array (meist Position \texttt{0})
        \item Nimm jeweils den nächsten Eintrag im Array und füge ihn an der richtigen Stelle im sortierten Bereich ein
        \item Wiederholen bis das Array sortiert ist
    \end{enumerate}

    Insertion-Sort ist in den meisten Fällen der schnellste elementare Suchalgorithmus - auch, weil er potentielle Vorsortierung ausnutzt.

    Insertion-Sort ist stabil.
\end{algo}

\begin{code}{Insertion-Sort}
    \lstinputlisting{insertionsort.java}
\end{code}

\begin{example}{Insertion-Sort}
    \centering

    \begin{tikzpicture}[
        start chain,
        node distance = 0em,
        ArrayBlock/.style={draw, minimum width=2em, minimum height=2em, outer sep=0pt, on chain},
        sorted/.style={fill=blue!25},
        i/.style={fill=teal!25, label=above:\texttt{i}},
        ]
        { start chain = going right

        \node [ArrayBlock, i] (0_3) {$3$};
        \node [ArrayBlock] (0_8a) {$8_a$};
        \node [ArrayBlock] (0_8b) {$8_b$};
        \node [ArrayBlock] (0_6) {$6$};
        \node [ArrayBlock] (0_4) {$4$};
        \node [ArrayBlock] (0_2) {$2$};

        { [continue chain = going right]
        \chainin (0_3);

        \node [ArrayBlock, xshift=-2em, yshift=-4em, sorted] (1_3) {$3$};
        \node [ArrayBlock, i] (1_8a) {$8_a$};
        \node [ArrayBlock] (1_8b) {$8_b$};
        \node [ArrayBlock] (1_6) {$6$};
        \node [ArrayBlock] (1_4) {$4$};
        \node [ArrayBlock] (1_2) {$2$};
        }

        { [continue chain = going right]
        \chainin (0_3);

        \node [ArrayBlock, xshift=-2em, yshift=-8em, sorted] (2_3) {$3$};
        \node [ArrayBlock, sorted] (2_8a) {$8_a$};
        \node [ArrayBlock, i] (2_8b) {$8_b$};
        \node [ArrayBlock] (2_6) {$6$};
        \node [ArrayBlock] (2_4) {$4$};
        \node [ArrayBlock] (2_2) {$2$};
        }

        { [continue chain = going right]
        \chainin (0_3);

        \node [ArrayBlock, xshift=-2em, yshift=-12em, sorted] (3_3) {$3$};
        \node [ArrayBlock, sorted] (3_8a) {$8_a$};
        \node [ArrayBlock, sorted] (3_8b) {$8_b$};
        \node [ArrayBlock, i] (3_6) {$6$};
        \node [ArrayBlock] (3_4) {$4$};
        \node [ArrayBlock] (3_2) {$2$};
        }

        { [continue chain = going right]
        \chainin (0_3);

        \node [ArrayBlock, xshift=-2em, yshift=-16em, sorted] (4_3) {$3$};
        \node [ArrayBlock, sorted] (4_6) {$6$};
        \node [ArrayBlock, sorted] (4_8a) {$8_a$};
        \node [ArrayBlock, sorted] (4_8b) {$8_b$};
        \node [ArrayBlock, i] (4_4) {$4$};
        \node [ArrayBlock] (4_2) {$2$};
        }

        { [continue chain = going right]
        \chainin (0_3);

        \node [ArrayBlock, xshift=-2em, yshift=-20em, sorted] (5_3) {$3$};
        \node [ArrayBlock, sorted] (5_4) {$4$};
        \node [ArrayBlock, sorted] (5_6) {$6$};
        \node [ArrayBlock, sorted] (5_8a) {$8_a$};
        \node [ArrayBlock, sorted] (5_8b) {$8_b$};
        \node [ArrayBlock, i] (5_2) {$2$};
        }

        { [continue chain = going right]
        \chainin (0_3);

        \node [ArrayBlock, xshift=-2em, yshift=-24em, sorted] (6_2) {$2$};
        \node [ArrayBlock, sorted] (6_3) {$3$};
        \node [ArrayBlock, sorted] (6_4) {$4$};
        \node [ArrayBlock, sorted] (6_6) {$6$};
        \node [ArrayBlock, sorted] (6_8a) {$8_a$};
        \node [ArrayBlock, sorted] (6_8b) {$8_b$};
        }

        \draw[<->] (0_3) -- ++(0,-1.5em) -| (0_3.south west);
        \draw[<->] (1_8a) -- ++(0,-1.5em) -| (1_8a.south west);
        \draw[<->] (2_8b) -- ++(0,-1.5em) -| (2_8b.south west);
        \draw[<->] (3_6) -- ++(0,-1.5em) -| (3_8a.south west);
        \draw[<->] (4_4) -- ++(0,-1.5em) -| (4_6.south west);
        \draw[<->] (5_2) -- ++(0,-1.5em) -| (5_3.south west);
        }
    \end{tikzpicture}
\end{example}

\begin{bonus}{Bewertung elementarer Sortierverfahren}
    \begin{itemize}
        \item Simple-Sort
              \begin{itemize}
                  \item Einfach zu implementieren
                  \item Langsam
              \end{itemize}
        \item Selection-Sort
              \begin{itemize}
                  \item Aufwand unabhängig von Vorsortierung
                  \item nie mehr als $\bigo(n)$ Vertauschungen nötig
              \end{itemize}
        \item Bubble-Sort
              \begin{itemize}
                  \item Stabil
                  \item Vorsortierung wird ausgenutzt
              \end{itemize}
        \item Insertion-Sort
              \begin{itemize}
                  \item Stabil
                  \item Vorsortierung wird ausgenutzt
                  \item schnell für eine elementares Suchverfahren ($\bigo(n^2)$)
              \end{itemize}
    \end{itemize}
\end{bonus}

\subsection{Höhere Sortierverfahren}

\begin{algo}{Heap-Sort}
    Der Heap ist Grundlage für das Sortierverfahren \emph{Heap-Sort}.

    \begin{itemize}
        \item Zu Beginn: Unsortiertes Feld
        \item Phase 1:
              \begin{itemize}
                  \item Alle Elemente werden nacheinander in einen Heap eingefügt
                  \item Resultat ist ein Heap, der in ein Feld eingebettet ist
              \end{itemize}
        \item Phase 2:
              \begin{itemize}
                  \item Die Elemente werden in absteigender Reihenfolge entfernt (Wurzel!)
                  \item Heap schrumpft immer weiter
              \end{itemize}
    \end{itemize}

    \begin{center}
        \begin{tikzpicture}
            [
                start chain,
                node distance = 0pt,
                HeapBlock/.style={draw, minimum width=2em, minimum height=2em, outer sep=0pt, on chain, fill=teal!30},
                SortedBlock/.style={draw, minimum width=2em, minimum height=2em, outer sep=0pt, on chain, fill=blue!30},
            ]

            { start chain = going right
                \node [HeapBlock, label=above:\texttt{root}] (1) {$0$};
                \node [HeapBlock] (2) {$1$};
                \node [HeapBlock] (3) {$2$};
                \node [HeapBlock] (dots) {$\ldots$};
                \node [HeapBlock] (n2) {$n-2$};
                \node [HeapBlock] (n1) {$n-1$};
                \node [HeapBlock] (n) {$n$};

                \draw[->, blue] ([yshift=1em] 1.north east) to[bend left=30] ([yshift=.5em] n.north);

                \draw [decorate,decoration={brace,amplitude=5pt,mirror,raise=.5em}]
                (1.south west) -- (n.south east) node[midway,yshift=-2em]{Heap};
                %\draw[->] (val.south) [out=-30, in=-150] to (4.south);
            }
        \end{tikzpicture}
        \hspace{4em}
        \begin{tikzpicture}
            [
                start chain,
                node distance = 0pt,
                HeapBlock/.style={draw, minimum width=2em, minimum height=2em, outer sep=0pt, on chain, fill=teal!30},
                SortedBlock/.style={draw, minimum width=2em, minimum height=2em, outer sep=0pt, on chain, fill=blue!30},
            ]

            { start chain = going right
                \node [HeapBlock, label=above:\texttt{root}] (1) {$0$};
                \node [HeapBlock] (2) {$1$};
                \node [HeapBlock] (3) {$2$};
                \node [HeapBlock] (dots) {$\ldots$};
                \node [HeapBlock] (n2) {$n-2$};
                \node [HeapBlock] (n1) {$n-1$};
                \node [SortedBlock] (n) {$n$};

                \draw[->, blue] ([yshift=1em] 1.north east) to[bend left=30] ([yshift=.5em] n1.north);

                \draw [decorate,decoration={brace,amplitude=5pt,mirror,raise=.5em}]
                (1.south west) -- (n1.south east) node[midway,yshift=-2em]{Heap};

                \draw [decorate,decoration={brace,amplitude=5pt,mirror,raise=.5em}]
                (n.south west) -- (n.south east) node[midway,yshift=-2em]{sortiertes Array};
                %\draw[->] (val.south) [out=-30, in=-150] to (4.south);
            }
        \end{tikzpicture}

        \begin{tikzpicture}
            [
                start chain,
                node distance = 0pt,
                HeapBlock/.style={draw, minimum width=2em, minimum height=2em, outer sep=0pt, on chain, fill=teal!30},
                SortedBlock/.style={draw, minimum width=2em, minimum height=2em, outer sep=0pt, on chain, fill=blue!30},
            ]

            { start chain = going right
                \node [HeapBlock, label=above:\texttt{root}] (1) {$0$};
                \node [HeapBlock] (2) {$1$};
                \node [HeapBlock] (3) {$2$};
                \node [HeapBlock] (dots) {$\ldots$};
                \node [HeapBlock] (n2) {$n-2$};
                \node [SortedBlock] (n1) {$n-1$};
                \node [SortedBlock] (n) {$n$};
                \node [xshift=2em, on chain] (dots2) {$\ldots$};

                \draw[->, blue] ([yshift=1em] 1.north east) to[bend left=30] ([yshift=.5em] n2.north);

                \draw [decorate,decoration={brace,amplitude=5pt,mirror,raise=.5em}]
                (1.south west) -- (n2.south east) node[midway,yshift=-2em]{Heap};

                \draw [decorate,decoration={brace,amplitude=5pt,mirror,raise=.5em}]
                (n1.south west) -- (n.south east) node[midway,yshift=-2em]{sortiertes Array};
                %\draw[->] (val.south) [out=-30, in=-150] to (4.south);
            }
        \end{tikzpicture}
        \hspace{1em}
        \begin{tikzpicture}
            [
                start chain,
                node distance = 0pt,
                HeapBlock/.style={draw, minimum width=2em, minimum height=2em, outer sep=0pt, on chain, fill=teal!30},
                SortedBlock/.style={draw, minimum width=2em, minimum height=2em, outer sep=0pt, on chain, fill=blue!30},
            ]

            { start chain = going right
                \node [SortedBlock] (1) {$0$};
                \node [SortedBlock] (2) {$1$};
                \node [SortedBlock] (3) {$2$};
                \node [SortedBlock] (dots) {$\ldots$};
                \node [SortedBlock] (n2) {$n-2$};
                \node [SortedBlock] (n1) {$n-1$};
                \node [SortedBlock] (n) {$n$};

                %\draw[->, blue] ([yshift=1em] 1.north) to[bend left=30] ([yshift=.5em] n2.north);

                %\draw [decorate,decoration={brace,amplitude=5pt,mirror,raise=.5em}]
                %(1.south west) -- (n2.south east) node[midway,yshift=-2em]{Heap};

                \draw [decorate,decoration={brace,amplitude=5pt,mirror,raise=.5em}]
                (1.south west) -- (n.south east) node[midway,yshift=-2em]{sortiertes Array};
                %\draw[->] (val.south) [out=-30, in=-150] to (4.south);
            }
        \end{tikzpicture}
    \end{center}
\end{algo}

\begin{algo}{Quick-Sort}
    \emph{Quick-Sort} ist ein rekursiver Algorithmus, der nach dem Prinzip von \glqq divide-and-conquer\grqq (\glqq teile und herrsche\grqq) arbeitet:
    \begin{enumerate}
        \item Müssen 0 oder 1 Elemente sortiert werden: Rekursionsabbruch
        \item Wähle ein Element als \emph{Pivot-Element} aus
        \item Teile das Feld in zwei Teile Teile:\footnote{Beachte: Die Elemente werden \emph{in-place} getauscht. Siehe Beispiel.}
              \begin{itemize}
                  \item Ein Teil mit den Elementen größer als das Pivot.
                  \item Ein Teil mit den Elementen kleiner als das Pivot.
              \end{itemize}
        \item Wiederhole rekursiv für beide Teilfelder.
    \end{enumerate}


    Die überwiegende Mehrheit der Programmbibliotheken benutzt Quick-Sort.
    In fast allen Fällen sind zwei Optimierungen eingebaut:
    \begin{itemize}
        \item \glqq Median of three\grqq
        \item \glqq Behandlung kleiner Teilfelder\grqq
    \end{itemize}
\end{algo}

\begin{example}{Quick-Sort}
    \textcolor{red}{ACHTUNG: Das Beispiel muss in Rücksprache mit Herrn Pflug noch angepasst werden.
        Der in der Vorlesung benutzte - und damit klausurrelevante - Quick-Sort nutzt ein \emph{in-place}-Verfahren zum Aufteilen der Elemente.
        Das Beispiel unten (noch) nicht.}

    \centering

    \begin{tikzpicture}[
            start chain,
            node distance = 0pt,
            pivot/.style = {draw, red, on chain, label=above:\texttt{pivot}, xshift=1em},
            array/.style = {draw, minimum width=2em, minimum height=2em, outer sep=0pt, on chain},
            sorted/.style = {array, fill=blue!25},
        ]
        { start chain = going right
            \node [array] (23) {$23$};
            \node [array] (01) {$1$};
            \node [array] (05) {$5$};
            \node [array] (20) {$20$};
            \node [array] (10) {$10$};
            \node [array] (18) {$18$};
            \node [array] (08) {$8$};
            \node [array] (46) {$46$};
            \node [array] (11) {$11$};
            \node [array] (03) {$3$};
            \node [array] (36) {$36$};
            \node [array] (07) {$7$};
            \node [array] (30) {$30$};
            \node [array] (09) {$9$};
        }
    \end{tikzpicture}

    \begin{tikzpicture}[
            start chain,
            node distance = 0pt,
            pivot/.style = {draw, red, on chain, label=above:\texttt{pivot}, xshift=1em},
            array/.style = {draw, minimum width=2em, minimum height=2em, outer sep=0pt, on chain},
            sorted/.style = {array, fill=blue!25},
        ]
        { start chain = going right
            \node [array] (23) {$23$};
            \node [array] (01) {$1$};
            \node [array] (05) {$5$};
            \node [array] (20) {$20$};
            \node [array] (10) {$10$};
            \node [array] (18) {$18$};
            \node [array] (08) {$8$};
            \node [array] (46) {$46$};
            \node [array] (11) {$11$};
            \node [array] (03) {$3$};
            \node [array] (36) {$36$};
            \node [array] (07) {$7$};
            \node [array] (30) {$30$};
            \node [array, pivot] (09) {$9$};
        }
    \end{tikzpicture}

    \begin{tikzpicture}[
            start chain,
            node distance = 0pt,
            shifted/.style = {xshift=1em},
            pivot/.style = {draw, red, on chain, label=above:\texttt{pivot}},
            array/.style = {draw, minimum width=2em, minimum height=2em, outer sep=0pt, on chain},
            sorted/.style = {array, fill=blue!25},
        ]
        { start chain = going right
            \node [array] (01) {$1$};
            \node [array] (05) {$5$};
            \node [array] (08) {$8$};
            \node [array] (03) {$3$};
            \node [array] (07) {$7$};
            %
            \node [array, pivot, shifted] (09) {$9$};
            %
            \node [array, shifted] (23) {$23$};
            \node [array] (20) {$20$};
            \node [array] (10) {$10$};
            \node [array] (18) {$18$};
            \node [array] (46) {$46$};
            \node [array] (11) {$11$};
            \node [array] (36) {$36$};
            \node [array] (30) {$30$};
        }

        \draw [decorate,decoration={brace,amplitude=5pt,raise=.5em}]
        (01.north west) -- (07.north east) node[midway,yshift=1.5em] {\texttt{<= 9}};
        \draw [decorate,decoration={brace,amplitude=5pt,raise=.5em}]
        (23.north west) -- (30.north east) node[midway,yshift=1.5em] {\texttt{> 9}};
    \end{tikzpicture}



    \begin{tikzpicture}[
            start chain,
            node distance = 0pt,
            shifted/.style = {xshift=1em},
            pivot/.style = {draw, red, on chain, label=above:\texttt{pivot}},
            array/.style = {draw, minimum width=2em, minimum height=2em, outer sep=0pt, on chain},
            sorted/.style = {array, fill=blue!25},
        ]
        { start chain = going right
            \node [array] (01) {$1$};
            \node [array] (05) {$5$};
            \node [array] (08) {$8$};
            \node [array] (03) {$3$};
            \node [array] (07) {$7$};
            %
            \node [array, sorted, shifted] (09) {$9$};
            %
            \node [array, shifted] (23) {$23$};
            \node [array] (20) {$20$};
            \node [array] (10) {$10$};
            \node [array] (18) {$18$};
            \node [array] (46) {$46$};
            \node [array] (11) {$11$};
            \node [array] (36) {$36$};
            \node [array] (30) {$30$};
        }
    \end{tikzpicture}

    \begin{tikzpicture}[
            start chain,
            node distance = 0pt,
            shifted/.style = {xshift=1em},
            pivot/.style = {draw, red, on chain, label=above:\texttt{pivot}},
            array/.style = {draw, minimum width=2em, minimum height=2em, outer sep=0pt, on chain},
            sorted/.style = {array, fill=blue!25},
        ]
        { start chain = going right
            \node [array] (01) {$1$};
            \node [array] (05) {$5$};
            \node [array] (08) {$8$};
            \node [array] (03) {$3$};
            %
            \node [array, pivot, shifted] (07) {$7$};
            %
            \node [array, shifted, sorted] (09) {$9$};
            %
            \node [array, shifted] (23) {$23$};
            \node [array] (20) {$20$};
            \node [array] (10) {$10$};
            \node [array] (18) {$18$};
            \node [array] (46) {$46$};
            \node [array] (11) {$11$};
            \node [array] (36) {$36$};
            \node [array, pivot, shifted] (30) {$30$};
        }
    \end{tikzpicture}

    \begin{tikzpicture}[
            start chain,
            node distance = 0pt,
            shifted/.style = {xshift=1em},
            pivot/.style = {draw, red, on chain, label=above:\texttt{pivot}},
            array/.style = {draw, minimum width=2em, minimum height=2em, outer sep=0pt, on chain},
            sorted/.style = {array, fill=blue!25},
        ]
        { start chain = going right
            \node [array] (01) {$1$};
            \node [array] (05) {$5$};
            \node [array] (03) {$3$};
            %
            \node [array, pivot, shifted] (07) {$7$};
            %
            \node [array, shifted] (08) {$8$};
            %
            \node [array, shifted, sorted] (09) {$9$};
            %
            \node [array, shifted] (23) {$23$};
            \node [array] (20) {$20$};
            \node [array] (10) {$10$};
            \node [array] (18) {$18$};
            \node [array] (11) {$11$};
            %
            \node [array, pivot, shifted] (30) {$30$};
            %
            \node [array, shifted] (46) {$46$};
            \node [array] (36) {$36$};
        }

        \draw [decorate,decoration={brace,amplitude=5pt,raise=.5em}]
        (01.north west) -- (03.north east) node[midway,yshift=1.5em] {\texttt{<= 7}};
        \draw [decorate,decoration={brace,amplitude=5pt,raise=.5em}]
        (08.north west) -- (08.north east) node[midway,yshift=1.5em] {\texttt{> 7}};

        \draw [decorate,decoration={brace,amplitude=5pt,raise=.5em}]
        (23.north west) -- (11.north east) node[midway,yshift=1.5em] {\texttt{<= 30}};
        \draw [decorate,decoration={brace,amplitude=5pt,raise=.5em}]
        (46.north west) -- (36.north east) node[midway,yshift=1.5em] {\texttt{> 30}};
    \end{tikzpicture}

    \begin{tikzpicture}[
            start chain,
            node distance = 0pt,
            shifted/.style = {xshift=1em},
            pivot/.style = {draw, red, on chain, label=above:\texttt{pivot}},
            array/.style = {draw, minimum width=2em, minimum height=2em, outer sep=0pt, on chain},
            sorted pivot/.style = {array, fill=blue!25},
            sorted end/.style = {array, fill=blue!10},
        ]
        { start chain = going right
            \node [array] (01) {$1$};
            \node [array] (05) {$5$};
            \node [array] (03) {$3$};
            %
            \node [array, sorted pivot, shifted] (07) {$7$};
            %
            \node [array, shifted, sorted end] (08) {$8$};
            %
            \node [array, shifted, sorted pivot] (09) {$9$};
            %
            \node [array, shifted] (23) {$23$};
            \node [array] (20) {$20$};
            \node [array] (10) {$10$};
            \node [array] (18) {$18$};
            \node [array] (11) {$11$};
            %
            \node [array, sorted pivot, shifted] (30) {$30$};
            %
            \node [array, shifted] (46) {$46$};
            \node [array] (36) {$36$};
        }
    \end{tikzpicture}

    \begin{tikzpicture}[
            start chain,
            node distance = 0pt,
            shifted/.style = {xshift=1em},
            pivot/.style = {draw, red, on chain, label=above:\texttt{pivot}},
            array/.style = {draw, minimum width=2em, minimum height=2em, outer sep=0pt, on chain},
            sorted pivot/.style = {array, fill=blue!25},
            sorted end/.style = {array, fill=blue!10},
        ]
        { start chain = going right
            \node [array] (01) {$1$};
            \node [array] (05) {$5$};
            %
            \node [array, pivot, shifted] (03) {$3$};
            %
            \node [array, sorted pivot, shifted] (07) {$7$};
            %
            \node [array, shifted, sorted end] (08) {$8$};
            %
            \node [array, shifted, sorted pivot] (09) {$9$};
            %
            \node [array, shifted] (23) {$23$};
            \node [array] (20) {$20$};
            \node [array] (10) {$10$};
            \node [array] (18) {$18$};
            %
            \node [array, pivot, shifted] (11) {$11$};
            %
            \node [array, sorted pivot, shifted] (30) {$30$};
            %
            \node [array, shifted] (46) {$46$};
            %
            \node [array, pivot, shifted] (36) {$36$};
        }
    \end{tikzpicture}

    \begin{tikzpicture}[
            start chain,
            node distance = 0pt,
            shifted/.style = {xshift=1em},
            pivot/.style = {draw, red, on chain, label=above:\texttt{pivot}},
            array/.style = {draw, minimum width=2em, minimum height=2em, outer sep=0pt, on chain},
            sorted pivot/.style = {array, fill=blue!25},
            sorted end/.style = {array, fill=blue!10},
        ]
        { start chain = going right
            \node [array] (01) {$1$};
            %
            \node [array, pivot, shifted] (03) {$3$};
            %
            \node [array, shifted] (05) {$5$};
            %
            \node [array, sorted pivot, shifted] (07) {$7$};
            %
            \node [array, shifted, sorted end] (08) {$8$};
            %
            \node [array, shifted, sorted pivot] (09) {$9$};
            %
            \node [array, shifted] (10) {$10$};
            %
            \node [array, pivot, shifted] (11) {$11$};
            %
            \node [array, shifted] (23) {$23$};
            \node [array] (20) {$20$};
            \node [array] (18) {$18$};
            %
            %
            \node [array, sorted pivot, shifted] (30) {$30$};
            %
            \node [array, pivot, shifted] (36) {$36$};
            %
            \node [array, shifted] (46) {$46$};
        }

        \draw [decorate,decoration={brace,amplitude=5pt,raise=.5em}]
        (01.north west) -- (01.north east) node[midway,yshift=1.5em] {\texttt{<= 3}};
        \draw [decorate,decoration={brace,amplitude=5pt,raise=.5em}]
        (05.north west) -- (05.north east) node[midway,yshift=1.5em] {\texttt{> 3}};

        \draw [decorate,decoration={brace,amplitude=5pt,raise=.5em}]
        (10.north west) -- (10.north east) node[midway,yshift=1.5em] {\texttt{<= 11}};
        \draw [decorate,decoration={brace,amplitude=5pt,raise=.5em}]
        (23.north west) -- (18.north east) node[midway,yshift=1.5em] {\texttt{> 11}};

        \draw [decorate,decoration={brace,amplitude=5pt,raise=.5em}]
        (46.north west) -- (46.north east) node[midway,yshift=1.5em] {\texttt{> 36}};
    \end{tikzpicture}

    \begin{tikzpicture}[
            start chain,
            node distance = 0pt,
            shifted/.style = {xshift=1em},
            pivot/.style = {draw, red, on chain, label=above:\texttt{pivot}},
            array/.style = {draw, minimum width=2em, minimum height=2em, outer sep=0pt, on chain},
            sorted pivot/.style = {array, fill=blue!25},
            sorted end/.style = {array, fill=blue!10},
        ]
        { start chain = going right
            \node [array, sorted end] (01) {$1$};
            %
            \node [array, sorted pivot, shifted] (03) {$3$};
            %
            \node [array, shifted, sorted end] (05) {$5$};
            %
            \node [array, sorted pivot, shifted] (07) {$7$};
            %
            \node [array, shifted, sorted end] (08) {$8$};
            %
            \node [array, shifted, sorted pivot] (09) {$9$};
            %
            \node [array, shifted, sorted end] (10) {$10$};
            %
            \node [array, sorted pivot, shifted] (11) {$11$};
            %
            \node [array, shifted] (23) {$23$};
            \node [array] (20) {$20$};
            \node [array] (18) {$18$};
            %
            %
            \node [array, sorted pivot, shifted] (30) {$30$};
            %
            \node [array, sorted pivot, shifted] (36) {$36$};
            %
            \node [array, shifted, sorted end] (46) {$46$};
        }
    \end{tikzpicture}

    \begin{tikzpicture}[
            start chain,
            node distance = 0pt,
            shifted/.style = {xshift=1em},
            pivot/.style = {draw, red, on chain, label=above:\texttt{pivot}},
            array/.style = {draw, minimum width=2em, minimum height=2em, outer sep=0pt, on chain},
            sorted pivot/.style = {array, fill=blue!25},
            sorted end/.style = {array, fill=blue!10},
        ]
        { start chain = going right
            \node [array, sorted end] (01) {$1$};
            %
            \node [array, sorted pivot, shifted] (03) {$3$};
            %
            \node [array, shifted, sorted end] (05) {$5$};
            %
            \node [array, sorted pivot, shifted] (07) {$7$};
            %
            \node [array, shifted, sorted end] (08) {$8$};
            %
            \node [array, shifted, sorted pivot] (09) {$9$};
            %
            \node [array, shifted, sorted end] (10) {$10$};
            %
            \node [array, sorted pivot, shifted] (11) {$11$};
            %
            \node [array, shifted] (23) {$23$};
            \node [array] (20) {$20$};
            %
            \node [array, pivot, shifted] (18) {$18$};
            %
            %
            \node [array, sorted pivot, shifted] (30) {$30$};
            %
            \node [array, sorted pivot, shifted] (36) {$36$};
            %
            \node [array, shifted, sorted end] (46) {$46$};
        }
    \end{tikzpicture}

    \begin{tikzpicture}[
            start chain,
            node distance = 0pt,
            shifted/.style = {xshift=1em},
            pivot/.style = {draw, red, on chain, label=above:\texttt{pivot}},
            array/.style = {draw, minimum width=2em, minimum height=2em, outer sep=0pt, on chain},
            sorted pivot/.style = {array, fill=blue!25},
            sorted end/.style = {array, fill=blue!10},
        ]
        { start chain = going right
            \node [array, sorted end] (01) {$1$};
            %
            \node [array, sorted pivot, shifted] (03) {$3$};
            %
            \node [array, shifted, sorted end] (05) {$5$};
            %
            \node [array, sorted pivot, shifted] (07) {$7$};
            %
            \node [array, shifted, sorted end] (08) {$8$};
            %
            \node [array, shifted, sorted pivot] (09) {$9$};
            %
            \node [array, shifted, sorted end] (10) {$10$};
            %
            \node [array, sorted pivot, shifted] (11) {$11$};
            %
            \node [array, pivot, shifted] (18) {$18$};
            %
            \node [array, shifted] (23) {$23$};
            \node [array] (20) {$20$};
            %
            %
            \node [array, sorted pivot, shifted] (30) {$30$};
            %
            \node [array, sorted pivot, shifted] (36) {$36$};
            %
            \node [array, shifted, sorted end] (46) {$46$};
        }

        \draw [decorate,decoration={brace,amplitude=5pt,raise=.5em}]
        (23.north west) -- (20.north east) node[midway,yshift=1.5em] {\texttt{> 18}};
    \end{tikzpicture}

    \begin{tikzpicture}[
            start chain,
            node distance = 0pt,
            shifted/.style = {xshift=1em},
            pivot/.style = {draw, red, on chain, label=above:\texttt{pivot}},
            array/.style = {draw, minimum width=2em, minimum height=2em, outer sep=0pt, on chain},
            sorted pivot/.style = {array, fill=blue!25},
            sorted end/.style = {array, fill=blue!10},
        ]
        { start chain = going right
            \node [array, sorted end] (01) {$1$};
            %
            \node [array, sorted pivot, shifted] (03) {$3$};
            %
            \node [array, shifted, sorted end] (05) {$5$};
            %
            \node [array, sorted pivot, shifted] (07) {$7$};
            %
            \node [array, shifted, sorted end] (08) {$8$};
            %
            \node [array, shifted, sorted pivot] (09) {$9$};
            %
            \node [array, shifted, sorted end] (10) {$10$};
            %
            \node [array, sorted pivot, shifted] (11) {$11$};
            %
            \node [array, sorted pivot, shifted] (18) {$18$};
            %
            \node [array, shifted] (23) {$23$};
            \node [array] (20) {$20$};
            %
            %
            \node [array, sorted pivot, shifted] (30) {$30$};
            %
            \node [array, sorted pivot, shifted] (36) {$36$};
            %
            \node [array, shifted, sorted end] (46) {$46$};
        }
    \end{tikzpicture}

    \begin{tikzpicture}[
            start chain,
            node distance = 0pt,
            shifted/.style = {xshift=1em},
            pivot/.style = {draw, red, on chain, label=above:\texttt{pivot}},
            array/.style = {draw, minimum width=2em, minimum height=2em, outer sep=0pt, on chain},
            sorted pivot/.style = {array, fill=blue!25},
            sorted end/.style = {array, fill=blue!10},
        ]
        { start chain = going right
            \node [array, sorted end] (01) {$1$};
            %
            \node [array, sorted pivot, shifted] (03) {$3$};
            %
            \node [array, shifted, sorted end] (05) {$5$};
            %
            \node [array, sorted pivot, shifted] (07) {$7$};
            %
            \node [array, shifted, sorted end] (08) {$8$};
            %
            \node [array, shifted, sorted pivot] (09) {$9$};
            %
            \node [array, shifted, sorted end] (10) {$10$};
            %
            \node [array, sorted pivot, shifted] (11) {$11$};
            %
            \node [array, sorted pivot, shifted] (18) {$18$};
            %
            \node [array, shifted] (23) {$23$};
            %
            \node [array, pivot, shifted] (20) {$20$};
            %
            \node [array, sorted pivot, shifted] (30) {$30$};
            %
            \node [array, sorted pivot, shifted] (36) {$36$};
            %
            \node [array, shifted, sorted end] (46) {$46$};
        }
    \end{tikzpicture}

    \begin{tikzpicture}[
            start chain,
            node distance = 0pt,
            shifted/.style = {xshift=1em},
            pivot/.style = {draw, red, on chain, label=above:\texttt{pivot}},
            array/.style = {draw, minimum width=2em, minimum height=2em, outer sep=0pt, on chain},
            sorted pivot/.style = {array, fill=blue!25},
            sorted end/.style = {array, fill=blue!10},
        ]
        { start chain = going right
            \node [array, sorted end] (01) {$1$};
            %
            \node [array, sorted pivot, shifted] (03) {$3$};
            %
            \node [array, shifted, sorted end] (05) {$5$};
            %
            \node [array, sorted pivot, shifted] (07) {$7$};
            %
            \node [array, shifted, sorted end] (08) {$8$};
            %
            \node [array, shifted, sorted pivot] (09) {$9$};
            %
            \node [array, shifted, sorted end] (10) {$10$};
            %
            \node [array, sorted pivot, shifted] (11) {$11$};
            %
            \node [array, sorted pivot, shifted] (18) {$18$};
            %
            \node [array, pivot, shifted] (20) {$20$};
            %
            \node [array, shifted] (23) {$23$};
            %
            \node [array, sorted pivot, shifted] (30) {$30$};
            %
            \node [array, sorted pivot, shifted] (36) {$36$};
            %
            \node [array, shifted, sorted end] (46) {$46$};
        }

        \draw [decorate,decoration={brace,amplitude=5pt,raise=.5em}]
        (23.north west) -- (23.north east) node[midway,yshift=1.5em] {\texttt{> 20}};
    \end{tikzpicture}

    \begin{tikzpicture}[
            start chain,
            node distance = 0pt,
            shifted/.style = {xshift=1em},
            pivot/.style = {draw, red, on chain, label=above:\texttt{pivot}},
            array/.style = {draw, minimum width=2em, minimum height=2em, outer sep=0pt, on chain},
            sorted pivot/.style = {array, fill=blue!25},
            sorted end/.style = {array, fill=blue!10},
        ]
        { start chain = going right
            \node [array, sorted end] (01) {$1$};
            %
            \node [array, sorted pivot, shifted] (03) {$3$};
            %
            \node [array, shifted, sorted end] (05) {$5$};
            %
            \node [array, sorted pivot, shifted] (07) {$7$};
            %
            \node [array, shifted, sorted end] (08) {$8$};
            %
            \node [array, shifted, sorted pivot] (09) {$9$};
            %
            \node [array, shifted, sorted end] (10) {$10$};
            %
            \node [array, sorted pivot, shifted] (11) {$11$};
            %
            \node [array, sorted pivot, shifted] (18) {$18$};
            %
            \node [array, sorted pivot, shifted] (20) {$20$};
            %
            \node [array, shifted, sorted end] (23) {$23$};
            %
            \node [array, sorted pivot, shifted] (30) {$30$};
            %
            \node [array, sorted pivot, shifted] (36) {$36$};
            %
            \node [array, shifted, sorted end] (46) {$46$};
        }
    \end{tikzpicture}
\end{example}

\begin{defi}{Zeitkomplexität von Quick-Sort}
    \begin{itemize}
        \item Best-Case:
              \begin{itemize}
                  \item Pivot-Wert ist immer Median der Teilliste $\implies$ Teillisten werden stets halbiert
                  \item $\log n$ Stufen nötig mit $n$ Elementen
                  \item $\bigo(n \log n)$
              \end{itemize}
        \item Worst-Case:
              \begin{itemize}
                  \item Pivot-Wert ist immer größtes oder kleinstes Element der Teilliste
                  \item $n-1$ Stufen nötig mit $n-i$ Elementen pro Stufe
                  \item $\bigo(n^2)$
              \end{itemize}
        \item Average-Case:
              \begin{itemize}
                  \item Berechnung ziemlich aufwändig
                  \item $\bigo(n \log n)$
              \end{itemize}
    \end{itemize}
\end{defi}

\begin{bonus}{Standard-Optimierung von Pivot-Elementen}
    Wählt man z.B. bei einem vorsortierten Array stets das letzte Element, hat man genau den Worst-Case.

    \emph{Median-of-three-Methode} zum Auswählen des Pivots:
    \begin{itemize}
        \item Es werden drei Elemente als Referenz-Elemente (Anfang, Mitte, Ende) gewählt, von denen dann der Median als Pivot verwendet wird.
        \item Kann auf mehr als drei Elemente ausgebaut werden.
    \end{itemize}
\end{bonus}

\begin{bonus}{Standard-Optimierung des Rekursionsabbruchs}
    Beim \glqq einfachen\grqq  Rekursionsabbruch (Teilliste enthält 0 oder 1 Elemente) sind die letzten Rekursionsdurchgänge nicht mehr effektiv.

    Daher kann die Rekursion z.B. schon früher abgebrochen werden und die dann vorhandene Teilliste mit Insertion-Sort sortiert werden.
    Die Grenze dafür ist nicht klar festgelegt (Empfehlung von Knuth und Sedgewick liegt bei 9, im Internet aber oft zwischen 3 und 32).
\end{bonus}

\begin{defi}{Dual-Pivot Quick-Sort}
    Der \emph{Dual-Pivot Quick-Sort} funktioniert analog zum normalen Quick-Sort.

    Hier werden jedoch zwei Pivot-Elemente verwendet, die dann die entsprechende Teilliste in drei Bereiche aufteilen.
\end{defi}

\begin{defi}{Internes und externes Sortieren}
    Bis jetzt wurde vorausgesetzt, dass schneller Zugriff auf einen beliebigen Datensatz (\emph{wahlfreier Zugriff}) möglich ist.
    Das ist zumeist möglich, wenn der Datensatz im Hauptspeicher liegt.
    Ist dies der Fall, spricht man von \emph{internem Sortieren}.

    Bei sehr großen Datenbeständen ist das aber oft nicht mehr möglich, da sie z.B. auf Hintergrundspeicher ausgelagert werden müssen.
    Hier hat man dann lediglich \emph{sequentiellen Zugriff}\footnote{das Gleiche gilt auch für verkettete Listen} und man spricht von \emph{externem Sortieren}.
\end{defi}

\begin{algo}{Merge-Sort}
    \emph{Merge-Sort} kann sowohl iterativ als auch rekursiv implementiert werden, wobei die rekursive Variante etwas schneller ist.

    Der rekursive Ablauf ist:
    \begin{enumerate}
        \item Teile die Daten in zwei Hälften
        \item Unterteilung wird fortgesetzt, bis nur noch ein Element in einer Menge vorhanden ist
        \item Teilstücke werden für sich sortiert
        \item sortierte Teilstücke werden zusammengeführt
    \end{enumerate}

    Merge-Sort ist stabil.

    Die Komplexität von Merge-Sort liegt in $\bigo(n \log n)$.
\end{algo}

\begin{example}{Merge-Sort (rekursiv)}
    \centering

    \begin{tikzpicture}[
            start chain,
            node distance = 0pt,
            shifted/.style = {xshift=1em},
            array/.style = {draw, minimum width=2em, minimum height=2em, outer sep=0pt, on chain},
            sorted/.style = {array, fill=blue!25},
        ]
        { start chain = going right
            \node [array] (38) {$38$};
            \node [array] (27) {$27$};
            \node [array] (43) {$43$};
            \node [array] (03) {$3$};
            \node [array] (09) {$9$};
            \node [array] (82) {$82$};
            \node [array] (10) {$10$};
        }

        %\draw [decorate,decoration={brace,mirror,amplitude=5pt,raise=.5em}]
        %(38.south west) -- (03.south east) node[midway,yshift=-1.5em] {};
        %\draw [decorate,decoration={brace,mirror,amplitude=5pt,raise=.5em}]
        %(09.south west) -- (10.south east) node[midway,yshift=-1.5em] {};
    \end{tikzpicture}

    \begin{tikzpicture}[
            start chain,
            node distance = 0pt,
            shifted/.style = {xshift=1em},
            array/.style = {draw, minimum width=2em, minimum height=2em, outer sep=0pt, on chain},
            sorted/.style = {array, fill=blue!25},
        ]
        { start chain = going right
            \node [array] (38) {$38$};
            \node [array] (27) {$27$};
            \node [array] (43) {$43$};
            \node [array] (03) {$3$};
            %
            \node [array, shifted] (09) {$9$};
            \node [array] (82) {$82$};
            \node [array] (10) {$10$};
        }

        %\draw [decorate,decoration={brace,mirror,amplitude=5pt,raise=.5em}]
        %(38.south west) -- (27.south east) node[midway,yshift=-1.5em] {};
        %\draw [decorate,decoration={brace,mirror,amplitude=5pt,raise=.5em}]
        %(43.south west) -- (03.south east) node[midway,yshift=-1.5em] {};
        %
        %\draw [decorate,decoration={brace,mirror,amplitude=5pt,raise=.5em}]
        %(09.south west) -- (82.south east) node[midway,yshift=-1.5em] {};
        %\draw [decorate,decoration={brace,mirror,amplitude=5pt,raise=.5em}]
        %(10.south west) -- (10.south east) node[midway,yshift=-1.5em] {};
    \end{tikzpicture}

    \begin{tikzpicture}[
            start chain,
            node distance = 0pt,
            shifted/.style = {xshift=1em},
            array/.style = {draw, minimum width=2em, minimum height=2em, outer sep=0pt, on chain},
            sorted/.style = {array, fill=blue!25},
        ]
        { start chain = going right
            \node [array] (38) {$38$};
            \node [array] (27) {$27$};
            %
            \node [array, shifted] (43) {$43$};
            \node [array] (03) {$3$};
            %
            \node [array, shifted] (09) {$9$};
            \node [array] (82) {$82$};
            %
            \node [array, shifted] (10) {$10$};
        }

        %\draw [decorate,decoration={brace,mirror,amplitude=5pt,raise=.5em}]
        %(38.south west) -- (38.south east) node[midway,yshift=-1.5em] {};
        %\draw [decorate,decoration={brace,mirror,amplitude=5pt,raise=.5em}]
        %(27.south west) -- (27.south east) node[midway,yshift=-1.5em] {};
        %
        %\draw [decorate,decoration={brace,mirror,amplitude=5pt,raise=.5em}]
        %(43.south west) -- (43.south east) node[midway,yshift=-1.5em] {};
        %\draw [decorate,decoration={brace,mirror,amplitude=5pt,raise=.5em}]
        %(03.south west) -- (03.south east) node[midway,yshift=-1.5em] {};
        %
        %\draw [decorate,decoration={brace,mirror,amplitude=5pt,raise=.5em}]
        %(09.south west) -- (09.south east) node[midway,yshift=-1.5em] {};
        %\draw [decorate,decoration={brace,mirror,amplitude=5pt,raise=.5em}]
        %(82.south west) -- (82.south east) node[midway,yshift=-1.5em] {};
    \end{tikzpicture}

    \begin{tikzpicture}[
            start chain,
            node distance = 0pt,
            shifted/.style = {xshift=1em},
            array/.style = {draw, minimum width=2em, minimum height=2em, outer sep=0pt, on chain},
            sorted/.style = {array, fill=blue!25},
            framed/.style = {draw, red, densely dashed, label=below:\texttt{sort}},
        ]
        { start chain = going right
            \node [array] (38) {$38$};
            %
            \node [array, shifted] (27) {$27$};
            %
            \node [array, shifted] (43) {$43$};
            %
            \node [array, shifted] (03) {$3$};
            %
            \node [array, shifted] (09) {$9$};
            %
            \node [array, shifted] (82) {$82$};
            %
            \node [array, shifted] (10) {$10$};
        }

        \node [framed, fit={(38) (27)}] {};
        \node [framed, fit={(43) (03)}] {};
        \node [framed, fit={(09) (82)}] {};
    \end{tikzpicture}

    \begin{tikzpicture}[
            start chain,
            node distance = 0pt,
            shifted/.style = {xshift=1em},
            array/.style = {draw, minimum width=2em, minimum height=2em, outer sep=0pt, on chain},
            sorted/.style = {array, fill=blue!25},
            framed/.style = {draw, red, densely dashed, label=below:\texttt{sort}},
        ]
        { start chain = going right
            \node [array] (27) {$27$};
            \node [array] (38) {$38$};
            %
            \node [array, shifted] (03) {$3$};
            \node [array] (43) {$43$};
            %
            \node [array, shifted] (09) {$9$};
            \node [array] (82) {$82$};
            %
            \node [array, shifted] (10) {$10$};
        }
    \end{tikzpicture}

    \begin{tikzpicture}[
            start chain,
            node distance = 0pt,
            shifted/.style = {xshift=1em},
            array/.style = {draw, minimum width=2em, minimum height=2em, outer sep=0pt, on chain},
            sorted/.style = {array, fill=blue!25},
            framed/.style = {draw, red, densely dashed, label=below:\texttt{sort}},
        ]
        { start chain = going right
            \node [array] (27) {$27$};
            \node [array] (38) {$38$};
            %
            \node [array, shifted] (03) {$3$};
            \node [array] (43) {$43$};
            %
            \node [array, shifted] (09) {$9$};
            \node [array] (82) {$82$};
            %
            \node [array, shifted] (10) {$10$};
        }

        \node [framed, fit={(27) (43)}] {};
        \node [framed, fit={(09) (10)}] {};
    \end{tikzpicture}

    \begin{tikzpicture}[
            start chain,
            node distance = 0pt,
            shifted/.style = {xshift=1em},
            array/.style = {draw, minimum width=2em, minimum height=2em, outer sep=0pt, on chain},
            sorted/.style = {array, fill=blue!25},
            framed/.style = {draw, red, densely dashed, label=below:\texttt{sort}},
        ]
        { start chain = going right
            \node [array] (03) {$3$};
            \node [array] (27) {$27$};
            \node [array] (38) {$38$};
            \node [array] (43) {$43$};
            %
            \node [array, shifted] (09) {$9$};
            \node [array] (10) {$10$};
            \node [array] (82) {$82$};
        }
    \end{tikzpicture}

    \begin{tikzpicture}[
            start chain,
            node distance = 0pt,
            shifted/.style = {xshift=1em},
            array/.style = {draw, minimum width=2em, minimum height=2em, outer sep=0pt, on chain},
            sorted/.style = {array, fill=blue!25},
            framed/.style = {draw, red, densely dashed, label=below:\texttt{sort}},
        ]
        { start chain = going right
            \node [array] (03) {$3$};
            \node [array] (27) {$27$};
            \node [array] (38) {$38$};
            \node [array] (43) {$43$};
            %
            \node [array, shifted] (09) {$9$};
            \node [array] (10) {$10$};
            \node [array] (82) {$82$};
        }

        \node [framed, fit={(03) (82)}] {};
    \end{tikzpicture}

    \begin{tikzpicture}[
            start chain,
            node distance = 0pt,
            shifted/.style = {xshift=1em},
            array/.style = {draw, minimum width=2em, minimum height=2em, outer sep=0pt, on chain},
            sorted/.style = {array, fill=blue!25},
            framed/.style = {draw, red, densely dashed, label=below:\texttt{sort}},
        ]
        { start chain = going right
            \node [array, sorted] (03) {$3$};
            \node [array, sorted] (09) {$9$};
            \node [array, sorted] (10) {$10$};
            \node [array, sorted] (27) {$27$};
            \node [array, sorted] (38) {$38$};
            \node [array, sorted] (43) {$43$};
            \node [array, sorted] (82) {$82$};
        }
    \end{tikzpicture}
\end{example}

\begin{defi}{Optimierung von Merge-Sort}
    \begin{itemize}
        \item Felduntergrenze
              \begin{itemize}
                  \item Feld wird nicht in Einzelelemente geteilt, sondern in Gruppen zu $n$ Elementen, die dann mit Insertion-Sort sortiert werden
                  \item gcj-Java nimmt 6 Elemente als Grenze
                  \item Python nimmt 64 Elemente als Grenze
              \end{itemize}
        \item Trivialfall
              \begin{itemize}
                  \item ist das kleinste Element der einen Teilfolge größer als das größte Element der anderen Teilfolge $\implies$ Zusammenfügen beschränkt sich auf Hintereinandersetzen der beiden Teilfolgen
                  \item genutzt von z.B. Java
                  \item Vorsortierung wird ausgenutzt
              \end{itemize}
        \item Natürlicher Merge-Sort
              \begin{itemize}
                  \item weitergehende Ausnutzung der Vorsortierung
                  \item jede Zahlenfolge besteht aus Teilstücken, die abwechselnd monoton steigend und monoton fallend sind
                  \item Idee ist, diese bereits sortierten Teilstücke als Ausgangsbasis des Merge-Sorts zu nehmen
                  \item bei nahezu sortierten Feldern werden die Teilstücke sehr groß und as Verfahren sehr schnell
                  \item Best-Case in $\bigo(n)$
              \end{itemize}
    \end{itemize}
\end{defi}

\subsection{Spezialisierte Sortierverfahren}

\begin{defi}{Radix-Sort}
    \emph{Radix-Sort} heißt eine Gruppe von Sortierverfahren mit folgenden Eigenschaften:
    \begin{itemize}
        \item Sortiervorgang erfolgt mehrstufig
        \item zunächst wird Grobsortierung vorgenommen, zu der nur ein Teil des Schlüssels (z.B. erstes Zeichen) herangezogen wird
        \item grob sortierte Bereiche dann feinsortiert, wobei schrittweise der restliche Teil des Schlüssels verwendet wird
    \end{itemize}
\end{defi}

\begin{bonus}{MSD-Radix-Sort}
    \begin{itemize}
        \item MSD = most significant digits
        \item In der 1. Stufe wird die erste Ziffer betrachtet.
        \item Benutzt Bucket-Sort mit Teillisten.
        \item Bucket-Sort wird für die einzelnen Buckets weiter genutzt, bis alle Werte sortiert sind
    \end{itemize}
\end{bonus}

\begin{bonus}{LSD-Radix-Sort}
    \begin{itemize}
        \item LSD = least significant digits
        \item In der 1. Stufe wird die letzte Ziffer betrachtet.
        \item Benutzt Bucket-Sort mit schlüsselindiziertem Zählen.
    \end{itemize}
\end{bonus}

\begin{algo}{Bucket-Sort}
    \emph{Bucket-Sort} sortiert für bestimmte Werte-Verteilungen einen Datensatz in linearer Zeit.

    Der Algorithmus ist in drei Phasen eingeteilt:
    \begin{enumerate}
        \item Verteilung der Elemente auf die Buckets
        \item Jeder Bucket wird mit einem weiteren Sortiertverfahren sortiert .
        \item Der Inhalt der sortierten Buckets wird konkateniert.
    \end{enumerate}

    Zu viele oder zu wenige Eimer können zu großen Laufzeitproblemen führen.
    Empfohlen werden:
    \begin{itemize}
        \item Sedgewick: Für 64-bit Schlüssel (\texttt{long}) $2^{16}$ Eimer
        \item Linux-related: Für 32-bit Schlüssel (\texttt{int}) $2^{11}$ Eimer
    \end{itemize}

    Der Einsatz von Bucket-Sort lohnt sich nur, wenn die Anzahl der zu sortierenden Werte deutlich größer ist, als die Anzahl der Eimer.
\end{algo}