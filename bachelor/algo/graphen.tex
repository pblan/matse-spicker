\section{Graphen}

\begin{defi}{Gerichteter Graph}
    Ein \emph{gerichteter Graph} $G = (V, E)$ besteht aus
    \begin{itemize}
        \item einer endlichen, nicht leeren Menge $V = \{v_1, \ldots, v_n\}$ von \emph{Knoten (vertices)} und
        \item einer Relation $E \subseteq V \times V$ von geordneten Paaren $e = (u, v)$ den \emph{Kanten (edges)}.
    \end{itemize}

    Jede Kante $(u,v) \in E$ hat einen Anfangsknoten $u$ und einen Enknoten $v$ und damit eine Richtung von $u$ nach $v$ ($u=v$ ist möglich).

    \begin{center}
        \begin{tikzpicture}[->]
            \node[draw, circle] (1) {1};
            \node[draw, circle, below=of 1] (2) {2};
            \node[draw, circle, right=of 2] (3) {3};
            \node[draw, circle, above=of 3] (4) {4};
            \node[draw, circle, right=of 4] (5) {5};

            %\draw[->] (1) to[bend left=15] (4);
            %\draw[->] (4) to[bend left=15] (1);
            %\draw[->] (1) to (2);
            %\draw[->] (2) to (3);
            %\draw[->] (3) to (4);
            %\draw[->] (4) to (5);

            \path (1) edge[bend left=15] (4);
            \path (4) edge[bend left=15] (1);
            \path (1) edge (2);
            \path (2) edge (3);
            \path (3) edge (4);
            \path (4) edge (5);
            \path (5) edge[loop above] (5);
        \end{tikzpicture}
    \end{center}
\end{defi}

\begin{defi}{Ungerichteter Graph}
    Ein \emph{ungerichteter Graph} $G = (V, E)$ besteht aus
    \begin{itemize}
        \item einer endlichen, nicht leeren Menge $V = \{v_1, \ldots, v_n\}$ von \emph{Knoten (vertices)} und
        \item einer symmetrischen Relation $E \subseteq V \times V$ von geordneten Paaren $e = (u, v) \iff (v, u)$ den \emph{Kanten (edges)}.
    \end{itemize}

    Jede Kante $(u,v) \in E$ hat einen Anfangsknoten $u$ und einen Enknoten $v$ und damit eine Richtung von $u$ nach $v$ ($u=v$ ist möglich).

    \vspace{1em}
    \begin{center}
        \begin{tikzpicture}
            \node[draw, circle] (1) {1};
            \node[draw, circle, below=of 1] (2) {2};
            \node[draw, circle, right=of 2] (3) {3};
            \node[draw, circle, above=of 3] (4) {4};
            \node[draw, circle, right=of 4] (5) {5};

            %\draw[->] (1) to[bend left=15] (4);
            %\draw[->] (4) to[bend left=15] (1);
            %\draw[->] (1) to (2);
            %\draw[->] (2) to (3);
            %\draw[->] (3) to (4);
            %\draw[->] (4) to (5);

            \path (1) edge (4);
            \path (1) edge (2);
            \path (2) edge (3);
            \path (3) edge (4);
            \path (4) edge (5);
        \end{tikzpicture}
    \end{center}
\end{defi}

\begin{defi}{Gewichteter Graph}
    Ein Graph heißt \emph{gewichtet}, wenn jeder Kante ein Wert als \emph{Gewicht} zugeordnet ist (z.B. Transportkosten, Entfernung).

    \begin{center}
        \begin{tikzpicture}[->]
            \node[draw, circle] (1) {1};
            \node[draw, circle, below=of 1] (2) {2};
            \node[draw, circle, right=of 2] (3) {3};
            \node[draw, circle, above=of 3] (4) {4};
            \node[draw, circle, right=of 4] (5) {5};

            %\draw[->] (1) to[bend left=15] (4);
            %\draw[->] (4) to[bend left=15] (1);
            %\draw[->] (1) to (2);
            %\draw[->] (2) to (3);
            %\draw[->] (3) to (4);
            %\draw[->] (4) to (5);

            \path (1) edge[bend left=15] node[above,scale=0.7] {2} (4);
            \path (4) edge[bend left=15] node[below,scale=0.7] {1} (1);
            \path (1) edge node[left,scale=0.7] {2} (2);
            \path (2) edge node[above,scale=0.7] {1} (3);
            \path (3) edge node[right,scale=0.7] {8} (4);
            \path (4) edge node[above,scale=0.7] {1} (5);
            \path (5) edge[loop above] node[above,scale=0.7] {2} (5);
        \end{tikzpicture}
    \end{center}
\end{defi}

\begin{defi}{Teilgraph}
    $G' = (V', E')$ heißt \emph{Teilgraph} von $G=(V, E)$, wenn gilt:
    $$
        V' \subseteq V \quad \text{und} \quad E' \subseteq E
    $$
\end{defi}

\begin{defi}{Weg}
    Sei $G = (V, E)$ ein Graph.

    Eine Folge von Knoten
    $$
        W := (v_1, v_2, \ldots, v_n)
    $$
    heißt \emph{Weg} oder \emph{Pfad} in $G$, falls gilt:
    $$
        \forall 1 \leq i \leq n-1 : (v_i, v_{i+1}) \in E
    $$
    (also eine Folge von zusammenhängenden Kanten)

    $\alpha(W) := v_1$ heißt \emph{Anfangsknoten} des Weges $W$.

    $\omega(W) := v_n$ heißt \emph{Endknoten} des Weges $W$.

    $\forall v_i \in V : (v_i)$ heißt \emph{trivialer Weg} und ist stets ein Weg in $G$.

    Die \emph{Länge eines Weges} ist $l(W) := n-1$, falls $n$ Knoten auf diesem Weg besucht werden.

    Ein Weg heißt \emph{einfacher Weg}, wenn kein Knoten (ausgenommen Start- und Endknoten) mehr als einmal vorkommt.

    Ein \emph{Zykel} oder \emph{Kreis} ist ein nicht-trivialer einfacher Weg mit der Bedingung $\alpha(W) = \omega(W)$.
\end{defi}

\begin{defi}{Adjazenz}
    Zwei Knoten heißen \emph{adjazent (benachbart)}, wenn sie eine Kante verbindet.
\end{defi}

\begin{defi}{Speicherung von Graphen}
    \begin{itemize}
        \item Kantenorientiert
              \begin{itemize}
                  \item Index für Kanten
                  \item für jede Kante speichern: Vorgänger-, Nachfolgerknoten (Markierung, Gewicht)
                  \item meist statische Darstellung, z.B. Kantenliste
              \end{itemize}
        \item Knotenorientiert
              \begin{itemize}
                  \item gebräuchlicher als kantenorientiert
                  \item in vielen Ausprägungen, z.B. Knotenliste, Adjazenzmatrix, Adjazenzliste
                  \item für Adjazenzmatrix gilt:
                        $$
                            A_{ij} = \begin{cases}
                                1 & \text{, falls} \ (i, j) \in E \\
                                0 & \text{, sonst}
                            \end{cases}
                        $$
              \end{itemize}
    \end{itemize}
\end{defi}

\begin{example}{Kantenliste}
    \begin{center}
        \begin{tikzpicture}[->]
            \node[draw, circle] (1) {1};
            \node[draw, circle, below left=of 1] (3) {3};
            \node[draw, circle, below right=of 1] (4) {4};
            \node[draw, circle, above right=of 4] (2) {2};
            \node[draw, circle, below=of 3] (5) {5};
            \node[draw, circle, below=of 4] (6) {6};

            \path (5) edge[loop below] (5);
            \path (1) edge[bend left=15] (3);
            \path (3) edge[bend left=15] (1);
            \path (4) edge (1);
            \path (1) edge (2);
            \path (3) edge (4);
            \path (5) edge (3);
            \path (3) edge (6);
            \path (6) edge (4);
            \path (6) edge (5);
            \path (6) edge (2);
        \end{tikzpicture}
    \end{center}

    Für eine Kantenliste gilt:
    \begin{itemize}
        \item Position \texttt{0}: Anzahl der Knoten
        \item Position \texttt{1}: Anzahl der Kanten
        \item Danach für jede Kante \texttt{c = (i, j)}:
              \begin{itemize}
                  \item Startknoten \texttt{i}
                  \item Endknoten \texttt{j}
              \end{itemize}
    \end{itemize}

    Für den Graphen $G$ oben gilt die Kantenliste:

    \centering

    \texttt{[6,11,1,2,1,3,3,1,4,1,3,4,3,6,5,3,5,5,6,5,6,2,6,4]}

\end{example}

\begin{example}{Knotenliste}
    \begin{center}
        \begin{tikzpicture}[->]
            \node[draw, circle] (1) {1};
            \node[draw, circle, below left=of 1] (3) {3};
            \node[draw, circle, below right=of 1] (4) {4};
            \node[draw, circle, above right=of 4] (2) {2};
            \node[draw, circle, below=of 3] (5) {5};
            \node[draw, circle, below=of 4] (6) {6};

            \path (5) edge[loop below] (5);
            \path (1) edge[bend left=15] (3);
            \path (3) edge[bend left=15] (1);
            \path (4) edge (1);
            \path (1) edge (2);
            \path (3) edge (4);
            \path (5) edge (3);
            \path (3) edge (6);
            \path (6) edge (4);
            \path (6) edge (5);
            \path (6) edge (2);
        \end{tikzpicture}
    \end{center}

    Für eine Knoten gilt:
    \begin{itemize}
        \item Position \texttt{0}: Anzahl der Knoten
        \item Position \texttt{1}: Anzahl der Kanten
        \item Danach für jeden Knoten \texttt{i}:
              \begin{itemize}
                  \item Ausgangsgrad des Knotens \texttt{i} (Anzahl der ausgehenden Kanten)
                  \item Alle Knoten \texttt{j} für die gilt $(i, j) \in E$
              \end{itemize}
    \end{itemize}

    Für den Graphen $G$ oben gilt die Knotenliste:

    \centering

    \texttt{[6,11,2,2,3,0,3,1,4,6,1,1,2,3,5,3,2,4,5]}

\end{example}

\begin{example}{Adjazenzmatrix}
    \begin{center}
        \begin{tikzpicture}[->]
            \node[draw, circle] (1) {1};
            \node[draw, circle, below left=of 1] (3) {3};
            \node[draw, circle, below right=of 1] (4) {4};
            \node[draw, circle, above right=of 4] (2) {2};
            \node[draw, circle, below=of 3] (5) {5};
            \node[draw, circle, below=of 4] (6) {6};

            \path (5) edge[loop below] (5);
            \path (1) edge[bend left=15] (3);
            \path (3) edge[bend left=15] (1);
            \path (4) edge (1);
            \path (1) edge (2);
            \path (3) edge (4);
            \path (5) edge (3);
            \path (3) edge (6);
            \path (6) edge (4);
            \path (6) edge (5);
            \path (6) edge (2);
        \end{tikzpicture}
    \end{center}

    Für den Graphen $G$ oben gilt die Adjazenzmatrix:
    $$
        A(G) = \vektor{
            0 & 1 & 1 & 0 & 0 & 0 \\
            0 & 0 & 0 & 0 & 0 & 0 \\
            1 & 0 & 0 & 1 & 0 & 1 \\
            1 & 0 & 0 & 0 & 0 & 0 \\
            0 & 0 & 1 & 0 & 1 & 0 \\
            0 & 1 & 0 & 1 & 1 & 0
        }
    $$
\end{example}

\begin{example}{Adjazenzliste}
    \begin{center}
        \begin{tikzpicture}[->]
            \node[draw, circle] (1) {1};
            \node[draw, circle, below left=of 1] (3) {3};
            \node[draw, circle, below right=of 1] (4) {4};
            \node[draw, circle, above right=of 4] (2) {2};
            \node[draw, circle, below=of 3] (5) {5};
            \node[draw, circle, below=of 4] (6) {6};

            \path (5) edge[loop below] (5);
            \path (1) edge[bend left=15] (3);
            \path (3) edge[bend left=15] (1);
            \path (4) edge (1);
            \path (1) edge (2);
            \path (3) edge (4);
            \path (5) edge (3);
            \path (3) edge (6);
            \path (6) edge (4);
            \path (6) edge (5);
            \path (6) edge (2);
        \end{tikzpicture}
    \end{center}

    Für den Graphen $G$ oben gilt die Adjazenzliste:

    \vspace{1em}

    \begin{center}
        \begin{tikzpicture}[
            start chain = going below,
            StackBlock/.style={minimum width=2em, minimum height=2em, outer sep=0pt, on chain},
            every node/.style={draw, label distance=0.5em},
            node distance=0em
            ]
            {
            \node [StackBlock] (k1) {$1$};
            \node [StackBlock,yshift=-0.5em] (k2) {$2$};
            \node [StackBlock,yshift=-0.5em] (k3) {$3$};
            \node [StackBlock,yshift=-0.5em] (k4) {$4$};
            \node [StackBlock,yshift=-0.5em] (k5) {$5$};
            \node [StackBlock,yshift=-0.5em] (k6) {$6$};

            % Chain 1
            { [continue chain = going right]
            \chainin (k1);

            \node [StackBlock, xshift=2em] (12) {$2$};
            \node [StackBlock] (12p) {};
            \node [StackBlock, xshift=2em] (13) {$3$};
            \node [StackBlock] (13p) {};
            }

            % Chain 3
            { [continue chain = going right]
            \chainin (k3);

            \node [StackBlock, xshift=2em] (31) {$1$};
            \node [StackBlock] (31p) {};
            \node [StackBlock, xshift=2em] (34) {$4$};
            \node [StackBlock] (34p) {};
            \node [StackBlock, xshift=2em] (36) {$6$};
            \node [StackBlock] (36p) {};
            }

            % Chain 4
            { [continue chain = going right]
            \chainin (k4);

            \node [StackBlock, xshift=2em] (41) {$1$};
            \node [StackBlock] (41p) {};
            }

            % Chain 5
            { [continue chain = going right]
            \chainin (k5);

            \node [StackBlock, xshift=2em] (53) {$3$};
            \node [StackBlock] (53p) {};
            \node [StackBlock, xshift=2em] (55) {$5$};
            \node [StackBlock] (55p) {};
            }

            % Chain 6
            { [continue chain = going right]
            \chainin (k6);

            \node [StackBlock, xshift=2em] (62) {$2$};
            \node [StackBlock] (62p) {};
            \node [StackBlock, xshift=2em] (64) {$4$};
            \node [StackBlock] (64p) {};
            \node [StackBlock, xshift=2em] (65) {$5$};
            \node [StackBlock] (65p) {};
            }

            \draw[->] (k1.east) [out=0, in=180] to (12.west);
            \draw[->] (k3.east) [out=0, in=180] to (31.west);
            \draw[->] (k4.east) [out=0, in=180] to (41.west);
            \draw[->] (k5.east) [out=0, in=180] to (53.west);
            \draw[->] (k6.east) [out=0, in=180] to (62.west);

            \draw[->] (12p.center) [out=0, in=180] to (13.west);

            \draw[->] (31p.center) [out=0, in=180] to (34.west);
            \draw[->] (34p.center) [out=0, in=180] to (36.west);

            \draw[->] (53p.center) [out=0, in=180] to (55.west);

            \draw[->] (62p.center) [out=0, in=180] to (64.west);
            \draw[->] (64p.center) [out=0, in=180] to (65.west);
            }
        \end{tikzpicture}
    \end{center}
\end{example}

\subsection{Suche}

\begin{algo}{Breitensuche}
    \begin{enumerate}
        \item Zunächst werden alle Knoten als \glqq noch nicht besucht\grqq markiert.
        \item Startpunkt $v$ wählen und als \glqq besucht\grqq markieren.
        \item Jetzt:
              \begin{enumerate}
                  \item alle von $v$ aus direkt erreichbaren (nicht besuchten) Knoten \glqq besuchen\grqq
                  \item alle von $v$ über zwei Kanten erreichbaren (nicht besuchten) Knoten \glqq besuchen\grqq
                  \item $\ldots$
              \end{enumerate}
        \item Wenn noch nicht alle Knoten besucht worden sind, wähle einen neuen unbesuchten Startpunkt $v$ und beginne bei Schritt 3
    \end{enumerate}

    Start bei Knoten 1.

    \begin{center}
        \begin{tikzpicture}[->]
            \node[draw, circle, current] (1) {1};
            \node[draw, circle, below left=of 1] (3) {3};
            \node[draw, circle, below right=of 1] (4) {4};
            \node[draw, circle, above right=of 4] (2) {2};
            \node[draw, circle, below=of 3] (5) {5};
            \node[draw, circle, below=of 4] (6) {6};

            \path (5) edge[loop below] (5);
            \path (1) edge[bend left=15] (3);
            \path (3) edge[bend left=15] (1);
            \path (4) edge (1);
            \path (1) edge (2);
            \path (3) edge (4);
            \path (5) edge (3);
            \path (3) edge (6);
            \path (6) edge (4);
            \path (6) edge (5);
            \path (6) edge (2);
        \end{tikzpicture}
        %
        \hspace{5em}
        %
        \begin{tikzpicture}[->]
            \node[draw, circle, visited] (1) {1};
            \node[draw, circle, below left=of 1, current] (3) {3};
            \node[draw, circle, below right=of 1] (4) {4};
            \node[draw, circle, above right=of 4, current] (2) {2};
            \node[draw, circle, below=of 3] (5) {5};
            \node[draw, circle, below=of 4] (6) {6};

            \path (5) edge[loop below] (5);
            \path (1) edge[bend left=15] (3);
            \path (3) edge[bend left=15] (1);
            \path (4) edge (1);
            \path (1) edge (2);
            \path (3) edge (4);
            \path (5) edge (3);
            \path (3) edge (6);
            \path (6) edge (4);
            \path (6) edge (5);
            \path (6) edge (2);
        \end{tikzpicture}
        %
        %\hspace{5em}
        %
        \begin{tikzpicture}[->]
            \node[draw, circle, visited] (1) {1};
            \node[draw, circle, below left=of 1, visited] (3) {3};
            \node[draw, circle, below right=of 1, current] (4) {4};
            \node[draw, circle, above right=of 4, visited] (2) {2};
            \node[draw, circle, below=of 3] (5) {5};
            \node[draw, circle, below=of 4, current] (6) {6};

            \path (5) edge[loop below] (5);
            \path (1) edge[bend left=15] (3);
            \path (3) edge[bend left=15] (1);
            \path (4) edge (1);
            \path (1) edge (2);
            \path (3) edge (4);
            \path (5) edge (3);
            \path (3) edge (6);
            \path (6) edge (4);
            \path (6) edge (5);
            \path (6) edge (2);
        \end{tikzpicture}
        %
        \hspace{5em}
        %
        \begin{tikzpicture}[->]
            \node[draw, circle, visited] (1) {1};
            \node[draw, circle, below left=of 1, visited] (3) {3};
            \node[draw, circle, below right=of 1, visited] (4) {4};
            \node[draw, circle, above right=of 4, visited] (2) {2};
            \node[draw, circle, below=of 3, current] (5) {5};
            \node[draw, circle, below=of 4, visited] (6) {6};

            \path (5) edge[loop below] (5);
            \path (1) edge[bend left=15] (3);
            \path (3) edge[bend left=15] (1);
            \path (4) edge (1);
            \path (1) edge (2);
            \path (3) edge (4);
            \path (5) edge (3);
            \path (3) edge (6);
            \path (6) edge (4);
            \path (6) edge (5);
            \path (6) edge (2);
        \end{tikzpicture}
        %
        %\hspace{5em}
        %
        \begin{tikzpicture}[->]
            \node[draw, circle, visited] (1) {1};
            \node[draw, circle, below left=of 1, visited] (3) {3};
            \node[draw, circle, below right=of 1, visited] (4) {4};
            \node[draw, circle, above right=of 4, visited] (2) {2};
            \node[draw, circle, below=of 3, visited] (5) {5};
            \node[draw, circle, below=of 4, visited] (6) {6};

            \path (5) edge[loop below] (5);
            \path (1) edge[bend left=15] (3);
            \path (3) edge[bend left=15] (1);
            \path (4) edge (1);
            \path (1) edge (2);
            \path (3) edge (4);
            \path (5) edge (3);
            \path (3) edge (6);
            \path (6) edge (4);
            \path (6) edge (5);
            \path (6) edge (2);
        \end{tikzpicture}
    \end{center}
\end{algo}

\begin{algo}{Tiefensuche}
    \begin{enumerate}
        \item Zunächst alle Knoten als \glqq noch nicht besucht\grqq markieren.
        \item Startpunkt $v$ wählen und als \glqq besucht\grqq markieren.
        \item Von dort aus möglichst langen Pfad entlang gehen; dabei nur bisher nicht besuchte Knoten \glqq besuchen\grqq
        \item Wenn dann noch nicht alle Knoten besucht worden sind, wähle einen neuen unbesuchten Startpunkt $v$ und beginne bei Schritt 3
    \end{enumerate}

    Start bei Knoten 1.

    \begin{center}
        \begin{tikzpicture}[->]
            \node[draw, circle, current] (1) {1};
            \node[draw, circle, below left=of 1] (3) {3};
            \node[draw, circle, below right=of 1] (4) {4};
            \node[draw, circle, above right=of 4] (2) {2};
            \node[draw, circle, below=of 3] (5) {5};
            \node[draw, circle, below=of 4] (6) {6};

            \path (5) edge[loop below] (5);
            \path (1) edge[bend left=15] (3);
            \path (3) edge[bend left=15] (1);
            \path (4) edge (1);
            \path (1) edge (2);
            \path (3) edge (4);
            \path (5) edge (3);
            \path (3) edge (6);
            \path (6) edge (4);
            \path (6) edge (5);
            \path (6) edge (2);
        \end{tikzpicture}
        %
        \hspace{5em}
        %
        \begin{tikzpicture}[->]
            \node[draw, circle, visited] (1) {1};
            \node[draw, circle, below left=of 1] (3) {3};
            \node[draw, circle, below right=of 1] (4) {4};
            \node[draw, circle, above right=of 4, current] (2) {2};
            \node[draw, circle, below=of 3] (5) {5};
            \node[draw, circle, below=of 4] (6) {6};

            \path (5) edge[loop below] (5);
            \path (1) edge[bend left=15] (3);
            \path (3) edge[bend left=15] (1);
            \path (4) edge (1);
            \path (1) edge (2);
            \path (3) edge (4);
            \path (5) edge (3);
            \path (3) edge (6);
            \path (6) edge (4);
            \path (6) edge (5);
            \path (6) edge (2);
        \end{tikzpicture}
        %
        %\hspace{5em}
        %
        \begin{tikzpicture}[->]
            \node[draw, circle, visited] (1) {1};
            \node[draw, circle, below left=of 1, current] (3) {3};
            \node[draw, circle, below right=of 1] (4) {4};
            \node[draw, circle, above right=of 4, visited] (2) {2};
            \node[draw, circle, below=of 3] (5) {5};
            \node[draw, circle, below=of 4] (6) {6};

            \path (5) edge[loop below] (5);
            \path (1) edge[bend left=15] (3);
            \path (3) edge[bend left=15] (1);
            \path (4) edge (1);
            \path (1) edge (2);
            \path (3) edge (4);
            \path (5) edge (3);
            \path (3) edge (6);
            \path (6) edge (4);
            \path (6) edge (5);
            \path (6) edge (2);
        \end{tikzpicture}
        %
        \hspace{5em}
        %
        \begin{tikzpicture}[->]
            \node[draw, circle, visited] (1) {1};
            \node[draw, circle, below left=of 1, visited] (3) {3};
            \node[draw, circle, below right=of 1, current] (4) {4};
            \node[draw, circle, above right=of 4, visited] (2) {2};
            \node[draw, circle, below=of 3] (5) {5};
            \node[draw, circle, below=of 4] (6) {6};

            \path (5) edge[loop below] (5);
            \path (1) edge[bend left=15] (3);
            \path (3) edge[bend left=15] (1);
            \path (4) edge (1);
            \path (1) edge (2);
            \path (3) edge (4);
            \path (5) edge (3);
            \path (3) edge (6);
            \path (6) edge (4);
            \path (6) edge (5);
            \path (6) edge (2);
        \end{tikzpicture}
        %
        %\hspace{5em}
        %
        \begin{tikzpicture}[->]
            \node[draw, circle, visited] (1) {1};
            \node[draw, circle, below left=of 1, visited] (3) {3};
            \node[draw, circle, below right=of 1, visited] (4) {4};
            \node[draw, circle, above right=of 4, visited] (2) {2};
            \node[draw, circle, below=of 3] (5) {5};
            \node[draw, circle, below=of 4, current] (6) {6};

            \path (5) edge[loop below] (5);
            \path (1) edge[bend left=15] (3);
            \path (3) edge[bend left=15] (1);
            \path (4) edge (1);
            \path (1) edge (2);
            \path (3) edge (4);
            \path (5) edge (3);
            \path (3) edge (6);
            \path (6) edge (4);
            \path (6) edge (5);
            \path (6) edge (2);
        \end{tikzpicture}
        %
        \hspace{5em}
        %
        \begin{tikzpicture}[->]
            \node[draw, circle, visited] (1) {1};
            \node[draw, circle, below left=of 1, visited] (3) {3};
            \node[draw, circle, below right=of 1, visited] (4) {4};
            \node[draw, circle, above right=of 4, visited] (2) {2};
            \node[draw, circle, below=of 3, current] (5) {5};
            \node[draw, circle, below=of 4, visited] (6) {6};

            \path (5) edge[loop below] (5);
            \path (1) edge[bend left=15] (3);
            \path (3) edge[bend left=15] (1);
            \path (4) edge (1);
            \path (1) edge (2);
            \path (3) edge (4);
            \path (5) edge (3);
            \path (3) edge (6);
            \path (6) edge (4);
            \path (6) edge (5);
            \path (6) edge (2);
        \end{tikzpicture}
        %
        %\hspace{5em}
        %
        \begin{tikzpicture}[->]
            \node[draw, circle, visited] (1) {1};
            \node[draw, circle, below left=of 1, visited] (3) {3};
            \node[draw, circle, below right=of 1, visited] (4) {4};
            \node[draw, circle, above right=of 4, visited] (2) {2};
            \node[draw, circle, below=of 3, visited] (5) {5};
            \node[draw, circle, below=of 4, visited] (6) {6};

            \path (5) edge[loop below] (5);
            \path (1) edge[bend left=15] (3);
            \path (3) edge[bend left=15] (1);
            \path (4) edge (1);
            \path (1) edge (2);
            \path (3) edge (4);
            \path (5) edge (3);
            \path (3) edge (6);
            \path (6) edge (4);
            \path (6) edge (5);
            \path (6) edge (2);
        \end{tikzpicture}
    \end{center}
\end{algo}

\begin{defi}{Level-Order-Baumdurchlauf}
    Führt man eine Breitensuche bei Bäumen aus, stellt man fest:
    \begin{itemize}
        \item Es ist nicht nötig, die Knoten zu markieren.
        \item Der Baum wird Ebene für Ebene durchlaufen,
    \end{itemize}

    Diesen Durchlauf nennt man \emph{Level-Order-Durchlauf}.
\end{defi}

\begin{defi}{Pre-Order-Baumdurchlauf}
    Führt man eine Tiefensuche bei Bäumen aus, nennt man diesen Durchlauf \emph{Pre-Order-Durchlauf}.

    Dabei wird stets der linke Teilbaum zuerst durchlaufen.
\end{defi}

\begin{defi}{Post-Order-Baumdurchlauf}
    \begin{enumerate}
        \item Durchlaufe linken Teilbaum
        \item Durchlaufe rechten Teilbaum
        \item Betrachte die Wurzel
    \end{enumerate}
\end{defi}

\begin{defi}{In-Order-Baumdurchlauf}
    \begin{enumerate}
        \item Durchlaufe linken Teilbaum
        \item Betrachte die Wurzel
        \item Durchlaufe rechten Teilbaum
    \end{enumerate}
\end{defi}

\begin{example}{Baumdurchläufe}
    \begin{center}
        \begin{forest}
            baseline,anchor=north,
            for tree={circle, draw,
                    minimum size=2em, % <-- added
                    inner sep=1pt}
                [
                    45
                        [
                            36
                                [
                                    25
                                        [
                                            2
                                        ]
                                        [
                                            3
                                        ]
                                ]
                                [
                                    26
                                        [
                                            19
                                        ]
                                        [, draw=none, edge={draw=none}]
                                ]
                        ]
                        [
                            17
                                [
                                    7
                                ]
                                [
                                    1
                                ]
                        ]
                ]
        \end{forest}
    \end{center}

    Für den gegebenen Baum gelten folgende Baumdurchläufe:

    \centering
    \begin{tabular}{|l|cccccccccc|}
        \hline
        Pre-Order   & 45 & 36 & 25 & 2  & 3  & 26 & 19 & 17 & 7  & 1  \\
        \hline
        In-Order    & 2  & 25 & 3  & 19 & 26 & 36 & 45 & 7  & 17 & 1  \\
        \hline
        Post-Order  & 2  & 3  & 25 & 19 & 26 & 36 & 7  & 1  & 17 & 45 \\
        \hline
        Level-Order & 45 & 36 & 17 & 25 & 26 & 7  & 1  & 2  & 3  & 19 \\
        \hline
    \end{tabular}
\end{example}


\subsection{Entwurfsprinzipien}

\begin{defi}{Traveling-Salesman-Problem}
    Das \emph{Traveling-Salesman-Problem} ist ein Optimierungsproblem.
    Die Aufgabe besteht darin, eine Reihenfolge für den Besuch mehrerer Orte so zu wählen, dass:
    \begin{itemize}
        \item kein Ort doppelt besucht wird und
        \item die Reisestrecke minimal ist.
    \end{itemize}
\end{defi}

\begin{defi}{Greedy}
    Wähle immer den Schritt, desse unmittelbarer Folgezustand das beste Ergebnis liefert.
    Denke nicht an die weiteren Schritte (greedy = gierig).

    Ob ein Greedy-Algorithmus die optimale Lösung liefert oder nicht, ist vom konkreten Problem abhängig.

    Für das Traveling-Salesman-Problem ist eine Greedy-Lösung z.B. die \emph{Nearest-Neighbour-Lösung}:
    \begin{itemize}
        \item Gehe in jedem Schritt zur nächstliegenden Stadt, die noch nicht besucht ist.
        \item $\bigo(n^2)$
        \item liefert nicht die optimale Lösung (kann sogar beliebig schlecht werden)
        \item arbeitet meistens einigermaßen gut
    \end{itemize}
\end{defi}

\begin{bonus}{Random Insertion}
    Wähle immer einen zufälligen Knoten.

    \begin{itemize}
        \item $\bigo(n^2)$
        \item keine Garantie, dass die Lösung irgendeinem Gütekriterium genügt
        \item im Allgmeinen ist Lösung ziemlich gut
        \item Vorteil: Durch den Zufallsfaktor kann man das Verfahren beliebig oft wiederholen und sich die beste Lösung heraussuchen.
    \end{itemize}
\end{bonus}

\begin{bonus}{Lösung mit minimalem Spannbaum}
    \begin{enumerate}
        \item Konstruktion eines minimalen Spannbaums nach Prim
        \item Durchlaufen mit Tiefensuche
        \item jede Kante wird zweimal durchlaufen
        \item summierte Gewichte des minimalen Spannbaums müssen kleiner sein als das optimale TSP-Ergebnis
        \item Optimierung:
              \begin{itemize}
                  \item Knoten schon einmal besucht: beim nächsten Mal überspringen
                  \item bei einem Schritt mehrere Möglichkeiten: zuerst kürzeren Weg wählen (greedy)
              \end{itemize}
    \end{enumerate}
\end{bonus}

\begin{defi}{Backtracking}
    Durchlauf eines Lösungsbaum in Pre-Order-Reihenfolge.

    Situation: Mehrere Alternativen sind in bestimmten Schritten des Algorithmus möglich.

    Lösung mit \emph{Backtracking}:
    \begin{enumerate}
        \item Wähle eine Alternative und verfolge dieses Weg weiter.
        \item Falls man so eine Lösung des Problems findet, ist man fertig.
        \item Ansonsten gehe einen Schritt zurück und verfolge rekursiv eine andere (nicht versuchte) Alternative in diesem Schritt.
        \item Falls alle Alternativen erfolglos probiert wurden, einen Schritt zurückgehen $\ldots$
    \end{enumerate}

    Backtracking-Algorithmen können exponentiellen Aufwand haben.

    Tipps:
    \begin{itemize}
        \item Durch Einführung von Zusatzbedingungen möglichst viele Sackgassen ausschließen.
        \item Symmetriebedingungen ausnutzen.
    \end{itemize}
\end{defi}

\begin{defi}{Branch and Bound}
    \begin{enumerate}
        \item Ermitteln einer oberen Schranke (greedy)
        \item Falls dieser Wert überschritten wird, kann man die Suche auf diesem Pfad abbrechen.
        \item Bessere Lösungen nutzen, um die Schranke zu verbessern.
    \end{enumerate}
\end{defi}


\subsection{Graphalgorithmen}

\begin{defi}{Bipartiter Graph}
    Ein Graph $G=(V, E)$ heißt \emph{bipartit}, wenn man $V$ in disjunkte Mengen $U$ und $W$ zerlegen kann, so dass alle Kanten zwischen $U$ und $W$ verlaufen.

    Die Knoten eines bipartiten Graphen lassen sich mit zwei Farben so einfärben, dass zwei Nachbarknoten immer unterschiedlich eingefärbt sind.
\end{defi}

\begin{algo}{Prüfung ob ein Graph bipartit ist}
    \begin{enumerate}
        \item Einfärben des 1. Knotens
        \item Durchlauf mit Breitensuche
        \item Abwechselndes Färben
    \end{enumerate}

    Start bei Knoten 1.

    \vspace{1em}

    \begin{center}
        \begin{tikzpicture}
            \node[draw, circle, text=white, fill = blue!50] (1) {1};
            \node[draw, circle, right=of 1] (2) {2};
            \node[draw, circle, right=of 2] (3) {3};
            \node[draw, circle, below=of 1] (4) {4};
            \node[draw, circle, right=of 4] (5) {5};
            \node[draw, circle, below=of 4] (6) {6};
            \node[draw, circle, right=of 6] (7) {7};

            \path (1) edge (2);
            \path (2) edge (3);
            \path (1) edge (4);
            \path (4) edge (5);
            \path (2) edge (5);
            \path (4) edge (6);
            \path (6) edge (7);
            \path (3) edge[bend left=30] (7);
        \end{tikzpicture}
        %
        \hspace{5em}
        %
        \begin{tikzpicture}
            \node[draw, circle, text=white, fill = blue!50] (1) {1};
            \node[draw, circle, text=white, right=of 1, fill = red!50] (2) {2};
            \node[draw, circle, right=of 2] (3) {3};
            \node[draw, circle, text=white, below=of 1, fill = red!50] (4) {4};
            \node[draw, circle, right=of 4] (5) {5};
            \node[draw, circle, below=of 4] (6) {6};
            \node[draw, circle, right=of 6] (7) {7};

            \path (1) edge (2);
            \path (2) edge (3);
            \path (1) edge (4);
            \path (4) edge (5);
            \path (2) edge (5);
            \path (4) edge (6);
            \path (6) edge (7);
            \path (3) edge[bend left=30] (7);
        \end{tikzpicture}

        \vspace{1em}

        \begin{tikzpicture}
            \node[draw, circle, text=white, fill = blue!50] (1) {1};
            \node[draw, circle, text=white, right=of 1, fill = red!50] (2) {2};
            \node[draw, circle, text=white, right=of 2, fill = blue!50] (3) {3};
            \node[draw, circle, text=white, below=of 1, fill = red!50] (4) {4};
            \node[draw, circle, text=white, right=of 4, fill = blue!50] (5) {5};
            \node[draw, circle, text=white, below=of 4, fill = blue!50] (6) {6};
            \node[draw, circle, right=of 6] (7) {7};

            \path (1) edge (2);
            \path (2) edge (3);
            \path (1) edge (4);
            \path (4) edge (5);
            \path (2) edge (5);
            \path (4) edge (6);
            \path (6) edge (7);
            \path (3) edge[bend left=30] (7);
        \end{tikzpicture}
        %
        \hspace{5em}
        %
        \begin{tikzpicture}
            \node[draw, circle, text=white, fill = blue!50] (1) {1};
            \node[draw, circle, text=white, right=of 1, fill = red!50] (2) {2};
            \node[draw, circle, text=white, right=of 2, fill = blue!50] (3) {3};
            \node[draw, circle, text=white, below=of 1, fill = red!50] (4) {4};
            \node[draw, circle, text=white, right=of 4, fill = blue!50] (5) {5};
            \node[draw, circle, text=white, below=of 4, fill = blue!50] (6) {6};
            \node[draw, circle, text=white, right=of 6, fill = red!50] (7) {7};

            \path (1) edge (2);
            \path (2) edge (3);
            \path (1) edge (4);
            \path (4) edge (5);
            \path (2) edge (5);
            \path (4) edge (6);
            \path (6) edge (7);
            \path (3) edge[bend left=30] (7);
        \end{tikzpicture}
    \end{center}
\end{algo}

\begin{defi}{Spannbaum}
    Ein \emph{Spannbaum} verbindet alle Knoten eines ungerichteten Graphen miteinander, hat jedoch keine Kreise.

    Der \emph{minimale Spannbaum} ist der Spannbaum, dessen Kanten das kleinste summierte Gewicht haben.
\end{defi}

\begin{algo}{Prim-Algorithmus}
    \begin{enumerate}
        \item Wähle einen beliebigen Startknoten für den minimalen Spannbaum $T$
        \item Solange $T$ noch nicht alle Knoten enthält:
              \begin{itemize}
                  \item Wähle eine Kante $e$ mit minimalem Gewicht aus, die einen noch nicht in $T$ enthaltenen Knoten $v$ mit $T$ verbindet.
                  \item Füge $e$ und $v$ dem Graphen $T$ hinzu.
              \end{itemize}
    \end{enumerate}

    Start bei Knoten $A$.

    \vspace{1em}
    \begin{center}
        \begin{tikzpicture}
            \node[draw, circle, prim node] (A) {$A$};
            \node[draw, circle, below right=of A] (B) {$B$};
            \node[draw, circle, above right=of B] (C) {$C$};
            \node[draw, circle, below left=of B] (D) {$D$};
            \node[draw, circle, below right=of B] (E) {$E$};
            \node[draw, circle, below=of D] (F) {$F$};
            \node[draw, circle, below=of E] (G) {$G$};

            \path (A) edge node[weight,above right] {7} (B);
            \path (B) edge node[weight,above left] {8} (C);
            \path (D) edge node[weight,right] {5} (A);
            \path (D) edge node[weight,above left] {9} (B);
            \path (E) edge node[weight,above right] {7} (B);
            \path (C) edge node[weight,right] {5} (E);
            \path (D) edge node[weight,above] {15} (E);
            \path (D) edge node[weight,right] {6} (F);
            \path (E) edge node[weight,above left] {8} (F);
            \path (E) edge node[weight,right] {11} (G);
            \path (F) edge node[weight,above] {9} (G);
        \end{tikzpicture}
        %
        \hspace{3em}
        %
        \begin{tikzpicture}
            \node[draw, circle, prim node] (A) {$A$};
            \node[draw, circle, below right=of A] (B) {$B$};
            \node[draw, circle, above right=of B] (C) {$C$};
            \node[draw, circle, below left=of B, prim node] (D) {$D$};
            \node[draw, circle, below right=of B] (E) {$E$};
            \node[draw, circle, below=of D] (F) {$F$};
            \node[draw, circle, below=of E] (G) {$G$};

            \path (A) edge node[weight,above right] {7} (B);
            \path (B) edge node[weight,above left] {8} (C);
            \path (D) edge[prim edge] node[weight,right] {5} (A);
            \path (D) edge node[weight,above left] {9} (B);
            \path (E) edge node[weight,above right] {7} (B);
            \path (C) edge node[weight,right] {5} (E);
            \path (D) edge node[weight,above] {15} (E);
            \path (D) edge node[weight,right] {6} (F);
            \path (E) edge node[weight,above left] {8} (F);
            \path (E) edge node[weight,right] {11} (G);
            \path (F) edge node[weight,above] {9} (G);
        \end{tikzpicture}
        %
        \hspace{3em}
        %
        \begin{tikzpicture}
            \node[draw, circle, prim node] (A) {$A$};
            \node[draw, circle, below right=of A] (B) {$B$};
            \node[draw, circle, above right=of B] (C) {$C$};
            \node[draw, circle, below left=of B, prim node] (D) {$D$};
            \node[draw, circle, below right=of B] (E) {$E$};
            \node[draw, circle, below=of D, prim node] (F) {$F$};
            \node[draw, circle, below=of E] (G) {$G$};

            \path (A) edge node[weight,above right] {7} (B);
            \path (B) edge node[weight,above left] {8} (C);
            \path (D) edge[prim edge] node[weight,right] {5} (A);
            \path (D) edge node[weight,above left] {9} (B);
            \path (E) edge node[weight,above right] {7} (B);
            \path (C) edge node[weight,right] {5} (E);
            \path (D) edge node[weight,above] {15} (E);
            \path (D) edge[prim edge] node[weight,right] {6} (F);
            \path (E) edge node[weight,above left] {8} (F);
            \path (E) edge node[weight,right] {11} (G);
            \path (F) edge node[weight,above] {9} (G);
        \end{tikzpicture}
        %
        \vspace{1em}
        %

        \begin{tikzpicture}
            \node[draw, circle, prim node] (A) {$A$};
            \node[draw, circle, below right=of A, prim node] (B) {$B$};
            \node[draw, circle, above right=of B] (C) {$C$};
            \node[draw, circle, below left=of B, prim node] (D) {$D$};
            \node[draw, circle, below right=of B] (E) {$E$};
            \node[draw, circle, below=of D, prim node] (F) {$F$};
            \node[draw, circle, below=of E] (G) {$G$};

            \path (A) edge[prim edge] node[weight,above right] {7} (B);
            \path (B) edge node[weight,above left] {8} (C);
            \path (D) edge[prim edge] node[weight,right] {5} (A);
            \path (D) edge node[weight,above left] {9} (B);
            \path (E) edge node[weight,above right] {7} (B);
            \path (C) edge node[weight,right] {5} (E);
            \path (D) edge node[weight,above] {15} (E);
            \path (D) edge[prim edge] node[weight,right] {6} (F);
            \path (E) edge node[weight,above left] {8} (F);
            \path (E) edge node[weight,right] {11} (G);
            \path (F) edge node[weight,above] {9} (G);
        \end{tikzpicture}
        %
        \hspace{3em}
        %
        \begin{tikzpicture}
            \node[draw, circle, prim node] (A) {$A$};
            \node[draw, circle, below right=of A, prim node] (B) {$B$};
            \node[draw, circle, above right=of B] (C) {$C$};
            \node[draw, circle, below left=of B, prim node] (D) {$D$};
            \node[draw, circle, below right=of B, prim node] (E) {$E$};
            \node[draw, circle, below=of D, prim node] (F) {$F$};
            \node[draw, circle, below=of E] (G) {$G$};

            \path (A) edge[prim edge] node[weight,above right] {7} (B);
            \path (B) edge node[weight,above left] {8} (C);
            \path (D) edge[prim edge] node[weight,right] {5} (A);
            \path (D) edge node[weight,above left] {9} (B);
            \path (E) edge[prim edge] node[weight,above right] {7} (B);
            \path (C) edge node[weight,right] {5} (E);
            \path (D) edge node[weight,above] {15} (E);
            \path (D) edge[prim edge] node[weight,right] {6} (F);
            \path (E) edge node[weight,above left] {8} (F);
            \path (E) edge node[weight,right] {11} (G);
            \path (F) edge node[weight,above] {9} (G);
        \end{tikzpicture}
        %
        \hspace{3em}
        %
        \begin{tikzpicture}
            \node[draw, circle, prim node] (A) {$A$};
            \node[draw, circle, below right=of A, prim node] (B) {$B$};
            \node[draw, circle, above right=of B, prim node] (C) {$C$};
            \node[draw, circle, below left=of B, prim node] (D) {$D$};
            \node[draw, circle, below right=of B, prim node] (E) {$E$};
            \node[draw, circle, below=of D, prim node] (F) {$F$};
            \node[draw, circle, below=of E] (G) {$G$};

            \path (A) edge[prim edge] node[weight,above right] {7} (B);
            \path (B) edge node[weight,above left] {8} (C);
            \path (D) edge[prim edge] node[weight,right] {5} (A);
            \path (D) edge node[weight,above left] {9} (B);
            \path (E) edge[prim edge] node[weight,above right] {7} (B);
            \path (C) edge[prim edge] node[weight,right] {5} (E);
            \path (D) edge node[weight,above] {15} (E);
            \path (D) edge[prim edge] node[weight,right] {6} (F);
            \path (E) edge node[weight,above left] {8} (F);
            \path (E) edge node[weight,right] {11} (G);
            \path (F) edge node[weight,above] {9} (G);
        \end{tikzpicture}
        %
        \vspace{1em}
        %

        \begin{tikzpicture}
            \node[draw, circle, prim node] (A) {$A$};
            \node[draw, circle, below right=of A, prim node] (B) {$B$};
            \node[draw, circle, above right=of B, prim node] (C) {$C$};
            \node[draw, circle, below left=of B, prim node] (D) {$D$};
            \node[draw, circle, below right=of B, prim node] (E) {$E$};
            \node[draw, circle, below=of D, prim node] (F) {$F$};
            \node[draw, circle, below=of E, prim node] (G) {$G$};

            \path (A) edge[prim edge] node[weight,above right] {7} (B);
            \path (B) edge node[weight,above left] {8} (C);
            \path (D) edge[prim edge] node[weight,right] {5} (A);
            \path (D) edge node[weight,above left] {9} (B);
            \path (E) edge[prim edge] node[weight,above right] {7} (B);
            \path (C) edge[prim edge] node[weight,right] {5} (E);
            \path (D) edge node[weight,above] {15} (E);
            \path (D) edge[prim edge] node[weight,right] {6} (F);
            \path (E) edge node[weight,above left] {8} (F);
            \path (E) edge node[weight,right] {11} (G);
            \path (F) edge[prim edge] node[weight,above] {9} (G);
        \end{tikzpicture}
    \end{center}
\end{algo}

\begin{defi}{Shortest-Path-Probleme}
    Eigentlich: Suche nach \emph{günstigsten Wegen} in gewichteten Graphen: Gewichte $\simeq$ Kosten

    Bei Anwendung auf ungewichtete Graphen ergibt sich: \emph{kürzeste Wege}.

    Beispiele:
    \begin{itemize}
        \item Single-Source-Shortest-Path
              \begin{itemize}
                  \item Dijkstra-Algorithmus (nicht-negative Kantengewichte)
                  \item Bellman-Ford-Algorithmus (keine negativen Zykel)
              \end{itemize}
        \item All-Pairs-Shortest-Path
              \begin{itemize}
                  \item Floyd-Warshall-Algorithmus
              \end{itemize}
        \item One-Pair
              \begin{itemize}
                  \item A*-Algorithmus
              \end{itemize}
    \end{itemize}
\end{defi}

\begin{algo}{Dijkstra-Algorithmus}
    Gegeben: Graph $G = (V, E)$, dessen Bewertungsfunktion die Eigenschaften hat:
    \begin{itemize}
        \item Jede Kante von $v_i$ nach $v_j$ hat nicht-negative Kosten: $C(i, j) \geq 0$
        \item Falls keine Kante zwischen $v_i$ und $v_j$: $C(i, j) = \infty$
        \item Diagonalelemente: $C(i, i) = 0$
    \end{itemize}

    Menge $S$: die Knoten, deren günstigste Wegekosten von der vorgegebenen Quelle (Startknoten) bereits bekannt sind.

    \begin{enumerate}
        \item Initialisierung: $S = \{ \text{Startknoten} \}$
        \item Beginnend mit Quelle alle ausgehenden Kanten betrachten (analog Breitensuche). Nachfolgerknoten $v$ mit günstigster Kante zu $S$ hinzunehmen.
        \item Jetzt: Berechnen, ob die Knoten in $V \setminus S$ günstiger über $v$ als Zwischenweg erreichbar sind, als ohne Umweg über $v$.
        \item Danach: Denjenigen Knoten $v'$ zu $S$ hinzunehmen, der nun am günstigsten zu erreichen ist. Bei zwei gleich günstigen Knoten wird ein beliebiger davon ausgewählt.
        \item Ab Schritt 3 wiederholen, bis alle Knoten in $S$ sind.
    \end{enumerate}

    Zeitkomplexität (bei Speicherung des Graphen mit Adjazenzmatrix): $\bigo(\abs{V}^2)$
\end{algo}

\begin{example}{Dijkstra-Algorithmus}
    \begin{center}
        \begin{tikzpicture}[->]
            \node[draw, circle] (A) {$A$};
            \node[draw, circle, below right=of A] (D) {$D$};
            \node[draw, circle, right=of D] (E) {$E$};
            \node[draw, circle, above right=of E] (B) {$B$};
            \node[draw, circle, below right=of B] (H) {$H$};
            \node[draw, circle, below left=of H] (G) {$G$};
            \node[draw, circle, below left=of A] (F) {$F$};
            \node[draw, circle, below right=of F] (C) {$C$};

            \path (A) edge node[weight, below left] {4} (D);
            \path (A) edge node[weight, above right] {2} (E);
            \path (A) edge node[weight, above right] {12} (B);
            \path (A) edge node[weight, above left] {30} (F);
            \path (E) edge node[weight, above] {1} (D);
            \path (E) edge node[weight, below right] {8} (C);
            \path (E) edge node[weight, above left] {8} (B);
            \path (C) edge node[weight, above right] {3} (F);
            \path (C) edge node[weight, above right] {3} (F);
            \path (C) edge node[weight, above] {12} (G);
            \path (G) edge node[weight, right] {4} (B);
            \path (G) edge node[weight, below right] {5} (H);
            \path (H) edge node[weight, above right] {2} (B);
        \end{tikzpicture}

        \vspace{1em}

        \begin{tabular}{|c|ccccccc|ccccccc|}
            \hline
            \multirow{2}{*}{$v_i$} & $d[2]$      & $d[3]$      & $d[4]$     & $d[5]$     & $d[6]$      & $d[7]$      & $d[8]$      & $p[2]$ & $p[3]$ & $p[4]$ & $p[5]$ & $p[6]$ & $p[7]$ & $p[8]$ \\ \cline{2-15}
                                   & B           & C           & D          & E          & F           & G           & H           & B      & C      & D      & E      & F      & G      & H      \\
            \hline
            A                      & 12          &             & 4          & \textbf{2} & 30          &             &             & A      &        & A      & A      & A      &        &        \\
            E                      & 10          & 10          & \textbf{3} & \textbf{2} & 30          &             &             & E      & E      & E      & A      & A      &        &        \\
            D                      & \textbf{10} & 10          & \textbf{3} & \textbf{2} & 30          &             &             & E      & E      & E      & A      & A      &        &        \\
            B                      & \textbf{10} & \textbf{10} & \textbf{3} & \textbf{2} & 30          &             &             & E      & E      & E      & A      & A      &        &        \\
            C                      & \textbf{10} & \textbf{10} & \textbf{3} & \textbf{2} & \textbf{13} & 22          &             & E      & E      & E      & A      & C      & C      &        \\
            F                      & \textbf{10} & \textbf{10} & \textbf{3} & \textbf{2} & \textbf{13} & \textbf{22} &             & E      & E      & E      & A      & C      & C      &        \\
            G                      & \textbf{10} & \textbf{10} & \textbf{3} & \textbf{2} & \textbf{13} & \textbf{22} & \textbf{27} & E      & E      & E      & A      & C      & C      & G      \\
            H                      & \textbf{10} & \textbf{10} & \textbf{3} & \textbf{2} & \textbf{13} & \textbf{22} & \textbf{27} & E      & E      & E      & A      & C      & C      & G      \\
            \hline
        \end{tabular}
    \end{center}

    \vspace{1em}

    \url{https://youtu.be/4pBP2hbnGso} (Herleitung und Erklärung)
\end{example}

\begin{algo}{Bellman-Ford-Algorithmus}
    Gegeben: Graph $G = (V, E)$, dessen Bewertungsfunktion die Eigenschaften hat:
    \begin{itemize}
        \item Falls keine Kante zwischen $v_i$ und $v_j$: $C(i, j) = \infty$
        \item Diagonalelemente: $C(i, i) = 0$
    \end{itemize}

    \begin{enumerate}
        \item Initialisierung: Startknoten $s$, Distanz zu $V \setminus \{s\}$ auf $\infty$ setzen
        \item Für jede Kante $(i, j)$:
              \begin{itemize}
                  \item Falls Distanz zu $v_i$ bekannt:
                        \subitem Falls $d(v_i) + C(i, j) < d(v_j)$, setze $d(v_j)$ auf $d(v_i) + C(i, j)$
              \end{itemize}
    \end{enumerate}

    Zeitkomplexität: $\bigo(\abs{V} \cdot \abs{E})$
\end{algo}

\begin{example}{Bellman-Ford-Algorithmus}
    \begin{center}
        \begin{tikzpicture}[->,baseline,anchor=north]
            \node[draw, circle] (S) {$S$};
            \node[draw, circle, above right=of S] (A) {$A$};
            \node[draw, circle, right=of A] (B) {$B$};
            \node[draw, circle, below right=of S] (C) {$C$};
            \node[draw, circle, right=of C] (D) {$D$};

            \path (S) edge node[weight, above left] {2} (A);
            \path (S) edge node[weight, below left] {5} (C);
            \path (A) edge node[weight, above] {4} (B);
            \path (C) edge node[weight, above] {2} (D);
            \path (C) edge node[weight, left] {-4} (A);
            \path (D) edge node[weight, right] {8} (B);
            \path (B) edge[bend left=10] node[weight, below right] {1} (C);
            \path (C) edge[bend left=10] node[weight, above left] {6} (B);
        \end{tikzpicture}
        %
        \hspace{5em}
        %
        \begin{tabular}{|c||c|c|c|c|c|}
            \hline
              & S & A        & B        & C        & D        \\
            \hline
            0 & 0 & $\infty$ & $\infty$ & $\infty$ & $\infty$ \\
            1 & 0 & 2        & $\infty$ & 5        & $\infty$ \\
            2 & 0 & 1        & 6        & 5        & 7        \\
            3 & 0 & 1        & 5        & 5        & 7        \\
            4 & 0 & 1        & 5        & 5        & 7        \\
            \hline
        \end{tabular}

        Keine Änderungen nach Schritt 3 $\implies$ Fertig!
    \end{center}
\end{example}

\begin{algo}{Floyd-Algorithmus}
    Gegeben: Graph $G = (V, E)$, dessen Bewertungsfunktion die Eigenschaften hat:
    \begin{itemize}
        \item Jede Kante von $v_i$ nach $v_j$ hat nicht-negative Kosten: $C(i, j) \geq 0$
        \item Falls keine Kante zwischen $v_i$ und $v_j$: $C(i, j) = \infty$
        \item Diagonalelemente: $C(i, i) = 0$
    \end{itemize}

    Im Kontrast zum Dijkstra-Algorithmus bestimmt der \emph{Floyd-Algorithmus} für alle geordneten Paare $(v, w)$ den kürzesten Weg von $v$ nach $w$.

    $\abs{V}$ Iterationen:
    \begin{enumerate}
        \item Vergleiche Kosten von
              \begin{itemize}
                  \item direkter Verbindung von Knoten $i$ zu Knoten $j$
                  \item Umweg über Knoten $1$ (also: von $i$ nach $1$ + von $1$ nach $j$)
                  \item Falls Umweg günstiger: alten Weg durch Umweg ersetzen
              \end{itemize}
        \item Umwege über Knoten $2$ betrachten.
        \item[$k$.] Umwege über Knoten $k$ betrachten.
    \end{enumerate}

    Der Floyd-Algorithmus nutzt eine $\abs{V} \times \abs{V}$-Matrix, um die Kosten der günstigsten Wege zu speichern:
    \begin{center}
        $A_k[i][j] :=$ minimale Kosten, um in Schritt $k$ über irgendwelche der Knoten in $V$ vom Knoten $i$ zum Knoten $j$ zu gelangen
    \end{center}

    \begin{itemize}
        \item Initialisierung: $A_0[i][j]= C(i, j)$
        \item $\abs{V}$ Iterationen mit \glqq dynamischer Programmierung\grqq:
              \subitem Iterationsformel zur Aktualisierung von $A[i][j]$:
              $$
                  A_k[i][j] = \min \{ A_{k-1}[i][j] , A_{k-1}[i][k] + A_{k-1}[k][j] \}
              $$
    \end{itemize}

    Der Floyd-Algorithmus ist ein wichtiges Beispiel für dynamische Programmierung.
\end{algo}

\begin{code}{Floyd-Algorithmus}
    \lstinputlisting{floyd.java}
\end{code}

\begin{example}{Floyd-Algorithmus}
    \newcolumntype{g}{>{\columncolor{gray!25}}c}

    $A_0$ (direkter Weg):
    \begin{center}
        \begin{tikzpicture}[baseline,anchor=north]
            \node[draw, circle] (1) {$1$};
            \node[draw, circle, right=of 1] (2) {$2$};
            \node[draw, circle, below=of 1] (3) {$3$};
            \node[draw, circle, right=of 3] (4) {$4$};

            \path (1) edge node[weight, above] {2} (2);
            \path (1) edge node[weight, left] {3} (3);
            \path (3) edge node[weight, below] {7} (4);
            \path (2) edge node[weight, right] {1} (4);
        \end{tikzpicture}
        %
        \hspace{5em}
        %
        \begin{tabular}[t]{|c|cccc|}
            \hline
              & 1                   & 2                   & 3                   & 4                   \\
            \hline
            1 & \textcolor{gray}{0} & 2                   & 3                   & $\infty$            \\
            2 & 2                   & \textcolor{gray}{0} & $\infty$            & 1                   \\
            3 & 3                   & $\infty$            & \textcolor{gray}{0} & 7                   \\
            4 & $\infty$            & 1                   & 7                   & \textcolor{gray}{0} \\
            \hline
        \end{tabular}
        %
        \hspace{5em}
        %
        \begin{tabular}[t]{|c|cccc|}
            \hline
              & 1                   & 2                   & 3                   & 4 \\
            \hline
            1 & \textcolor{gray}{-} & -                   & -                   & - \\
            2 & -                   & \textcolor{gray}{-} & -                   & - \\
            3 & -                   & -                   & \textcolor{gray}{-} & - \\
            4 & -                   & -                   & -                   & - \\
            \hline
        \end{tabular}
    \end{center}

    $A_1$ (Umweg über 1):
    \begin{center}
        \begin{tikzpicture}[baseline,anchor=north]
            \node[draw, circle] (1) {$1$};
            \node[draw, circle, right=of 1] (2) {$2$};
            \node[draw, circle, below=of 1] (3) {$3$};
            \node[draw, circle, right=of 3] (4) {$4$};

            \path (1) edge node[weight, above] {2} (2);
            \path (1) edge node[weight, left] {3} (3);
            \path (3) edge node[weight, below] {7} (4);
            \path (2) edge node[weight, right] {1} (4);
            %\path (1) edge[blue, bend left=45] node[weight, above, blue] {2} (2);
            %\path (1) edge[blue, bend right=45] node[weight, left, blue] {3} (3);
            \path (2) edge[dashed, red, bend right] node[weight, above left, pos=1] {5} (1.north west);
            \path (1.north west) edge[dashed, red, bend right] (3);
        \end{tikzpicture}
        %
        \hspace{5em}
        %
        \begin{tabular}[t]{|c|gccc|}
            \hline
              & 1                   & 2                   & 3                   & 4                   \\
            \hline
            \rowcolor{gray!25}
            1 & \textcolor{gray}{0} & 2                   & 3                   & $\infty$            \\
            2 & 2                   & \textcolor{gray}{0} & \textcolor{red}{5}  & 1                   \\
            3 & 3                   & \textcolor{red}{5}  & \textcolor{gray}{0} & 7                   \\
            4 & $\infty$            & 1                   & 7                   & \textcolor{gray}{0} \\
            \hline
        \end{tabular}
        %
        \hspace{5em}
        %
        \begin{tabular}[t]{|c|cccc|}
            \hline
              & 1                   & 2                   & 3                   & 4 \\
            \hline
            1 & \textcolor{gray}{-} & -                   & -                   & - \\
            2 & -                   & \textcolor{gray}{-} & 1                   & - \\
            3 & -                   & 1                   & \textcolor{gray}{-} & - \\
            4 & -                   & -                   & -                   & - \\
            \hline
        \end{tabular}
    \end{center}

    $$A_0[2][3] > A_0[3][1] + A_0[1][2] \implies A_1[2][3] = A_0[3][1] + A_0[1][2] = 3 + 2 = 5$$
    $$A_0[3][2] > A_0[2][1] + A_0[1][3] \implies A_1[3][2] = A_0[2][1] + A_0[1][3] = 2 + 3 = 5$$


    $A_2$ (Umweg über 2):
    \begin{center}
        \begin{tikzpicture}[baseline,anchor=north]
            \node[draw, circle] (1) {$1$};
            \node[draw, circle, right=of 1] (2) {$2$};
            \node[draw, circle, below=of 1] (3) {$3$};
            \node[draw, circle, right=of 3] (4) {$4$};

            \path (1) edge node[weight, above] {2} (2);
            \path (1) edge node[weight, left] {3} (3);
            \path (3) edge node[weight, below] {7} (4);
            \path (2) edge node[weight, right] {1} (4);

            % old edge 2-1-3
            \path (2) edge[dashed, blue, bend right] node[weight, above left, pos=1] {5} (1.north west);
            \path (1.north west) edge[dashed, blue, bend right] (3);
            % new edge 1-2-4
            \path (1) edge[dashed, red, bend left] node[weight, above right, pos=1] {3} (2.north east);
            \path (2.north east) edge[dashed, red, bend left] (4);
            % old edge 3-2-4
            \path (3) edge[dashed, red, bend left] node[weight, below right, pos=0.5] {6} (2.south west);
            \path (2.south west) edge[dashed, red, bend right] (4);
        \end{tikzpicture}
        %
        \hspace{5em}
        %
        \begin{tabular}[t]{|c|cgcc|}
            \hline
              & 1                   & 2                   & 3                   & 4                   \\
            \hline
            1 & \textcolor{gray}{0} & 2                   & 3                   & \textcolor{red}{3}  \\
            \rowcolor{gray!25}
            2 & 2                   & \textcolor{gray}{0} & 5                   & 1                   \\
            3 & 3                   & 5                   & \textcolor{gray}{0} & \textcolor{red}{6}  \\
            4 & \textcolor{red}{3}  & 1                   & \textcolor{red}{6}  & \textcolor{gray}{0} \\
            \hline
        \end{tabular}
        %
        \hspace{5em}
        %
        \begin{tabular}[t]{|c|cccc|}
            \hline
              & 1                   & 2                   & 3                   & 4 \\
            \hline
            1 & \textcolor{gray}{-} & -                   & -                   & 2 \\
            2 & -                   & \textcolor{gray}{-} & 1                   & - \\
            3 & -                   & 1                   & \textcolor{gray}{-} & 2 \\
            4 & 2                   & -                   & 2                   & - \\
            \hline
        \end{tabular}
    \end{center}

    $$A_1[1][4] > A_1[1][2] + A_1[2][4] \implies A_2[2][3] = A_1[1][2] + A_1[2][4] = 2 + 1 = 3$$
    $$A_1[4][1] > A_1[4][2] + A_1[2][1] \implies A_2[3][2] = A_1[4][2] + A_1[2][1] = 1 + 2 = 3$$
    $$A_1[3][4] > A_1[3][2] + A_1[2][4] \implies A_2[3][4] = A_1[3][2] + A_1[2][4] = 5 + 1 = 6$$
    $$A_1[4][3] > A_1[4][2] + A_1[2][3] \implies A_2[4][3] = A_1[4][2] + A_1[2][3] = 1 + 5 = 6$$

    $A_3$ (Umweg über 3):
    \begin{center}
        \begin{tikzpicture}[baseline,anchor=north]
            \node[draw, circle] (1) {$1$};
            \node[draw, circle, right=of 1] (2) {$2$};
            \node[draw, circle, below=of 1] (3) {$3$};
            \node[draw, circle, right=of 3] (4) {$4$};

            \path (1) edge node[weight, above] {2} (2);
            \path (1) edge node[weight, left] {3} (3);
            \path (3) edge node[weight, below] {7} (4);
            \path (2) edge node[weight, right] {1} (4);

            % old edge 2-1-3
            \path (2) edge[dashed, blue, bend right] node[weight, above left, pos=1] {5} (1.north west);
            \path (1.north west) edge[dashed, blue, bend right] (3);
            % old edge 1-2-4
            \path (1) edge[dashed, blue, bend left] node[weight, above right, pos=1] {3} (2.north east);
            \path (2.north east) edge[dashed, blue, bend left] (4);
            % old edge 3-2-4
            \path (3) edge[dashed, blue, bend left] node[weight, below right, pos=0.5] {6} (2.south west);
            \path (2.south west) edge[dashed, blue, bend right] (4);
        \end{tikzpicture}
        %
        \hspace{5em}
        %
        \begin{tabular}[t]{|c|ccgc|}
            \hline
              & 1                   & 2                   & 3                   & 4                   \\
            \hline
            1 & \textcolor{gray}{0} & 2                   & 3                   & 3                   \\
            2 & 2                   & \textcolor{gray}{0} & 5                   & 1                   \\
            \rowcolor{gray!25}
            3 & 3                   & 5                   & \textcolor{gray}{0} & 6                   \\
            4 & 3                   & 1                   & 6                   & \textcolor{gray}{0} \\
            \hline
        \end{tabular}
        %
        \hspace{5em}
        %
        \begin{tabular}[t]{|c|cccc|}
            \hline
              & 1                   & 2                   & 3                   & 4 \\
            \hline
            1 & \textcolor{gray}{-} & -                   & -                   & 2 \\
            2 & -                   & \textcolor{gray}{-} & 1                   & - \\
            3 & -                   & 1                   & \textcolor{gray}{-} & 2 \\
            4 & 2                   & -                   & 2                   & - \\
            \hline
        \end{tabular}
    \end{center}

    $A_4$ (Umweg über 4):
    \begin{center}
        \begin{tikzpicture}[baseline,anchor=north]
            \node[draw, circle] (1) {$1$};
            \node[draw, circle, right=of 1] (2) {$2$};
            \node[draw, circle, below=of 1] (3) {$3$};
            \node[draw, circle, right=of 3] (4) {$4$};

            \path (1) edge node[weight, above] {2} (2);
            \path (1) edge node[weight, left] {3} (3);
            \path (3) edge node[weight, below] {7} (4);
            \path (2) edge node[weight, right] {1} (4);

            % old edge 2-1-3
            \path (2) edge[dashed, blue, bend right] node[weight, above left, pos=1] {5} (1.north west);
            \path (1.north west) edge[dashed, blue, bend right] (3);
            % old edge 1-2-4
            \path (1) edge[dashed, blue, bend left] node[weight, above right, pos=1] {3} (2.north east);
            \path (2.north east) edge[dashed, blue, bend left] (4);
            % old edge 3-2-4
            \path (3) edge[dashed, blue, bend left] node[weight, below right, pos=0.5] {6} (2.south west);
            \path (2.south west) edge[dashed, blue, bend right] (4);
        \end{tikzpicture}
        %
        \hspace{5em}
        %
        \begin{tabular}[t]{|c|cccg|}
            \hline
              & 1                   & 2                   & 3                   & 4                   \\
            \hline
            1 & \textcolor{gray}{0} & 2                   & 3                   & 3                   \\
            2 & 2                   & \textcolor{gray}{0} & 5                   & 1                   \\
            3 & 3                   & 5                   & \textcolor{gray}{0} & 6                   \\
            \rowcolor{gray!25}
            4 & 3                   & 1                   & 6                   & \textcolor{gray}{0} \\
            \hline
        \end{tabular}
        %
        \hspace{5em}
        %
        \begin{tabular}[t]{|c|cccc|}
            \hline
              & 1                   & 2                   & 3                   & 4 \\
            \hline
            1 & \textcolor{gray}{-} & -                   & -                   & 2 \\
            2 & -                   & \textcolor{gray}{-} & 1                   & - \\
            3 & -                   & 1                   & \textcolor{gray}{-} & 2 \\
            4 & 2                   & -                   & 2                   & - \\
            \hline
        \end{tabular}
    \end{center}
\end{example}

\begin{algo}{Warshall-Algorithmus}
    Gegeben: Graph $G = (V, E)$, egal ob gerichtet oder nicht.

    Aufgabenstellung: Prüfe für alle geordneten Paare $(v_i, v_j)$, ob ein Weg von $v_i$ nach $v_j$ existiert.

    Ziel: Berechne Adjazenzmatrix $A$ zu \emph{transitivem Abschluss} von $G$:
    \begin{center}
        $A[i][j] = $ \texttt{true} $\iff$ es existiert ein nicht-trivialer Weg (Länge $>0$) von $v_i$ nach $v_j$
    \end{center}

    Iterationsformel zur Aktualisierung von $A[i][j]$:
    $$
        A_k[i][j] = A_{k-1}[i][j] \lor (A_{k-1}[i][k] \land A_{k-1}[k][j])
    $$

    Der Algorithmus funktioniert also analog zum Floyd-Algorithmus.
\end{algo}

\begin{code}{Warshall-Algorithmus}
    \lstinputlisting{warshall.java}
\end{code}