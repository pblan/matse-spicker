\section{Sockets}

\begin{defi}{Socket}
    \emph{Sockets} bilden eine plattformunabhängige standardisierte Schnittstelle (API) zwischen der Netzwerkprotokoll-Implementierung des Betriebssystems und der eigentlichen Anwendungssoftware.

    Ein Computerprogramm fordert einen Socket vom Betriebssystem an.
    Das Betriebssystem hat die Aufgabe, alle benutzten Sockets sowie die zugehörigen Verbindungsinformationen zu verwalten.
\end{defi}

\begin{defi}{Internet-Socket}
    \emph{Internet-Sockets} ermöglichen Kommunikation mittels bestimmter Kommunikationsprotokolle.

    Generell kann man unterscheiden zwischen:
    \begin{itemize}
        \item \emph{Stream-Sockets}:
              \begin{itemize}
                  \item kommunizieren über einen Zeichen-Datenstrom
                  \item verwenden meist TCP
              \end{itemize}
        \item \emph{Datagramm-Sockets}:
              \begin{itemize}
                  \item kommunizieren über einzelne Nachrichten
                  \item verwenden meist UDP
              \end{itemize}
    \end{itemize}
\end{defi}

\begin{defi}{Socket-Kommunikation (Stream-Socket)}
    Client-seitig:
    \begin{enumerate}
        \item Socket erstellen
        \item erstellten Socket mit einer Server-Adresse verbinden, von welcher Daten angefordert werden sollen
        \item Senden und Empfangen von Daten
        \item evtl. Socket herunterfahren
        \item Verbindung trennen, Socket schließen
    \end{enumerate}

    Server-seitig:
    \begin{enumerate}
        \item Server-Socket erstellen
        \item Binden des Sockets an eine Adresse (Port), über welche Anfragen akzeptiert werden
        \item auf Anfragen warten
        \item Anfrage akzeptieren und damit ein neues Socket-Paar für diesen Client erstellen
        \item Bearbeiten der Client-Anfrage auf dem neuen Client-Socket
        \item Client-Socket wieder schließen
    \end{enumerate}
\end{defi}

\begin{defi}{Socket-Kommunikation (Datagramm-Socket)}
    Client-seitig:
    \begin{enumerate}
        \item Socket erstellen
        \item An Adresse senden
    \end{enumerate}

    Server-seitig:
    \begin{enumerate}
        \item Socket erstellen
        \item Socket binden
        \item warten auf Pakete
    \end{enumerate}
\end{defi}

\begin{example}{Socket (Client)}
    \begin{lstlisting}[language=Java]
        import java.io.*;
        import java.net.*;

        class TCPClient {
            public static void main(String args[]) throws Exception {
                BufferedReader input = new BufferedReader(new InputStreamReader(System.in));

                // Erstelle Client-Socket, baue Verbindung auf
                Socket socket = new Socket("192.168.0.1", 591);

                // Erstelle Datenstrom fuer den Socket
                DataOutputStream sender = new DataOutputStream(
                    socket.getOutputStream()
                );

                // Erstelle Datenstrom aus dem Socket
                BufferedReader receiver = new BufferedReader(
                    new InputStreamReader(socket.getInputStream())
                );

                String msg = input.readLine();

                // Sende an den Server
                sender.writeBytes(msg + '\n');

                // Empfange vom Server
                String modified_msg = receiver.readLine();

                System.out.println("FROM SERVER: " + modified_msg);

                socket.close();
            }
        }
    \end{lstlisting}
\end{example}

\begin{example}{Socket (Server)}
    \begin{lstlisting}[language=Java]
        import java.io.*;
        import java.net.*;

        class TCPServer {
            public static void main(String args[]) throws Exception {
                // Erstelle Socket
                ServerSocket socket = new ServerSocket(591);

                while (true){
                    // Warte auf eingehende Verbindungswuensche
                    Socket connection = socket.accept();

                    // Starte Thread um auf weitere Anfragen reagieren zu koennen.
                    new Thread() {
                        public void run() {
                            // Verknuepfe Buffer mit dem Socket
                            BufferedReader receiver = new BufferedReader(
                                new InputStreamReader(connection.getInputStream())
                            );
        
                            // Verknuepfe Datenstrom mit dem Socket
                            DataOutputStream sender = new DataOutputStream(
                                connection.getOutputStream()
                            );
        
                            // Lesen vom Socket
                            String msg = receiver.readLine();
        
                            // Schreiben auf den Socket
                            sender.writeBytes(msg.toUpperCase() + '\n');
                        }
                    }.start();
                }
            }
        }
    \end{lstlisting}
\end{example}