\section{Unsupervised Learning}

\subsection{Dimensionsreduktion}

\begin{defi}{Dimensionsreduktionstechnik}
    \emph{Dimensionsreduktionstechniken} reduzieren die Anzahl der Features (Merkmalen), die für Supervised oder Unsupervised Learning zur Verfügung stehen.
    \footnote{
        Oft mit dem Ziel, wenig Informationen dabei zu verlieren.
    }

    Ein Risiko besteht darin, dass prädikative Informationen für das Supervised Learning verloren gehen können.

    Chancen sind aber:
    \begin{itemize}
        \item Schnellere Berechnungen durch Weglassen irrelevanter Informationen
        \item Ermöglichung von Visualisierung der Daten (und damit der Gewinnung von Einsichten) im Rahmen einer explorativen Analyse (Data Science)
        \item Ermöglichung besserer Generalisierung von Lernmodellen durch Entgegenwirken der Curse of Dimensionality (Fluch der Dimensionalität)
    \end{itemize}
\end{defi}

\begin{bonus}{Curse of Dimensionality}
    Der \emph{Curse of Dimensionality} im Kontext des Machine Learnings das Phänomen, nach dem das Volumen des Feature-Raumes dramatisch zunimmt mit der Anzahl der Dimensionen, so dass die $N$ Datenpunkte (der Trainingsmenge) nur noch spärlich (sparse) im Raum liegen.

    Typische Folgen sind:
    \begin{itemize}
        \item Overfitting-Tendenz steigt
        \item Rechenzeit steigt
        \item Lernmodelle generalisieren schlecht
    \end{itemize}

    Wenn Daten nur noch spärlich den Feature-Raum ausfüllen, dann wird es schwierig, Modelle an die Daten anzupassen, die den Raum adäquat beschreiben
\end{bonus}

\begin{defi}{Principal Component Analysis}
    Die \emph{Hauptkomponentenzerlegung} (\emph{Principal Component Analysis}, PCA) zählt zu den bekanntesten Dimensionsreduktionsverfahren.

    PCA findet neue Achsen (Komponenten) für die Daten, so dass die Daten bezüglich dieser Achsen eine \emph{möglichst große Varianz} aufweisen.

    Achsen, bezüglich derer die Daten kaum (oder keine) Varianz anweisen, können später weggelassen werden (Dimensionsreduktion).

    Neue Achsen werden durch orthogonale Transformation erzeugt, d.h. Vektorlängen und Winkel bleiben erhalten.
    \footnote{
        Präziser: das innere Produkt bleibt erhalten.
    }

    Sei die \emph{Kovarianzmatrix} $S$ gegeben mit:
    \[
        S = \frac{1}{N} \sum_{n=1}^N (\mathbf{x}_n - \conj{\mathbf{x}_n}) (\mathbf{x}_n - \conj{\mathbf{x}_n})^T
    \]

    Dann gilt:
    \begin{itemize}
        \item Eigenwertzerlegung der Kovarianzmatrix $S$ liefert Eigenwerte $\lambda_j$ und -vektoren $\mathbf{u}_j$.
        \item Wir sortieren nach Größe der Eigenwerte und erhalten die Richtungen der PCA über die zu den Eigenwerten assoziierten Eigenvektoren von S.
    \end{itemize}

    PCA kann als Transformation interpretiert werden, die die Achsen des Koordinatensystems so rotiert, dass die Varianz der Projektion der Daten auf der ersten Achse maximiert wird.

    Der Mittelwert der Daten beeinflusst die PCA-Koordinaten:
    \begin{itemize}
        \item Daten mit Schwerpunkt (Mittelwert), der nicht dem Koordinatenursprung entspricht, haben nach PCA-Transformation in den neuen Koordinaten typischerweise Offsets, die meist nicht gut interpretierbar sind.
        \item PCA-Richtungen ändern sich nicht unter Änderung des Schwerpunkts der Daten. Grund: Kovarianzmatrix ist invariant\footnote{Die Kovarianzmatrix verändert sich nicht.} unter Transformation der Mittelwerte.
        \item Manche Software-Implementierungen nutzen für die PCA nicht die Kovarianzmatrix. Dort kann die PCA von den Mittelwerten der Daten abhängen.
    \end{itemize}

    \emph{Jedes Merkmal sollte auf den Mittelwert $0$ zentriert werden}:
    \[
        \tilde{x}_i = x_i - \conj{x}, \forall i
    \]

    Die Skalierung der Daten beeinflusst PCA:
    \begin{itemize}
        \item Die Skalen, in der ein Merkmal angegeben wird beeinflusst die Varianz dieses Merkmals.
        \item Die Ergebnisse der PCA ist abhängig von den Varianzen der Merkmale.
    \end{itemize}

    \emph{Jedes Merkmal sollte auf eine Varianz von $1$ skaliert werden, damit die Ergebnisse der PCA nicht von der (eventuell) willkürlichen Wahl der Skalen der Merkmale abhängig ist}:
    \[
        \tilde{x}_i = \frac{x_i}{\sigma}, \forall i
    \]
\end{defi}

\begin{bonus}{Standardisierung}
    \emph{Standardisierung} bzw. \emph{z-scoring} beschreibt die Transformation eines Merkmals $X$, so dass es Mittelwert $0$ und Varianz $1$ aufweist.
    Das Merkmal wird also \emph{zentriert} und \emph{skaliert}.

    Sei $\conj{X}$ der Mittelwert und $\sigma$ die Standardabweichung von $X$.

    Das \emph{standardisierte Merkmal} bzw. der \emph{z-score} $Z$ wird erzeugt durch:
    \[
        Z = \frac{X - \conj{X}}{\sigma}
    \]

    In der Praxis liegen sind Daten oft in Matrizen organisiert, in denen jede Spalte einem Merkmal entsprechen.
    In diesem Fall standardisiert man jedes Merkmal, indem jede Spalte separat auf Mittelwert 0 und Varianz 1 transformiert wird.
\end{bonus}

\begin{defi}{Proportion of Variance Explained}
    \emph{Proportion of Variance Explained} (PVE) beschreibt den Bruchteil der Gesamtvarianz der Merkmale, die über die PCA-Komponenten repräsentiert wird.

    Die Varianz der $j$-ten Komponente ist:\footnote{Entspricht also dem jeweiligen Eigenwert.}
    \[
        \Var(Z_j) = \Var(\mathbf{u}_j^T \mathbf{x}) = \mathbf{u}_j^T \mathbf{S} \mathbf{u}_j = \mathbf{u}_j^T \lambda_j \mathbf{u}_j = \lambda_j
    \]

    Die Gesamtvarianz über alle Merkmale ist dann:
    \[
        \Var_\text{total} = \sum_{j=1}^D \Var(Z_j) = \sum_{j=1}^D \lambda_j
    \]

    Dann ist die $PVE(j)$ der $j$-ten PCA-Komponente definiert als:
    \[
        \PVE(j) = \frac{\Var(Z_j)}{\Var_\text{total}} = \frac{\lambda_j}{\sum_{j=1}^D \lambda_j}
    \]
\end{defi}

\begin{defi}{Kernel PCA}
    PCA ist oft nicht hilfreich, wenn Daten nichtlinear strukturiert sind.

    Die Idee von \emph{Kernel PCA} ist, dass ähnlich wie bei Support Vector Machines der Kernel Trick angewandt wird.

    Der Vorgang ist wie folgt:
    \begin{enumerate}
        \item Wählen eines Kernels, z.B. Gaußscher RBF Kernel:
              \[
                  K(\mathbf{x}_i, \mathbf{x}_j) = \exp\left( -\gamma \norm{\mathbf{x}_i - \mathbf{x}_j}^2 \right)
              \]
        \item Bestimmen und zentrieren der Kernelmatrix:
              \[
                  \tilde{K} = K - \mathbf{1}_N K - K \mathbf{1}_N + \mathbf{1}_N K \mathbf{1}_N \quad \text{mit} \quad (K)_{ij} = K(\mathbf{x}_i, \mathbf{x}_j), \, (\mathbf{1}_N)_{ij} = \frac{1}{N}
              \]
        \item Lösen des Eigenwertproblems:
              \[
                  \tilde{K} \vec{\alpha}_j = N \lambda_j \vec{\alpha}_j = \tilde{\lambda}_j \vec{\alpha}_j
              \]
        \item Normieren der Eigenvektoren:
              \[
                  \vec{\alpha}_j \to \tilde{\vec{\alpha}}_j = \frac{1}{\sqrt{\tilde{\lambda}_j}} \frac{\vec{\alpha}_j}{\norm{\vec{\alpha}_j}} \quad \forall j
              \]
        \item Projizieren der Daten auf die neue $j$-te kPCA-Achse:
              \[
                  \Phi(\mathbf{x})^T \tilde{\mathbf{u}}_j = \sum_l \tilde{\alpha}_{jl} K(\mathbf{x}, \mathbf{x}_l)
              \]
    \end{enumerate}

    kPCA erlaubt nichtlineare Dimensionsreduktion!
\end{defi}

% \begin{defi}{Kernel PCA}
%     PCA ist oft nicht hilfreich, wenn Daten nichtlinear strukturiert sind.
%     Die Idee von \emph{Kernel PCA} ist, dass ähnlich wie bei Support Vector Machines der Kernel Trick angewandt wird.

%     Der Vorgang ist wie folgt:
%     \begin{enumerate}
%         \item Umformulieren der PCA mit nichtlinearer Transformation:
%               \begin{itemize}
%                   \item $N$ Datensätze mit $D$ Merkmalen
%                   \item Kovarianzmatrix $S$ wird umformuliert mit $\mathbf{x}_i \to \Phi(\mathbf{x}_i)$. Dann ist $S$:\footnote{Die Differenz mit $\conj{\mathbf{x}}$ bzw. $\conj{\Phi}(\mathbf{x})$ fällt weg, da der Mittelwert $0$ sein wird aufgrund der folgenden Standardisierung.}
%                         \[
%                             S = \frac{1}{N} \sum_{i=1}^N (\Phi(\mathbf{x}_i) - \conj{\Phi}(\mathbf{x})) (\Phi(\mathbf{x}_i) - \conj{\Phi}(\mathbf{x}))^T = \frac{1}{N} \sum_{i=1}^N \Phi(\mathbf{x}_i) \Phi(\mathbf{x}_i)^T
%                         \]
%               \end{itemize}
%         \item Darstellung der Eigenvektoren als Linearkombination der transformierten Features:
%               \begin{itemize}
%                   \item Wir erhalten die Achsen der (klassischen) PCA durch die Eigenvektoren $u_j$ der Kovarianzmatrix. Es gilt:
%                         \[
%                             S \mathbf{u}_j = \lambda_j \mathbf{u}_j \quad \text{und} \quad \mathbf{u}_j = \sum_l \alpha_{jl} \Phi(\mathbf{x}_l)
%                         \]
%                   \item[$\implies$] Das Auffinden der PCA-Richtungen entspricht dem Bestimmen der Koeffizienten.
%               \end{itemize}
%         \item Finden der Eigenvektoren:
%               \begin{itemize}
%                   \item Es gilt ($\tilde{\lambda}_j := N \lambda_j$):
%                         \[
%                             K \vec{\alpha}_j = N \lambda_j \vec{\alpha}_j = \tilde{\lambda}_j \vec{\alpha}_j
%                         \]
%                   \item Die Koeffizienten $\alpha_{jl}$ entsprechen also den Komponenten der Eigenvektoren der Kernel-Matrix und sind dadurch bestimmbar.
%               \end{itemize}
%         \item Normierung der Eigenvektoren:
%               \begin{itemize}
%                   \item Aus der Normierung der Eigenvektoren (bei der klassischen PCA) ergibt sich eine Normierungsbedingung für die Koeffizienten $\alpha_{jl}$:
%                         \[
%                             \vec{\alpha}_j \to \tilde{\vec{\alpha}}_j = \frac{1}{\sqrt{\tilde{\lambda}_j}} \frac{\vec{\alpha}_j}{\norm{\vec{\alpha}_j}} \quad \forall j
%                         \]
%               \end{itemize}
%         \item Projektion auf kPCA Achsen:
%               \begin{itemize}
%                   \item Wir betrachten die Projektion eines beliebigen Featurevektors $\mathbf{x}$ aus dem Featureraum auf die $j$-te Achse der Kernel PCA (kPCA):
%                         \[
%                             \Phi(\mathbf{x})^T \tilde{\mathbf{u}}_j = \sum_l \tilde{\alpha}_{jl} K(\mathbf{x}, \mathbf{x}_l)
%                         \]
%               \end{itemize}
%         \item Zentrieren der Features:
%               \begin{itemize}
%                   \item Normieren der Features:
%                         \[
%                             \tilde{\Phi}(\mathbf{x}_i) = \Phi(\mathbf{x}_i) - \frac{1}{N} \sum_l \Phi(\mathbf{x}_l)
%                         \]
%                   \item Zentrieren der Kernelmatrix (mit $\mathbf{1}_N$ $N \times N$ Matrix, in der jeder Eintrag $\nicefrac{1}{N}$ ist):
%                         \[
%                             \tilde{K} = K - \mathbf{1}_N K - K \mathbf{1}_N + \mathbf{1}_N K \mathbf{1}_N
%                         \]
%               \end{itemize}
%     \end{enumerate}
% \end{defi}

% \begin{defi}{Linear Discriminant Analysis}
%     Gegeben seien:
%     \begin{itemize}
%         \item Daten $(\mathbf{x}_i, y_i)$ ($i=1,\ldots,N$), wobei $(\mathbf{x}_i)_j$ das $j$-te Feature des $i$-ten Datenpunktes ist.
%         \item Zwei Klassen $\mathcal{C}_1$ und $\mathcal{C}_2$ (Klassifikationsproblem)
%         \item $N_k$ Anzahl der Datenpunkte in Klasse $K$
%     \end{itemize}

%     Ziel der \emph{Linear Discriminant Analysis} (LDA) ist es, eine Richtung $\mathbf{u}_1$ zu finden, die die Klassen $C_1$ und $C_2$ gut trennt.

%     Sei $\mathbf{u}_1$ ein Richtungsvektor.
%     Die Projektion $\tilde{\mathbf{x}}_i$ eines Datenpunktes $\mathbf{x}_i$ auf $\mathbf{u}_1$ ist:
%     \[
%         \tilde{\mathbf{x}}_i = \mathbf{u}_1^T \mathbf{x}_i
%     \]

%     Sei $\mathbf{m}^{(k)}$ der mittlere Vektor (Mittelpunkt) aller Datenpunkte der Klasse $k$:
%     \[
%         \mathbf{m}^{(k)} = \frac{1}{N_k} \sum_{n \in \mathcal{C}_k}^N \mathbf{x}_n \quad \text{mit} \quad \tilde{\mathbf{m}}^{(k)} = \mathbf{u}_1^T \tilde{\mathbf{m}}^{(k)}
%     \]

%     Der Abstand der Klassenmittelpunkte ist dann:
%     \[
%         \Delta \tilde{\mathbf{m}} = \tilde{\mathbf{m}}^{(1)} - \tilde{\mathbf{m}}^{(2)}
%     \]

%     Gesucht werden soll dann die Richtung $\mathbf{u}_1$, die den Abstand zwischen beiden projizierten Mittelpunkten maximiert.
%     Wichtig ist aber auch eine Quantifikation der \emph{Streuung} (\emph{scatter}).

%     Die Streuung (scatter) der Punkte der $k$-ten Klasse um ihren Mittelpunkt herum ist:
%     \[
%         s_k^2 = \sum_{n \in \mathcal{C}_k} \left( \tilde{\mathbf{x}}_n - \tilde{\mathbf{m}}^{(k)} \right)^2 = \mathbf{u}_1^T \underbrace{\left( \sum_{n \in \mathcal{C}_k} (\mathbf{x}_n - \mathbf{m}^{(k)}) (\mathbf{x}_n - \mathbf{m}^{(k)})^T \right) }_{S_{wk}} \mathbf{u}_1^T = \mathbf{u}_1^T S_{wk} \mathbf{u}_1^T
%     \]

%     $S_{wk}$ bezeichnet die \emph{within-scatter Matrix der Klasse} $k$.

%     Die Summe der Streuungen beider Klassen, projiziert auf $u_1$ ist dann:
%     \[
%         (\tilde{\mathbf{m}}^{(1)} - \tilde{\mathbf{m}}^{(2)})^2 = \mathbf{u}_1^T \underbrace{(\mathbf{m}^{(1)} - \mathbf{m}^{(2)}) (\mathbf{m}^{(1)} - \mathbf{m}^{(2)})^T}_{S_b} \mathbf{u}_1^T = \mathbf{u}_1^T S_b \mathbf{u}_1^T
%     \]

%     $S_{b}$ bezeichnet die \emph{between-scatter Matrix der Daten}.


% \end{defi}

\begin{defi}{Linear Discriminant Analysis}
    Gegeben seien:
    \begin{itemize}
        \item Daten $(\mathbf{x}_i, y_i)$ ($i=1,\ldots,N$), wobei $(\mathbf{x}_i)_j$ das $j$-te Feature des $i$-ten Datenpunktes ist.
        \item Zwei Klassen $\mathcal{C}_1$ und $\mathcal{C}_2$ (Klassifikationsproblem)
        \item $N_k$ Anzahl der Datenpunkte in Klasse $K$
    \end{itemize}

    Ziel der \emph{Linear Discriminant Analysis} (LDA) ist es, eine Richtung $\mathbf{u}_1$ zu finden, die die Klassen $C_1$ und $C_2$ gut trennt.

    Sei $\mathbf{u}_1$ ein Richtungsvektor.
    Die Projektion $\tilde{\mathbf{x}}_i$ eines Datenpunktes $\mathbf{x}_i$ auf $\mathbf{u}_1$ ist:
    \[
        \tilde{\mathbf{x}}_i = \mathbf{u}_1^T \mathbf{x}_i
    \]

    Sei $\mathbf{m}^{(k)}$ der mittlere Vektor (Mittelpunkt) aller Datenpunkte der Klasse $k$:
    \[
        \mathbf{m}^{(k)} = \frac{1}{N_k} \sum_{n \in \mathcal{C}_k}^N \mathbf{x}_n \quad \text{mit} \quad \tilde{\mathbf{m}}^{(k)} = \mathbf{u}_1^T \tilde{\mathbf{m}}^{(k)}
    \]

    Das Vorgehen ist wie folgt:
    \begin{enumerate}
        \item Berechnung der \emph{between-scatter Matrix} $S_{b}$ zur Quantifikation des \emph{Abstandes zwischen Mittelpunkten} der Klassen:
              \[
                  (\tilde{\mathbf{m}}^{(1)} - \tilde{\mathbf{m}}^{(2)})^2 = \mathbf{u}_1^T \underbrace{(\mathbf{m}^{(1)} - \mathbf{m}^{(2)}) (\mathbf{m}^{(1)} - \mathbf{m}^{(2)})^T}_{S_b} \mathbf{u}_1 = \mathbf{u}_1^T S_b \mathbf{u}_1
              \]
        \item Berechnung der \emph{within-scatter Matrix} $S_{w}$ zur Quantifikation der \emph{Streuung} der Klassen:
              \[
                  s_k^2 = \sum_{n \in \mathcal{C}_k} \left( \tilde{\mathbf{x}}_n - \tilde{\mathbf{m}}^{(k)} \right)^2 = \mathbf{u}_1^T \underbrace{\left( \sum_{n \in \mathcal{C}_k} (\mathbf{x}_n - \mathbf{m}^{(k)}) (\mathbf{x}_n - \mathbf{m}^{(k)})^T \right) }_{S_{wk}} \mathbf{u}_1 = \mathbf{u}_1^T S_{wk} \mathbf{u}_1
              \]
              \[
                  s_1^2 + s_2^2 = \mathbf{u}_1^T S_{w1} \mathbf{u}_1 + \mathbf{u}_1^T S_{w2} \mathbf{u}_1 = \mathbf{u}_1^T \underbrace{(S_{w1} + S_{w2})}_{S_w} \mathbf{u}_1  = \mathbf{u}_1^T S_w \mathbf{u}_1
              \]
        \item Eigenwertzerlegung der Matrix $(S_w^{-1}S_b)$:
              \[
                  \underbrace{(S_w^{-1}S_b)}_{\text{Matrix}} \mathbf{u}_1 = \underbrace{J(\mathbf{u}_1)}_{\text{Skalar}} \mathbf{u}_1
              \]
        \item Eigenvektoren sind dann die Richtungen der LDA.
    \end{enumerate}

    \begin{itemize}
        \item LDA nutzt gelabelte Daten für Dimensionsreduktion.
        \item Streuung innerhalb der Klassen wird minimiert
        \item Abstände zwischen Klassen wird maximiert
        \item Verallgemeinerungen auf Multiklassenprobleme wurden entwickelt und sind in gängigen Softwarelösungen implementiert.
    \end{itemize}
\end{defi}

\begin{bonus}{Fisher-Diskriminante}
    Die \emph{Fisher-Diskriminante} ist definiert als:
    \[
        J(\mathbf{u}_1) := \frac{(\tilde{\mathbf{m}}^{(1)} - \tilde{\mathbf{m}}^{(2)})^2}{s_1^2 + s_2^2} = \frac{\mathbf{u}_1^T S_b \mathbf{u}_1}{\mathbf{u}_1^T S_w \mathbf{u}_1}
    \]

    Sie quantifiziert also die \emph{Trennbarkeit} zwischen den Klassen durch den Abstand zwischen den Mittelpunkten der Klassen $S_b$ normiert über die Streuung innerhalb der Klassen $S_w$.

    Maximieren von $J(\mathbf{u}_1)$ findet die Richtung $\mathbf{u}_1$ mit möglichst großen Abständen zwischen den Mittelwerten und kleiner Streuung innehalb der Klassen:
    \[
        \mathbf{u}_1 = \arg\max_{\mathbf{u}_1} J(\mathbf{u}_1)
    \]

    Es stellt sich heraus, dass der Richtungsvektor $\mathbf{u}_1$, der $J(\mathbf{u}_1)$ maximiert, der Eigenvektor zum größten Eigenwert der Matrix $(S_w^{-1}S_b)$ ist.
\end{bonus}

\subsection{Clusteringverfahren}

\begin{defi}{Cluster-Analysis}
    Die \emph{Cluster-Analysis} zählt zur Verfahrensklasse der Unsupervised Learning Verfahren.
    Sie findet \emph{Gruppen} (Cluster) ähnlicher Datenpunkte, ohne dass Labels vorliegen müssen.

    \emph{Clustering} ist also Partitionierung (Einteilung) der Datenpunkte in Gruppen.

    Viele Cluster-Analyseverfahren existieren und unterscheiden sich:
    \begin{itemize}
        \item in der Definition von Ähnlichkeit,
        \item in der Definition von Gruppen (Clustern),
        \item im algorithmischen Vorgehen.
    \end{itemize}

    Es gilt:
    \begin{itemize}
        \item Cluster-Analyseverfahren finden immer Cluster in den Daten.
        \item Verschiedene Verfahren liefern verschiedene Cluster für dieselben Daten.
        \item Cluster-Verfahren interpretieren die Daten hinsichtlich der den Verfahren (teils implizit) zugrundeliegenden Annahmen.
        \item Wie Clusterings bewertet werden können (also z.B. als sinnvoll/nicht sinnvoll eingestuft werden können), ist aktuelles und im Allgemeinen ungelöstes Forschungsproblem.
    \end{itemize}
\end{defi}

\begin{defi}{K-Means Clustering}
    Seien $C_1, \ldots, C_K$ die Mengen der Indizes der Datenpunkte jedes Clusters.

    Die Mengen sollen zwei Eigenschaften erfüllen:
    \begin{enumerate}
        \item Jeder Datenpunkt gehört zu mindestens einem Cluster:
              \[
                  C_1 \cup C_2 \cup \ldots \cup C_K = \{ 1, \ldots, N \}
              \]
        \item Die Cluster überlappen nicht.
              Jeder Datenpunkt gehört nicht mals als einem Cluster an.
              \[
                  C_k \cap C_{k'} = \emptyset, \, \forall k \neq k'
              \]
    \end{enumerate}

    Wenn der $i$-te Datenpunkt zum $k$-ten Cluster gehört, dann gilt: $i \in C_k$.

    Die Hauptidee hinter \emph{K-Means Clustering} ist, dass das Clustering gut ist, wenn die \emph{Intra-Cluster-Variation} minimiert wird.

    Bezeichne $W(C_k)$ die Intra-Cluster-Variation des $k$-ten Clusters.
    Sie ist definiert als Summe der quadratischen euklidischen Distanzen zwischen allen Datenpunkten eines Clusters:
    \[
        W(C_k) = \frac{1}{\abs{C_k}} \sum_{i, i' \in C_k} \sum_{j=1}^p (x_{ij} - x'_{i'j})^2
    \]

    Es ist die Summe der Intra-Cluster-Variationen zu minimieren:
    \[
        \min_{C_1, \ldots, C_k} \, \left\{ \sum_{k=1}^K W(C)_k \right\} = \min_{C_1, \ldots, C_k} \, \left\{ \sum_{k=1}^K \frac{1}{\abs{C_k}} \sum_{i, i' \in C_k} \sum_{j=1}^p (x_{ij} - x'_{i'j})^2 \right\}
    \]

    Das Vorgehen ist also wie folgt:
    \begin{enumerate}
        \item Weise allen Datenpunkten zufällige Clusterzugehörigkeiten $1, \ldots, K$ zu.
        \item Iteriere, bis die Cluster sich nicht mehr ändern:
              \begin{enumerate}
                  \item Für jeden der K Cluster, berechne den \emph{geometrischen Schwerpunkt} (\emph{centroid}).

                        Der Schwerpunkt ist der Featurevektor, der sich aus der Mittelung aller Featurevektoren der dem Cluster zugehörigen Datenpunkte ergibt.
                  \item Weise jedem Datenpunkt dem Cluster zu, dessen Schwerpunkt dem Datenpunkt am nächsten liegt (im euklidischem Sinne).
              \end{enumerate}
    \end{enumerate}

    Es gilt:
    \begin{itemize}
        \item K-Means Clustering hängt ab von der zufälligen Anfangsinitialisierung der Clusterzugehörigkeiten.

              Der 1. Schritt findet lokale Minima unserer Optimierungsfunktion, nicht notwendigerweise das globale Minimum.
        \item Clusteranzahl K muss zu Beginn festgelegt werden.
              Verschiedene Werte für $K$ können ganz unterschiedliche Cluster liefern.
        \item Keine Möglichkeit, Ausreißer in den Daten zu erkennen und getrennt zu behandeln.
    \end{itemize}

    Eine Empfehlung ist, K-Means mehrfach mit zufälligen initialen Clusterzugehörigkeiten zu versuchen und dann das Ergebnis mit dem kleinsten Wert für die summierte Intra-Cluster-Variation zu wählen.
\end{defi}