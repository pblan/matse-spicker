\subsection{Sichten}

\begin{defi}{Sicht}
    Eine \emph{Sicht} ist eine logische Relation in einem Datenbanksystem.

    Diese logische Relation wird über eine im DBMS gespeicherte Abfrage definiert.
    Der Benutzende kann eine Sicht wie eine normale Tabelle abfragen.

    Wann immer eine Abfrage diese Sicht benutzt, wird diese zuvor durch das DBMS berechnet.

    Eine Sicht stellt im Wesentlichen einen Alias für eine Abfrage dar.

    Man kann unterscheiden zwischen:
    \begin{itemize}
        \item \emph{temporären Sichten}:
              \begin{itemize}
                  \item im Kontext von \texttt{WITH}-Definitionen
                  \item überdauern die Anfrage nicht
              \end{itemize}
        \item \emph{statischen Sichten}:
              \begin{itemize}
                  \item Datenschutz
                  \item Vereinfachung von Aufgaben
                  \item Darstellung der Generalisierung
              \end{itemize}
    \end{itemize}

    Sichten arbeiten auf den Daten der logischen Gesamtsicht.
    Somit sind entsprechende Transformationsregeln notwendig.

    Änderungen einer Sicht müssen immer noch auf die Basisrelationen abgebildet werden.
    Dies ist je nach Sichtendefinition nicht immer möglich.

    Auch können Änderungen von Tupeln ein \enquote{Löschen} des Tupels aus der entsprechenden Sicht bewirken.

    Damit eine Sicht änderbar ist, muss sie durch eine einzelne \texttt{SELECT}-Anweisung definiert sein.\footnote{d.h. kein \texttt{JOIN}, \texttt{UNION}, etc.}

    Weiterhin gilt für Änderbarkeit:\footnote{Mehr dazu: \href{https://wikis.gm.fh-koeln.de/Datenbanken/Aenderbare-Sicht}{edb-Online-Lexikon, TH Köln}}
    \begin{itemize}
        \item Die \texttt{SELECT}-Klausel darf nur Attributnamen enthalten und jeden Namen nur einmal, keine Aggregatfunktionen, berechnete oder konstante Ausdrücke, ebenso kein \texttt{DISTINCT}.
        \item Die \texttt{FROM}-Klausel darf nur einen einzigen Relationsnamen enthalten.
              Dieser muss eine Basisrelation oder eine änderbare Sicht bezeichnen.
        \item Falls die \texttt{WHERE}-Klausel geschachtelte Anfragen beinhaltet, darf in deren \texttt{FROM}- Klauseln der Relationsname auch \texttt{FROM} nicht auftauchen, d.h. keine korrelierten Unterabfragen.
    \end{itemize}
\end{defi}

\begin{sql}{VIEW}
    \rowcolors{2}{gray!15}{white}
    Eine \texttt{VIEW} erzeugt aus einem \texttt{SELECT} eine temporäre Tabelle.

    Dementsprechend kann man wie folgt eine temporäre Tabelle aller Feuer-Pokémon erstellen\footnote{Das Schlüsselwort \texttt{OR REPLACE} überschreibt eine eventuell existierende Tabelle}, um bei den folgenden Abfragen Laufzeit zu sparen.\footnote{Wenn der innere \texttt{SELECT} z.B. mit einem oder mehreren \texttt{JOIN}-Befehlen kombiniert wird oder die Bedingung in der Selektion eine gewisse Komplexität überschreitet und die selbe Abfrage mehrfach benötigt wird}

    \begin{lstlisting}[language=mysql]
        -- Erstelle View feuerpokemon
        CREATE OR REPLACE VIEW feuerpokemon
        AS (
            SELECT *
            FROM pokemon
            WHERE
                PrimaerTyp = 'Feuer'
                OR SekundaerTyp = 'Feuer'
        );
        -- Gebe Entitaeten der View feuerpokemon aus
        SELECT * FROM feuerpokemon;
    \end{lstlisting}

    \begin{tabular}{I|ITIIITT}
        \rowcolor{gray!35}
                                   & \multicolumn{1}{T}{ID}    & \multicolumn{1}{T}{Name} & \multicolumn{1}{T}{Groesse} & \multicolumn{1}{T}{Gewicht} & \multicolumn{1}{T}{Generation} & \multicolumn{1}{T}{PrimaerTyp} & \multicolumn{1}{T}{SekundaerTyp} \\\hline
        1                          & 4                         & Glumanda                 & 0.6                         & 8.5                         & 1                              & Feuer                          & NULL                             \\
        2                          & 5                         & Glutexo                  & 1.1                         & 19                          & 1                              & Feuer                          & NULL                             \\
        3                          & 6                         & Glurak                   & 1.7                         & 90.5                        & 1                              & Feuer                          & Flug                             \\
        \multicolumn{1}{c|}{\dots} & \multicolumn{7}{c}{\dots}                                                                                                                                                                                             \\
        71                         & 851                       & Infernopod               & 3                           & 120                         & 8                              & Feuer                          & Käfer                            \\
    \end{tabular}
    \vspace{1em}

    Um die View zu entfernen, nutzt man \texttt{DROP}:

    \begin{lstlisting}[language=mysql]
        DROP VIEW feuerpokemon;
    \end{lstlisting}
\end{sql}