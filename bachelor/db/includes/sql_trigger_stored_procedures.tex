\subsection{Trigger, Stored Procedures}

\begin{sql}{DELIMITER}
    Da wir in \texttt{PROCEDURE}s einen \texttt{DELIMITER} nutzen, welcher von dem DBMS fälschlicherweise als Ende für den Befehl zum Erstellen der \texttt{PROCEDURE} erkannt wird, können wir diesen ändern.

    \begin{lstlisting}[language=mysql]
        DELIMITER <Trennzeichen>
    \end{lstlisting}
\end{sql}

\begin{sql}{PROCEDURE}
    \rowcolors{2}{gray!15}{white}
    Eine \texttt{PROCEDURE} wird genutzt um SQL-Befehle methodenähnlich zum Auszuführen zu speichern.

    Diese \texttt{PROCEDURE}s sind vergleichbar mit \texttt{VIEW}s.
    Jedoch wird bei einer \texttt{VIEW} ein Snapshot der aktuellen Tabelle erstellt und bei \texttt{PROCEDURE} jedes mal erneut die Datenbank aufgerufen.

    Sollte man nun eine \texttt{PROCEDURE} zum Anzeigen aller Feuer-Pokémon erstellen wollen, nutzt man:

    \begin{lstlisting}[language=mysql]
        -- Aendern des Trennzeichens
        DELIMITER //
        -- Procedure anlegen
        CREATE PROCEDURE ShowFeuerpokemon()
        BEGIN
            SELECT *
            FROM pokemon
            WHERE
                PrimaerTyp = 'Feuer'
                OR SekundaerTyp = 'Feuer';
        END //
        -- Aendern des Trennzeichens rueckgaengig machen
        DELIMITER ;
    \end{lstlisting}

    Um diese \texttt{PROCEDURE} nun aufzurufen nutzen wir:

    \begin{lstlisting}[language=mysql]
        CALL ShowFeuerpokemon();
    \end{lstlisting}

    \setcounter{rownum}{0}
    \begin{tabular}{I|TTTTTTT}
        \rowcolor{gray!35}
                                   & \multicolumn{1}{T}{ID}    & \multicolumn{1}{T}{Name} & \multicolumn{1}{T}{Groesse} & \multicolumn{1}{T}{Gewicht} & \multicolumn{1}{T}{Generation} & \multicolumn{1}{T}{PrimaerTyp} & \multicolumn{1}{T}{SekundaerTyp} \\\hline
        1                          & 4                         & Glumanda                 & 0.6                         & 8.5                         & 1                              & Feuer                          & NULL                             \\
        2                          & 5                         & Glutexo                  & 1.1                         & 19                          & 1                              & Feuer                          & NULL                             \\
        3                          & 6                         & Glurak                   & 1.7                         & 90.5                        & 1                              & Feuer                          & Flug                             \\
        \multicolumn{1}{c|}{\dots} & \multicolumn{7}{c}{\dots}                                                                                                                                                                                             \\
        71                         & 851                       & Infernopod               & 3                           & 120                         & 8                              & Feuer                          & Kaefer                           \\
    \end{tabular}
    \vspace{1em}

    Zum Löschen einer \texttt{PROCEDURE} nutzen wir:

    \begin{lstlisting}[language=mysql]
        DROP PROCEDURE ShowFeuerpokemon;
    \end{lstlisting}
\end{sql}

\begin{sql}{PROCEDURE (mit Parameter)}
    \rowcolors{2}{gray!15}{white}
    Alternativ kann man Parameter mitgeben:

    \begin{lstlisting}[language=mysql]
        DELIMITER //
        CREATE PROCEDURE ShowPokemonFromType(typ varchar(255))
        BEGIN
            SELECT *
            FROM pokemon
            WHERE
                PrimaerTyp = typ
                OR SekundaerTyp = typ;
        END //
        DELIMITER ;
    \end{lstlisting}

    Wenn man erneut alle Feuer-Pokémon ausgeben möchte, erhält man äquivalentes Ergebnis wie folgt:

    \begin{lstlisting}[language=mysql]
        CALL ShowPokemonFromType('Feuer');
    \end{lstlisting}

    Jedoch können wir uns ebenfalls jeden anderen Typen ausgeben lassen:

    \begin{lstlisting}[language=mysql]
        CALL ShowPokemonFromType('Pflanze');
    \end{lstlisting}

    \begin{tabular}{I|ITIIITT}
        \rowcolor{gray!35}
                                   & \multicolumn{1}{T}{ID}    & \multicolumn{1}{T}{Name} & \multicolumn{1}{T}{Groesse} & \multicolumn{1}{T}{Gewicht} & \multicolumn{1}{T}{Generation} & \multicolumn{1}{T}{PrimaerTyp} & \multicolumn{1}{T}{SekundaerTyp} \\\hline
        1                          & 1                         & Bisasam                  & 0.7                         & 6.9                         & 1                              & Pflanze                        & Gift                             \\
        2                          & 2                         & Bisaknosp                & 1                           & 13                          & 1                              & Pflanze                        & Gift                             \\
        3                          & 3                         & Bisaflor                 & 2                           & 100                         & 1                              & Pflanze                        & Gift                             \\
        \multicolumn{1}{c|}{\dots} & \multicolumn{7}{c}{\dots}                                                                                                                                                                                             \\
        107                        & 898                       & Coronospa                & 1.1                         & 7.7                         & 8                              & Psycho                         & Pflanze                          \\
    \end{tabular}
    \vspace{1em}

    \begin{lstlisting}[language=mysql]
        CALL ShowPokemonFromType('Wasser');
    \end{lstlisting}

    \setcounter{rownum}{0}
    \begin{tabular}{I|TTTTTTT}
        \rowcolor{gray!35}
                                   & \multicolumn{1}{T}{ID}    & \multicolumn{1}{T}{Name} & \multicolumn{1}{T}{Groesse} & \multicolumn{1}{T}{Gewicht} & \multicolumn{1}{T}{Generation} & \multicolumn{1}{T}{PrimaerTyp} & \multicolumn{1}{T}{SekundaerTyp} \\\hline
        1                          & 7                         & Schiggy                  & 0.5                         & 9                           & 1                              & Wasser                         & NULL                             \\
        2                          & 8                         & Schillok                 & 1                           & 22.5                        & 1                              & Wasser                         & NULL                             \\
        3                          & 9                         & Turtok                   & 1.6                         & 85.5                        & 1                              & Wasser                         & NULL                             \\
        \multicolumn{1}{c|}{\dots} & \multicolumn{7}{c}{\dots}                                                                                                                                                                                             \\
        141                        & 883                       & Pescryodon               & 2                           & 175                         & 8                              & Wasser                         & Eis                              \\
    \end{tabular}
\end{sql}

\begin{sql}{TRIGGER}
    Um auf Eingaben zu reagieren kann man \texttt{TRIGGER} verwenden.

    Diese lösen automatisch aus, sollte ein bestimmtes Event stattfinden.

    \begin{lstlisting}[language=mysql]
        DELIMITER $$
        CREATE TRIGGER <Triggername>
        <Triggerzeitpunkt> <Triggerevent>
        ON <Tabellenname>
        FOR EACH ROW BEGIN
            <TODO>
        END $$
        DELIMITER ;
    \end{lstlisting}

    Zulässige Werte sind dabei:

    \begin{itemize}
        \item Triggerzeitpunkt

              \begin{itemize}
                  \item \texttt{BEFORE}
                  \item \texttt{AFTER}
              \end{itemize}
        \item Triggerevent

              \begin{itemize}
                  \item \texttt{INSERT}
                  \item \texttt{UPDATE}
                  \item \texttt{DELETE}
              \end{itemize}
        \item In der Aufgabe verwendbare Keywords

              \begin{itemize}
                  \item \texttt{OLD}

                        \begin{itemize}
                            \item die zu löschende Entität oder
                            \item die zu verändernde Entität vor der Änderung
                        \end{itemize}
                  \item \texttt{NEW}

                        \begin{itemize}
                            \item die einzufügende Entität oder
                            \item die zu verändernde Entität nach der Änderung
                        \end{itemize}
              \end{itemize}
    \end{itemize}

    Sollte man beispielsweise beim Einfügen von neuen Typen die Effektivitätsliste direkt anpassen wollen, indem man den neuen Typen mit allen existierenden kreuzt, so nutzt man:

    \begin{lstlisting}[language=mysql]
        DELIMITER $$
        CREATE TRIGGER after_insert_typ
        BEFORE INSERT
        ON typ
        FOR EACH ROW BEGIN
            INSERT INTO effektivitaet (Angreifend, Verteidigend)
                SELECT NEW.Bezeichnung, typ.Bezeichnung FROM typ;
            INSERT INTO effektivitaet (Angreifend, Verteidigend)
                SELECT typ.Bezeichnung, NEW.Bezeichnung FROM typ;
        END $$
        DELIMITER ;
    \end{lstlisting}
\end{sql}