\subsection{Transaktionen}

\begin{defi}{ACID}
    Der Begriff \emph{ACID} beschreibt Regeln und Eigenschaften zur Durchführung von Transaktionen in Datenbankmanagementsystemen.

    Dabei steht ACID für:
    \begin{itemize}
        \item \emph{Atomicity} (Atomarität)
              \begin{itemize}
                  \item atomar in dem Sinne, dass entweder alle Operationen der Transaktion bedingten Änderungen in der Datenbank umgesetzt werden, oder gar keine
              \end{itemize}
        \item \emph{Consistency} (Konsistenz)
              \begin{itemize}
                  \item Ausgangspunkt ist konsistente Datenbank vor Beginn einer Transaktion
                  \item am Ende einer Transaktion ist Datenbank in konsistentem Zustand, d.h.
                        \begin{itemize}
                            \item entweder jede Integritätsregel ist erfüllt\footnote{z.B. referentielle Integrität (Fremdschlüssel)}
                            \item Transaktion wird komplett zurückgesetzt und Datenbank befindet sich wieder in dem (konsistenten) Zustand wie zu Beginn
                        \end{itemize}
              \end{itemize}
        \item \emph{Isolation} (Nebenläufigkeit)
              \begin{itemize}
                  \item parallele bzw. gleichzeitig ausgeführte Transaktionen beeinflussen sich nicht\footnote{Fehlerfälle sind z.B. \enquote{Lost Updates} oder \enquote{Dirty Reads}}
                  \item Locking (Lese- bzw. Schreibsperre)
              \end{itemize}
        \item \emph{Durability)} (Dauerhaftigkeit)
              \begin{itemize}
                  \item nach Abschluss einer Transaktion haben ausgeführte Änderungen dauerhaft Bestand
              \end{itemize}
    \end{itemize}

    ACID wird häufig zur Charakterisierung eines DBMS genannt, insbesondere auch für Nicht-Relationale DBMS.
\end{defi}

\subsubsection{Locking}

\begin{defi}{Lost Update}
    Zwei Transaktionen modifizieren parallel denselben Datensatz und nach Ablauf dieser beiden Transaktionen wird nur die Änderung von einer von ihnen übernommen.
\end{defi}

\begin{example}{Lost Update}
    Wir haben zwei nebenläufige Transaktionen, die denselben Datensatz modifizieren:
    \begin{itemize}
        \item $T_1$: Buche Betrag $300$ von Konto $A$ auf Konto $B$
        \item $T_2$: Schreibe Konto $A$ $3\%$ Zinsen zu
    \end{itemize}

    \begin{center}
        \begin{tabular}{|I|T|T|}
            \hline
               & $T_1$         & $T_2$          \\
            \hline
            01 & a1 = read(A)  &                \\
            02 & a1 = a1 - 300 &                \\
            03 &               & a2 = read(A)   \\
            04 &               & a2 = a2 * 1.03 \\
            05 &               & write(A, a2)   \\
            06 & write(A, a1)  &                \\
            07 & b1 = read(B)  &                \\
            08 & b1 = b1 + 300 &                \\
            09 & write(B, b1)  &                \\
            \hline
        \end{tabular}
    \end{center}

    Die Änderung von $T_2$ in Zeile 5 geht verloren, da $T_1$ im nächsten Schritt die Änderung überschreibt.
\end{example}

\begin{defi}{Dirty Read}
    Daten einer noch nicht abgeschlossenen Transaktion werden von einer anderen Transaktion gelesen.
\end{defi}

\begin{example}{Dirty Read}
    Wir haben zwei nebenläufige Transaktionen, die denselben Datensatz modifizieren:
    \begin{itemize}
        \item $T_1$: Buche Betrag $300$ von Konto $A$ auf Konto $B$
        \item $T_2$: Schreibe Konto $A$ $3\%$ Zinsen zu
    \end{itemize}

    \begin{center}
        \begin{tabular}{|I|T|T|}
            \hline
               & $T_1$         & $T_2$          \\
            \hline
            01 & a1 = read(A)  &                \\
            02 & a1 = a1 - 300 &                \\
            03 & write(A, a1)  &                \\
            04 &               & a2 = read(A)   \\
            05 &               & a2 = a2 * 1.03 \\
            06 &               & write(A, a2)   \\
            07 & b1 = read(B)  &                \\
            08 & b1 = b1 + 300 &                \\
            09 & write(B, b1)  &                \\
            10 & ...           &                \\
            n  & abort         &                \\
            \hline
        \end{tabular}
    \end{center}

    Sollte $T_1$ abbrechen (\texttt{abort}), oder eine andere Änderung an $A$ vornehmen, war die Sicht auf $A$ in Schritt 3 für $T_2$ falsch.
\end{example}

\begin{defi}{Locking}
    Unter \emph{Locking} versteht man das Sperren des Zugriffs auf ein Betriebsmittel.

    Eine solche Sperre ermöglicht den exklusiven Zugriff eines Prozesses auf eine Ressource, d. h. mit der Garantie, dass kein anderer Prozess diese Ressource liest oder verändert, solange die Sperre besteht.

    Regeln beim Locking:
    \begin{itemize}
        \item Jedes Objekt muss vor der Benutzung gesperrt werden.
        \item Eine Transaktion fordert eine Sperre, die sie schon besitzt, nicht erneut an.
        \item Eine Transaktion respektiert vorhandene Sperren und wird ggf. in eine Warteschlange eingereiht.
        \item Jede Transaktion durchläuft zwei Phasen:
              \begin{itemize}
                  \item Wachstumsphase (nur Sperren anfordern)
                  \item Freigabephase (nur Sperren freigeben)
              \end{itemize}
        \item Spätestens bei Transaktionsende muss eine Transaktion all ihre Sperren zurückgeben.
    \end{itemize}
\end{defi}

\begin{bonus}{Lesesperre}
    Besitzt eine Ressource eine \emph{Lesesperre} (\enquote{Shared Lock}), so möchte der Prozess, der diese Sperre gesetzt hat, von der Ressource nur lesen.

    Somit können auch andere Prozesse auf diese Ressource lesend zugreifen, dürfen diese aber nicht verändern.
\end{bonus}

\begin{bonus}{Schreibsperre}
    Besitzt eine Ressource eine \emph{Schreibsperre} (\enquote{Exclusive Lock}), wird verhindert, dass die Ressource von anderen Prozessen gelesen oder geschrieben wird, da der Prozess, der den Lock gesetzt hat, die Ressource verändern möchte.

    Es wird vorausgesetzt, dass keine Lesesperre auf die Ressource gesetzt ist.
\end{bonus}

\begin{sql}{TRANSACTION}
    Um eine Transaktion zu starten, nutzt man:

    \begin{lstlisting}[language=mysql]
        START TRANSACTION;
    \end{lstlisting}

    Nun kann man Daten einfügen, löschen oder verändern bzw. die DBMS-spezifische Struktur anpassen.

    Sollten dabei etwaige Fehler auftreten, welche irreparable Schäden an der Datenbank verursachen, ist erst mal nur die Session der Verursachenden betroffen.

    Um die Transaktion zu abzuschließen, nutzt man:

    \begin{lstlisting}[language=mysql]
        COMMIT;
    \end{lstlisting}

    Alternativ bricht man die Transaktion folgendermaßen ab:

    \begin{lstlisting}[language=mysql]
        ROLLBACK;
    \end{lstlisting}
\end{sql}

\begin{sql}{AUTOCOMMIT}
    Um jeden Befehl sofort zu verarbeiten muss die Einstellung \texttt{AUTOCOMMIT} auf \texttt{1} gesetzt werden.

    Standardmäßig ist der Wert bereits auf \texttt{1}, jedoch nutzen wir um sicher zu gehen:

    \begin{lstlisting}[language=mysql]
        SET AUTOCOMMIT = 1;
    \end{lstlisting}

    Wenn der Wert auf \texttt{0} gesetzt wurde, werden Änderungen nur in der eigenen \texttt{SESSION} vorgenommen.
    Wenn man seine Änderungen veröffentlichen möchte, nutzt man \texttt{COMMIT;}.
\end{sql}

\begin{sql}{FOREIGN\_KEY\_CHECKS}
    Der folgende Befehl schaltet das automatische Überprüfen von Fremdschlüssel-Beziehungen aus:

    \begin{lstlisting}[language=mysql]
        SET FOREIGN_KEY_CHECKS = 0
    \end{lstlisting}

    Analog schaltet man die Überprüfung folgendermaßen wieder ein:

    \begin{lstlisting}[language=mysql]
        SET FOREIGN_KEY_CHECKS = 1
    \end{lstlisting}

    Dieser Befehl birgt auf den ersten Blick große Gefahren, ist jedoch vor allem bei Selbstrelationen nötig, wenn sich noch nicht existierende Entitäten gegenseitig referenzieren.
\end{sql}